\documentclass[11pt]{article}
\usepackage{amsmath,graphicx,color,epsfig,physics}
%\usepackage{pstricks}
\usepackage{float}
\usepackage{subfigure}
\usepackage{slashed}
\usepackage{color}
\usepackage{multirow}
\usepackage{feynmp}
\usepackage[top=1in, bottom=1in, left=1.2in, right=1.2in]{geometry}
\def\del{{\partial}}

\begin{document}
\title{Particle physics HW6}
\author{Yang Ma}

\maketitle

\section{ }
Consider $A \times B = C$ is a linear (matrix) representation of $a\times b = c$, then we can also have
\begin{itemize}
    \item Take the complex conjugate: $(A \times B)^* = C^*$, i.e. $A^* \times B^* = C^*$,
    \item Since $(A \times B)^{-1} =B^{-1} A^{-1} = C^{-1}$, we have $(A^{-1})^T \times (B^{-1})^T = (B^{-1} A^{-1})^T = (C^{-1})^T$,
    \item Take the complex conjugate again, $((A^{-1})^T \times (B^{-1})^T)^* = (C^{-1})^T$, which can be rewritten as $A^{-1})^\dagger \times (B^{-1})^\dagger = (C^{-1})^\dagger$.
\end{itemize}

\section{ }
We can write $p^\mu p_\mu$ in matrix form as
\begin{eqnarray}
    p^\mu p_\mu =
    \begin{pmatrix}
        E&p_x&p_y&p_z
    \end{pmatrix}
    \begin{pmatrix}
        E \\ -p_x \\-p_y\\-p_z
    \end{pmatrix}
    = E^2- p_x^2-p_y^2-p_z^2.
\end{eqnarray}
After the transformation $p^\mu \to L p^\mu$, $p_\mu \to L' p_\mu$, we have
\begin{eqnarray}
    {p'}^\mu p'_\mu &=&
    \begin{pmatrix}
        E'&p'_x&p'_y&p'_z
    \end{pmatrix}
    \begin{pmatrix}
        E' \\ -p'_x \\-p'_y\\-p'_z
    \end{pmatrix} \nonumber \\
    &=&
    (L p^\mu)^T L' p_\mu = (p^\mu)^T L^T L' p_\mu
    = E^2- p_x^2-p_y^2-p_z^2,
\end{eqnarray}
is invariant. To satisty this invariance,
\begin{eqnarray}
    L'=(L^T)^{-1}=(L^{-1})^T.
\end{eqnarray}

\section{ }
We now transform our ``old'' matrix representation into the new tensor representation via
\begin{eqnarray}
    &&q_i \to q^i ~~~ U_{ij} \to U^i_j\\
    &&q_i^* \to q_i ~~~U^*_{ij} \to U^j_i, 
\end{eqnarray}
and have
transformations of $q^i$ and $q_i$ can be expressed as
\begin{eqnarray}
 && q^i \to (q')^i = U^i_j q^j,\\
 && q_i \to (q')_i = U_i^j q_j.
\end{eqnarray}
The invariance of the norm is then written as 
\begin{eqnarray}
q_i q^i \to  (q')_i (q')^i
= (U_i^j q_j) (U^i_k q^k)
=  U_i^j U^i_k  q_j q^k
=  \delta ^j_k   q_j q^k
=  q_k q^k
\end{eqnarray}

\section{ }
The $SU(2)$ transformations are
\begin{eqnarray}
    && \phi \to U \phi ~~ \phi^c \to U \phi^c \\
    && {\phi^c}^* \to U^* {\phi^c}^* ~~~ L \to U L 
\end{eqnarray}
and then we see the invariance of  $(\phi^c)^\dagger L$
\begin{eqnarray}
    (\phi^c)^\dagger L \to (\phi^c)^\dagger U^\dagger U L = (\phi^c)^\dagger L.
\end{eqnarray}
Recall the definition of $\psi^c=i\sigma^2 \psi^*$, we have
\begin{eqnarray}
    (\phi^c)^\dagger L = (i\sigma^2 \psi^*)^\dagger L= (-i\sigma^2 \psi)^T L= \psi^T (-i \sigma^2)L,
\end{eqnarray}
so $\psi^T (-i \sigma^2)L$ is also invariant under $SU(2)$ transformation.

\section{ }
For $\phi = (\phi^+, \phi^0)^T$ and $L = (\nu_L, l_L)^T$, we have
\begin{eqnarray}
    (\phi^c)^\dagger L & = & \phi^T (-i\sigma^2) L = 
    \begin{pmatrix}
        \phi^+ & \phi^0
    \end{pmatrix}
    \begin{pmatrix}
        0 & 1 \\ -1 & 0
    \end{pmatrix}
    \begin{pmatrix}
        \nu_L \\ l_L
    \end{pmatrix}\\
    &=&
    \begin{pmatrix}
        \phi^+ & \phi^0
    \end{pmatrix}
    \begin{pmatrix}
        -l_L\\\nu_L
    \end{pmatrix}
    = -\psi^+ l_L + \psi^0 \nu_L
\end{eqnarray}

\section{ }
For
\begin{eqnarray}
    {\cal L}  =  (\del^\mu \phi)^\dagger (\del_\mu \phi) -m^2 \phi^\dagger \phi,
\end{eqnarray}
which is invariant under $U(1)$ transformation, we write
\begin{eqnarray}
    0 = \delta {\cal L} &=& {\cal L}(\phi', \del_\mu \phi') - {\cal L}(\phi, \del_\mu \phi)\\
  &=&  [\delta {\cal L} / \delta{\phi}] \delta\phi +  [\delta {\cal L} / \delta{\del_\mu \phi}] \delta{\del_\mu \phi}\\
  &=& (\del_\mu [\delta L/\delta{\del_\mu \phi}]) \delta\phi +  [\delta L / \delta{\del_\mu \phi}] (\del_\mu \delta\phi) \\
  &=& \del_\mu ([\delta L/\delta{\del_\mu \phi}] \delta\phi) \\&=&  \del_\mu j(x)^\mu,
\end{eqnarray}
where we used the equation of motion
\begin{eqnarray}
    [\delta {\cal L} / \delta{\phi}]  =  \del_\mu [\delta {\cal L}/\delta{\del_\mu \phi}].
\end{eqnarray}
For infinitesimal transformation
\begin{eqnarray}
    \phi \to (1+i\theta )\phi, ~~~ \phi^\dagger  \to (1-i\theta )\phi^\dagger,
\end{eqnarray}
we have
\begin{eqnarray}
    j^\mu &=& \frac{\delta {\cal L}}{\delta \del_\mu \phi } \delta \phi +\frac{\delta {\cal L}}{\delta \del_\mu \phi^\dagger } \delta \phi^\dagger \\
    &=& i\theta [ (\del_\mu \phi)^\dagger \delta \phi - (\del_mu \phi) \phi^\dagger].
\end{eqnarray}
The charge is
\begin{eqnarray}
    Q=\int d^3x j^0= i\theta \int d^3 x [ (\del_0 \phi)^\dagger \delta \phi - (\del0 \phi) \phi^\dagger].
\end{eqnarray}

\section{ }
We first use a $3\times 3$ complex matrix to represent $O$, so there are $18$ matrix elements. For $O=O^*$, we see that all the elements should be real, i.e. there are only $9$ elements. $O^T=O$ gives that, $O_{ij}=O_{ji}$ and all diagnal elements equal to zero, then there are only $3$ real numbers to paramterize the $SO(3)$ group.

\section{ }

\begin{enumerate}
    \item 
    \begin{itemize}
        \item $O(x,0,0)$, a rotation by an angle x about the x axis,
        \begin{eqnarray}
            O(x,0,0)=
            \begin{pmatrix}
                1& 0         & 0\\
                0&\cos x & \sin x\\
                0&-\sin x & \cos x 
            \end{pmatrix},
        \end{eqnarray}
        whose generator is
        \begin{eqnarray}
            J_x=\frac{1}{i}\frac{\del O(x,0,0)}{\del x}|_{x=0} =
            \begin{pmatrix}
                0&0&0\\
                0&0&i\\
                0&-i&0
            \end{pmatrix} 
        \end{eqnarray}
        \item $O(0,y,0)$, a rotation by an angle y about the y axis,
        \begin{eqnarray}
            O(0,y,0)=
            \begin{pmatrix}
                \cos y& 0    & -\sin y\\
                0&1 & 0\\
                \sin y&0 & \cos y
            \end{pmatrix},
        \end{eqnarray}
        whose generator is
        \begin{eqnarray}
            J_y=\frac{1}{i}\frac{\del O(0,y,0)}{\del y}|_{y=0} =
            \begin{pmatrix}
                0&0&-i\\
                0&0&0\\
                i&0&0
            \end{pmatrix} 
        \end{eqnarray}
        \item $O(0,0,z)$, a rotation by an angle z about the z axis,
        \begin{eqnarray}
            O(0,0,z)=
            \begin{pmatrix}
                \cos z& \sin z    & 0\\
                -\sin z&\cos z & 0\\
               0 &0 & 1
            \end{pmatrix},
        \end{eqnarray}
        whose generator is
        \begin{eqnarray}
            J_z=\frac{1}{i}\frac{\del O(0,0,z)}{\del z}|_{z=0} =
            \begin{pmatrix}
                0&i&0\\
                -i&0&0\\
                0&0&0
            \end{pmatrix} 
        \end{eqnarray}
      \end{itemize} 
    \item
    \begin{eqnarray}
        &&[ J_x, J_y ] = 
        \begin{pmatrix}
            0&0&0\\
            0&0&i\\
            0&-i&0
        \end{pmatrix}
        \begin{pmatrix}
            0&0&-i\\
            0&0&0\\
            i&0&0
        \end{pmatrix}
        -
        \begin{pmatrix}
            0&0&-i\\
            0&0&0\\
            i&0&0
        \end{pmatrix}
        \begin{pmatrix}
            0&0&0\\
            0&0&i\\
            0&-i&0
        \end{pmatrix}
        =i J_z \\
        &&[ J_y, J_z ] = 
        \begin{pmatrix}
            0&0&-i\\
            0&0&0\\
            i&0&0
        \end{pmatrix}
        \begin{pmatrix}
            0&i&0\\
            -i&0&0\\
            0&0&0
        \end{pmatrix} 
        -
        \begin{pmatrix}
            0&i&0\\
            -i&0&0\\
            0&0&0
        \end{pmatrix} 
        \begin{pmatrix}
            0&0&-i\\
            0&0&0\\
            i&0&0
        \end{pmatrix}
        =i J_x \\
        &&[ J_z, J_x ] = 
        \begin{pmatrix}
            0&i&0\\
            -i&0&0\\
            0&0&0
        \end{pmatrix} 
        \begin{pmatrix}
            0&0&0\\
            0&0&i\\
            0&-i&0
        \end{pmatrix}
        -
        \begin{pmatrix}
            0&0&0\\
            0&0&i\\
            0&-i&0
        \end{pmatrix}
        \begin{pmatrix}
            0&i&0\\
            -i&0&0\\
            0&0&0
        \end{pmatrix}
        =i J_y
    \end{eqnarray}

    \item
    Now we can take $J_x$ to obtain the normalization factor
    \begin{eqnarray}
        Tr\{ J_x j_x\} = Tr\{ 
            \begin{pmatrix}
                0&0&0\\
                0&0&i\\
                0&-i&0
            \end{pmatrix}
            \begin{pmatrix}
                0&0&0\\
                0&0&i\\
                0&-i&0
            \end{pmatrix}
            \} =2,
    \end{eqnarray}
    the same result will be derieved if we take $J_y$ or $J_z$.
    \item 
 
    \begin{eqnarray}
        O(x,0,0)= e^{-ixJ_x}= 1-ixJ_x+\frac{1}{2!} (-ixJ_x)^2+\dots=
        \begin{pmatrix}
            1& 0         & 0\\
            0&\cos x & \sin x\\
            0&-\sin x & \cos x 
        \end{pmatrix},
    \end{eqnarray}
   \begin{eqnarray}
        O(0,y,0)= e^{-iyJ_y}= 1-iyJ_y+\frac{1}{2!} (-iyJ_y)^2+\dots=
        \begin{pmatrix}
            \cos y& 0    & -\sin y\\
            0&1 & 0\\
            \sin y&0 & \cos y
        \end{pmatrix},
   \end{eqnarray}
   \begin{eqnarray}
    O(0,0,z)=e^{-izJ_z}= 1-izJ_z+\frac{1}{2!} (-izJ_z)^2+\dots=
    \begin{pmatrix}
        \cos z& \sin z    & 0\\
        -\sin z&\cos z & 0\\
       0 &0 & 1
    \end{pmatrix}.
    \end{eqnarray}
\end{enumerate}

\section{ }
\begin{enumerate}
    \item $2 \times 2$ complex has $8$ degree of freedom (DOF). The requirement $J_k=J_k^\dagger$ reduces the DOF to $4$ by requiring the diagnal elements to be real and the anti-diagnal ones to be equal. $Tr\{ J_k\}=0$ relates the diagnal elements and then the DOF is reduced to $3$, i.e. elements of $SU(2)$ group could be paramterized by $3$ real numbers.
    \item With $J_k = \sigma_k/2$, we have
    \begin{eqnarray}
        [J_i,J_j]=\frac{1}{4} [\sigma_i,\sigma_j]=2i\epsilon_{ijk}\sigma_k,
    \end{eqnarray}
    which can be written as
    \begin{eqnarray}
        &&[ J_x, J_y ] = i J_z, \\
        &&[ J_y, J_z ] = i J_x, \\
        &&[ J_z, J_x ] = i J_y.
    \end{eqnarray}
    \item 
    \begin{eqnarray}
        U(x,0,0)= e^{-ixJ_x}= 1-ixJ_x+\frac{1}{2!} (-ixJ_x)^2+\dots=
        \begin{pmatrix}
            \cos\frac{x}{2}& i\sin\frac{x}{2} \\
            i\sin\frac{x}{2}&\cos\frac{x}{2}
        \end{pmatrix},
    \end{eqnarray}
    \begin{eqnarray}
        U(0,y,0)= e^{-iyJ_y}= 1-iyJ_y+\frac{1}{2!} (-iyJ_y)^2+\dots=
        \begin{pmatrix}
            \cos \frac{y}{2}&  -\sin \frac{y}{2}\\
            \sin \frac{y}{2}&\cos \frac{y}{2}
        \end{pmatrix},
   \end{eqnarray}
   \begin{eqnarray}
    U(0,0,z)&=&e^{-izJ_z}= 1-izJ_z+\frac{1}{2!} (-izJ_z)^2+\dots\\
    &=&
    \begin{pmatrix}
        \cos \frac{z}{2}-i\sin \frac{z}{2}    & 0\\
               0 &\cos\frac{z}{2}-i\sin \frac{z}{2}  
    \end{pmatrix}.
    \end{eqnarray}
\end{enumerate}

\section{ }
Here we need the {\bf Baker–Campbell–Hausdorff formula}
\begin{eqnarray}
    e^X e^Y = e^{X+Y+\frac{1}{2}[X,Y]+\frac{1}{12}[X,[X,Y]]}.
\end{eqnarray}
and then have
\begin{eqnarray}
    e^{ix_k T_k}e^{ix'_l T'_l}=e^{ix_k T_k+ix'_l T'_l+\frac{1}{2}[ix_k T_k,ix'_l T'_l]+\cdots}.
\end{eqnarray}
We see
\begin{eqnarray}
    [ix_k T_k,ix'_l T'_l]=x_k x'_l [ T'_l, T_k]=ix_k x'_l f_{lk}^mT_m
\end{eqnarray}
is a linear combination of $i T_m$, which indicates that  $e^{ix_k T_k}e^{ix'_l T'_l}$ is also a transformation.

\section{ }
\begin{itemize}
    \item Using {\bf Jacobi's formula}
    \begin{eqnarray}
        {\rm det}\{e^{tB}\}=e^{Tr\{tB\}},
    \end{eqnarray}
    we see if ${\rm det}\{A\}={\rm det}\{e^{x_kT_k}\}=1$, then $e^{Tr\{x_kT_k\}}=1$ is true for arbitrary $x_k$, implying $Tr\{T_k\}=0$.
    \item With
    \begin{eqnarray}
        A^\dagger = e^{-ix_kT_k^\dagger}= A^{-1}=e^{-x_kT_k},
    \end{eqnarray}
    we see $T^\dagger _k = T_k$.
\end{itemize}

\section{ }
A complex matrix $M$ has $2n^2$ elements. The condition $M^\dagger =M$ will reduce the DOF to $n^2$ by requiring $M_{ij}=M_{ji}^*$. Then $Tr\{M\}=0$ will reduce DOF to $n^2-1$ since one of the diagnal element could be written as a linear combination of the rest diagnal elements.


\end{document}
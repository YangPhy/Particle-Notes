\documentclass[11pt]{article}
\usepackage{amsmath,graphicx,color,epsfig,physics}
%\usepackage{pstricks}
\usepackage{float}
\usepackage{subfigure}
\usepackage{slashed}
\usepackage{color}
\usepackage{multirow}
\usepackage{feynmp}
\usepackage[top=1in, bottom=1in, left=1.2in, right=1.2in]{geometry}
\begin{document}
\title{Particle physics HW4}
\author{Yang Ma}

\maketitle

\section{ }
Write $Q\to Q'= UQ$ as
\begin{eqnarray}
  \begin{pmatrix}
    Q_{1k}\\ Q_{2k} \\ Q_{3k}
  \end{pmatrix}
    \to 
  \begin{pmatrix}
    U_{11} & U_{12} & U_{13} \\
    U_{21} & U_{32} & U_{23} \\
    U_{31} & U_{32} & U_{33}
  \end{pmatrix}
  \begin{pmatrix}
    Q_{1k}\\ Q_{2k} \\ Q_{3k}
  \end{pmatrix},
\end{eqnarray}
we then have 
\begin{eqnarray}
  \begin{pmatrix}
    Q'_{1k}\\ Q'_{2k} \\ Q'_{3k}
  \end{pmatrix}
  =
  \begin{pmatrix}
    U_{11} Q_{1k}+ U_{12} Q_{2k} +U_{13} Q_{3k} \\
    U_{21} Q_{1k}+ U_{22} Q_{2k} +U_{23} Q_{3k} \\
    U_{31} Q_{1k}+ U_{32} Q_{2k} +U_{13} Q_{3k} 
  \end{pmatrix}
\end{eqnarray}

\section{ }
For $SU(2)$
\begin{eqnarray}
  U(\theta_1,\theta_2,\theta_3) = \exp{ i \sum_{k=1}^3 T^k \theta^k},
\end{eqnarray}
where $T^k= \sigma^k/2$. 
Since we roved 
\begin{eqnarray}
  e^{i \sigma^2 \theta}=
  \begin{pmatrix}
    \cos \theta & -\sin \theta\\
    \sin \theta & \cos \theta
  \end{pmatrix},
\end{eqnarray}
 we can write
\begin{eqnarray}
  U(0,\pi,0)&=&e^{i \pi \sigma_2 /2} \\
  &=&
  \begin{pmatrix}
    \cos \pi/2 & -\sin \pi/2\\
    \sin \pi/2 & \cos \pi/2
  \end{pmatrix}
  =  \begin{pmatrix}
    0 & -1\\
    1 & 0
  \end{pmatrix},
\end{eqnarray}
and
\begin{eqnarray}
  U(0,\pi,0) Q =
  \begin{pmatrix}
    0 & -1\\
    1 & 0
  \end{pmatrix}
  \begin{pmatrix}
    u_L\\
    d_L
  \end{pmatrix}
  =
  \begin{pmatrix}
    -d_L\\
    u_L
  \end{pmatrix}
\end{eqnarray}

\section{ }
Under $U(1)_Y$, all 6 components of Q transform exactly the same way with $Y=1/6$, 
\begin{eqnarray}
  Q_{ik} \to Q'_{ik} = U Q_{ik}
                 = e^{iY\theta} Q_{ik}
                 = e^{i(1/6)\theta} Q_{ik}.
\end{eqnarray}
Then we have $U(3\pi)=e^{i\pi/2}=i$ and can write out
\begin{eqnarray}
  &&  Q \to Q'  = U_Y(3\pi) Q = iQ\\
  && u_R \to u_R' = U_Y(3\pi) u_R = i u_R\\
  && d_R \to d_R' = U_Y(3\pi) d_R = i d_R\\
  && L  \to L'  = U_Y(3\pi) L= iL\\
  && l_R \to l_R' = U_Y(3\pi) =  il_R.
\end{eqnarray}

\section{ }
\begin{itemize}
  \item Since $\phi$ is singlet under $SU(3)$ transformation, i.e., $U \phi \to \phi$, it is obvious that $\phi^\dagger \phi$ in invariatn under this transformation.
  \item For $SU(2)$, where $ U^\dagger U=1$ we see
  \begin{eqnarray}
    \phi & \to & \phi'=U \phi \\
    \phi^\dagger & \to & {\phi^\dagger}'= \phi^\dagger U^\dagger,
  \end{eqnarray}
  so
  \begin{eqnarray}
    {\phi^\dagger}' \phi' = \phi^\dagger U^\dagger U \phi = \phi^\dagger \phi.
  \end{eqnarray}
  \item For $U(1)$,
  \begin{eqnarray}
    \phi \to \phi' = U \phi = e^{i \theta Y} \phi, \\
    \phi^* \to {\phi^*}' = U^* \phi^* = e^{-i \theta Y} \phi^*,
  \end{eqnarray}
  where $Y=1/2$. It is not hard to see
  \begin{eqnarray}
    {\phi^\dagger}' \phi' ={\phi^*}' \phi' = e^{-i \theta Y} \phi^* e^{i \theta Y} \phi = \phi^* \phi= {\phi^\dagger}\phi
  \end{eqnarray}
\end{itemize}

\section{ }
For
\begin{eqnarray}
  \phi^c &=& i \sigma^2 \phi^* \\
  {\phi^c}^\dagger &=& - i  \phi^T {\sigma^2} ^\dagger,
\end{eqnarray}
we write
\begin{eqnarray}
  {\phi^c}^\dagger \phi^c &=&  - i  \phi^T {\sigma^2} ^\dagger i \sigma^2 \phi^* \\
  &=& \phi^T \phi^*= \phi^\dagger \phi.
\end{eqnarray}
In the last step I used the relation $\sigma^2 \sigma^2 =1$.

\section{ }
Under a new global $SU(2)$ transformation
\begin{eqnarray}
  \begin{pmatrix}
    \phi \\ \phi^c
  \end{pmatrix}
  \to U
  \begin{pmatrix}
    \phi \\ \phi^c
  \end{pmatrix},
\end{eqnarray}
one has 
\begin{eqnarray}
  \begin{pmatrix}
    \phi \\ \phi^c
  \end{pmatrix} ^\dagger U^\dagger U 
  \begin{pmatrix}
    \phi \\ \phi^c
  \end{pmatrix}
  =
  \begin{pmatrix}
    \phi \\ \phi^c
  \end{pmatrix} ^\dagger
  \begin{pmatrix}
    \phi \\ \phi^c
  \end{pmatrix},
\end{eqnarray}
where $U^\dagger U=1$.

\section{ }
The $SU(3)$ transformation satisfies $U^\dagger U=1$.
\begin{itemize}
  \item $y_u Q^\dagger u_R \phi^c$:
  \begin{eqnarray}
    y_u Q^\dagger u_R \phi^c \to y_u Q^\dagger U^\dagger U u_R \phi^c = y_u Q^\dagger u_R \phi^c.
  \end{eqnarray}
  \item $y_d Q^\dagger d_R \phi$:
  \begin{eqnarray}
    y_d Q^\dagger d_R \phi \to y_d Q^\dagger U^\dagger U d_R \phi = y_d Q^\dagger d_R \phi.
  \end{eqnarray}
  \item $y_l L^\dagger l_R \phi$:
  \begin{eqnarray}
    y_l L^\dagger l_R \phi \to y_l L^\dagger U^\dagger U l_R \phi = y_l L^\dagger l_R \phi.
  \end{eqnarray}
\end{itemize}

\section{ }
The $SU(2)$ transformation satisfies $U^\dagger U=1$.
\begin{itemize}
  \item $y_u Q^\dagger u_R \phi^c$:
  \begin{eqnarray}
    y_u Q^\dagger u_R \phi^c \to y_u Q^\dagger U^\dagger u_R U \phi^c = y_u Q^\dagger u_R \phi^c.
  \end{eqnarray}
  \item $y_d Q^\dagger d_R \phi$:
  \begin{eqnarray}
    y_d Q^\dagger d_R \phi \to y_d Q^\dagger U^\dagger d_R U \phi = y_d Q^\dagger d_R \phi.
  \end{eqnarray}
  \item $y_l L^\dagger l_R \phi$:
  \begin{eqnarray}
    y_l L^\dagger l_R \phi \to y_l L^\dagger U^\dagger l_R U \phi = y_l L^\dagger l_R \phi.
  \end{eqnarray}
\end{itemize}

\section{ }
The $U(1)$ transformation $U=e^{i Y \theta}$satisfies $U^* U=1$.
\begin{itemize}
  \item $y_u Q^\dagger u_R \phi^c$:
  \begin{eqnarray}
    y_u Q^\dagger u_R \phi^c \to y_u e^{-i \theta/6} Q^\dagger e^{2i \theta/3} u_R e^{-i \theta/2}\phi^c = y_u Q^\dagger u_R \phi^c.
  \end{eqnarray}
  \item $y_d Q^\dagger d_R \phi$:
  \begin{eqnarray}
    y_d Q^\dagger d_R \phi \to y_d e^{-i \theta/6} Q^\dagger e^{-i \theta/3} d_R e^{i \theta/2} \phi = y_d Q^\dagger d_R \phi.
  \end{eqnarray}
  \item $y_l L^\dagger l_R \phi$:
  \begin{eqnarray}
    y_l L^\dagger l_R \phi \to y_l e^{i \theta/2} L^\dagger e^{-i \theta} l_R e^{i \theta/2}\phi = y_l L^\dagger l_R \phi.
  \end{eqnarray}
\end{itemize}

\section{ }
\begin{eqnarray}
  (y_u Q^\dagger u_R \phi^c )^\dagger &=& y_u^*  u_R^* {\phi^c}^\dagger Q \\
  (y_d Q^\dagger d_R \phi )^\dagger &=& y_d^*  d_R^* {\phi}^\dagger Q \\
  (y_l L^\dagger l_R \phi )^\dagger &=& y_l^*  l_R^* {\phi}^\dagger L.
\end{eqnarray}

\section{ }
\begin{itemize}
  \item Since both $\phi$ and $L$ are singlet under $SU(3)$ transformation, ${\phi^c}^\dagger L$ is invariant.
  \item For $SU(2)$,
  \begin{eqnarray}
    {{\phi^c}^\dagger}' L'= {\phi^c}^\dagger U^\dagger U L={\phi^c}^\dagger L.
  \end{eqnarray}
  \item For $U(1)$,
  \begin{eqnarray}
    {{\phi^c}^\dagger}' L'= {\phi^c}^\dagger e^{i\theta/2} e^{-i\theta/2} L={\phi^c}^\dagger L.
  \end{eqnarray}
\end{itemize}

\section{ }
By definition
\begin{eqnarray}
  \phi^c=i \sigma^2 \phi^* = i 
  \begin{pmatrix}
    0 & -i \\ i & 0
  \end{pmatrix}
  \begin{pmatrix}
    \phi^- \\ {\phi^0}^*
  \end{pmatrix}
  =
  \begin{pmatrix}
    {\phi^0}^* \\ -\phi^- 
  \end{pmatrix},
\end{eqnarray}
then we wcan write
\begin{eqnarray}
  {\phi^c}^\dagger = 
  \begin{pmatrix}
    \phi^0 & -(\phi^-)^*
  \end{pmatrix}
  =
  \begin{pmatrix}
    \phi^0 & -\phi^+
  \end{pmatrix},
\end{eqnarray}
and
\begin{eqnarray}
  {\phi^c}^\dagger L 
  =
  \begin{pmatrix}
    \phi^0 & -\phi^+
  \end{pmatrix}
  \begin{pmatrix}
    \nu_L \\ l_L
  \end{pmatrix}
  = \phi^0 \nu_L - \phi^\dagger l_L.
\end{eqnarray}

\section{ }
\begin{eqnarray}
  y_\nu [ (\phi^c)^\dagger L) ]^2/2\Lambda &=& y_\nu (\phi^0 \nu_L - \phi^\dagger l_L)^2/2\Lambda\\
  &=& y_\nu((\phi^0)^2 \nu_L^2+ (\phi^\dagger)^2 l_L^2 -2 \phi^0\phi^\dagger\nu_L\l_L)/2\Lambda.
\end{eqnarray}




\end{document}
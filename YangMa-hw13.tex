\documentclass[11pt]{article}
\usepackage{amsmath,graphicx,color,epsfig,physics}
%\usepackage{pstricks}
\usepackage{float}
\usepackage{subfigure}
\usepackage{slashed}
\usepackage{color}
\usepackage{multirow}
\usepackage{feynmp}
\usepackage[top=1in, bottom=1in, left=1.2in, right=1.2in]{geometry}
\def\del{{\partial}}
\begin{document}
\title{Particle physics HW13}
\author{Yang Ma}

\maketitle

\section{ }
\begin{eqnarray}
    S_L(\theta_k,\eta_k)
  = e^{\sum_k \sigma^k (-i\theta_k -\eta_k)/2} = e^{-i\frac{\sigma^k}{2}(\theta_k-i\eta_k)} =L_A \\ 
  S_R(\theta_k,\eta_k)
  = e^{\sum_k \sigma^k (-i\theta_k +\eta_k)/2} =e^{-i\frac{\sigma^k}{2}(\theta_k+i\eta_k)} =L_B
\end{eqnarray}

\section{ }
\begin{eqnarray}
    &&(S_L)^\dagger (S_R) = e^{\sum_k {\sigma^k}^\dagger (i\theta_k -\eta_k)/2} e^{\sum_k \sigma^k (-i\theta_k +\eta_k)/2} = 1 \\
    &&(S_R)^\dagger (S_L) = e^{\sum_k {\sigma^k}^\dagger (i\theta_k +\eta_k)/2}e^{\sum_k \sigma^k (-i\theta_k -\eta_k)/2}=1
\end{eqnarray}

\section{ }
\begin{itemize}
    \item  Apply the $U(1)_{EM}$ transformation:
    \begin{eqnarray}
    &&\psi_L \to \psi_L' = e^{iQ\theta} \psi_L   \\
    &&\psi_R \to  \psi_R' = e^{iQ\theta} \psi_R 
    \end{eqnarray}
    to the Dirac mass term 
    \begin{eqnarray}
        \psi_L^\dagger \psi_R + \psi_R^\dagger \psi_L \to \psi_L^\dagger e^{-iQ\theta}e^{iQ\theta}\psi_R + \psi_R^\dagger e^{-iQ\theta}e^{iQ\theta} \psi_L =  \psi_L^\dagger \psi_R + \psi_R^\dagger \psi_L,
    \end{eqnarray}
    we see it is invariant.
    \item Under the Lorentz transformation
    \begin{eqnarray}
        \psi_L   \to \psi_L'   = S_L \psi_L ,
    \end{eqnarray}
    we have 
    \begin{eqnarray}
        \psi_L^c &=& i \sigma^2 \psi^*_L \to {\psi_L^c}' = i \sigma^2  (S_L \psi_L)^* = i \sigma^2  S_L^* \psi_L^* \\
        &=&i \sigma^2  S_L^* (-i \sigma^2)(i \sigma^2)\psi_L^* = \sigma^2 S_L^*\sigma^2 \psi_L^c \\
        &=& e^{-\frac{\sigma^k}{2}(i\theta_k-\eta_k)}\psi_L^c= S_R \psi_L^c 
    \end{eqnarray}
\end{itemize}

\section{ }
By doing the phase transformation
\begin{eqnarray}
    && \psi_L   \to \psi_L'   = e^{ iQ\theta} \psi_L,    \\
    && \psi_L^c \to {\psi_L^c}' = e^{-iQ\theta} \psi_L^c , 
\end{eqnarray}
we see
\begin{eqnarray}
    &&{\psi_L^c}^\dagger \psi_L \to {\psi_L^c}^\dagger e^{i\theta} e^{i\theta} \psi_L = e^{ iQ\theta} {\psi_L^c}^\dagger \psi_L, ~~~Q=2;\\ 
    &&\psi_L^\dagger {\psi_L^c}  \to \psi_L^\dagger e^{-i\theta} e^{-i\theta} {\psi_L^c} = e^{ iQ\theta} \psi_L^\dagger {\psi_L^c}, ~~~Q=-2.
\end{eqnarray}

\section{ }
With the Dirac Lagrangian
\begin{eqnarray}
    {\cal L}_{Dirac} = \psi_L^\dagger i\del_\mu \sigma_-^\mu \psi_L
    + \psi_R^\dagger i\del_\mu \sigma_+^\mu \psi_R -m (\psi_L^\dagger \psi_R + \psi_R^\dagger \psi_L), 
\end{eqnarray}
we have
\begin{eqnarray}
    \frac{\del {\cal L}_{Dirac} }{\del \psi_L^\dagger } = i\del^\mu \sigma_-^\mu \psi_L -m \psi_R, ~~~ \frac{\del {\cal L}_{Dirac} }{\del \del_\mu \psi_L^\dagger }=0; \\
    \frac{\del {\cal L}_{Dirac} }{\del \psi_R^\dagger } = i\del^\mu \sigma_+^\mu \psi_R -m \psi_L, ~~~ \frac{\del {\cal L}_{Dirac} }{\del \del_\mu \psi_R^\dagger }=0,
\end{eqnarray}
so the e.o.m is
\begin{eqnarray}
    &&i\del_\mu \sigma_-^\mu \psi_L = m \psi_R ,  \\
    && i\del_\mu \sigma_+^\mu \psi_R = m \psi_L .
\end{eqnarray}

\section{ }
Now we do the Lorentz transformation on ${\cal L}_{kin}$ 
\begin{eqnarray}
    \psi_L^\dagger i \del_\mu \sigma_-^\mu \psi_L \to \psi_L^\dagger e^{-iQ\theta} i \del_\mu \sigma_-^\mu e^{iQ\theta} \psi_L =\psi_L^\dagger i \del_\mu \sigma_-^\mu \psi_L, \\
    \psi_R^\dagger i \del_\mu \sigma_+^\mu \psi_R \to \psi_R^\dagger e^{-iQ\theta} i \del_\mu \sigma_+^\mu e^{iQ\theta} \psi_R =\psi_R^\dagger i \del_\mu \sigma_+^\mu \psi_R,
\end{eqnarray}
and see it is invariant.

\section{ }
With the $6$ generators of Lorentz transformation
\begin{eqnarray}
    J_1 = 
    \begin{pmatrix}
      0 & 0 & 0 & 0\\
      0 & 0 & 0 & 0 \\
      0 & 0 & 0 & -i \\
      0 & 0 & i & 0
    \end{pmatrix}
    ,~~ J_2 =
    \begin{pmatrix}
      0 & 0 & 0 & 0 \\
      0 & 0 & 0 & i \\
      0 & 0 & 0 & 0 \\
      0 & -i & 0 & 0 
    \end{pmatrix}
    ,~~J_3=
    \begin{pmatrix}
      0 & 0 & 0 & 0 \\
      0 & 0 & -i & 0 \\
      0 & i & 0 & 0 \\
      0 & 0 & 0 & 0 
    \end{pmatrix}
  \end{eqnarray}
  \begin{eqnarray}
    K_1 =
    \begin{pmatrix}
       0 & i & 0 & 0\\ 
       i & 0 & 0 & 0 \\
       0 & 0 & 0 & 0 \\
       0 & 0 & 0 & 0 
    \end{pmatrix}
    ,~~K_2 =
    \begin{pmatrix}
      0 & 0 & i & 0\\ 
      0 & 0 & 0 & 0 \\
      i & 0 & 0 & 0 \\
      0 & 0 & 0 & 0
    \end{pmatrix}
     ,~~K_3=
     \begin{pmatrix}
      0 & 0 & 0 & i \\
      0 & 0 & 0 & 0 \\
      0 & 0 & 0 & 0 \\
      i & 0 & 0 & 0 
     \end{pmatrix} 
  \end{eqnarray}
we can write the infinitesimal Lorentz transformation
\begin{eqnarray}
    L(\theta_k,\eta_k)
    &=& e^{-i( J_1 \theta_1 + J_2 \theta_2 + J_3 \theta_3
            +K_1 \eta_1   + K_2 \eta_2   + K_3 \eta_3  )} \\
    &=& 1  -i( J_1 \theta_1 + J_2 \theta_2 + J_3 \theta_3
            +K_1 \eta_1   + K_2 \eta_2   + K_3 \eta_3 ) + h.o.\\
    &=& 
    \begin{pmatrix}
        1 & \eta_1 & \eta_2 & \eta_3 \\
        \eta_1 & 1 & -\theta_3 & \theta_2 \\
        \eta_2 & \theta_3 & 1 & \theta_2 \\
        \eta_3 & -\theta_2 & \theta_1 & 1
    \end{pmatrix}.     
  \end{eqnarray}

\section{ }
\begin{itemize}
    \item The commutation and anti-commutation relations
    For
    \begin{eqnarray}
        \sigma_-^\mu=(1,-\sigma^1,-\sigma^2,-\sigma^3),
    \end{eqnarray}
    we see
    \begin{enumerate}
        \item $\mu=0$
        \begin{eqnarray}
            &&[\sigma^k,\sigma_-^0]=[\sigma^k,I]=0, \\
            &&\{\sigma^k,\sigma_-^0\}= \{\sigma^k,I \}= 2 \sigma^k.
        \end{eqnarray}
        \item $\mu=j\neq 0$
        \begin{eqnarray}
            &&[\sigma^k,\sigma_-^j]=[\sigma^k,-\sigma^j]=2i\epsilon_{jkl}\sigma^l, \\
            &&\{\sigma^k,\sigma_-^j\}= \{\sigma^k,-\sigma^j \}= -2 \delta_{jk}.
        \end{eqnarray}
    \end{enumerate}
    \item Matrix representation of infinitesimal Lorentz transformation
    \begin{enumerate}
        \item $\mu=0$
        \begin{eqnarray}
            {\sigma_-'}^0 =\sigma_-^0-\sum_{k=1}^3\sigma^k\eta_k=\sigma_-^0+\sum_{k=1}^3\sigma_-^k\eta_k
        \end{eqnarray}
        \item $\mu=1$
        \begin{eqnarray}
            {\sigma_-^1}'=\eta_1\sigma_-^0+\sigma_-^1-\theta_3\sigma_-^2+\theta_2\sigma_-^3
        \end{eqnarray}
        \item $\mu=2$
        \begin{eqnarray}
            {\sigma_-^2}'=\eta_2\sigma_-^0+\theta_3\sigma_-^1+\sigma_-^2-\theta_1\sigma_-^3
        \end{eqnarray}
        \item $\mu=3$
        \begin{eqnarray}
            {\sigma_-^3}'=\eta_3\sigma_-^0-\theta_2\sigma_-^1+\theta_1\sigma_-^2+\sigma_-^3
        \end{eqnarray}
    \end{enumerate}
    As a summary, we see
    \begin{eqnarray}
        {\sigma_-^\mu}'=
        \begin{pmatrix}
            1 & \eta_1 & \eta_2 & \eta_3 \\
            \eta_1 & 1 & -\theta_3 & \theta_2 \\
            \eta_2 & \theta_3 & 1 & \theta_2 \\
            \eta_3 & -\theta_2 & \theta_1 & 1
        \end{pmatrix}
        \sigma_-^\mu.
    \end{eqnarray}
\end{itemize}


\section{ }
Now we move to check
\begin{eqnarray}
    J_R^\mu
    &=& \psi_R^\dagger \sigma_+^\mu \psi_R
    \to J_R'^\mu
    = \psi_R'^\dagger \sigma_+^\mu \psi_R'
    = (S_R \psi_R)^\dagger \sigma_+^\mu (S_R \psi_R)
    = \psi_R^\dagger (S_R)^\dagger \sigma_+^\mu (S_R) \psi_R,
  \end{eqnarray}
  where
  \begin{eqnarray}
    \sigma_+^\mu=(1,\sigma^1,\sigma^2,\sigma^3),
\end{eqnarray}
and see
\begin{eqnarray}
    \sigma_+^\mu
      \to {\sigma_+^\mu}'
      &=& (S_R)^\dagger \sigma_+^\mu (S_R) \\
      &=& (1 + \sum_k \sigma^k (+i\theta_k +\eta_k)/2 )
        \sigma_+^\mu
        (1 + \sum_k \sigma^k (-i\theta_k +\eta_k)/2 ) \\ 
      &=& \sigma_+^\mu
       +\sum_k \frac{i\theta_k}{2} [\sigma^k,\sigma_-^\mu]
       +\sum_k \frac{\eta_k}{2}  \{ \sigma^k,\sigma_-^\mu \} + h.l. 
\end{eqnarray}
Using the commutation and anti-commutation relations we have shown in {\bf hw13-8}, we write
\begin{enumerate}
    \item $\mu=0$
    \begin{eqnarray}
        {\sigma_+'}^0 =\sigma_+^0+\sum_{k=1}^3\sigma^k\eta_k=\sigma_+^0+\sum_{k=1}^3\sigma_+^k\eta_k
    \end{eqnarray}
    \item $\mu=1$
    \begin{eqnarray}
        {\sigma_+^1}'=\eta_1\sigma_+^0+\sigma_+^1-\theta_3\sigma_+^2+\theta_2\sigma_+^3
    \end{eqnarray}
    \item $\mu=2$
    \begin{eqnarray}
        {\sigma_+^2}'=\eta_2\sigma_+^0+\theta_3\sigma_+^1+\sigma_+^2-\theta_1\sigma_+^3
    \end{eqnarray}
    \item $\mu=3$
    \begin{eqnarray}
        {\sigma_+^3}'=\eta_3\sigma_+^0-\theta_2\sigma_+^1+\theta_1\sigma_+^2+\sigma_+^3
    \end{eqnarray}
\end{enumerate}
The results agree with
\begin{eqnarray}
    {\sigma_+^\mu}'=
    \begin{pmatrix}
        1 & \eta_1 & \eta_2 & \eta_3 \\
        \eta_1 & 1 & -\theta_3 & \theta_2 \\
        \eta_2 & \theta_3 & 1 & \theta_2 \\
        \eta_3 & -\theta_2 & \theta_1 & 1
    \end{pmatrix}
    \sigma_+^\mu.
\end{eqnarray}

\section{ }
To make sure one combination $A$ transforms as a vector, it should satisfy
\begin{eqnarray}
    A \to A'=
    \begin{pmatrix}
        1 & \eta_1 & \eta_2 & \eta_3 \\
        \eta_1 & 1 & -\theta_3 & \theta_2 \\
        \eta_2 & \theta_3 & 1 & \theta_2 \\
        \eta_3 & -\theta_2 & \theta_1 & 1
    \end{pmatrix}
    A.
\end{eqnarray}
However, none of these combinations have the right sign to satisfy above relation.

\section{ }
Define $B=V_\mu \sigma_+^\mu\psi_R$ and we have $\psi_R^\dagger B$ is Lorentz invariant
\begin{eqnarray}
    \psi_R^\dagger B \to \psi_R\dagger S_R^\dagger B',
\end{eqnarray}
which implies that $B\to B'=S_L B$ under Lorentz transformation, same as $\psi_L\to\psi_L'=S_L \psi_L$.

\section{ }
Recall the definition
\begin{eqnarray}
    \Psi(x) =
    \begin{pmatrix}
      \psi_L(x) \\ \psi_R(x)
    \end{pmatrix},
  \end{eqnarray}
we have
\begin{eqnarray}
    {\cal L}_{Dirac} &=& {\overline \Psi} i\del \gamma \Psi - m {\overline \Psi} \Psi \\
    &=&
    \begin{pmatrix}
        \psi_R^\dagger & \psi_L^\dagger
    \end{pmatrix}
    i\del_\mu 
    \begin{pmatrix}
        0 & \sigma_+^\mu \\
        \sigma_-^\mu &0
    \end{pmatrix}
    \begin{pmatrix}
        \psi_L \\ \psi_R
    \end{pmatrix}
    -m
    \begin{pmatrix}
        \psi_R^\dagger & \psi_L^\dagger
    \end{pmatrix}
    \begin{pmatrix}
        \psi_L \\ \psi_R
    \end{pmatrix}\\
    &=&\psi_L^\dagger i\del_\mu \sigma_-^\mu \psi_L 
    + \psi_R^\dagger i\del_\mu \sigma_+^\mu \psi_R
    -m (\psi_L^\dagger \psi_R + \psi_R^\dagger \psi_L) 
\end{eqnarray}

\section{ }
We now rewrite the equation $( i\del \gamma - m ) \Psi(x) = 0 $
\begin{eqnarray}
    ( i\del \gamma - m ) \Psi(x) &=& 
    i\del_\mu 
    \begin{pmatrix}
        0 & \sigma_+^\mu \\
        \sigma_-^\mu &0
    \end{pmatrix}
    \begin{pmatrix}
        \psi_L \\ \psi_R
    \end{pmatrix}
    -m
    \begin{pmatrix}
        \psi_L \\ \psi_R
    \end{pmatrix} \\
    &=&i\del_\mu
    \begin{pmatrix}
        \sigma_+^\mu \psi_R \\
        \sigma_-^\mu \psi_L
    \end{pmatrix}
    -m
    \begin{pmatrix}
        \psi_L \\ \psi_R
    \end{pmatrix}
    =0
\end{eqnarray}
and see it is equivalent to
\begin{eqnarray}
    &&i\del^\mu \sigma_-^\mu \psi_L = m \psi_R  \\
    && i\del^\mu \sigma_+^\mu \psi_R = m \psi_L
\end{eqnarray}

\section{ }
Since we know that multiplying $i\del \gamma + m $ to the left of
\begin{eqnarray}
    ( i\del \gamma - m ) \Psi(x) = 0 
\end{eqnarray}
leads to
\begin{eqnarray}
    [ i\del \gamma + m ] [ i\del \gamma - m ] \Psi(x)
    = [ \del_\mu \del_\nu \gamma^\mu \gamma^\nu + m^2 ] \Psi(x) = [ \del_\mu \del^\mu + m^2 ] \Psi(x) = 0. 
\end{eqnarray}
We see
\begin{eqnarray}
    \del_\mu \del^\mu=\del_\mu \del_\nu \gamma^\mu \gamma^\nu = \frac{1}{2}(\del_\mu \del_\nu \gamma^\mu \gamma^\nu+\del_\nu \del_\mu \gamma^\nu \gamma^\mu)=\frac{1}{2} \del_\mu \del_\nu \{\gamma^\mu,\gamma^\nu\},
\end{eqnarray}
implying
\begin{eqnarray}
    \{\gamma^\mu,\gamma^\nu\}=2g^{\mu\nu}
\end{eqnarray}

\section{ }
First we prove $\sigma_+^a \sigma_-^b + \sigma_+^b \sigma_-^a = 2 g^{ab}$:
\begin{enumerate}
    \item $a=b=1$: $\sigma_+^a \sigma_-^b + \sigma_+^b \sigma_-^a = 1+1=2$;
    \item $a=1$, $b\neq 1$: $\sigma_+^a \sigma_-^b + \sigma_+^b \sigma_-^a = -\sigma^b+\sigma^b=0$;
    \item $a\neq 1$, $b=1$: $\sigma_+^a \sigma_-^b + \sigma_+^b \sigma_-^a = \sigma^a-\sigma^a=0$;
    \item $a\neq 1$, $b\neq 1$: $\sigma_+^a \sigma_-^b + \sigma_+^b \sigma_-^a = -\{ \sigma^a,\sigma^b\}=-a\delta_{ab}$.
\end{enumerate}
Likelywise, we can also prove $\sigma_-^a \sigma_+^b + \sigma_-^b \sigma_+^a = 2 g^{ab}$.

In Weyl representation
\begin{eqnarray}
    \{\gamma^\mu,\gamma^\nu\}&=&\gamma^\mu\gamma^\nu+\gamma^\nu\gamma^\mu =
    \begin{pmatrix}
        0 & \sigma_+^\mu \\
        \sigma_-^\mu & 0
    \end{pmatrix}
    \begin{pmatrix}
        0 & \sigma_+^\nu \\
        \sigma_-^\nu & 0
    \end{pmatrix}
    +
    \begin{pmatrix}
        0 & \sigma_+^\nu \\
        \sigma_-^\nu & 0
    \end{pmatrix}
    \begin{pmatrix}
        0 & \sigma_+^\mu \\
        \sigma_-^\mu & 0
    \end{pmatrix}\\
    &=&
    \begin{pmatrix}
         \sigma_+^\mu\sigma_-^\nu+ \sigma_+^\nu\sigma_-^\mu  & 0\\
        0& \sigma_-^\mu\sigma_+^\nu+ \sigma_-^\nu\sigma_+^\mu
    \end{pmatrix}
    =2g^{\mu\nu}
\end{eqnarray}

\section{ }
In Weyl representation
\begin{eqnarray}
    &&{\gamma^0}^\dagger =
    \begin{pmatrix}
        0 & 1\\
        1 & 0
    \end{pmatrix}^\dagger
    =
    \begin{pmatrix}
        0 & 1\\
        1 & 0
    \end{pmatrix}
    =\gamma^0 \\
    && {\gamma^i}^\dagger =
    \begin{pmatrix}
        0 &\sigma^i\\
        -\sigma^i & 0
    \end{pmatrix}^\dagger
    =
    \begin{pmatrix}
        0 &-{\sigma^i}^\dagger\\
        {\sigma^i}^\dagger & 0
    \end{pmatrix}
    =-
    \begin{pmatrix}
        0 &\sigma^i\\
        -\sigma^i & 0
    \end{pmatrix}
    =-\gamma^i,
\end{eqnarray}
where we used ${\sigma^i}^\dagger =\sigma^i$ in the last step.

\section{ }
\begin{itemize}
    \item For a given value of $\mu$, we can anti-commute the $\gamma^\mu$ in each term all the way to the corresponding $\gamma$ in $\gamma^5$. In each anti-commutation, we pick up a negative sign. There are even anti-commutations in one term, and odd in the other, and thus they always cancel. Take $\mu = 0$ as an example
    \begin{eqnarray}
        \{\gamma^5,\gamma^0\}=i(\gamma^0 \gamma^1 \gamma^2 \gamma^3 \gamma^0+\gamma^0 \gamma^0 \gamma^1 \gamma^2 \gamma^3)=i(-\gamma^0 \gamma^0 \gamma^1 \gamma^2 \gamma^3+\gamma^0 \gamma^0 \gamma^1 \gamma^2 \gamma^3)=0.
    \end{eqnarray}
    \item 
    \begin{eqnarray}
        {\gamma^5}^2&=&-\gamma^0 \gamma^1 \gamma^2 \gamma^3 \gamma^0 \gamma^1 \gamma^2 \gamma^3 =-(-1)^3\gamma^0 \gamma^0 (-1)^2 \gamma^1 \gamma^1 (-1)\gamma^2 \gamma^2 \gamma^3 \gamma^3 \\ 
        &=& -\gamma^0 \gamma^0  \gamma^1 \gamma^1 \gamma^2 \gamma^2 \gamma^3 \gamma^3 =1,
    \end{eqnarray}
    where we used ${\gamma\mu}^2=-1$ in the last step.
\end{itemize}

\section{ }
\begin{itemize}
    \item In Chiral representation (Weyl representation),
    \begin{eqnarray}
        \gamma^5 &=& i \gamma^0 \gamma^1 \gamma^2 \gamma^3 = i
        \begin{pmatrix}
            0 & 1\\
            1 & 0
        \end{pmatrix}
        \begin{pmatrix}
            0 & \sigma^1\\
            -\sigma^1 & 0
        \end{pmatrix}
        \begin{pmatrix}
            0 & \sigma^2\\
            -\sigma^2 & 0
        \end{pmatrix}
        \begin{pmatrix}
            0 & \sigma^3\\
            -\sigma^3 & 0
        \end{pmatrix}\\
        &=&i
        \begin{pmatrix}
            -\sigma^1&0\\
            0& \sigma^1 
        \end{pmatrix}
        \begin{pmatrix}
            -\sigma^2\sigma^3&0\\
            0&-\sigma^2\sigma^3
        \end{pmatrix} \\
        &=& -i \sigma^1\sigma^2\sigma^3
        \begin{pmatrix}
            -1 & 0\\
            0 & 1
        \end{pmatrix}
        =
        \begin{pmatrix}
            -1 & 0\\
            0 & 1
        \end{pmatrix}.
    \end{eqnarray}
    \item 
    \begin{eqnarray}
        &&P_L^2=
        \begin{pmatrix}
            1 & 0\\
            0 & 0
        \end{pmatrix}
        \begin{pmatrix}
            1 & 0\\
            0 & 0
        \end{pmatrix}
        =
        \begin{pmatrix}
            1 & 0\\
            0 & 0
        \end{pmatrix}\\
        &&P_R^2=
        \begin{pmatrix}
            0 & 0\\
            0 & 1
        \end{pmatrix}
        \begin{pmatrix}
            0 & 0\\
            0 & 1
        \end{pmatrix}
        =        
        \begin{pmatrix}
            0 & 0\\
            0 & 1
        \end{pmatrix}\\
        &&P_LP_R=
        \begin{pmatrix}
            1 & 0\\
            0 & 0
        \end{pmatrix}
        \begin{pmatrix}
            0 & 0\\
            0 & 1
        \end{pmatrix}
        =0 \\
        &&P_L+P_R=
        \begin{pmatrix}
            1 & 0\\
            0 & 0
        \end{pmatrix}
        +
        \begin{pmatrix}
            0 & 0\\
            0 & 1
        \end{pmatrix}
        =1.
    \end{eqnarray}
\end{itemize}



\end{document}
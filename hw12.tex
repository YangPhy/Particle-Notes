\documentclass[12pt]{article}
\usepackage{amsmath,graphicx,color,epsfig,physics}
\usepackage{float}
\usepackage{subfigure}
\usepackage{slashed}
\usepackage{color}
\usepackage{multirow}
\usepackage{feynmp}
\textheight=9.5in \voffset=-1.0in \textwidth=6.5in \hoffset=-0.5in
\parskip=0pt
\def\del{{\partial}}


\begin{document}

\begin{center}
{\large\bf HW12 for Advanced Particle Physics} \\

\end{center}

\vskip 0.2 in

Dear Students,\\

  By the last week, we learned the basic symmetry structure of the
  Standard Model, and how the spontaneous symmetry breaking in the
  minimum SM gives masses to W and Z bosons (in the Higgs Lagrangian)
  while keeping the massless photon, and to all the fermions (in the
  Yukawa Lagrangian), giving rise to CKM and MNS mixings in the charged
  currents while keeping the diagonality of all the neutral currents.

  This week, we are studying the Lorentz transformations. We started
  with the transformations of the coordinates and four-vectors, mainly
  in the matrix representation. Today, we introduce a fermion, with
  its simplest representation, called Weyl fermion. It will be in my
  next lectures, where I introduce the Dirac fermion, which is a pair
  of two Weyl fermions, the pairing as a result of the symmetry breaking
  in the SM.  In short, all the quarks and leptons (including neutrino's)
  are Weyl fermions before the symmetry breaking.

  Let us start with the algebra of the Lorentz group (28):
\begin{eqnarray}
    &&[ J_i, J_j ] =  i \epsilon_{ijk} J_k \label{eq.12Jijk}\\
    &&[ J_i, K_j ] =  i \epsilon_{ijk} K_k \label{eq.12JKijk}\\
    && [ K_i, K_j ] = -i \epsilon_{ijk} J_k \label{eq.12Kijk}
\end{eqnarray}
  which you proved in the last homework. Now, let us introduce two
  combinations of the $6$ generators
\begin{eqnarray}
    &&A_k = (J_k + i K_k)/2 \label{eq.12AJK}\\
    &&B_k = (J_k - i K_k)/2 \label{eq.12BJK}
\end{eqnarray}
  Please note that both $A_k$ and $B_k$ are Hermetian, since $J_k$ are
  Hermetian and $K_k$ are anti-Hermetian.

{\bf hw12-1}: Please show that the commutation relations Eq.(\ref{eq.12Jijk},\ref{eq.12JKijk},\ref{eq.12Kijk}) become
\begin{eqnarray}
    && [ A_i, A_j ] = i \epsilon_{ijk} A_k \label{eq.12Aijk}\\
    && [ B_i, B_j ] = i \epsilon_{ijk} B_k \label{eq.12Bijk}\\
    && [ A_i, B_j ] = 0 \label{eq.12ABijk}
\end{eqnarray}
  The above algebra tells us that, the Lorentz algebra Eq.(\ref{eq.12Jijk},\ref{eq.12JKijk},\ref{eq.12Kijk}) of the
  6 generatros can be split into two $SU(2)$ algebra, which are
  independent. Therefore, the algebra of the Lorentz group is
  identical to the algebra of $SU(2) \times SU(2)$.

  Although the algebra Eq.(\ref{eq.12Aijk},\ref{eq.12Bijk},\ref{eq.12ABijk}) of the generators $A_i$ and $B_i$ is
  $SU(2) \times SU(2)$, the Lorentz transformations which are generated
  by them are NOT $SU(2) \times SU(2)$, because the ``parameters'' of
  the transformations become complex.

{\bf hw12-2}: By inserting the inversion of Eq.(\ref{eq.12AJK},\ref{eq.12BJK}),
\begin{eqnarray}
    && J_k = (A_k + B_k) \\
    && K_k = (A_k - B_k)/i
\end{eqnarray}
  into our Lorentz transformation matrix
\begin{eqnarray}
    L(\theta_1, \theta_2, \theta_3, \eta_1, \eta_2, \eta_3)
  = e^{-i[ J_1\theta_1 +J_2\theta_2 +J_3\theta_3 +K_1 \eta_1 +K_2 \eta_2 +K_3 \eta_3 ]} \label{eq.12Lmatrx}
\end{eqnarray}
  please obtain the representation of the same transformation by
  using the generators $A_k$ and $B_k$ ($k=1,2,3$).

  Please note the minus sign in Eq.\ref{eq.12Lmatrx}, which shows the Lorentz
  transformation of the objects (such as four-vectors and the fermion
  spin wave functions) rather than that of the co-ordinates.

  The solution of {\bf hw12-2} should read
\begin{eqnarray}
    &&L(\theta_1, \theta_2, \theta_3, \eta_1, \eta_2, \eta_3)
  =  e^{-i[ J_1\theta_1 +J_2\theta_2 +J_3\theta_3 +K_1\eta_1 +K_2\eta_2 +K_3\eta_3 ]} \\
  &=& e^{-i[ A_1(\theta_1-i\eta_1) +A_2(\theta_2-i\eta_2) +A_3(\theta_3-i\eta_3)
          +B_1(\theta_1+i\eta_1) +B_2(\theta_2+i\eta_2) +B_3(\theta_3+i\eta_3) ]} \label{eq.12LJK}
\end{eqnarray}
 More importantly, because the generators $A_k$'s and $B_i$'s commute in Eq.\ref{eq.12ABijk}, we can express the Lorentz transformation as
\begin{eqnarray}
    L(\theta_1, \theta_2, \theta_3, \eta_1, \eta_2, \eta_3)
   = L_A(\theta_1, \theta_2, \theta_3, \eta_1, \eta_2, \eta_3)
   \times L_B(\theta_1, \theta_2, \theta_3, \eta_1, \eta_2, \eta_3) \label{eq.12LLAB}
\end{eqnarray}
with
\begin{eqnarray}
    L_A(\theta_1, \theta_2, \theta_3, \eta_1, \eta_2, \eta_3)
  &=& e^{-i[ A_1(\theta_1-i\eta_1) +A_2(\theta_2-i\eta_2) +A_3(\theta_3-i\eta_3) ]} \\
  &=& e^{ A_1 (-i\theta_1-\eta_1) +A_2 (-i\theta_2-\eta_2) +A_3(-i\theta_3-\eta_3) } \label{eq.12LA}
\end{eqnarray}
\begin{eqnarray}
    L_B(\theta_1, \theta_2, \theta_3, \eta_1, \eta_2, \eta_3)
  &=& e^{-i[B_1(\theta_1+i\eta_1) +B_2(\theta_2+i\eta_2) +B_3(\theta_3+i\eta_3)]} \\
  &=& e^{ B_1 (-i\theta_1+\eta_1) +B_2 (-i\theta_2+\eta_2) +B_3(-i\theta_3+\eta_3) } \label{eq.12LB}
\end{eqnarray}
  The four-vectors transform as Eq.\ref{eq.12LJK} or as Eq.\ref{eq.12LLAB}, i.e., both by $L_A$ Eq.\ref{eq.12LA} and $L_B$ Eq.\ref{eq.12LB} simultaneously. We wonder if there are objects which
  transform only under $L_A$, but not under $L_B$, and vice versa.

  We can easily find such an object, which is a column vector of two
  complex numbers (a two spinor), since we know that the algebra
  Eq.\ref{eq.12Aijk}, or Eq.\ref{eq.12Bijk}, are satisfied by
\begin{eqnarray}
    A_k = \frac{\sigma^k}{2}, ~~~ B_k = \frac{\sigma^k}{2}
\end{eqnarray}
 Let us first find the solution which transform only under $L_A$ (Eq.\ref{eq.12LA}).

  We may denote the eigen state of the operator $A_3 = \sigma^3/2$ with
  the eigenvalue $-1/2$
\begin{eqnarray}
  A_3 u_L = \frac{\sigma^3}{2} u_L = -\frac{1}{2} u_L \label{eq.12A3}
\end{eqnarray}
  as a two component column vector $u_L$:
\begin{eqnarray}
  u_L = \sqrt{2E} (0,1)^T \label{eq.12ul}
\end{eqnarray}
   The reason for the normalization factor $\sqrt{2E}$, where $E$ denotes
  the energy of the state, will become clear when we find the equation
  of motion of this state.

  Let us examine its tranformation under the boost along the $z$-axis:
\begin{eqnarray}
  L_A(0,0,0,0,0,\eta_3)
  &=& e^{ A_3       (0-\eta_3) }
  = e^{ \frac{\sigma^3}{2} (-\eta_3) }
  = e^{ \sigma^3 (-\eta_3/2) } \\
  &=&
  \begin{pmatrix}
    e^{-\eta_3/2} & 0 \\
    0 & e^{\eta_3/2}
  \end{pmatrix} \label{eq.12bostz}
\end{eqnarray}
   We find
\begin{eqnarray}
  &&L_A(0,0,0,0,0,\eta_3) u_L = u_L' = \sqrt{2E'} (0,1)^T \label{eq.12bostul} \\ 
  &&E' = e^{\eta_3} E = (\cosh\eta_3 + \sinh\eta_3) E = \gamma(1+\beta) E \label{eq.12boste}
\end{eqnarray}

{\bf hw12-3}: Show Eq.(\ref{eq.12bostz}, \ref{eq.12bostul}, \ref{eq.12boste}).
  If we regard the state Eq.\ref{eq.12ul} as that of a ``particle'' with four momentum
\begin{eqnarray}
  p^\mu = (E, 0, 0, E)^T \label{eq.12pu}
\end{eqnarray}
  then its normalization scales as $\sqrt{E}$ when we make a boost along
  the momentum direction. The quantization condition Eq.\ref{eq.12A3} can be
  regarded as the condition that the helicity of the state is -1/2,
  because $A_3 = \sigma^3/2$ can be regarded as the helicity operaator,
\begin{eqnarray}
  h = \frac {\vec{J} \cdot \vec{p}}{|\vec{p}|}, \label{eq.12h}
\end{eqnarray}
  the angular momentum along the momentum direction, because
\begin{eqnarray}
  &&\vec{J} = (\frac{\sigma^1}{2}, \frac{\sigma^2}{2}, \frac{\sigma^3}{2})^T \label{eq.12agm} \\  &&\vec{p} = ( 0, 0, E )^T
\end{eqnarray}
 
{\bf hw12-4}: Show that the quantization condition Eq.\ref{eq.12A3} is obtained from
\begin{eqnarray}
  h u_L = -\frac{1}{2} u_L \label{eq.12_69}
\end{eqnarray}
 by using the definition of the helicity operator Eq.\ref{eq.12h}. Eq.(\ref{eq.12bostul}, \ref{eq.12boste}) tells that when $\eta_3 > 0$, both the energy of the ``particle''
  and the normalization of its ``wave function'' grows.  We will later
  learn that this is the property of the Left-handed Weyl spinors.

  We can now obtain the spin wave function of this ``object'', the
  left-handed Weyl spinor in an arbitrary Lorentz frame, by using
  the transformation
\begin{eqnarray}
  u_L' = L_A(\theta_1,\theta_2,\theta_3,\eta_1,\eta_2,\eta_3) u_L
\end{eqnarray}
 
{\bf hw12-5}: Please obtain the wave function of Left-handed Weyl fermion,
  in the Lorentz frame where its four momentum is
\begin{eqnarray}
  {p'}^\mu
  = E' (1, \sin\theta\cos\phi, \sin\theta\sin\phi, \cos\theta)^T \label{eq.12puLTed}
\end{eqnarray}
  hint: The Lorentz transformation from Eq.\ref{eq.12pu} to Eq.\ref{eq.12puLTed} is
\begin{eqnarray}
  R_z(\phi)  R_y(\theta)  B_z(\eta)
  =
  L(0,0,\theta_3=\phi,0,0,0) L(0,\theta_2=\theta,0,0,0,0) L(0,0,0,0,0,\eta_3=\eta)
\end{eqnarray}
  where $e^\eta = E'/E$, and note that the Left-handed Weyl fermion
  transforms under $L_A$ only (by our assumption).

{\bf hw12-6}: By using the solution of the above problem, please calculate
  its helicity after Lorentz transformation.

hint: The helicity is the angular momentum along the momentum direction,
  which is defined Eq.\ref{eq.12h}, as an innerproduct of the two three vectors.
  For an object which transform under $L_A$, the angular momentum
  operators are Eq.\ref{eq.12agm}, and the three momentum of the state is
\begin{eqnarray}
  \frac{\vec p}{|\vec p|}
  = (\sin\theta\cos\phi, \sin\theta\sin\phi, \cos\theta)^T\label{eq.12pvp}
\end{eqnarray}
  If both the problem and your solution are correct, we proved that the
  ``helicity'' of Weyl fermions are coserved under Lorentz transformations.

  This is an example of the general conservatin law of all the massless
  states under Poincare symmetry, which will be studied in the next week.

{\bf hw12-7}: Show that our Weyl fermion wave function $u_L$ satisfies the
  equation:
\begin{eqnarray}
  p_\mu \sigma_-^\mu u_L = 0 \label{eq.12ule}
\end{eqnarray}
 with
\begin{eqnarray}
  \sigma_-^\mu = (1, -\sigma_1, -\sigma_2, -\sigma_3) \label{eq.12_75}
\end{eqnarray}
  in an arbitrary Lorentz frame where its momentum is Eq.\ref{eq.12pvp},
\begin{eqnarray}
  p^\mu
  = |\vec {p}| (1, \sin\theta\cos\phi, \sin\theta\sin\phi, \cos\theta)^T. \label{eq.12pu76}
\end{eqnarray}
hint: The solution of {\bf hw12-6} can be expressed as
\begin{eqnarray}
  h u_L = -\frac{1}{2} u_L \label{eq.12helicon}
\end{eqnarray}
  in an arbitrary frame, or equivalently as
\begin{eqnarray}
  (1 + 2h) u_L = 0.
\end{eqnarray}
  We will find soon that Eq.\ref{eq.12ule} is the equation of motion of the
  Left-handed Weyl fermions, which is a building block of the SM,
  and hence of the Dirac equation after the symmetry breakdown.
  Please observe that the e.o.m. Eq.\ref{eq.12ule} simply gives the helicity
  conservation, Eq.\ref{eq.12helicon}, in an arbitrary Lorentz frame.

  Now, let us find the wave function for its anti-paritlce. As in the
  case of our $SU(2)$ doublet studies, let us inroduce a charge conjugate
  state as
\begin{eqnarray}
  u_L^c = (i\sigma_2) u_L^* \label{eq.12ulcdef}
\end{eqnarray}
  and study its Lorentz transformation. Let us start again in the frame
  where the particle momentum is chosen along the positive z-axis Eq.\ref{eq.12pu}. We immediately find
\begin{eqnarray}
  u_L^c=
  \begin{pmatrix}
    0 & 1 \\ 
    -1 & 0 
  \end{pmatrix}
  \sqrt{2E}
  \begin{pmatrix}
    0 \\ 1
  \end{pmatrix}
  = \sqrt{2E}
  \begin{pmatrix}
    1 \\ 0 
  \end{pmatrix}
\end{eqnarray}
  and that its helicity is now $+1/2$
\begin{eqnarray}
  h u_L^c = (1/2) u_L^c. \label{eq.12hulc}
\end{eqnarray}

{\bf hw12-8}: Although it may be trivial to you by now, please confirm Eq.\ref{eq.12hulc} by using the definition of the helicity Eq.\ref{eq.12h}.
  The helicity of the anti-particle of Left-handed Weyl fermion is
  positive.  We say that the anti-particle of Left-handed Weyl fermion
  is a Right-handed Weyl fermion.

  How do they transform under Lorentz transformation?

  Let's first assume that it is transformed by $L_A$, just like the
  Left-handed Weyl fermions.  We immediately find a trouble, since
\begin{eqnarray}
  {u_L^c}'
  = L_A(0,0,0,0,0,\eta_3) u_L^c
  = \sqrt{ 2e^{-\eta} E } (1,0)^T \label{eq.12LTulc}
\end{eqnarray}

{\bf hw12-9}: Show Eq.\ref{eq.12LTulc} by following the steps Eq.(\ref{eq.12bostz}, \ref{eq.12bostul}, \ref{eq.12boste}).

  This is a problem in our interpretation that the wave function
  $u_L^c$ (Eq.\ref{eq.12ulcdef}) describes the state of the ``anti-particle'' of our Weyl fermion, because energy should be $E' = e^\eta E$ in the new frame, rather than $e^{-\eta} E$, in the squareroot of Eq.\ref{eq.12LTulc}.

  The solution to this problem is instantly found, since the
  transformation matrix $L_B$ (Eq.\ref{eq.12LB}) reverses the sign of all boost
  parameters, $\eta_1$, $\eta_2$, $\eta_3$, the rapidities.

  If we assume that $u_L^c$ transform under $L_B$, but not under $L_A$, then
  the problem is solved:
\begin{eqnarray}
  (u_L^c)'
  = L_B(0,0,0,0,0,\eta_3) u_L^c
  = \sqrt{2e^{\eta} E } (1,0)^T
  = \sqrt{ 2E' }         (1,0)^T \label{eq.12LTBulc}
\end{eqnarray}

{\bf hw12-10}: Show Eq.\ref{eq.12LTBulc}.

{\bf hw12-11}: Please find the general expression of ${u_L^c}'$ in an arbitrary
  Lorentz frame where the four momentum is Eq.\ref{eq.12pu76}.

hint: For $u_L^c$, the transformation should be
\begin{eqnarray}
  u_L^c \to (u_L^c)'
  =
  L_B(0,0,\phi,0,0,0) L_B(0,\theta,0,0,0,0) L_B(0,0,0,0,0,\eta) u_L^c
\end{eqnarray}
  but the rotations are exactly the same between $L_A$ and $L_B$.

{\bf hw12-12}: Show that the state ${u_L^c}'$ obtained above has the positive
  helicity in an arbitrary Lorentz frame:
\begin{eqnarray}
  h u_L^c = \frac{1}{2} u_L^c \label{eq.12_85}
\end{eqnarray}
   and that it satisfies the equation of motion:
\begin{eqnarray}
  p_\mu \sigma_+^\mu u_L^c = 0 \label{eq.12_86}
\end{eqnarray}
  with
\begin{eqnarray}
  \sigma_+^\mu = (1, +\sigma_1, +\sigma_2, +\sigma_3), \label{eq.12_87}
\end{eqnarray}
  which is called the right-handed chiral four vector of $\sigma$ matrices.
  Please note that the sign diffeence between Eq.\ref{eq.12_75} and Eq.\ref{eq.12_87} simply refrects the sign of the helicities of the Weyl fermions.

  Just like the helicity of Left-handed Weyl fermions are always
  left-handed, that of its anti-paritlce is always right-handed.
  Just like the e.o.m. for $u_L$ (Eq.\ref{eq.12ule}) gives the helicity conservation Eq.\ref {eq.12_69}, the e.o.m. for $u_L^c$ (Eq.\ref{eq.12_86}) gives the helicity conservation (Eq.\ref{eq.12_85}).  The difference between $\sigma_-^\mu$ (Eq.\ref{eq.12_75}) in Eq.\ref{eq.12ule} and $\sigma_+^\mu$ (Eq.\ref{eq.12_87}) in Eq.\ref{eq.12_86} simply reflects the fact that $u_L$ has always $-1/2$ helicity, while $u_L^c$ has always $+1/2$ helicity.


  Let us summarize the transformation properties our two Weyl spinors:
\begin{eqnarray}
  && u_L   \to   u_L'    = L_A u_L \label{eq.12_88a} \\
  && u_L^c \to  {u_L^c}' = L_B u_L^c \label{eq.12_88b}
\end{eqnarray}
   Now, let us show the most striking property of the above two states:
  the product
\begin{eqnarray}
  {u_L^c}^\dagger u_L \label{eq.12_89}
\end{eqnarray}
  is invariant under full Lorentz transformations!  More explicitly,
\begin{eqnarray}
  {u_L^c}^\dagger u_L
  \to 
   {u_L^c}'^\dagger u_L'
 = (L_B u_L^c)^\dagger L_A u_L
 = (u_L^c)^\dagger (L_B)^\dagger L_A u_L
 = (u_L^c)^\dagger u_L \label{eq.12_90}
\end{eqnarray}
  or simply
\begin{eqnarray}
  L_B^\dagger L_A = 1 \label{eq.12_91}
\end{eqnarray}

{\bf hw12-13}: Show Eq.\ref{eq.12_91} for an arbitrary Lorentz transformation,
  $L(\theta_1,\theta_2,\theta_3,\eta_1,\eta_2,\eta_3)$.

  I think that you can easily show this, because the rotation parts
  ($\theta_1,\theta_2,\theta_3$) are exactly like the $SU(2)$ group, both in $L_A$ and
  $L_B$, whereas the boost part ($\eta_1,\eta_2,\eta_3$) are real and opposite
  between $L_A$ and $L_B$.

  When Eq.\ref{eq.12_91} is satisfied, we notice that
\begin{eqnarray}
  L_B = [(L_A)^{-1}]^\dagger \label{eq.12_92}
\end{eqnarray}
   or, $L_B$ is the inverse-dagger matrix of of $L_A$.

  Now, we arrive at what Majorana found, who showed that a spin $1/2$
  particle can have Lorentz invariant mass term,
\begin{eqnarray}
  m (u_L^c)^\dagger (u_L) = m u_L^T (-i\sigma_2) u_L \label{eq.12_93}
\end{eqnarray}
  and once we put it into the QFT language, we obtain the e.o.m.
\begin{eqnarray}
  p_\mu \sigma_-^\mu u_L = m u_L^c \label{eq.12_94}
\end{eqnarray}
  in an arbitrary Lorentz frame where the four momentum is
\begin{eqnarray}
  p^\mu = (E, p_x, p_y, p_z), ~~~E = \sqrt{m^2+p_x^2+p_y^2+p_z^2} \label{eq.12_95}
\end{eqnarray}

  Although Eq.\ref{eq.12_93} gives the equation of motion of a massive fermion,
  it does not describe that of a particle with conserved charge,
  since its right-hand side $(u_L^c)$ has the opposite charge with that
  of $u_L$ in the left-hand side.

  Although you can prove Eq.\ref{eq.12_94}, which may be called Majorana equation, without much difficulty, I think it is sufficient for you to
  appreciate the Lorentz invariance of the product, Eq.\ref{eq.12_89}, and in more general the fundamental relations Eq.\ref{eq.12_91}, between the two Lorentz transformations $L_A$ and $L_B$ we found in Eq.(\ref{eq.12LA}, \ref{eq.12LB}).


  Now we are ready to learn how Dirac found his equation, Dirac equation,
  for the electron, which has both the conserved charge and a finite
  invariant mass.

  We have already shown that Weyl fermions can have charges, but cannot
  have a mass, while Majorana fermion can have a mass, but cannot have
  a conserved charge.  The solution is only one small step away.

  However, I always feel that only a true genius like Dirac could find
  the solution in his days, when none of the above steps which we
  studied one by one were known.  In fact, contrary to the ordering of
  simplest (Weyl) to complex (Dirac) in my lecture, Weyl found his
  equation (1929) one year after Dirac (1928), and Majorana found
  his equation in 1937.

  The SM of particle physics tells us that all the quarks and leptons
  are just Weyl fermions before the symmetry breakdown.  In this sence,
  the nature chooses the simplest realization of the Lorentz symmetry
  for fermions.  I might want to add, that even for the gauge groups
  $SU(3)$ and $SU(2)$, the nature seems to give us only their simplest
  representations for fermions.  It is only because of the symmetry
  breakdown, a complicated object like the electron appears in front
  of us.  Dirac did not know any of what I explained above.

  Please let me start my next lecture by deriving Dirac equations.

Best regards,\\

Kaoru




\end{document}

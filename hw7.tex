\documentclass[12pt]{article}
\usepackage{amsmath,graphicx,color,epsfig,physics}
\usepackage{float}
\usepackage{subfigure}
\usepackage{slashed}
\usepackage{color}
\usepackage{multirow}
\usepackage{feynmp}
\textheight=9.5in \voffset=-1.0in \textwidth=6.5in \hoffset=-0.5in
\parskip=0pt
\def\del{{\partial}}


\begin{document}

\begin{center}
{\large\bf HW7 for Advanced Particle Physics} \\

\end{center}

\vskip 0.2 in

Dear students,\\

  This week, I introduce the Higgs mechanism of the SM.
  Before giving the homework for the Higgs mechanism, let
  me give two comments regarding the past homeworks.
  First, about the vector ($3 \times 3$ matrix) representation
  of rotation, $SO(3)$.
 \begin{eqnarray}
  O(\theta_1,\theta_2,\theta_3) = e^{ i (\theta_1 J_1 + \theta_2 J_2 + \theta_3 J_3) }
 \end{eqnarray}
   The above oprators transform the coordinate system,
\begin{eqnarray}
  (x,y,z)^T \to (x',y',z')^T = O(\theta_1,\theta_2,\theta_3) (x,y,z)^T
\end{eqnarray}
  The object, $v = (v_x,v_y,v_z)^T$, then transform as
\begin{eqnarray}
  v \to v' = (v'_x,v'_y,v'_z)^T
  = O(\theta_1,\theta_2,\theta_3)^{-1} v
  = e^{ -i (\theta_1 J_1 + \theta_2 J_2 + \theta_3 J_3) } v
\end{eqnarray}
   Therefore, when you obtain the generators from the
  transformations of the right-handed coordinates,
  you should use
\begin{eqnarray}
  (x,y,z)^T \to (x',y',z')^T = O(\theta_1,\theta_2,\theta_3) (x,y,z)^T
\end{eqnarray}
   Also, please note that the right handed coordinate
  rotate e.g. as
\begin{eqnarray}
  (x,y,z)^T \to (x',y',z)^T = O(0,0,\theta) (x,y,z)^T
                          = e^{ i \theta J_3 } (x,y,z)^T
\end{eqnarray}
 with
 \begin{eqnarray}
   x' &=& \cos\theta x+ \sin \theta y \\
   y' &=& -\sin \theta x + \cos \theta y  
 \end{eqnarray}
This determines $J_3$ as $(J_3)_{12} = -i$.
Then, $J_1$ and $J_2$ are obtained by cyclic rotation,
\begin{eqnarray}
  (J_1)_{23} = (J_2)_{31} = (J_3)_{12} = -i
\end{eqnarray}
   $J_1$, $J_2$, $J_3$ are now fixed by their Hermiticity.

  The correctness should be checked by the fundamental
  commutation relation for the right-handed rotations:
\begin{eqnarray}
  [ J_i, J_j ] = i \epsilon_{ijk} J_k
\end{eqnarray}
   or
\begin{eqnarray}
  &&[{J_1},{J_2}]= i {J_3}\\
  &&[{J_2},{J_3}]= i {J_1}\\
  &&[{J_3},{J_1}]= i {J_2}
\end{eqnarray}
 
  Some of you made the sign errors e.g. for $J_1$ and $J_2$,
  and then the first commutation relation holds, but not
  the other two.

  It is very very important to note that there is no
  sign ambiguity for the rotation generators, because
  of the fundamental commutation relations.

  As for the relation between $U(1)$ and $SO(2)$, all what
  I wanted you to recall is the well known relationship
  between the rotation in the real $(x,y)$ coordinate and
  the phase transformation of the complex number z=x+iy.

  In short, $U(1)$ transformation of $z$,
\begin{eqnarray}
  z \to z' = e^{i\theta} z
  = e^{i \theta} |z| e^{i\arg(z)}
  = |z| e^{i(\arg(z)+\theta)}
\end{eqnarray}
  can be expressed as
\begin{eqnarray}
  x+iy \to  x'+iy' &=& e^{i\theta} (x+iy)
              = (\cos\theta+i\sin\theta)(x+iy) \\
              &=& (\cos\theta x -\sin\theta y)
              +i(\sin\theta x +\cos\theta x)
\end{eqnarray}
  which is an $SO(2)$ rotation,
\begin{eqnarray}
  (x,y)^T \to (x',y')^T = O (x,y)^T.
\end{eqnarray}

  This is the right-handed rotation of an object
  (rather than the coordinate), since it corresponds
  to the rotation of a vector denoted by the complex
  number, $z = x+iy$, in the $(x,y)$ plane.

  This is the reason why we find
\begin{eqnarray}
  O(\theta) = e^{ -i \sigma^2 \theta}
\end{eqnarray}
   with the $-i$ factor in front of the $SO(2)$ generator $\sigma^2$,
  which is just the $2 \times 2$ ($x$ and $y$) component of the $SO(3)$
  generator $J_3$.



{\bf Now, let me give the homework 07 for the Higgs mechanism.}

  I introduced the SM Lagrangian as
  \begin{eqnarray}
    L_{SM} = L_{gauge} + L_{fermion} + L_{Higgs} + L_{Yukawa} \label{eq.7SM}
  \end{eqnarray}
    and explained how $L_{gauge}$, $L_{fermion}$, $L_{Higgs}$ and $L_{Yukawa}$
  are invariant under the SM gauge group,
\begin{eqnarray}
  SU(3)_{color} \times SU(2)_L \times U(1)_Y \label{eq.7G}
\end{eqnarray}
  
  In this hw 07, we review $L_{Higgs}$, and
  obtain the mass-eigenstate weak bosons (the charged weak
  boson $W^+$ and $W^-$, the neutral weak boson $Z$, and the photon $A$
  We will then come back to $L_{fermion}$, and find its expressions
  in terms of physical gauge bosons, $W$, $Z$, $A$ and gluons ($A^a$).

  First, let us review how $SU(2)_L \times U(1)_Y$ gauge
  symmetry can be broken spontaneously down to $U(1)_{EM}$
  by a very simple $L_{Higgs}$ of the MSM.

  $L_{Higgs}$ consists of just one Higgs doublet in the
  minimal SM (MSM), which Weinberg introduced in 1967
  as a theory of leptons.  Its extention to quarks
  was achieved in 1971 by Glasho-Illiopoulos-Maiani (GIM)
  by postulating the 4'th quark, the charm.  Weinberg's
  MSM was not able to explain the smallness of
  flavor-changing (strangeness-changing) neutral current,
  or the extreme smallness of the $K^0-\overline{K^0}$ mixing.
  Weinberg therefore proposed his model as a model for
  leptonic weak interactions, since he recognized the problem
  of applying his model to hadrons.  GIM showed that
  if there is a 4'th quark in addition to the 3 known
  ones (up, down, and strange), the observed smallness
  of the FCNC can be explained within Weinberg's MSM.
  You should understand all these details during the
  course of my lectures.  To understand them is to
  understand the SM, or the world as we see it now.

  The Higgs doublet of the SM has the
  $SU(3)_{color} \times SU(2)_L \times U(1)_Y$ quantum numbers
  $(1, 2, +1/2)$:
  \begin{eqnarray}
    \phi= 
    \begin{pmatrix}
      \phi^+ \\ \phi^0
    \end{pmatrix} \label{eq.7Higgs}
  \end{eqnarray}
   where the charge assignment follows the {\bf Gell-Mann Nishijima
  relation} 
\begin{eqnarray}
  Q = T_3 + Y \label{eq.7charge}
\end{eqnarray}
  The gauge invariant Lagrangian $L_{Higgs}$ of the doublet Eq.\ref{eq.7Higgs} is
  \begin{eqnarray}
    L_{Higgs} = (D_\mu \phi)^\dagger (D^\mu \phi) - V(\phi)
  \end{eqnarray}
   The invariance of the kinetic part under the gague
  transformation
\begin{eqnarray}
  \phi(x) \to U(x) \phi(x)
\end{eqnarray}
  should be clear by now from the transformation property of
  the covariant derivative
\begin{eqnarray}
  D_\mu \to D'_\mu = U(x) D_\mu U(x)^{-1}
\end{eqnarray}
  The potential $V(\phi)$ is a function of $\phi$, without
  derivatives.  It should be invariant under the SM gauge
  transformation
\begin{eqnarray}
  V(\phi) \to V(\phi') = V(\phi)
\end{eqnarray}
 for all $SU(2)_L \times U(1)_Y$ transformations,
 \begin{eqnarray}
  \phi \to \phi' = U(\theta^1,\theta^2,\theta^3,\theta^0) \phi
 \end{eqnarray}
  with
\begin{eqnarray}
  U = e^{i(T^1\theta^2 +T^2\theta^2 +T^3\theta^3 +Y\theta^0)}
\end{eqnarray}
  We can easily find the solution by noting that $\phi^\dagger \phi$
  is invariant under all the transformations. We can now make
  the gauge invariant potential as follows:
\begin{eqnarray}
  V(\phi) = \frac{\lambda}{4} (\phi^\dagger \phi)^2
  + \mu^2 (\phi^\dagger \phi)
\end{eqnarray}
 Note that this potential has a minimum when $\lambda > 0$

  Let us denote the location of the minimum (vacuum) as
\begin{eqnarray}
  <\phi> = ( <\phi_u>, <\phi_d> )^T \label{eq.7vacuum}
\end{eqnarray}
 Here, I wrote
\begin{eqnarray}
  \phi = 
  \begin{pmatrix}
    \phi_u \\ \phi_d
  \end{pmatrix}
\end{eqnarray}
  instead of Eq.\ref{eq.7Higgs}, in order to show that the definition
  of the electric charge, Eq.\ref{eq.7charge}, follows from Eq.\ref{eq.7vacuum} by an $SU(2)_L \times U(1)_Y$ rotation:
  \begin{eqnarray}
    \phi \to \phi' = U(\theta_2) U(\theta_3) U(\theta_0) \phi \label{eq.721ro}
  \end{eqnarray}
   where $U(\theta_2) = U(0,\theta_2,0,0)$, $U(\theta_3) = U(0,0,\theta_3,0)$, $U(\theta_0) = U(0,0,0,\theta_0)$, with the notation
\begin{eqnarray}
  U(\theta_1,\theta_2,\theta_3,\theta_0)
  = e^{ i \sum_k \theta_k T^k }
\end{eqnarray}
 where $T^k = \sigma^k/2 (k=1,2,3)$ and $T^0 = {\rm diag}\{1,1\}/2$.

  For instance, if
\begin{eqnarray}
  <\phi_u> &=& \frac{v}{\sqrt 2} \sin\theta e^{i\alpha} \nonumber \\
  <\phi_d> &=& \frac{v}{\sqrt 2} \cos\theta e^{i\beta} \label{eq.7egcon}
\end{eqnarray}
   with $v>0$, then the transformation Eq.\ref{eq.721ro} with
\begin{eqnarray}
  \theta_0 &=& -(\alpha+\beta)\\
  \theta_3 &=& -(\alpha-\beta)\\
  \theta_2 &=& -2\theta
\end{eqnarray}
  gives
\begin{eqnarray}
  <\phi> \to <\phi'>
    &=& U(\theta_2) U(\theta_3) U(\theta_0)
    \begin{pmatrix}
      <\phi_u> \\ <\phi_d>
    \end{pmatrix} \\
    &=& U(\theta_2) U(\theta_3) 
    \begin{pmatrix}
      v \sin\theta e^{i(\alpha-\beta)/2} /\sqrt 2 \\
      v \cos\theta e^{i(\beta-\alpha)/2} /\sqrt 2
    \end{pmatrix}
     = U(\theta_2) 
     \begin{pmatrix}
       \frac{\sin\theta v}{\sqrt 2} \\
       \frac{\cos\theta v}{\sqrt 2}
     \end{pmatrix} \\
    &=&
    \begin{pmatrix}
      0 \\ \frac{v}{\sqrt 2}
    \end{pmatrix} \label{eq.7eg}
\end{eqnarray}


{\bf hw07-1}: Show the above derivation, from Eq.\ref{eq.7egcon} to Eq.\ref{eq.7eg}.

  This exercise is important for you to understand that the
  preservation of electromagnetic gauge invariance is automatic
  in MSM.

  Because the Lagrangian is $SU(2)_L \times U(1)_Y$ invariant,
  we can always perform the transformation (Eq.\ref{eq.7eg}) to arrive at
  the definition of the $T^3$ axis, such that the v.e.v. exists
  only in the lower ($T^3 = -1/2$) component of the $Y=1/2$ Higgs
  doublet, where the Nishijima-GellMann relation (Eq.\ref{eq.7charge}) tells
  that only the $Q_{EM}=0$ component of the Higgs doublet has
  the v.e.v..  This ensures that the $U(1)_{EM}$ invariance holds
  even after the symmetry breaking.  We express this fact as
\begin{eqnarray}
  SU(2)_L \times U(1)_Y  \to  U(1)_{EM}
\end{eqnarray}
    Now, if the Higgs potential is minimized at Eq.\ref{eq.7egcon} or Eq.\ref{eq.7eg}, then the potential should have zero derivative there:
\begin{eqnarray}
\left (\frac{\del V(\phi)}{\del \phi} \right )_{ \phi = <\phi>} = 0 \label{eq.7pd}
\end{eqnarray}
  From the MSM potential (Eq.\ref{eq.7vacuum}), we find
\begin{eqnarray}
  [\frac{\lambda}{2} (<\phi>^\dagger <\phi>) + \mu^2] <\phi>^\dagger = 0
\end{eqnarray}
  Because $<\psi>^\dagger <\psi>$ is positive semi-definite,
  we find
  \begin{enumerate}
    \item If $\mu^2 > 0$, then $<\phi>^\dagger = <\phi> = 0$
    is the only solution. \label{it.7s1}
    \item If $\mu^2 < 0$, then in addition to $<\phi> = 0$,
    $<\phi>^\dagger <\phi> = -2\mu^2/\lambda$ also is a solution. \label{it.7s2}
  \end{enumerate}


{\bf hw07-2}: Show above \ref{it.7s1} and \ref{it.7s2}.  Especially, when \ref{it.7s2} holds, please compare the energy of the two solutions and find the true vacuum (the lowest energy state of the potential).

  When \ref{it.7s1} holds, the potential is minimized at the symmetric
  point $<\phi> = ( 0, 0 )^T$, and there is no breakdown of the
  symmetry.  The Higgs doublet (the charged boson and the complex
  neutral boson) has a common mass of $\mu$, all the weak bosons
  are massless, and all the quarks and leptons are massless.

  When \ref{it.7s2} holds, we find the solution
\begin{eqnarray}
  <\phi>^\dagger <\phi> = |<\phi_u>|^2 + |<\phi_d>|^2
                            = \frac{-2\mu^2}{\lambda}
                            = \frac{v^2}{2} \label{eq.7so}
\end{eqnarray}
  gives the lowest energy state, and hence the vacuum.
  The meaning of the factor of 1/2 in the definition of v in
  Eq.\ref{eq.7so} will become clear later.

  From the potential parameters, only the sum of the absolute
  value squareds of the upper and lower components of the v.e.v.
  of the Higgs doublet $\phi$ is determined.  The most general
  parametrization of the v.e.v. should hence be as in Eq.\ref{eq.7vacuum}
  and Eq.\ref{eq.7egcon}.  As shown in Eq.\ref{eq.7eg}, the new $T^3$ axis can be chosen along the direction of the v.e.v., such that the v.e.v. resides
  solely in the $T^3 = -1/2$ component,
\begin{eqnarray}
  <\phi> = 
  \begin{pmatrix}
    0 \\ \frac{v}{\sqrt 2}
      \end{pmatrix}
\end{eqnarray}
  Most importantly, with this choice of isospin axis,
  the Gell-Mann Nishijima relation
\begin{eqnarray}
  Q <\phi> = (T^3 + Y) <\phi> = 0 \label{eq.7charge0}
\end{eqnarray}
  or equivalently,
\begin{eqnarray}
  U_{EM}(\theta) <\phi> = e^{iQ\theta} <\phi> = <\phi>. \label{eq.7charge0e}
\end{eqnarray}


{\bf hw07-3}: Show Eq.\ref{eq.7charge0} and Eq.\ref{eq.7charge0e}.

  When the generator of a symmetry ($Q$ in case of EM $U(1)$ symmetry)
  vanishes the vacuum Eq.\ref{eq.7charge0}, the vacuum does not change (symmetric) under the symmetry transformation.

  All the other 3 generators of the $SU(2)\times U(1)$ do not vanish
  the vacuum,
  $T^1$,$T^2$, $T^3-Y$
  and the symmetry associated with the above 3 generators are
  broken by the vacuum (spontaneously broken).

  Let us first obtain the mass-eigen states of the massive
  weak bosons by inserting Eq.\ref{eq.7pd} into $L_{Higgs}$
\begin{eqnarray}
  L_{Higgs}(\phi=<\phi>)
  = (D_\mu \phi)^\dagger (D^\mu \phi)|_{\phi=<\phi>} \label{eq.7LHiggs}
\end{eqnarray}
  up to cnstants, since $V(\phi=<\phi>)$ is just a constant.
  (In QFT, the minimum/vacuum energy is set to zero,
  $V(<\phi>)=0$, by adding/subtracting the constant.)
\begin{eqnarray}
  D_\mu <\phi>
  &=& D_\mu
  \begin{pmatrix}
    0 \\ \frac{v}{\sqrt 2}
  \end{pmatrix}
  =(\del_\mu + i g T^k W^k_\mu + i g' Y B_\mu)
  \begin{pmatrix}
    0 \\ \frac{v}{\sqrt 2}
  \end{pmatrix} \\ 
  &=&i( g T^k W^k_\mu + g' Y B_\mu)
  \begin{pmatrix}
    0 \\ \frac{v}{\sqrt 2}
  \end{pmatrix} \\
  &=&i[g(T^1W^1_\mu+T^2W^2_\mu) + (gT^3W^3_\mu+g'YB_\mu)]
  \begin{pmatrix}
    0 \\ \frac{v}{\sqrt 2}
  \end{pmatrix} \\
  &=&i[(g/\sqrt2)(T^+W^+_\mu+T^-W^-_\mu)+(1/2)(gW^3_\mu \sigma^3+g'B_\mu)]
  \begin{pmatrix}
    0 \\ \frac{v}{\sqrt 2}
  \end{pmatrix} \\
  &=&\frac{v}{\sqrt 2}
  \begin{pmatrix}
    \frac{g}{\sqrt 2} W^+_\mu \\ (-gW^3_\mu+g'B_\mu)/2
  \end{pmatrix} 
   =\frac{iv}{\sqrt 2} 
   \begin{pmatrix}
    \frac{g}{\sqrt 2} W^+_\mu \\ -(g_z/2) Z_\mu
  \end{pmatrix}\\
   &=& i
   \begin{pmatrix}
    \frac{gv}{2} W^+_\mu \\ -\frac{g_zv}{2\sqrt2} Z_\mu
   \end{pmatrix} \label{eq.7ppsi}
\end{eqnarray}

{\bf hw07-4}: Show Eq.\ref{eq.7ppsi}, by using Eq.(\ref{eq.7use1},\ref{eq.7use2},\ref{eq.7use3},\ref{eq.7use4},\ref{eq.7use5}).

  In the above, I used
  \begin{eqnarray}
    T^1 W^1_\mu + T^2 W^2_\mu
    &=& (T^1+iT^2)(W^1_\mu-iW^2_\mu)/2 + (T^1-iT^2)(W^1_\mu+iW^2_\mu)/2 \nonumber \\
    &=& T^+(W^1_\mu-iW^2_\mu)/2 + T^-(W^1_\mu+iW^2_\mu)/2  \nonumber \\
    &=& (T^+ W^+_\mu)/\sqrt2 + (T^- W^-_\mu)/\sqrt2 \label{eq.7use1}
  \end{eqnarray}
  where
\begin{eqnarray}
  W^+_\mu &=& (W^1_\mu - iW^2_\mu)/\sqrt2 \\ \label{eq.7use2}
  W^-_\mu &=& (W^1_\mu + iW^2_\mu)/\sqrt2 \label{eq.7use3}
\end{eqnarray}

  Please note the sign in the r.h.s..  In QFT, the operator $W^+_\mu$
  annihilates the $W^+$ boson, and creates the $W^-$ boson.  The factor
  of $1/\sqrt2$ appears since $W^1$ and $W^2$ (together with $W^3$) are real
  fields, while $W^+$ and $W^-$ are complex fields.

  As for the neutral weak boson, I introduce the definition
\begin{eqnarray}
  Z_\mu = (g W^3_\mu - g' B_\mu)/\sqrt{ g^2 + g'^2 } \label{eq.7use4}
\end{eqnarray}
  and also introduced the coupling
\begin{eqnarray}
  g_z = \sqrt{ g^2 + g'^2 } \label{eq.7use5}
\end{eqnarray} 

  Now, we find
\begin{eqnarray}
  L_{Higgs}(\phi=<\phi>)
  &=& (gv/2)^2 W^-_\mu W^{+ \mu} + (g_z v/2\sqrt2)^2 Z_\mu Z^\mu \nonumber \\
  &=& (gv/2)^2 W^-_\mu W^{+ \mu} + (g_z v/2)^2 Z_\mu Z^\mu/2 \nonumber \\
  &=& m_W^2 W^-_\mu W^{+ \mu} + m_Z^2 Z_\mu Z^\mu/2 \label{eq.7LH}
\end{eqnarray}


{\bf hw07-5}: Show Eq.\ref{eq.7LH}.
  Please note the $1/2$ factor for the $Z$ boson mass term, which
  appears because the $Z$ boson is a real field. From Eq.\ref{eq.7so}, we find
\begin{eqnarray}
  m_W = g v/2 \\
  m_Z = g_z v/2
\end{eqnarray}
  Now, we introduce the Weinberg rotation of the neautral weak
  bosons:
\begin{eqnarray}
  \begin{pmatrix}
    W^3_\mu \\  B_\mu
  \end{pmatrix}
  =
  \begin{pmatrix}
    \cos\theta_W & \sin\theta_W \\
    -\sin\theta_W & \cos\theta_W
  \end{pmatrix}
  \begin{pmatrix}
    Z_\mu \\ A_\mu
  \end{pmatrix}
\end{eqnarray}
 or its inverse,
 \begin{eqnarray}
  \begin{pmatrix}
    Z_\mu \\  A_\mu
  \end{pmatrix}
  =
  \begin{pmatrix}
    \cos\theta_W & -\sin\theta_W \\
    \sin\theta_W & \cos\theta_W
  \end{pmatrix}
  \begin{pmatrix}
    W^3_\mu \\ B_\mu
  \end{pmatrix}
\end{eqnarray}

  By comparing Eq.\ref{eq.7LHiggs} with Eq.\ref{eq.7use4}, we find
\begin{eqnarray}
  \cos\theta_W =  g/g_z\\
  \sin\theta_W = g'/g_z
\end{eqnarray}

  Finally, let us obtain the explicit form of the EW covariant
  derivative in terms of the $W$, $Z$, $A$ bosons:
\begin{eqnarray}
  D_\mu
  &=&\del_\mu +i g T^k W^k_\mu + i g' Y B_\mu
  =\del_\mu +ig(T^1W^1_\mu+T^2W^2_\mu+T^3W^3_\mu) +ig'YB_\mu \nonumber \\
  &=&\del_\mu +i(g/\sqrt2)(T^+W^+_\mu + T^-W^-_\mu)
            +igT^3 W^3_\mu
            +ig'Y  B_\mu \nonumber \\
  &=&\del_\mu +i(g/\sqrt2)(T^+W^+_\mu + T^-W^-_\mu)
            +igT^3 ( \cos\theta_W Z_\mu + \sin\theta_W A_\mu) \nonumber \\
           &&+ig'Y  (-\sin\theta_W Z_\mu + \cos\theta_W A_\mu)\nonumber \\
  &=&\del_\mu +i(g/\sqrt2)(T^+W^+_\mu + T^-W^-_\mu)
            +i(g\cos\theta_W T^3 -g'\sin\theta_W Y) Z_\mu\nonumber \\
            &&+i(g\sin\theta_W T^3 +g'\cos\theta_W Y) A_\mu \nonumber \\
  &=&\del_\mu +i(g/\sqrt2)(T^+W^+_\mu + T^-W^-_\mu)
            +ig_Z(\cos^2\theta_W T^3 -\sin^2\theta_W Y) Z_\mu
            \nonumber \\ &&+ie(T^3 + Y) A_\mu \nonumber \\
  &=&\del_\mu +i(g/\sqrt2)(T^+W^+_\mu + T^-W^-_\mu)
            +ig_Z(T^3 -\sin^2\theta_W Q) Z_\mu
            +ieQ A_\mu \label{eq.7del}
\end{eqnarray}

{\bf hw07-6}: Reproduce the above derivation Eq.\ref{eq.7del}.
  In the last two steps, I introduced the new coupling
\begin{eqnarray}
  e = g\sin\theta_W = g'\cos\theta_W \label{eq.7e}
\end{eqnarray}
  and used the Nishijima-GellMann relation
  \begin{eqnarray}
    Q = T^3 + Y \label{eq.7charge3}
  \end{eqnarray}
   to replace $Y$ with $Q$. The last form of the covariant derivative
  is most convenient for computing the weak interaction amplitudes
  in the SM.  It shows that the massless gauge boson $A_\mu$ couples
  to the matter with the coupling strength e of Eq.\ref{eq.7use5} and the
  charge $Q$ according to Eq.\ref{eq.7charge3}.

  Since we know the magnitude of $e$, which is $\sqrt{4\pi\alpha}$,
\begin{eqnarray}
  e = \sqrt{ 4\pi \alpha }
        = \sqrt{ 4\pi /137 }
        = \sqrt{ 0.092 }
        = 0.30
\end{eqnarray}
  it is useful to note
\begin{eqnarray}
  g   &=& e/\sin\theta_W \\
  g'  &=& e/\cos\theta_W \\
  g_Z &=& g/\cos\theta_W = e/(\sin\theta_W \cos\theta_W) 
\end{eqnarray}
  For $\sin^2\theta_W = 0.233$, the coupling strengths are:
\begin{eqnarray}
  g   = 0.62\\
  g'  = 0.34 \\
  g_Z = 0.71
\end{eqnarray}
  I memorize Eq.\ref{eq.7e} as one set of equations:
\begin{eqnarray}
  e = g \sin\theta_W
    = g' \cos\theta_W
    = g_Z \sin\theta_W \cos\theta_W
\end{eqnarray}

  With the above knowledge of the electroweak coupling strengths
  and the last expression of the covariant derivative Eq.\ref{eq.7del},
  which I copy below:
\begin{eqnarray}
  D_\mu =\del_\mu +i(g/\sqrt2)(T^+W^+_\mu + T^-W^-_\mu)
                         +ig_Z(T^3 -\sin^2\theta_W Q) Z_\mu
                         +ieQ A_\mu
\end{eqnarray}

  we can express all the gauge interactions of the quarks
  and leptons in $L_{fermion}$ and also in $L_{gauge}$.

  I will come back to them after showing the custodial $SU(2)$
  invariance of the MSM Higgs potential and its violation
  by the Hypercharge gauge interactions and the Yukawa
  interactions in the next lectures.

  That's all for hw07.\\

Best regards,\\

Kaoru


\end{document}

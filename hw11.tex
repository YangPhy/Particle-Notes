\documentclass[12pt]{article}
\usepackage{amsmath,graphicx,color,epsfig,physics}
\usepackage{float}
\usepackage{subfigure}
\usepackage{slashed}
\usepackage{color}
\usepackage{multirow}
\usepackage{feynmp}
\textheight=9.5in \voffset=-1.0in \textwidth=6.5in \hoffset=-0.5in
\parskip=0pt
\def\del{{\partial}}


\begin{document}

\begin{center}
{\large\bf HW11 for Advanced Particle Physics} \\

\end{center}

\vskip 0.2 in

Dear students,\\

  The last lecture concluded my introduction to the Standard Model
  (SM) of particle physics. I believe that all the most fundamental
  aspects of the SM gauge symmetry and its spontaneous breakdown
  have been explained.  Whenever you become unsure about the subject,
  you may come back to my homework, and I hope that they will be
  useful for you, as they have been for myself.

  Throughout my past lectures, I assumed that the Lagrangian
  density we have been studying is Lorentz invariant.  You might
  be able to fool yourself to believe that it is Lorentz invariant,
  by noting Einstein's tensor notations (which gives Lorentz
  invariants when the upper and lower Lorentz indices are
  contracted), may wonder what it means for $\sigma_\pm^\mu
  = (1, \pm \sigma^k)$, and for Majorana and Dirac mass terms.

  This is the subject of my lecture today, and the following two.



  Lorentz transformations express change of reference coordinate
  system, parametrized by space-time 4 vectors,
\begin{eqnarray}
    x^\mu = (x^0, x^1, x^2, x^3)^T = (t, x, y, z)^T \label{eq.11xmu1}
\end{eqnarray}
  as
\begin{eqnarray}
    x^\mu \to {x'}^\mu = L^\mu_\nu x^\nu
                  = (e^{ i \omega_{ab} M^{ab}/2 })^\mu_\nu x^\nu
                  = x^\mu - \omega^\mu_\nu x^\nu + {\cal O}(\omega_{ab}^2) \label{eq.11LT1}
\end{eqnarray}

  Unfortunately, thanks to Einstein, the Lorentz transformation is
  usually expressed as above, in the tensor notation. Once you
  understand the tensor notation well, it is very convenient (since
  we can avoid errors), but I find it rather difficult to explain it
  clearly to most of the starting graduate students.

  I therefore, introduce Matrix representations for the Lorentz
  transformation, whenever possible. I start with connecting the
  above tensor representation and the matrix representation.

  In the matrix representation, $x^\mu$ is expressed as a column vector
  as above Eq.\ref{eq.11xmu1}, and $x_\mu$ is expressed as another column vector
\begin{eqnarray}
    x_\mu = (x_0, x_1, x_2, x_3)^T = (t, -x, -y, -z)^T \label{eq.11xmu2}
\end{eqnarray}
  The Lorentz transformation of $x^\mu$ in Eq.\ref{eq.11LT1} is expressed as
\begin{eqnarray}
    (x^0, x^1, x^2, x^3)^T \to ({x'}^0, {x'}^1, {x'}^2, {x'}^3)^T
                            = L (x^0, x^1, x^2, x^3)^T, \label{eq.11LT1m}
\end{eqnarray}
  where $L$ is the $4\times 4$ matrix representation of $L^\mu_\nu$.  On the other
  hand, the Lorentz transformation of $x_\mu$ (Eq.\ref{eq.11xmu2}) may be expressed as
\begin{eqnarray}
    (x_0, x_1, x_2, x_3)^T \to  (x_0', x_1', x_2', x_3')^T
                            = L' (x_0, x_1, x_2, x_3)^T, \label{eq.11LT2m}
\end{eqnarray}
  where
\begin{eqnarray}
    L' = (L^{-1})^T \label{eq.11Lp}
\end{eqnarray}

  In the Tensor representation, Eq.\ref{eq.11LT2m} is written simply as
\begin{eqnarray}
    x_\mu \to x'_\mu = L_\mu^\nu x_\nu \label{eq.11LT2}
\end{eqnarray}
  and you may not notice the difference between Eq.\ref{eq.11LT1} and Eq.\ref{eq.11LT2}.  When we
  compare the matrix representation Eq.\ref{eq.11LT1m} and Eq.(\ref{eq.11LT2m},\ref{eq.11Lp}), we find
\begin{eqnarray}
    L^\mu_\nu &=& { L          }_{\mu,\nu}\\
    L_\mu^\nu &=& { (L^{-1})^T }_{\mu,\nu}
\end{eqnarray}
  The Lorentz invariance,
\begin{eqnarray}
    a_\mu b^\mu \to a'_\mu b'^\mu
               = L_\mu^\alpha L^\mu_\beta a_\alpha b^\beta
               = \delta^\alpha_\beta    a_\alpha b^\beta
               = a_\alpha b^\alpha
\end{eqnarray}
  tells
\begin{eqnarray}
    L_\mu^\alpha L^\mu_\beta = \delta^\alpha_\beta \label{eq.11Ldt}
\end{eqnarray}
  corresponding to
\begin{eqnarray}
    ((L^{-1})^T)^T L = L^{-1} L = 1. \label{eq.11LLp}
\end{eqnarray}

{\bf hw11-1}: Show Eq.\ref{eq.11LLp}, when we write the 4-component column vector
  representations of $a_\mu$ and $b^\mu$ as ($a_\mu$) and ($b^\mu$):
\begin{eqnarray}
  (a_\mu) &=& (a_0, a_1, a_2, a_3)^T = (a^0, -a^1, -a^2, -a^3)^T \\
  (b^\mu) &=& (b^0, b^1, b^2, b^3)^T
\end{eqnarray}
  and express the Lorentz transformation as
\begin{eqnarray}
  (b^\mu) \to ({b'}^\mu) &=&      L     (b^\mu) \\
  (a_\mu) \to ({a'}_\mu) &=& (L^{-1})^T (a_\mu)
\end{eqnarray}
   The identity Eq.\ref{eq.11LLp} should then follow from the invariance
\begin{eqnarray}
  (a'_\mu)^T (b'^\mu) = (a_\mu)^T (b^\mu)
\end{eqnarray}
  Although this may be trivial, it is useful for you to know the
  relationship between the Matrix representation and the tensor
  representation, since unlike $L^{-1} L = 1$ in the Matrix representation,
  the Tensor representation condition Eq.\ref{eq.11Ldt} is not so trivial. It is the identity behind the Einstein's convention of contracting repeated
  upper and lower indices to reduce the rank (the number of Lorentz
  indices) of a tensor, while keeping Lorentz covariance/invariance.


  In the above, $L^\mu _\nu$ or its matrix representation $L_{\mu,\nu}$ gives
  the Lorentz transformation, $\omega_{ab}$ ($a,b=0,1,2,3$) are real
  numbers which measure the magnitude of the transformations,
  $M^{ab}$ denote the generators of the transformation. In the last step
  of Eq.\ref{eq.11LT1}, infinitesimal transformations are given.  Most importantly,
  the generators $M^{ab}$ are anti-symmetric in the exchange of their
  Lorentz indices
\begin{eqnarray}
  M^{ba} = - M^{ab} \label{eq.11Mantis}
\end{eqnarray}
  and therefore
  \begin{eqnarray}
    \omega_{ba} = - \omega_{ab} \label{eq.11omeantis}
  \end{eqnarray}
 Hence there are 6 independent generators and their parameters:
\begin{eqnarray}
  \frac{1}{2} \omega_{ab} M^{ab}
  &=& \omega_{23} M^{23} +\omega_{31} M^{31} +\omega_{12} M^{12} +\omega_{01} M^{01} +\omega_{02} M^{02} +\omega_{03} M^{03} \\
  &=& \theta_1 J_1  +\theta_2 J_2  +\theta_3 J_3  +y_1 K_1  +y_2  K_2  +y_3  K_3, \label{eq.11gepara}
\end{eqnarray}
 The 6 generators are named as
\begin{eqnarray}
  M^{ij} = \epsilon_{ijk} J_k,  ~~ \omega_{ij} = \epsilon_{ijk} \theta_k, ~~ M^{0k} = K_k,  ~~              \omega_{0k} = y_k, \label{eq.11LTgener}
\end{eqnarray}
  where $a$,$b$,$c$ counts $0$ to $3$, while $i$,$j$,$k$ counts $1$ to $3$.

{\bf hw11-2}: Derive Eq.\ref{eq.11gepara} from Eq.(\ref{eq.11Mantis},\ref{eq.11omeantis},\ref{eq.11LTgener}). $J_k$ gives the right-handed rotation about the k-axis, by the angle $\theta_k$, and $K_k$ gives the boost along the k-axis, by a real number $y_k$, which is called the rapidity, that measures the relative velocity between the original and the boosted frame along the k-axis, just like $\theta_k$ measures the relative azimuthal angle about the k-axis between the original and the rotated frame.

  In my lecture today, I introduced the coordinate transformations,
  and obtain explicit matrix form of all the 6 generators.  It is
  then easy to obtain the commutation relations among all
  the 6 generators.

  We found that there is no ambiguity in the overall magnitude
  and the sign of the the 3 generators of rotations, since they
  should satisfy ($k,l,m = 1,2,3$)
\begin{eqnarray}
  [ J_k, J_l ] = i\epsilon_{klm} J_m
\end{eqnarray}
  where the anti-symmetric tensor $\epsilon_{123}=1$ defines the right-handed rotation. The 3 generators of rotations make the $SU(2)$ algebra, and hence the spatial rotations and the $SU(2)$ transformations have the same group structure.

{\bf hw11-3}: Please show Eq.\ref{eq.11defthet3}, Eq.\ref{eq.11defy3} by using Eq.(\ref{eq.11vatran1},\ref{eq.11vatran2},\ref{eq.11omegaanti}).

\begin{eqnarray}
\omega_{12}  
= -\omega_{21}
=  \omega^{12}
= -\omega^1_{~2}
=  \omega^2_{~1} \equiv  \theta_3 \label{eq.11defthet3} 
\end{eqnarray}
\begin{eqnarray}
 \omega_{03}     
  = -\omega_{30}
  = -\omega^{03}
  =  \omega^0_{~3}
  =  \omega^3_{~0}
  \equiv   y_3 \label{eq.11defy3}
\end{eqnarray}
by using the metric tensor $g^{ab} = g_{ab} = {\rm diag}(1,-1,-1,-1)$ which
  dictates the transformation between the covariant and the
  contra-variant vector indices ($a,b = 0$ to $3$):
\begin{eqnarray}
 v^a &=& g^{ab} v_b \label{eq.11vatran1}\\
 v_a &=& g_{ab} v^b \label{eq.11vatran2}
\end{eqnarray}
  and by using the anti-symmetricity
\begin{eqnarray}
  \omega_{ab} = -\omega_{ba} \label{eq.11omegaanti}
\end{eqnarray}
   From this exercise, you should confirm that
\begin{eqnarray}
  \omega^i_j = -\omega^j_i~~~   (i,j=1,2,3)
\end{eqnarray}
  for rotations and
\begin{eqnarray}
  \omega^0_k = \omega^k_0  ~~~ (k=1,2,3)
\end{eqnarray}
  for boosts, which distinguishes the two transformations in the general
  expression for Lorentz transformations,
\begin{eqnarray}
  x^\mu \to {x'}^\mu = L^\mu_\nu x^\nu
  = (e^{ i/2 \omega_{ab} M^{ab} })^\mu_\nu x^\nu
  = x^\mu - \omega^\mu_\nu x^\nu + {\cal O}((\omega_{ab})^2)
\end{eqnarray}
  With exactly the same token, you should observe
\begin{eqnarray}
  J_3 &=& M^{12} = -M^{21} = M_{12} = -M^1_2 = M^2_1 = \cdots \\
  J_2 &=& M^{31} = -M^{13} = M_{31} = -M^3_1 = M^1_3 = \cdots \\
  K_3 &=& M^{03} = -M^{30} = -M_{03} = -M^3_0 = -M^0_3 = \cdots
\end{eqnarray}
  from which you can show
\begin{eqnarray}
  \frac{1}{2} \omega_{ab} M^{ab}
  &=& \omega_{23} M^{23} +\omega_{31} M^{31} +\omega_{12} M^{12} +\omega_{01} M^{01} +\omega_{02} M^{02} +\omega_{03} M^{03} \\
  &=& \theta_1 J_1  +\theta_2 J_2  +\theta_3 J_3  +y_1 K_1  +y_2  K_2  +y_3  K_3, 
\end{eqnarray}
when one sums over all the four-vector indices ($a,b = 0$ $to 3$).

{\bf hw11-4}: Confirm the above expressions (24), (25), (28) with the
  help of (26), (27).

  This lengthy introduction shows how to connect the tensor and the
  matrix notations.  From now on, I will use only matrix notations
  to explain everything you need to know about Lorentz symmetry,
  and its representations.

  Let us first obtain the matrix representation of the rotation operators:
\begin{eqnarray}
  J_1 = 
  \begin{pmatrix}
    0 & 0 & 0 & 0\\
    0 & 0 & 0 & 0 \\
    0 & 0 & 0 & -i \\
    0 & 0 & i & 0
  \end{pmatrix}
  ,~~ J_2 =
  \begin{pmatrix}
    0 & 0 & 0 & 0 \\
    0 & 0 & 0 & i \\
    0 & 0 & 0 & 0 \\
    0 & -i & 0 & 0 
  \end{pmatrix}
  ,~~J_3=
  \begin{pmatrix}
    0 & 0 & 0 & 0 \\
    0 & 0 & -i & 0 \\
    0 & i & 0 & 0 \\
    0 & 0 & 0 & 0 
  \end{pmatrix}
\end{eqnarray}
  It is easy to remember the above matrices as
\begin{eqnarray}
  (J_1)_{23} = (J_2)_{31} = (J_3)_{12} = -i
\end{eqnarray}
  or as
\begin{eqnarray}
  (J_i)_{jk} = -i \epsilon_{ijk}. \label{eq.11Jjk}
\end{eqnarray}
We showed that the above 3 matrices satisfy the commutation relations
of the rotations:
\begin{eqnarray}
  [ J_j, J_k ] = i\epsilon_{jkl} J_l \label{eq.11Jjk2}
\end{eqnarray}
 in one of our past home works (in the $3 \times 3$ matrix form).

{\bf hw11-5}: Please show Eq.\ref{eq.11Jjk2} again here, since now our generators are $4 \times 4$
  Hremetian matrices, making them a part of Lorentz transformations.

  Let us confirm again here, that there is no ambiguity in the sign or
  the magnitude of the above generators.

{\bf hw11-6}: Although it is trivial, please obtain the commutation relations
  among $M_1$, $M_2$, $M_3$ when $M_k=2J_k$ ($k=1,2,3$), and also among
  $N_1$, $N_2$, $N_3$ when $N_k = -J_k$ ($i=1,2,3$).

  The first example shows that the commutation relations Eq.\ref{eq.11Jjk2} fixes the
  overall normalization of the generators, which led to quantization
  of angular momentum in quantum mechanics.  The second example shows
  that the sign of the generators are also fixed. In fact, the
  generators with the negative sign give left-handed rotations.

  Our $4 \times 4$ generators for $J_k$ ($k=1,2,3$) are often called the vector
  representation of rotations, or the adjoint representation of the
  rotation group, $SU(2)$. There is one useful identity which I often
  used in my research, regarding the identity Eq.\ref{eq.11Jjk}.  Please note that
  the r.h.s. of Eq.\ref{eq.11Jjk} is minus $i$ times the structure constant of $SU(2)$.

  This can be generalized.  Let the $SU(n)$ commutators be
\begin{eqnarray}
  [ T^a, T^b ] = i f_{abc} T^c \label{eq.11tatb}
\end{eqnarray}
  The adjoint representation of the generators is then
\begin{eqnarray}
  (T^a(A))_{bc} = (F^a)_{bc} = -i f_{abc}
\end{eqnarray}

{\bf hw11-7}: Please write down the commutation relation Eq.\ref{eq.11tatb} explicitly in
  the adjoint representation, by using $(F^a)_{bc}$ notation, and confirm
  that it is nothing but the Jacobi identity,
\begin{eqnarray}
  [A,[B,C]] + [B,[C,A]] + [C,[A,B]] = 0 \label{eq.11Jacobi}
\end{eqnarray}
hint: In the ajoint representation, the commutation relation (24) can
  be written as
\begin{eqnarray}
  [ F^a, F^b ]_{de} = i f_{abc} F^c_{de}
\end{eqnarray}
  Please also prove the Jacobi identity Eq.\ref{eq.11Jacobi}.

{\bf hw11-8}: Please show that the following $3$ matrices
\begin{eqnarray}
  K_1 =
  \begin{pmatrix}
     0 & i & 0 & 0\\ 
     i & 0 & 0 & 0 \\
     0 & 0 & 0 & 0 \\
     0 & 0 & 0 & 0 
  \end{pmatrix}
  ,~~K_2 =
  \begin{pmatrix}
    0 & 0 & i & 0\\ 
    0 & 0 & 0 & 0 \\
    i & 0 & 0 & 0 \\
    0 & 0 & 0 & 0
  \end{pmatrix}
   ,~~K_3=
   \begin{pmatrix}
    0 & 0 & 0 & i \\
    0 & 0 & 0 & 0 \\
    0 & 0 & 0 & 0 \\
    i & 0 & 0 & 0 
   \end{pmatrix} \label{eq.11k123matx}
\end{eqnarray}
  satisfy the following commutation relations among themselves and
  with $J_k$:
\begin{eqnarray}
  &&[ K_l, K_m ] = -i\epsilon_{lmn} J_n \label{eq.11klm}\\
  &&[ J_l, K_m ] =  i\epsilon_{lmn} K_n \label{eq.11jlkm}
\end{eqnarray}
  The set of the above commutation relations and that of rotations
\begin{eqnarray}
  [ J_l, J_m ] =  i\epsilon_{lmn} J_n \label{eq.11jlm}
\end{eqnarray}
  gives the complete algebra of the Lorentz transformation.  Because
  all the commutators of the 6 generators of the Lorentz transformations
  form an algebra (the commutation relations), the transformations
  generated by them, the Lorentz transformations form a group. It is
  called the Lorentz group.
  Please note that the boost generators $K_l$ are NOT Hermetian,
\begin{eqnarray}
  (K_l)^\dagger = -K_l
\end{eqnarray}
  (they are anti-Hermetian), but it is symmetric:
\begin{eqnarray}
  (K_l)^T =  K_l
\end{eqnarray}
  From the above commutation relations Eq.(\ref{eq.11klm},\ref{eq.11jlkm},\ref{eq.11jlm}), the sign of the boost
  generatrors $K_l$'s are not fixed, because the replacement of $K_l$ by
  $-K_l$ does not change the algebra. This fact has a very important
  consequences when we study fermionic representations of the Lorentz
  transformations.

  The function of boost generators, $K_i (i=1,2,3)$ in Eq.\ref{eq.11k123matx} is understood by the coordinate transformation:
\begin{eqnarray}
  L(y_3 = y)
  = e^{ i K_3 y }=
  \begin{pmatrix}
    \cosh y & 0 & 0 & -\sinh y \\
    0 & 1 & 0 & 0 \\
    0 & 0 & 1 & 0 \\
    -\sinh y & 0 & 0 & \cosh y
  \end{pmatrix}
\end{eqnarray}

{\bf hw11-9}: Please obtain the above boost matrix by using the definition
  of the exponential function of a matrix.
  Now the Lorentz boost of the coordinate system along the z-axis by
  the rapidity y makes:
\begin{eqnarray}
  \begin{pmatrix}
    t' \\ x' \\ y' \\ z'
  \end{pmatrix}
  = L(y_3=y)
  \begin{pmatrix}
    t \\ x \\ y \\ z
  \end{pmatrix}
  =
  \begin{pmatrix}
    t\cosh y -z\sinh y \\ x \\ y \\ z\cosh y -t\sinh y
  \end{pmatrix}
\end{eqnarray}

{\bf hw11-10}:
  Please show that the rapidity $y$ is related to the relative velocity
  $\beta$ between the two coordinate system as
\begin{eqnarray}
  \beta = \tanh y , ~{\rm or}~ y = \tanh^{-1}(\beta)
\end{eqnarray}
  by noting,
\begin{eqnarray}
  \cosh y &=& \gamma      =     1/\sqrt{1-\beta^2} \\
  \sinh y &=& \gamma\beta = \beta/\sqrt{1-\beta^2}
\end{eqnarray}
  Just like the angles are summed up when two rotations about the same
  axis are performed,
\begin{eqnarray}
  L(\theta_3=\phi_a) L(\theta_3=\phi_b)
   = e^{ i\phi_a J_3 } e^{ i\phi_b J_3 }
   = e^{ i(\phi_a+\phi_b) J_3 }
   = L(\theta_3 = \phi_a + \phi_b ) \label{eq.11rotatphia}
\end{eqnarray}
  the rapidities add up when two boosts along the same axis are performed
\begin{eqnarray}
  L(y_3 = a) L(y_3 = b)
  = e^{ i a K_3 } e^{ i b K_3 }
  = e^{ i (a+b) K_3 }
  = L(y_3 = a+b ) \label{eq.11boosta}
\end{eqnarray}
  Here, $L(\theta _3=\phi_a)$ means that all the other parameters are zero
  ($\theta_1=\theta_2=y_1=y_2=y_3=0$).

{\bf hw11-11}: Show Eq.\ref{eq.11rotatphia} and Eq.\ref{eq.11boosta}.
  In this regard, the rapidity is called the angular variable of the
  boost operations.  Please note the basic property of the hyperbolic
  functions,
\begin{eqnarray}
  \cosh^2 y - \sinh^2 y = 1 \label{eq.11coshsinh}
\end{eqnarray}
that ensures the invariance
\begin{eqnarray}
{t'}^2 - {x'}^2 - {y'}^2 - {z'}^2
= t^2   -  x^2   -  y^2   -  z^2
\end{eqnarray}
 The identity Eq.\ref{eq.11coshsinh} is related to the property of triangular functions
\begin{eqnarray}
  \cos^2\theta + \sin^2\theta = 1
\end{eqnarray}
  that ensures the conservation of the norm under rotations.
  Finally, let us define the Lorentz transformation of vectors (objects)
  when the coordinate system is transformed as above.

  The only basic property which you should understand is that the
  vectorial objectfs (four-momentum vector of a particle, vector fields,
  etc) transform in the opposite direction of the coordinate
  transformations.

  Since we will study the Lorentz transformation of the particle
  four-momentum and the vector/fermionic fields in the following
  lectures, let me define the rotation and the boost operators of
  the objects along the $m$-axis as:
\begin{eqnarray}
  R_m(\theta_m) &=& e^{ -i \theta_m J_m } ~~~ (m=1,2,3 ~{\rm or}~ x,y,z) \\
  B_m(y_m)      &=& e^{ -i y_m      K_m } ~~~(m=1,2,3 ~{\rm or}~x,y,z)
\end{eqnarray}
  Please note the minus sign in the exponent, which does not exist for
  the coordinate transformations.

{\bf hw11-12}:
  Starting from the four-momentum vector in the rest frame of a massive
  particle,
\begin{eqnarray}
  p^\mu &=& 
  \begin{pmatrix}
    m \\ 0 \\ 0 \\ 0
  \end{pmatrix}
\end{eqnarray}
  Please show
\begin{eqnarray}
  {p'}^\mu &=& B_z(y) p^\mu =
  \begin{pmatrix}
    m \cosh y \\ 0 \\ 0 \\ m \sinh y 
  \end{pmatrix}
  =
  \begin{pmatrix}
   E \\ 0 \\ 0 \\ p
  \end{pmatrix} \\
 {p{''}}^\mu &=& R_y(\theta)B_z(y) p^\mu =
  \begin{pmatrix}
    E \\ p \sin\theta \\ 0 \\ p\cos\theta
  \end{pmatrix} \\
  {p{'''}}^\mu &=& R_z(\phi)R_y(\theta)B_z(y) p^\mu =
  \begin{pmatrix}
    E \\ p\sin\theta\cos\phi \\ p\sin\theta\sin\phi \\ p\cos\theta
  \end{pmatrix}
\end{eqnarray}
  The combined Lorentz transformation
\begin{eqnarray}
  L(\theta_3=\phi) L(\theta_2=\theta) L(y_3=y)
  = R_z(\phi)          R_y(\theta)          B_z(y)
  = e^{ -i\phi J_3 } e^{ -i\theta J_2 } e^{ -iy K_3 }
\end{eqnarray}
gives the standard (polar-coordinate) parametrization of the 4-momentum.\\

That's all for hw11.\\

Best regards,\\

Kaoru


\end{document}

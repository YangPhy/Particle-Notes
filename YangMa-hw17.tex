\documentclass[11pt]{article}
\usepackage{amsmath,graphicx,color,epsfig,physics}
%\usepackage{pstricks}
\usepackage{float}
\usepackage{subfigure}
\usepackage{slashed}
\usepackage{color}
\usepackage{multirow}
\usepackage{feynmp}
\usepackage[top=1in, bottom=1in, left=1.2in, right=1.2in]{geometry}
\def\del{{\partial}}
\def\dgr{\dagger}
\def\eps{\epsilon}
\def\lmd{\lambda}

\begin{document}
\title{Particle physics HW17}
\author{Yang Ma}

\maketitle

\section{ }
\begin{eqnarray}
    \frac{1}{n!}T\left[i\int d^4x {\cal L}_{int}(x)\right]^n & = &
    \frac{1}{n!} T \int \prod_{k=1}^n d^4x_k i {\cal L}_{int}(x_k)\\
    & =&\frac{1}{n!} \int T \left[\prod_{k=1}^{n}d^4x_k i{\cal L}_{int}(x_k)\right]\\
    & = &\frac{1}{n!}\prod_{k=1}^{n-1} \int i\mathcal{L}_{int} (x_k) \Theta(x_k^0 - x_{k+1}^0) \int d^4 x_n i\mathcal{L}_{int}(x_n)
\end{eqnarray}

\section{ }
\begin{eqnarray}
   && d^4x' =\det({L^\mu}_\nu) d^4x = d^4x\\
   && {\cal L}'_{int}(x') = {\cal L}_{int}(x)\\
   && \int d^4x' {\cal L}'_{int}(x') = \int d^4x {\cal L}_{int}(x)
\end{eqnarray}

\section{ }
Skip.
\section{ }
\begin{eqnarray}
\bra{f}1\ket{i}=\bra{0}a(p_\tau h_\tau) b(p_{\overline \tau},h_{\overline \tau})a^\dagger (p_H) \ket{0}=\bra{0}a^\dagger (p_H) a(p_\tau,h_\tau)b(p_{\overline \tau},h_{\overline \tau})\ket{0} =0
\end{eqnarray}

\section{ }
Recall {\bf hw10-5}, by making the replacement for lepton mass
\begin{eqnarray}
    m_l \to m_l (1+H/v)
  \end{eqnarray}
in
\begin{eqnarray}
    {\cal L}_{Yukawa} &=& l_L^\dagger M^l l_R + h.c. \\
    &=& - (e_L^\dagger, \mu_L^\dagger, \tau_L^\dagger) {\rm diag}\{m_e,m_\mu,m_\tau\} (e_R,\mu_R,\tau_R)^T
    +
    h.c. \\
    &=& - \sum_l g_{Hll} m_l (l_L^\dagger l_R + l_R^\dagger l_L),
  \end{eqnarray}
  we see
  \begin{eqnarray}
    {\cal L}_{Hll} &=& - \sum_l  m_l (1+H/v) (l_L^\dagger l_R + l_R^\dagger l_L)
 \end{eqnarray}
see $g_{H\tau\tau}=m_\tau/v$ in 
\begin{eqnarray}
    {\cal L}_{H\tau\tau} = - g_{H\tau\tau} H (l_{\tau,L}^\dagger l_{\tau,R} + l_{\tau,R}^\dagger l_{\tau,L})=- g_{H\tau\tau} H {\overline \Psi}_\tau \Psi_\tau.
\end{eqnarray}

\section{ }
\begin{itemize}
    \item Quark and lepton fields are proportional to $\phi$ (or $\phi^c$) in ${\cal L}_{Yukawa}$ so the mass terms should be modified as $m\to m(1+H(x)/v)$ by $v \to v+H(x) = v(1+H(x)/v)$. 
    \item Neutrino fields are proportional to $\phi^2$ in ${\cal L}_{Yukawa}$ so the mass terms should be modified as $m\to m(1+H(x)/v)^2$ by $v \to v+H(x) = v(1+H(x)/v)$.
    \item Recall
    \begin{eqnarray}
        D_\mu = \del_\mu +\frac{ig}{\sqrt 2} (T^+ W_\mu^+ + T^-W_\mu^-)+ig_Z(T^3-Q\sin^2\theta_W)Z_\mu + iQeA_\mu,
    \end{eqnarray}
    note that $Q=0$ for Higgs, so we have
    \begin{eqnarray}
        (D_\mu \phi)^\dagger (D^\mu \phi) =  \frac{1}{ 2}\del_\mu H \del^\mu H + \frac{g^2}{2}W_\mu^-{W^+}^\mu \frac{(v+H)^2}{2}+\frac{g_Z^2}{4} \frac{(v+H)^2}{2} Z_\mu Z^\mu. 
    \end{eqnarray}
    after $v \to v+H(x) = v(1+H(x)/v)$.
    Then we see 
    \begin{eqnarray}
        {\cal L}_{Higgs} &=& (D_\mu \phi)^\dagger (D^\mu \phi) - V(\phi) \\ 
        &=& \frac{1}{2}\del_\mu H \del^\mu H + \frac{g^2}{2}W_\mu^-{W^+}^\mu \frac{(v+H)^2}{2}+\frac{g_Z^2}{4} \frac{(v+H)^2}{2} Z_\mu Z^\mu -V(\phi) \\
        &=& \frac{1}{2} \del_\mu H \del^\mu H+
        m_W^2 W^+_\mu {W^-}^\mu
        + \frac{m_Z^2}{2} Z_\mu Z^\mu -\frac{m_H^2}{2} H^2
        + \cdots 
    \end{eqnarray}
    The $W$ and $Z$ mass terms should be modified as $m^2\to m^2(1+H(x)/v)^2$
\end{itemize}
Recall {\bf hw10-3}, {\bf hw10-4}, we can write the Higgs potential
\begin{eqnarray}
    &&{\cal L}_{Higgs~potential}=-V(\phi) =-\frac{\lambda}{4} (\phi^\dagger \phi)^2 - \mu^2 (\phi^\dagger \phi) 
    =-\frac{\lambda}{4} (\phi^\dagger \phi)^2 + \frac{\lambda v^2}{4}(\phi^\dagger \phi) \\
    &=& -\frac{\lambda}{4} \frac{(v+H)^4}{4} +\frac{\lambda v^2}{4}\frac{(v+H)^2}{2} 
    = -\frac{\lambda}{16}(H^4+4vH^3+4v^2H^2-v^4) \\
    &=&  -\frac{\lambda v^2}{2}  \frac{H^2}{2}
    -\frac{3 \lambda v}{2}   \frac{H^3}{3!}
    -\frac{3 \lambda }{2}   \frac{H^4}{4!},
\end{eqnarray}

If we replace $\frac{1}{2} m_H^2 H^2$ by $m_H^2 \rightarrow m_H^2(1+H/v)^2$ we see:
\begin{eqnarray}
    -\frac{1}{2} m_H^2 H^2  \to -\frac{1}{2} m_H^2 H^2 (1+\frac{H}{v})^2 &=& -\frac{1}{2} m_H^2 H^2 - \frac{m_H^2}{v} H^3 - \frac{m_H^2}{2v^2} H^4 \\
    & =& -\frac{\lambda v^2}{4} H^2 - \frac{\lambda v}{2} H^3 - \frac{\lambda}{4} H^4
\end{eqnarray}
So we get the wrong $H^3$ and $H^4$ terms in the Higgs potential.


\section{ }
To calculate
\begin{eqnarray}
  S_{fi}= -i\frac{m_\tau}{v} \int d^4 x
  \bra{0} b(p_{\overline \tau},h_{\overline \tau}) a(p_\tau,h_\tau)
      H(x) {\overline \Psi}_\tau(x) \Psi_\tau(x) a(p_H)^\dgr \ket{0} ,
\end{eqnarray}
we need to recall the three fields,
\begin{eqnarray}
  H(x)  &=& \int \frac{d^3p}{(2E)(2\pi)^3} [ a({\vec p}) e^{-ipx} + a(\vec p)^\dgr e^{ipx} ]  \\
\Psi(x)  &=& \int \frac{d^3p}{(2E)(2\pi)^3}
  \sum_{h=\pm1/2} [a({\vec p},h)     e^{-ipx} u({\vec p},h)
  + b({\vec p},h)^\dgr e^{ipx}  v({\vec p},h)]  \\ 
{\overline \Psi}(x) &=& \int \frac{d^3p}{(2E)(2\pi)^3} \sum_{h=\pm1/2}
  { a({\vec p},\lambda)^\dgr e^{ipx}  {\overline u}({\vec p},\lambda)
  + b({\vec p},\lambda)      e^{-ipx} {\overline v}({\vec p},\lambda) }
\end{eqnarray}
and pluig them into $S_{fi}$
\begin{eqnarray}
  S_{fi} &=& -i\frac{m_\tau}{v} \int d^4 x
  \bra{0} b(p_{\overline \tau},h_{\overline \tau}) a(p_\tau,h_\tau)
  \int \frac{d^3p}{(2E)(2\pi)^3} a_H(p) e^{-ipx}
  \nonumber \\&&\int \frac{d^3p'}{(2E')(2\pi)^3} \sum_{h'}
      a_\tau(p',h')^\dgr e^{ip'x}  {\overline u}(p',h')
      \nonumber \\&& \int \frac{d^3p{''}}{(2E{''})(2\pi)^3} \sum_{h{''}}
      b_{\overline \tau}(p{''},h{''})^\dgr e^{ip{''}x} v(p{''},h{''})
  a(p_H)^\dgr \ket{0} 
\end{eqnarray}
By using the commutation and anti-commutations among operators we finally obtain
\begin{eqnarray}
  S_{fi}
  &=& -i\frac{m_\tau}{v} \int d^4 x
  \bra{0} e^{-ix p_H}
      e^{ ix p_\tau} {\overline u}(p_\tau,h_\tau)
      e^{ ix p_{\overline \tau}} v(p_{\overline \tau},h_{\overline \tau}) \ket{0} \\
      &=&-i\frac{m_\tau}{v} \int d^4 x
      e^{ix(p_\tau+p_{\overline \tau}-p_H)}
      {\overline u}(p_\tau,h_\tau) v(p_{\overline \tau},h_{\overline \tau})  \bra{0}\ket{0} \\
      &=& -i\frac{m_\tau}{v} (2\pi)^4 \delta^4( p_\tau+p_{\overline \tau}-p_H )
      {\overline u}(p_\tau,h_\tau) v(p_{\overline \tau},h_{\overline \tau}) 
\end{eqnarray}

\section{ }
Since we know
\begin{eqnarray}
    S^\dagger S = (1 + iT)^\dagger (1+iT)
                    = 1 -i(T^\dagger - T) + T^\dagger T
                    = 1 ,
\end{eqnarray}
one then have 
\begin{eqnarray}
    -i(T^\dagger - T) + T^\dagger T=0,
\end{eqnarray}
which is 
\begin{eqnarray}
    -i(T-T^\dagger) = T^\dagger T 
\end{eqnarray}


\section{ }
Since we know, we can take the matrix element
\begin{eqnarray}
    \bra{i} -i(T-T^\dagger) - T^\dagger T \ket{i} = -i(T-T*) - (T^\dagger T)_{ii} =0,
\end{eqnarray}
which leads to 
\begin{eqnarray}
    2{\rm Im}T_{ii}=(T^\dagger T)_{ii} = \sum_{f}\bra{i}T^\dagger \ket{f}\bra{i}T\ket{i} =\sum_f |T_{fi}|^2.
\end{eqnarray}




















\end{document}
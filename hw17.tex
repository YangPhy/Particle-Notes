\documentclass[12pt]{article}
\usepackage{amsmath,graphicx,color,epsfig,physics}
\usepackage{float}
\usepackage{subfigure}
\usepackage{slashed}
\usepackage{color}
\usepackage{multirow}
\usepackage{feynmp}
\textheight=9.5in \voffset=-1.0in \textwidth=6.5in \hoffset=-0.5in
\parskip=0pt
\def\del{{\partial}}
\def\dgr{\dagger}
\def\eps{\epsilon}

\begin{document}

\begin{center}
{\large\bf HW17 for Advanced Particle Physics} \\

\end{center}

\vskip 0.2 in

Dear students,
We are now ready to introduce QFT in covariant perturbation theory.

  First, the full Lagrangian of a QFT is divided into the free part
  and the interaction part,
\begin{eqnarray}
  {\cal L} = {\cal L}_{free} + {\cal L}_{int} \label{eq.17_1}
\end{eqnarray}
  We solve the equation of motions of the free Lagrangian, and use the
  solutions (the free field expansions for spin $0$, $1/2$, and $1$ particles
  which we derived in the past lectures) as the basis of evaluating
  the effects of the interactions in ${\cal L}_{int}$.  In this basis (the
  interaction basis), the S-matrix (Scattering matrix) is defined as
\begin{eqnarray}
  S = {\cal T} e^{ i\int d^4x {\cal L}_{int}(x) } \label{eq.17_2}
\end{eqnarray}
    where the exponential factor is defined as usual
\begin{eqnarray}
  e^{ i\int d^4x {\cal L}_{int}(x) }
  =
  \sum_{n=0}^\infty \frac{1}{n!} [i\int d^4x {\cal L}_{int}(x)]^n \label{eq.17_3}
\end{eqnarray}
    whereas the $T$ simbol stands for Time-ordering, whenever the operators
  with two different space-time points ($x^\mu$) are multiplied, such that
  an operator in the left-hand-side applies after the operator in the
  right-hand-side. For instance,
\begin{eqnarray}
  {\cal T} A(x) B(y) =   \Theta(x^0-y^0) A(x) B(y)
                 \pm \Theta(y^0-x^0) B(y) A(x) \label{eq.17_4}
\end{eqnarray}
  where the plus sign is chosen when $A(x)$ and $B(y)$ commute, while the
  minus sign is chosen when they anti-commute.  Please note that when
  $A(x)$ and $B(y)$ are ${\cal L}_{int}(x)$ and ${\cal L}_{int}(y)$, they always commute because
  the Lagrangian density is a scalar (bosonic).

  Therefore,
\begin{eqnarray}
  S &=& {\cal T} e^{ i\int d^4x {\cal L}_{int}(x) }
       = {\cal T} \sum_{n=0}^\infty (1/n!) [i\int d^4x {\cal L}_{int}(x)]^n \nonumber \\
       &=& 1
       + i\int d^4x {\cal L}_{int}(x)
       + \frac{1}{2} {\cal T} [i\int d^4x {\cal L}_{int}(x)] [i\int d^4y{\cal L}_{int}(y)]
       + \cdots \nonumber \\
       &&+ \frac{1}{n!} {\cal T} [i\int d^4x_1 {\cal L}_{int}(x_1)] [i\int d^4x_2 {\cal L}_{int}(x_2)]
                            \cdots           [i\int d^4x_n {\cal L}_{int}(x_n)]
       + \cdots \label{eq.17_5}
\end{eqnarray}
  where the time-ordering applies for the $n=2$ and higher power terms.
  Let us first note that the $1/n!$ factor in the n'th power cancels once
  we introduce the time-ordering
\begin{eqnarray}
  \frac{1}{n!} {\cal T} [i\int d^4x {\cal L}_{int}(x)]^n  &=& 
  [i\int d^4x_1{\cal L}_{int}(x_1)] \Theta(x_1^0-x_2^0) 
  [i\int d^4x_2 {\cal L}_{int}(x_2)] \Theta(x_2^0-x_3^0) \nonumber \\
  &&[i\int d^4x_3 {\cal L}_{int}(x_3)] \Theta(x_3^0-x_4^0) 
  \cdots 
  [i\int d^4x_n {\cal L}_{int}(x_n)]  \label{eq.17_6}
\end{eqnarray}
 because there are $n!$ combinations of our labeling of
  $\{x_1^\mu, x_2^\mu, x_3^\mu, ..., x_n^\mu\}$, while all labeling cases
  give exactly the same results, cancelling the $1/n!$ factor.

{\bf hw17-1}: Show Eq.\ref{eq.17_6}, or equivalently, Eq.\ref{eq.17_8} below.

hint: Please note that the region of integration over $x_k^0$ is
  $\infty > x_k^0 > -\infty$ for all $k=1$ to $n$. In the r.h.s. of Eq.\ref{eq.17_6}, the integration region is restricted to
\begin{eqnarray}
  \infty > x_1^0 > x_2^0 > \cdots > x_{n-1}^0 > x_n^0 > -\infty \label{eq.17_7}
\end{eqnarray}
  We can generalize the above theorem to a $n$'th power of an arbitrary
  integral in the region $b > t > a$:
\begin{eqnarray}
  \frac{1}{n!} [ \int_a^b dt F(t) ]^n &=& [ \int dt_1 F(t_1)]  \Theta(b-t_1) \cdot \Theta(t_1-t_2)  [ \int dt_2 F(t_2)]  \Theta(t_2-t_3) \cdots \nonumber \\&&  [ \int dt_{n-1} F(t_{n-1})] \Theta(t_{n-1}-t_n)  [ \int dt_n F(t_n)]   \Theta(t_n-a) \label{eq.17_8}
\end{eqnarray}
  When $F(t)>0$ in the whole region of $b>t>a$ (positive definite), it leads
  to a powerful tool to simulate exponential of complicated functions
  (such as a Sudakov form factor in QED and QCD) by a Monte Carlo method.
  When you hear ``ordering parameter'', you may recall this theorem.

  The scattering amplitudes in perturbative QFT are defined as the matrix
  elements of the above S-matrix Eq.\ref{eq.17_2}, sandwiched between the initial state from the right-hand-side and the final state from the left-hand-side:
\begin{eqnarray}
  S_{fi}
  &=& \bra{final~state} S \ket{initial~state}
  = \bra{final~state} T e^{ i\int d^4x {\cal L}_{int}(x) } \ket{initial~state} \\
  &=& \sum_n \bra{final~state} \frac{1}{n!}T[i\int d^4x {\cal L}_{int}(x)]^n \ket{initial~state} \label{eq.17_9}
\end{eqnarray}
  Here, the initial and the final states are, respectively, a set of
  particles with definite three-momentum and on-shell energy
  ($E = \sqrt{|{\vec p}|^2+m^2}$), such as
\begin{eqnarray}
  &&\ket {in}=a^\dgr(p_1) b^\dgr(p_2) \ket{0} \label{eq.17_10a} \\
  &&\ket {in}= a^\dgr(p_3) b^\dgr(p_4) \ket{0} \label{eq.17_10b} 
\end{eqnarray}
  for a pair of a particle and an anti-particle.  All the external particle
  energies are on their mass-shell, $p_k^0 = E_k = \sqrt{|{\vec p}_k|^2+m_k^2}$, and hence their wave functions are plane waves,
\begin{eqnarray}
  e^{ -ip_k x } = e^{ -i(E_k t - {\vec p}\cdot {\vec x}) } \label{eq.17_11}
\end{eqnarray}
  The time-ordering of the S-matrix Eqs.(\ref{eq.17_2}.\ref{eq.17_5},\ref{eq.17_6}) tells that only the right-most operators with the earliest time can match the initial state operators, while the left-most operators with the last time can match the final state operators. If the order of the expansion is $n=2$ or larger, field operators inside several ${\cal L}_{int}(x_k)$'s can be connected within the time-ordering constraint.

  The S-matrix is Poincare invariant because
\begin{eqnarray}
  \int d^4x {\cal L}_{int}(x) \to \int d^4x' {\cal L}_{int}(x')' = \int d^4x {\cal L}_{int}(x). \label{eq.17_12}
\end{eqnarray}
  and also because the time-ordering is independent of the origin of time.

{\bf hw17-2}: Can you show the invariance Eq.\ref{eq.17_12} for the transformation Eq.\ref{eq.17_13}.
\begin{eqnarray}
  x^\mu \to x'^\mu = L^\mu_\nu x^\nu - a^\mu \label{eq.17_13}
\end{eqnarray}
  Because of the Poincare invariance, the matrix elements $S_{fi}$ is
  nonzero only when the four momentum are conserved. In our example of
  {\bf (19a,b), don't know what that is, to be determinded} ;
\begin{eqnarray}
  p_1^\mu + p_2^\mu = p_3^\mu + p_4^\mu ~(\mu = 0,1,2,3) \label{eq.17_14}
\end{eqnarray}
  In the next several lectures, we will learn the above most basic
  properties of the S-matrix and the scattering amplitudes by using
  the first two expansions, $n=1$ and $2$, of the SM interactions.

  Let me now simply list all the relevant free field expansions. For
  simplicity, I give those expressions for non-Majorana particles, i.e.,
  for those particles whose anti-particle is different from the particle.
  Expressions for Majorana, or real particles are obtained simply by
  replacing all b-operators by $a$-operators.
\begin{itemize}
  \item Spin $s=0$:
  \begin{eqnarray}
    \Phi(x)
  = \int \frac{d^3p}{(2E)(2\pi)^3} [ a({\vec p}) e^{-ipx} + b({\vec p})^\dgr e^{ipx} ]\label{eq.17_15}
  \end{eqnarray}
  \item Spin $s=1$ (massive):
  \begin{eqnarray}
    V^\mu(x)
  = \int  \frac{d^3p}{(2E)(2\pi)^3}
    \sum_{h=\pm1,0}
    [ a({\vec p},\lambda e^{-ipx} \eps^\mu({\vec p},h)
    + b({\vec p},\lambda)^\dgr e^{ipx}  \eps^\mu({\vec p},h)^* ] \label{eq.17_16}
  \end{eqnarray}
  \item Spin $s=1/2$:
  \begin{eqnarray}
    \Psi(x) &=& \int \frac{d^3p}{(2E)(2\pi)^3}
    \sum_{h=\pm1/2}
     [a({\vec p},h)      e^{-ipx} u({\vec p},h)
    + b({\vec p},h)^\dgr e^{ipx}  v({\vec p},h) ] \label{eq.17_18a} \\
    {\overline \Psi}(x)
  &=& \int \frac{d^3p}{(2E)(2\pi)^3}
    \sum_{h=\pm1/2}
    [a({\vec p},\lambda)^\dgr e^{ipx}  {\overline u}({\vec p},\lambda)
    + b({\vec p},\lambda)      e^{-ipx} {\overline v}({\vec p},\lambda)] \label{eq.17_18b}
  \end{eqnarray}
  where I show in Eq.\ref{eq.17_18b} for Dirac conjugate spinors, where the difference in the Lorentz transformation properties of the two-component
  $\psi_L$ and $\psi_R$ spinors in the Dirac four-spinor,
\begin{eqnarray}
    \Psi(x) =
  \begin{pmatrix}
    \psi_L(x) \\ \psi_R(x)
  \end{pmatrix} \label{eq.17_19}
\end{eqnarray}
are accounted for by taking the Dirac conjugates
\begin{eqnarray}
  {\overline \Psi}(x) = \Psi(x)^\dgr \gamma^0
  = ( \psi_L(x)^\dgr, \psi_R(x)^\dgr ) 
  \begin{pmatrix}
    0 & 1 \\
    1 & 0
  \end{pmatrix}
  = ( \psi_R(x)^\dgr, \psi_L(x)^\dgr ) \label{eq.17_20}
\end{eqnarray}
for both the mass term
\begin{eqnarray}
  m  {\overline \Psi} \Psi = m (\psi_R^\dgr \psi_L + \psi_L^\dgr \psi_R) \label{eq.17_21}
\end{eqnarray}
and for the kinetic terms
\begin{eqnarray}
  && {\overline \Psi} \gamma^\mu P_L \Psi = \psi_L^\dgr \sigma_-^\mu \psi_L \label{eq.17_22a} \\
  && {\overline \Psi} \gamma^\mu P_R \Psi = \psi_R^\dgr \sigma_+^\mu \psi_R \label{eq.17_22b}
\end{eqnarray}
\end{itemize}
 
{\bf hw17-3}: Please refresh your memeory on Eq.\ref{eq.17_15} to Eqs.(\ref{eq.15_22a},\ref{eq.15_22b}).
  If they all look deadly clear to you, please skip this homework.

  With the above preparation, we are now ready to compute scattering
  amplitudes ($S$-matrix elements) in the SM.

  Our first example is the $S$-matrix elements for the process
\begin{eqnarray}
  H \to \tau {\overline \tau} \label{eq.17_23}
\end{eqnarray}
  since it seems to be the simplest possible transition amplitudes
  which we can calculate, and since it is an interesting process.

  We prepare the initial state, which is just a single Higgs boson with
  momentum $p_H$:
\begin{eqnarray}
  \ket{initial~state}= a^\dgr(p_H) \ket{0} \label{eq.17_24}
\end{eqnarray}
  and also the final state, which has two particles,
\begin{eqnarray}
  \ket{final~state}= a(p_\tau,h_\tau)^\dgr b(p_{\overline \tau},h_{\overline \tau})^\dgr \ket{0} \label{eq.17_25}
\end{eqnarray}
   i.e., a $\tau$ lepton with momentum $p_\tau$, and helicity $h_\tau$, and an
  anti-$\tau$ lepton (${\overline \tau}$) with momentum $p_{\overline \tau}$, and helicity $h_{\overline \tau}$.

  The $S$-matrix element for this transition reads
\begin{eqnarray}
  S_{fi}
  &=& \bra{final~state} S \ket{initial~state}
  = \bra{final~state}  T e^{ i\int d^4x {\cal L}_{int}(x) } \ket{initial~state}\\ 
  &=& \bra{0} b(p_{\overline \tau},h_{\overline \tau}) a(p_\tau,h_\tau)
                  T e^{ i\int d^4x {\cal L}_{int}(x) } a(p_H)^\dgr \ket{0} \label{eq.17_26}
\end{eqnarray}

Now, in the expansion of the $S$-matrix
\begin{eqnarray}
  S = T e^{ i\int d^4x {\cal L}_{int}(x) }
  = 1
  + i\int d^4x {\cal L}_{int}(x)
  + T \frac{1}{2!} (i\int d^4x {\cal L}_{int}(x))^2
  + \cdots \label{eq.17_27}
\end{eqnarray}
 we notice the $0$'th term, $1$, does not have any matrix elements.

{\bf hw17-4}: Prove $\bra{final~state} 1 \ket{initial~state} = 0$ by using the
  property of the creation and annihilation operators.

hint: The vacuum in perturbative QFT is defined by
\begin{eqnarray}
  a\ket{0}=b\ket{0}=0,~~~or~\bra{0}a^\dgr = \bra{0}b^\dgr = 0,\label{eq.17_28}
\end{eqnarray}
  for any particle or anti-particle operators.

  We find that the 1'st term has a non-vanishing matrix elements because
  ${\cal L}_{int}(x)$ contains the term
\begin{eqnarray}
  {\cal L}_{int}(x)  =  -\frac{m_\tau}{v} H(x) {\overline \Psi}_\tau(x) \Psi_\tau(x) + \cdots \label{eq.17_29}
\end{eqnarray}
  None of the other terms in ${\cal L}_{int}(x)$, denoted by $+\cdots$ above, contribute
  to the amplitudes at this order ($n=1$), because of the same reason in Eq.\ref{eq.17_28}.

{\bf hw17-5}: Go back to past homework, and find that the above Lagrangian
  term for the $H-\tau-\tau$ interactions exists in ${\cal L}_{Yukawa}$. Please
  check the sign and the magnitude of the coupling.

hint: All the Higgs boson interactions in the SM are obtained by
  replacing the Higgs v.e.v., $v$, by $v + H(x)$, since the Higgs boson is
  the quantum of the fluctuation of the Higgs field about the vacuum
  along the direction where the potential has the minimum (the quantum of
  the fluctuation along the directions where the potential is flat is
  massless, and called Nambu-Goldstone boson). Because the massless
  Goldstone bosons are absorbed by the gauge bosons to make them massive,
  they do not appear as an observable state. The gauge in which those
  un-observable states do not appear is called Unitary gauge.  In the
  Unitary gauge, all the Higgs boson interactions are obtained simply by
  replacing the Higgs doublet fields by
\begin{eqnarray}
  &&\phi   = 
  \begin{pmatrix}
    0 \\ (v+H(x))/\sqrt{2}
  \end{pmatrix}\label{eq.17_30a}\\
  &&\psi^c =
  \begin{pmatrix}
    (v+H(x))/\sqrt{2} \\ 0
  \end{pmatrix}\label{eq.17_30b}
\end{eqnarray}
  Since all the mass terms in the SM are proportional to this v.e.v., $v$,
  this tells that all the Higgs boson interaction terms are obtained
  from the mass terms, by the following replacements:
\begin{eqnarray}
  &&m   \to m   ( 1 + H(x)/v )  ~~~  for~ all ~the ~Dirac ~fermion ~masses\label{eq.17_31a}\\
  && m^2\to m^2 ( 1 + H(x)/v )^2  ~~~for~ m_W, ~m_Z \label{eq.17_31b}\\
  && m   \to m   ( 1 + H(x)/v )^2  ~~~for ~Majorana ~neutrino ~mass\label{eq.17_31c}
\end{eqnarray}
  It is only for the Higgs boson itself this rule doesn't work, since
  the Higgs potential is
\begin{eqnarray}
  &&{\cal L}_{Higgs~potential}
  = -V_{Higgs~potential}
  = -\frac{\lambda}{4} (\phi^\dagger \phi)^2 -\mu^2 (\phi^\dagger \phi)\\ 
  &=& -\frac{\lambda}{4}(\frac{(v+H)^2}{2})^2 -\mu^2 \frac{(v+H)^2}{2}
  = -\frac{\lambda}{16} (v+H)^4 -\frac{\mu^2}{2} (v+H)^2 \\
  &=& -\frac{\lambda}{16}(v^4 + 4v^3 H +6v^2 H^2 +4v H^3 +H^4)
    -\frac{\mu^2}{2} (v^2 + 2vH + H^2) \\
  &=& -\frac{\lambda}{16} v^4 -\frac{\mu^2}{2} v^2
   -(\frac{\lambda}{16} 4v^3 + \frac{\mu^2}{2} 2v) H
   -(\frac{\lambda}{16} 6v^2 + \frac{\mu^2}{2}) H^2
   -(\frac{\lambda}{16} 4v) H^3
   -(\frac{\lambda}{16}) H^4 \label{eq.17_32}
\end{eqnarray}
   The term proportional to $H$ should vanish (the potential minimum)
\begin{eqnarray}
  && \frac{\lambda}{4} v^2 + \mu^2 = 0 \label{eq.17_33a} \\
  && \mu^2 = -\frac{\lambda}{4} v^2 \label{eq.17_33b}
\end{eqnarray}
  and hence the Higgs mass should be
\begin{eqnarray}
  m_H^2 = \mu^2 + \frac{\lambda}{4} 3v^2
  = \frac{\lambda}{4} (-v^2 + 3v^2)
  = \frac{\lambda}{2} v^2 \label{eq.17_34}
\end{eqnarray}
  The relevant terms for $H^3$ and $H^4$ couplings are hence
\begin{eqnarray}
  {\cal L}_{Higgs~potential}
  &=& -V_{Higgs~potential}
  = -\frac{\lambda v^2}{4} H^2
    -\frac{\lambda}{4} H^3
    -\frac{\lambda}{16} H^4 \\
  &=& -\frac{\lambda v^2}{2}  \frac{H^2}{2}
    -\frac{3 \lambda v}{2}   \frac{H^3}{3!}
    -\frac{3 \lambda }{2}   \frac{H^4}{4!} \label{eq.17_35}
\end{eqnarray}
  where I dropped the constant term in the potential.

{\bf hw17-6}: Please confirm the rule Eq.(\ref{eq.17_31a},\ref{eq.17_31b},\ref{eq.17_31c}), and also Eq.\ref{eq.17_35} for the SM
  Higgs potential. Confirm that the rule Eq.\ref{eq.17_31b} does not give the correct $H^3$ and $H^4$ term in the potential.

  Now, let us calculate the S-matrix elements Eq.\ref{eq.17_26}
\begin{eqnarray}
  S_{fi}
  &=& \bra{final~state} S \ket{initial~state}
  = \bra{final~state} T e^{ i\int d^4x {\cal L}_{int}(x) } \ket{initial~state} \\
  &=& \bra{0} b(p_{\overline \tau},h_{\overline \tau}) a(p_\tau,h_\tau) T e^{ i\int d^4x {\cal L}_{int}(x) } a(p_H)^\dgr \ket{0} \\
  &=& \bra{0} b(p_{\overline \tau},h_{\overline \tau}) a(p_\tau,h_\tau)  [1 + i\int d^4x {\cal L}_{int}(x) +\cdots] )a(p_H)^\dgr \ket{0} \label{eq.17_36}
\end{eqnarray}
  with
\begin{eqnarray}
  {\cal L}_{int} (x) = -\frac{m_\tau}{v} H(x) {\overline \Psi}_\tau(x) \Psi_\tau(x) \label{eq.17_37}
\end{eqnarray}
  We find
\begin{eqnarray}
  S_{fi}= -i\frac{m_\tau}{v} \int d^4 x
  \bra{0} b(p_{\overline \tau},h_{\overline \tau}) a(p_\tau,h_\tau)
      H(x) {\overline \Psi}_\tau(x) \Psi_\tau(x) a(p_H)^\dgr \ket{0} \label{eq.17_38}
\end{eqnarray}
  Since we have free field expansions for all the three fields,
\begin{eqnarray}
  H(x)  &=& \int \frac{d^3p}{(2E)(2\pi)^3} [ a({\vec p}) e^{-ipx} + a(\vec p)^\dgr e^{ipx} ] \label{eq.17_39a} \\
\Psi(x)  &=& \int \frac{d^3p}{(2E)(2\pi)^3}
  \sum_{h=\pm1/2} [a({\vec p},h)     e^{-ipx} u({\vec p},h)
  + b({\vec p},h)^\dgr e^{ipx}  v({\vec p},h)]  \label{eq.17_39b} \\ 
{\overline \Psi}(x) &=& \int \frac{d^3p}{(2E)(2\pi)^3} \sum_{h=\pm1/2}
  { a({\vec p},\lambda)^\dgr e^{ipx}  {\overline u}({\vec p},\lambda)
  + b({\vec p},\lambda)      e^{-ipx} {\overline v}({\vec p},\lambda) } \label{eq.17_39c}
\end{eqnarray}
we immediately find only the following term is non-vanishing:
\begin{eqnarray}
  S_{fi} &=& -i\frac{m_\tau}{v} \int d^4 x
  \bra{0} b(p_{\overline \tau},h_{\overline \tau}) a(p_\tau,h_\tau)
  \int \frac{d^3p}{(2E)(2\pi)^3} a_H(p) e^{-ipx}
  \nonumber \\&&\int \frac{d^3p'}{(2E')(2\pi)^3} \sum_{h'}
      a_\tau(p',h')^\dgr e^{ip'x}  {\overline u}(p',h')
      \nonumber \\&& \int \frac{d^3p{''}}{(2E{''})(2\pi)^3} \sum_{h{''}}
      b_{\overline \tau}(p{''},h{''})^\dgr e^{ip{''}x} v(p{''},h{''})
  a(p_H)^\dgr \ket{0} 
\end{eqnarray}
  By using the commutation and anti-commutations among operators we find
\begin{eqnarray}
  S_{fi}
  &=& -i\frac{m_\tau}{v} \int d^4 x
  \bra{0} e^{-ix p_H}
      e^{ ix p_\tau} {\overline u}(p_\tau,h_\tau)
      e^{ ix p_{\overline \tau}} v(p_{\overline \tau},h_{\overline \tau}) \ket{0} \\
      &=&-i\frac{m_\tau}{v} \int d^4 x
      e^{ix(p_\tau+p_{\overline \tau}-p_H)}
      {\overline u}(p_\tau,h_\tau) v(p_{\overline \tau},h_{\overline \tau})  \bra{0}\ket{0} \\
      &=& -i\frac{m_\tau}{v} (2\pi)^4 \delta^4( p_\tau+p_{\overline \tau}-p_H )
      {\overline u}(p_\tau,h_\tau) v(p_{\overline \tau},h_{\overline \tau})  \label{eq.17_41}
\end{eqnarray}
  As I will explain later, we usually express the amplitudes
  as T (transition) matrix elements, wich is defined as
\begin{eqnarray}
  &&\bra{f}S\ket{i}=\bra{f}(1+iT)\ket{i} \label{eq.17_42a}\\
  &&S_{fi} =          i T_{fi}\label{eq.17_42b}
\end{eqnarray}
  We then have a general expression for $T_{fi}$
\begin{eqnarray}
  T_{fi} = (2\pi)^4 \delta^4( p_f - p_i ) M_{fi} \label{eq.17_43}
\end{eqnarray}
   where $p_i$ and $p_f$ are the total sum of initial and final state four
  momenta.  Summing up, the S-matrix elements are usually expressed as
\begin{eqnarray}
  S_{fi} = i T_{fi}
             = i (2\pi)^4 \delta^4( p_f - p_i ) M_{fi} \label{eq.17_44}
\end{eqnarray}
   after factorizing out the common factor of $i$ and the four momentum
  conservation delta functions.  With this convention, our amplitudes
  are simply:
\begin{eqnarray}
  M(H(p_H) \to \tau(p_\tau,h_\tau) {\overline \tau}(p_{\overline \tau},h_{\overline \tau}))
  = -\frac{m_\tau}{v} {\overline u}(p_\tau,h_\tau) v(p_{\overline \tau},h_{\overline \tau}) \label{eq.17_45}
\end{eqnarray}
  
{\bf hw17-7}: Please obtain Eq.\ref{eq.17_41} from Eq.\ref{eq.17_38}.

  Before going on calculating various S-matrix elements (or scattering
  amplitudes), let me explain why we introduce Transition matrix, in
  addition to the S-matrix.

  The T-matrix is related to the S-matrix symbolically as
\begin{eqnarray}
  S = 1 + i T
\end{eqnarray}
  The unitarity of the S matrix is then expressed as
\begin{eqnarray}
  S^\dagger S = (1 + iT)^\dagger (1+iT)
                  = 1 -i(T^\dagger - T) + T^\dagger T
                  = 1 \label{eq.17_47}
\end{eqnarray}
  or
\begin{eqnarray}
  -i(T-T^\dagger) = T^\dagger T \label{eq.17_48}
\end{eqnarray}
  This is the most useful expression of the Unitarity of the S-matrix.

{\bf hw17-8}: Derive Eq.\ref{eq.17_48} from Eq.\ref{eq.17_47}.

  If we take the forward scattering ($\bra{final}= (\ket{initial})^\dagger$) matrix
  element of Eq.\ref{eq.17_48},
\begin{eqnarray}
  -i\bra{inital}(T-T^\dagger)\ket{inital} = \bra{inital}T^\dagger T\ket{inital} \label{eq.17_49}
\end{eqnarray}
  The left-hand-side of Eq.\ref{eq.17_49} is simply
\begin{eqnarray}
  -i\bra{inital}(T-T^\dagger)\ket{inital} = -i ( T_{ii} - (T_{ii})^* )
  = 2 \Im(T_{ii}) \label{eq.17_50}
\end{eqnarray}
  which is called the imaginary part of the forward scattering amplitudes.
  For the right-hand-side of Eq.\ref{eq.17_49}, we insert the complete set of states
\begin{eqnarray}
  1 = \sum_f \ket{f}\bra{f} \label{eq.17_51}
\end{eqnarray}
   between $T^\dagger$ and $T$, where the summation is taken over all possible
  final states and total integral over the whole phase space for each
  final state. We therefore find
\begin{eqnarray}
  2 \Im T_{ii} = \bra{i} T^\dagger \sum_f \ket{f}\bra{f} T \ket{i}
                  = \sum_f |T_{fi}|^2 \label{eq.17_52}
\end{eqnarray}
  This is called the optical theorem, which states that the the imaginary
  part of the forward scattering amplitude is proportional to the total
  cross section of the process.

{\bf hw17-9}: Reproduce the optical theorem Eq.\ref{eq.17_52}.

  I find it very useful to know that perturbative calculation of the
  S-matrix elements gives $S_{fi}$, whereas the T-matrix elements are
  related as $T_{fi} = -i S_{fi}$. Unitarity tells that the $T_{fi}$ are
  real in the tree-level approximation where final state interactions
  are neglected.  When we calculate a two-point function, with the same
  initial and the final states, the imaginary part of the T-matrix element
  should be positive according to the optical theorem.  I often used the
  above facts to test my calculation, such as the number of i's in tree
  level amplitudes, and the sign of loop functions.

  Please always be aware that the perturbative expansion gives the
  S-matrix elements, rather than the T-matrix elements.  Because of this,
  perturbative QFT computation of amplitudes gives $i$ times $T$ matrix
  elements. It is useful to remember this fact, and express the
  perturbatively calculated amplitudes as
\begin{eqnarray}
  \bra{f} S \ket{i} =  \bra{f} T e^{ i\int d^4x {\cal L}_{int}(x) } \ket{i} =   S_{fi}
  = i T_{fi}
  = i (2\pi)^4 \delta^4(\sum p_f - \sum p_i) M_{fi} \label{eq.17_53}
\end{eqnarray}
 and we expect that $T_{fi}$, and hence $M_{fi}$, should be real at the
  tree-level.
  Because the time-ordering of the S matrix should follow positive energy
  oscillation, it should behave as
\begin{eqnarray}
  S = T e^{ -i \int_{t=-\infty}^{\infty} dt {\cal H}_{int} }
        = T e^{  i \int_{t=-\infty}^{\infty} dt {\cal L}_{int}(x) }
        = T e^{  i \int d^4x  {\cal L}_{int}(x) } \label{eq.17_54}
\end{eqnarray}
  where ${\cal H}_{int} = -{\cal L}_{int}$ denotes the Hamiltonian after integration over
  space. Therefore, the sign of the exponent should be positive.
  I therefore choose the above sign conventions consistently.

  That's all for hw17.\\

Best regards,\\

Kaoru


\end{document}
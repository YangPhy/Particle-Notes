
\documentstyle[12pt]{article}
\textheight=9.5in \voffset=-1.0in \textwidth=6.5in \hoffset=-0.5in
\parskip=0pt
\begin{document}

\begin{center}
{\large\bf Cover letter and HW1 for Advanced Particle Physics} \\

\end{center}

\vskip 0.2 in

Dear students, \\

  In the first lecture, I ask you to list up the names the
  properties of the elementary particles that you know.
  Although there are infinitely many particle physics phenomena
  of interests to us, and hence are as many particle physics
  experiments, all the known phenomena are described well by
  the Standard Model (SM) of particle physics, which has only
  small number of elementary (fundamental) particles.
  The matter particles (quarks and leptons) are fermions with
  spin 1/2, while the force mediating particles (photon,
  weak bosons, gluons, and the Higgs boson) are bosons with
  integer spins.\\

  The SM is a model based on the quantum field theory (QFT).
  It is therefore absolutely necessary for all particle
  physics students to master QFT, including some details
  of its SM realization with its distinctive features such
  as Abelian and non-Abelian gauge symmetries, asymptotic
  freedom and confinement, spontaneous symmetry breaking
  and the Higgs mechanism, and various global symmetries
  which are either exact or partially violated.  Keys or
  hints that should lead us to physics beyond the SM
  should be hiding behind this SM framework that explains
  all the observed phenomena.\\

  The aim of my 3.5 months lectures is to help you start
  building up your own perspective on the present status
  of particle physics, based on which your pursuits for
  more fundamental understanding of nature should follow
  in the future.\\

  You are extremely lucky to start learning particle
  physics now, since all the above goal can be achieved
  by mastering just one QFT model, called the SM of
  particle physics, and also because the key experiments
  that will give us hints of physics beyond the SM
  are either in full operation or in preparation, such
  as the LHC in the high energy frontier as well as several
  precision experiments probing the flavor sector of the SM,
  both in the quark and the lepton sectors. In addition,
  cosmological and astrophysical observations, searches for
  various rare processes such as proton decay, neutrino-less
  double beta decay, and direct/indirect detection of the
  dark matter may tell us physics beyond the SM.\\

  I would like to help you prepare yourself toward
  contributing to the particle physics revolution
  which might be just around the corner.\\

  Here is the first homework.\\

  In this homework, I would like to introduce you the website
  of the Particle Data Group (PDG), from which you can learn
  about the particles, as well as about particle physics.
  The particle mass (the rest mass or the invariant mass),
  its lifetime and/or decay width, and its spin (the intrinsic
  angular momentum of a particle) are given in the tables.\\

{\bf hw01-1}:

  Please visit the web-page of Particle Data Group (PDG)
  and move to `Order PDG products'; and order
  `Particle Physics Booklet (328 pages)' if you don't have it yet.
  It will be delivered to you free of charge sometime this year.
  Review of Particle Physics (RPP) should be ordered only if
  you want to have its very fat and heavy hard copy.  All the
  RPP contents are available on the web.\\

{\bf hw01-2}:

  Go back to the PDG homepage.  Move to ``Summary Tables''.\\

  Check, ``Gauge and Higgs bosons'', ``Leptons'', and ``Quarks''.
  Please make your own table (1 page table) in which
  the following information are given.\\

  particle names (6 quarks, 6 leptons, 4 gauge bosons, 1 Higgs boson),
  their masses (with error, range, bounds), their lifetime or decay
  width, their electric charge, and their spin.\\

  Next, please visit `Baryons' and find data on p (proton) and
  n (neutron), and visit `Mesons' and find data on $\pi^+$ and
  $\pi^0$.  Please make a small table of the above particles with
  the same information (mass, lifetime, electric charge, spin).\\

  Please make your tables with your own format.
  Please make the table either by typing in a text format
  (as I write this e-mail text),
  or by using LaTeX.  I accept e-mail text,
  latex file (your-name-hw01.tex),
  post-script file (your-name-hw01.ps),
  and pdf file (your-name-hw01.pdf).  If you write your report
  by free hand, please scan it and send me its pdf file.\\

  After attending all my lectures, 26 lectures by the end of
  this April, I would like to suggest that you make your
  own table of elementary particles again.  It will be very
  different from the ones you make for this homework, and
  will serve as the starting point of your pursuit for
  physics beyond the SM.\\

{\bf hw01-3}:

  Show that the time evolution of a state $|\psi(t)>$
  that satisfies the Shroedinger equation
  \begin{eqnarray}
    i d/dt |\psi(t)> = H |\psi(t)> = E |\psi(t)>,
  \end{eqnarray}
  is
  \begin{eqnarray}
  |\psi(t)> = e^{-itE} |\psi(0)>.
  \end{eqnarray}

  If |psi(t)> is a single particle state in the vacuum,
  then its total energy is its invariant mass in the
  rest frame $p^\mu = (E, p_x, p_y, p_z) = (M, 0, 0, 0)$.
  The solution is then,
  \begin{eqnarray}
  |\psi(t)> = e^{-itM} |\psi(0)>.
  \end{eqnarray}
  If we give a small negative imaginary part to its mass
  \begin{eqnarray}
  M \to M - i/2 \Gamma (M, \Gamma > 0)
  \end{eqnarray}
  then,
  \begin{eqnarray}
  |\psi(t)> = e^{ -it(M - i/2 \Gamma) } |\psi(0)>
  \end{eqnarray}
  and
  \begin{eqnarray}
  | |\psi(t)> |^2 = | |\psi(0)> |^2 e^{-t\Gamma}
  \end{eqnarray}
  The norm of the state decreases with time.  The lifetime
  of the state is defined as
  \begin{eqnarray}
  \tau = \frac{\int_0^\infty dt t P(t)}{\int_0^\infty dt P(t)}
  \end{eqnarray}
  with the survival probability
  \begin{eqnarray}
  P(t) = ||psi(t)>|^2 / ||psi(0)>|^2 = e^{-t\Gamma}
  \end{eqnarray}
  Please show that $\tau = 1/\Gamma$.

  The sign of the imaginary part should be negative in order
  for the state not to blow up in the future ($t \to +\infty$).\\

  When you learn QFT in the latter part of my lectures, you
  will learn that this ``no blowing up'' requirement is essentially
  the condition that determines the sign of the imaginary part
  of Feynman's causal Green's function, the so called
  $i\epsilon$ prescription, for stable
  ($\tau \to \infty$, $\Gamma \to +0$) particles.\\

{\bf hw01-4}:

  Refer to the table of `Physical Constants' in RPP, which is
  given at the beginning of `Reviews, Tables, and Plots',
  under the first category `Constants, Units, ...', and find
  \begin{eqnarray}
  \hbar = h/2\pi
          = 6.6 \times 10^{-22} MeV s
  \end{eqnarray}
  (s stands for second).  Use this relation to convert between
  the lifetime and the width in your table, so that all the
  unstable particles (upper/lower bounds for `stable' particles)
  have both lifetime and width, in the natural unit
  \begin{eqnarray}
  c = \hbar = 1
  \end{eqnarray}
  which I adopt throughout my lectures.  I use GeV as units
  of energy.  Please give all the masses and widths in units
  of GeV.\\

  For muon, tau, neutron, pi+ and pi0, please also give the
  lifetime in units of cm, by using
  \begin{eqnarray}
  c = 3 \times 10^8 m/s = 3 \times 10^{10} cm/s
  \end{eqnarray}
  The mean flight length of a particle decay-in-flight is
  \begin{eqnarray}
  <l> = \tau c \gamma \beta,
  \end{eqnarray}
  where $\gamma \beta = p/m$ is the momentum of a particle with the rest mass m.\\

{\bf hw01-5}:

  Let us consider the Hamiltonian $H = H_0 + V$
  and look for its eigen-state
  \begin{eqnarray}
  H |\psi> = (H_0 + V) |\psi> = E |\psi>
  \end{eqnarray}
  Assume that we know the solution for
  \begin{eqnarray}
  H_0 |\psi^0_k> = E_k |\psi^0_k>
  \end{eqnarray}
  Then we can solve the equation perturbatively by using
  the Green's function method:
  \begin{eqnarray}
  (H_0 - E)|\psi> = -V |\psi>
  \end{eqnarray}
  \begin{eqnarray}
  |\psi> &=& -(H_0 - E)^{-1} V |\psi> \nonumber \\
        &=& - \sum_k 1/(E_k - E) |\psi^0_k><\psi^0_k| V |\psi>
  \end{eqnarray}
  and it can be solved perturbatively.
  Near $E = E_k$, the probability behaves as
  \begin{eqnarray}
  ||\psi>|^2 = constant/|E_k - E|^2
  \end{eqnarray}
  Please insert $E_k = M - i/2 \Gamma$, and show the
  E dependence of
  \begin{eqnarray}
  ||\psi>|^2 = constant/|(M - i/2 \Gamma) - E|^2
  \end{eqnarray}
  in the region $M - 4\Gamma < E < M + 4\Gamma$

  Please observe that $\Gamma$ is the width of the
  half maximum of the resonance peak.\\

{\bf hw01-6}:

  Feynman's causal Green's function has a relativistically
  invariant form:
  \begin{eqnarray}
  1/[ p^2 - m^2 + i\epsilon ]
  \end{eqnarray}
  Please observe that $\epsilon = m\Gamma$ if we make the
  replacement
  $m => m - i/2 \Gamma$.

  Therefore, the relativistic form of the resonance
  Green's function is usually expressed as
  \begin{eqnarray}
  1/[ p^2 - m^2 + im\Gamma ]
  \end{eqnarray}
  and the resonance peak is proportional to
  \begin{eqnarray}
    1/| p^2 - m^2 + im\Gamma |^2
  = 1/[ (p^2 - m^2)^2 + (m\Gamma)^2 ]
  \end{eqnarray}
  The above form is called Breit-Wigner resonance.
  Please observe that $m\Gamma$ is the half width of the
  resonance at the half maximum when it is expressed
  in terms of $p^2$.\\

  In the rest frame of the resonance, $p^\mu = (E,0,0,0)$,
  the Breit-Wigner resonance behaves as
  \begin{eqnarray}
    &&1/[ p^2 - m^2 + im\Gamma ] \\
  =&& 1/[ E^2 - m^2 + im\Gamma ] \\
  =&& 1/[ (E + m)(E-m) + im\Gamma ] \\
  =&& 1/[ 2m (E-m) + im\Gamma ] \\
  = &&(1/2m)/[ E - (m - i/2 \Gamma) ]\\
  \end{eqnarray}
  and we reproduce the non-relativistic Green's function
  for a resonance.\\

That is all for today.\\

Best regards,\\

Kaoru


\end{document}

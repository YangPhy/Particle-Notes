\documentclass[12pt]{article}
\usepackage{amsmath}
\textheight=9.5in \voffset=-1.0in \textwidth=6.5in \hoffset=-0.5in
\parskip=0pt
\def\del{{\partial}}


\begin{document}

\begin{center}
{\large\bf HW5 for Advanced Particle Physics} \\

\end{center}

\vskip 0.2 in

Dear students:\\

  Today, I introduce `gauge transformations', or `local
  phase transformations' of the SM gauge group, and showed
  how to obtain the gauge invariant Lagrangian of the SM. \\

  Let us start with the $U(1)$ gauge transformation:
   \begin{eqnarray}
    f(x) \to f'(x) = U(x) f(x) \\
    U(x) = e^{ i Q \theta(x) }
   \end{eqnarray} 
    $Q$ is the generator of U(1), and it takes the value of
  the charge (eigen value) of the complex field $f(x)$.
  Let us write down the covariant derivative as
  \begin{eqnarray}
    D_\mu = \del_\mu + i g Q A_\mu(x)
  \end{eqnarray}
  with an arbitrary positive number $g$ ($g>0$ is my convention).
  Since $\del_\mu$ transform as a vector under the Lorentz
  transformation, $A_\mu$ should also transform as a vector.

  Also note that since $\del_\mu$ has the mass dimension of 1,
  $g A_\mu(x)$ has the mass dimension of 1.  In 4 space-time
  dimensions, the coupling $g$ has the mass dimension of $0$,
  and the gauge field $A_\mu(x)$ has the mass dimension 1.

{\bf hw05-1}:

  Please show that $A_\mu(x)$ should be a real field
\begin{eqnarray}
    A_\mu(x)^\dagger = A_\mu(x)
\end{eqnarray}
  when $g$ is real and $Q$ is Hermitian ($Q^\dagger = Q$).

  hint: In quantum mechanics,
\begin{eqnarray}
    i\del_\mu = i d/dx^\mu = p_\mu
\end{eqnarray}
is an Hermetian operator.  Please note that since $p_\mu = (E, -p_x, -p_y, -p_z)$, and $x^\mu = (t,x,y,z)$, the above equation
\begin{eqnarray}
    p^\mu = i\del^\mu = g^{\mu\nu} i\del_\nu
\end{eqnarray}
gives $E = i d/dt$, $p_x = -i d/dx$, $p_y = -i d/dy$, $p_z = -i d/dz$.

hint2: You may need to insert the derivative operator
    inside of other fields, say, $A(x) i \del _\mu B(x)$
    in order to tell that it is an Hermetian operator.
\begin{eqnarray}
    [ A(x) i\del_\mu B(x) ]^\dagger &=&
     -i \del_\mu B(x)^\dagger A(x)^\dagger \\
    &=& 
    -i \del_\mu [ B(x)^\dagger A(x)^\dagger ]
    + B(x)^\dagger [ i \del_\mu ] A(x)^\dagger
\end{eqnarray}        

    If we ignore the total derivative part (which does
    not give physics in the local vicinity in QFT), the
    above identity tells that
\begin{eqnarray}
    (i \del_\mu)^\dagger = i \del_\mu
\end{eqnarray}
 where the derivative operates to its right hand side.

{\bf hw05-2}:
  Please obtain the transformation property of $A_\mu(x)$
\begin{eqnarray}
    A_\mu(x) \to A'_\mu(x) = A_\mu(x) + X_\mu(x)
\end{eqnarray}
  which ensures $D_\mu$ transforms as a covariant derivative,
\begin{eqnarray}
    D_\mu f(x) \to D'_\mu(x) f'(x) = U(x) D_\mu(x) f(x)
\end{eqnarray}
 or equivalently,
\begin{eqnarray}
    D_\mu \to D'_\mu(x) = U(x) D_\mu(x) U(x)^{-1}
\end{eqnarray}
  hint: $X_\mu(x)$ can be expressed in terms of the x-dependent
        (local) gauge transformation phase, $\theta(x)$.
  The combination of the transformations of $f(x)$ and $A_\mu(x)$
  that we obtained in the above two excercises
\begin{eqnarray}
    &&f(x) \to f'(x) = U(x) f(x) \\
    &&A_\mu(x) \to A'_\mu(x) = A_\mu(x) + X_\mu(x)
\end{eqnarray}
 is called the $U(1)$ gauge transformations.

{\bf hw05-3}:
  Please show that the commutator
  $[ D_\mu(x), D_\nu(x) ]$
  is invariant under the $U(1)$ gauge transformations.

{\bf hw05-4}:
  Please show that the Lagrangian
  \begin{eqnarray}
    L = -1/4 F_{\mu\nu} F^{\mu\nu}
      + \psi(x)^\dagger \gamma^0 i D_\mu(x) \gamma^\mu \psi(x)
      - m \psi(x)^\dagger \gamma^0 \psi(x)
  \end{eqnarray}
  is invariant under the gauge transformation. The Lorentz
  invariance of the above Lagrangian will be studied in the
  next lectures.

  hint: $F_{\mu\nu}(x)$ is proportional to $[ D_\mu(x), D_\nu(x) ]$
  hint2: $(\gamma^\mu)'$s ($\mu = 0,1,2,3$) are just numbers.
  The above Lagrangian is that of QED (Quantum Electro Dynamics),
  the prototype of the U(1) gauge theory, where the coupling is
\begin{eqnarray}
    g = e = \sqrt{4\pi\alpha} \approx \sqrt{4\pi/137} \approx 0.30
\end{eqnarray}
  and the charge $Q = -1$ for the electron.  The first successful
  quantum field theory, QED, is invariant under the $U(1)_{EM}$
  gauge transformation.

{\bf hw05-5}:
  Please express the QED Lagrangian by using the standard
  derivative $\del_\mu$ and $A_\mu(x)$.  It is now a function of
  $\psi(x)$, $\del_\mu \psi(x)$, $\psi(x)^\dagger$, $A_\mu(x)$, $\del_\mu A_\mu(x)$.

{\bf hw05-6}:
  Please obtain the equation of motion (E.O.M.) from the above
  Lagrangian for the field $\psi(x)$.

  hint: Equation of motion for a generic Lagranian density
  $L(\phi(x)$, $\del_\mu\phi(x))$
  is
\begin{eqnarray}
    \del^\mu \delta L/\delta(\del^\mu\phi(x))
    - \delta L/\delta(\phi(x))
    = 0
\end{eqnarray}
  where $\delta/\delta(\phi(x))$ and
  $\delta/\delta(\del^\mu\phi(x))$ are functional derivatives.
  [You can treat them as ordinary derivatives at a given
  spacetime point, x.]

  hint2:
  In case of the above QED Lagrangian, since it does not have
  the term proportional to $[\del \psi^\dagger(x)]$, the E.O.M.
  of $\psi(x)$ is obtained simply from
\begin{eqnarray}
    \delta L/\delta(\psi(x)^\dagger) = 0.
\end{eqnarray}
 This E.O.M. often appears when we study the electron behavior
  inside EM fields.

{\bf hw05-7}:
  Please obtain the equation of motion (E.O.M.) from the above
  Lagrangian for the field $A_\mu(x)$.

  hint: We find Maxwell's equations in presence of currents.
  In the SM, the $U(1)_Y$ phase transformations are NOT proportional
  to the electromagnetic charge $Q$, but they are proportional to
  the hypercharge, which we denote as $Y$.
\begin{eqnarray}
    f(x) \to f'(x) = U(x) f(x) = e^{i Y \theta(x)} f(x) 
\end{eqnarray}
  After the spontaneous breakdown of the $SU(2)_L \times U(1)_Y$
  symmetry, which will be studied in the next lectures, the
  $U(1)_{EM}$ gauge symmetry remains, and the electroweak part of
  the SM interactions reduce to the effective theory of dimension
  6 four fermion operators that describes the weak interactions
  at low energies (Fermi theory).

  Schematically, we write
\begin{eqnarray}
    L_{SM} \to L_{QCD} + L_{QED} + L_{Fermi}
\end{eqnarray}
   at low energies (energies far below the EW scale, v=246GeV).
  It is important to note that $L_{QCD}$ and $L_{QED}$ have only
  operators of dimension 4 or less, and hence they are
  `renormalizable', while $L_{Fermi}$ have 4 fermion fields,
  and hence their mass dimension are 6.  We say that $L_{Fermi}$
  are non-renormalizable interactions.  Nevertheless, the
  full Lagrangian above are invariant under the low energy
  gauge symmetry of the SM,
  $SU(3)_C \times U(1)_{EM}$.
  Such theories, with renormalizable QFT terms plus higher
  dimensional operators which preserve all the symmetries of
  the renormalizable part of QFT, are called effective theories.
  In effective theories, we cannot calculate radiative corrections
  due to non-renormalizable higher dimensional interactions, but
  we can calculate radiative corrections to all the effective
  operators.  In Fermi theory of weak interactions, QCD and
  QED radiative corrections play very important roles for
  determining the effective strengths of the weak interactions
  at low energies.\\
    
  Let us repeat the above derivations for the non-Abelian group
  $SU(n)$:
\begin{eqnarray}
    f(x) \to f'(x) = U(x) f(x) \\
    U(x) = e^{ i T^a \theta^a(x) }
\end{eqnarray}
  The covariant derivative is now expressed as
\begin{eqnarray}
    D_\mu(x) = I \del_\mu + i g T^a A^a_\mu(x)
\end{eqnarray}
  where I is the n times n unit matrix, and $T^a$ ($a = 1$ to $n^2-1$)
  are the generators of the group $SU(n)$.  $g>0$ is my convention.

{\bf hw05-8}:
  Please obtain the transformation property of $A^a_\mu(x)$
\begin{eqnarray}
    A^a_\mu(x)\to A'^a_\mu(x) = A^a_\mu(x) + X^a_\mu(x)
\end{eqnarray}
  such that $D_\mu$ transforms as the covariant derivative,
\begin{eqnarray}
    D_\mu(x) \to D'_\mu(x) = U(x) D_\mu(x) U(x)^{-1}
\end{eqnarray}
  Please use the $SU(n)$ algebra,
$ [ T^a, T^b ] = i f^{abc} T^c$
  with real structure constants $f^{abc}$.

  note: The above algebra should be read as follows:
        Comutators of all the generators are expressed
        as linear combination of the generators.  This is
        the necessary condition that an arbitrary product
        of two transformations is also a transformation
        (so that the transformations form a `group').

{\bf hw05-9}:
  Show that
  $Tr \{ [D_\mu(x), D_\nu(x)] [D^mu(x),D^\nu(x)] \}$
  is invariant under the SU(n) gauge transformations, and
  then express it interms of the derivative operator $\del_\mu$
  and the gauge fields $A^a_\mu(x)$.

  In the second step of the above excercise, please
  introduce the normalization of the generators,
  $Tr\{ T^a, T^b \} = T(R) \delta^{ab}$

  note: The normalization T(R) depends on the representation.
        I define
        $T(F) = 1/2$
        for the fundamental representation (F), or the
        n-representatio of SU(n), following the normalization
        of the rotation operator,
        $T^a(F) = (\sigma^a)/2$
        for the fundamental representation of $SU(2)$.

{\bf hw05-10}:
  Please obtain the coefficient c of the Yang-Mills Lagrangian
\begin{eqnarray}
    L_{YM} = c Tr\{ [D_\mu(x), D_\nu(x)] [D^\mu(x),D^\nu(x)] \}
\end{eqnarray}
  such that in the absence of non-linear terms (terms which
  contain 3 or 4 $A^a_\mu(x)$ fields] the Lagrangian reduces
  to $n^2-1$ times that of the $U(1)$ gauge fields.

{\bf hw05-11}:

  With the above normalization, please obtain the terms which
  contain 3 gauge fields (the terms proportional to g), and
  those which contain 4 gauge fields (the terms proportional
  to $g^2$).

  Those interactions are absent in the $U(1)$ theory.  The photons
  do not scatter with the other photons in the lowest order of
  the perturbation theory (which is called the classical level
  in QED).  In other words, the light-by-light scattering is
  a quantum effect in QED, which is strongly suppressed at
  energies below the electron mass scale.

  In non-Abelian gauge theories like $SU(3)_C$, the gauge bosons
  interact with the other gauge bosons at the tree-level.
  There is no scale below which the gauge bosons behave as
  classical fields.  This is the origin of the absence of
  classical gluon fields (confinement, infrared slavery)
  in QCD.

That's all for hw.05.\\

Best regards,\\

Kaoru

\end{document}

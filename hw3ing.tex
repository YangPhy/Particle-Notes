
\documentclass[12pt]{article}
\usepackage{amsmath,graphicx,color,epsfig,physics}
\usepackage{float}
\usepackage{subfigure}
\usepackage{slashed}
\usepackage{color}
\usepackage{multirow}
\usepackage{feynmp}
\textheight=9.5in \voffset=-1.0in \textwidth=6.5in \hoffset=-0.5in
\parskip=0pt
\begin{document}

\begin{center}
{\large\bf HW3 for Advanced Particle Physics} \\

\end{center}

\vskip 0.2 in

Dear students, \\

{\bf hw03-1}:

  In hw01, I asked you to give the invariant mass (rest mass)
  and the total decay width of several particles both in
  units of GeV.  In this way, we can tell the ratio,
  \begin{eqnarray}
    \Gamma/m = ({\rm total~decay~width})/({\rm invariant~ mass})
  \end{eqnarray}  
  as a dimensionless numerical value.  This ratio tells us a lot
  about the particle properties and their interactions.\\

  Please obtain the ratio for the following particles
  with two digit accuracy.  If the ratio is smaller
  than 0.0010, say, 0.00033, then quote the result as
  $3.3 x 10^{-4}$.
 
  top\\
  Higgs\\
  Z\\
  W\\
  tau\\ 
  mu\\
  n\\
  pi+\\
  pi0\\
  p (upper bound)\\
  e (upper bound)\\

{\bf hw03-2}:

  If you can explain the relative sizes of the above
  ratios, please explain them briefly.  Examples are:\\

  The difference between top and W/Z and Higgs\\
  The difference between top and tau\\
  The difference between tau and mu\\
  The difference between mu  and n\\
  The difference between pi+ and pi0\\
  The reason why the ratio is extremely small (upper
  bounds) for p and e.\\

                                *****

  Today, I introduced fundamental particles of the SM in
  a slightly more mathmatical way, by using their transformation
  properties under the SM symmetry, $SU(3)_{color} \times SU(2)_L \times U(1)_Y$.\\
  
  In this homework, I would like you to learn how the fundamental
  matter (fermion) multiplets and the Higgs multiplets transform
  under the SM gauge transformations. \\

  In order to write down the SM Lagrangian (which describes the
  interactions of all the fundamental particles) in a compact form,
  we introduce several short-hand notations. \\

  Matter (fermion) fields:\\

  We have 5 multiplets of matter fields, which transform under
  SU(3)xSU(2)xU(1) as\\

  Qi  with i = 1,2,3, which transform as (3, 2,  1/6)\\
  uRi with i = 1,2,3, which transform as (3, 1,  2/3)\\
  dRi with i = 1,2,3, which transform as (3, 1, -1/3)\\
  Li  with i = 1,2,3, which transform as (1, 2, -1/2)\\
  lRi with i = 1,2,3, which transform as (1, 1, -1  )\\

  The numbers in the right-hand-side parentheses are
  (SU(3) multiplet, SU(2) multiplet, $U(1)_Y$ charge).\\

  Each multiplet replicates 3 times, and I labeled them with an
  index i, which takes 3 values, 1, 2, 3.  The replication index
  i is referred to as `generation' or `family'.  We do not know
  why there are replications, and why there are 3 replications.\\

  Please note that Qi's are sextets, uRi's and dRi's are triplets,
  Li's are doublets, and lRi's are singlets.  Therefore, in
  each generation, there are 15 (=6+3+3+2+1) matter fields.\\

  All the 15 (times 3 by including 3 generations) matter fields
  transform as spin 1/2 particles under the Lorentz transformation.
  Among the 15 fermions in each generation, 8 (Q abd K) transform
  as left-handed fermions, while the remaining 7 (uR, dR, lR)
  transform as right-handed fermions.\\

  We will later learn that both left-handed and right-handed
  fermions transform as doublets (two-component spinors) under
  the Lorentz transformation.  We suppress the spinor indices
  untill we study Lorentz transformations.\\

  Gauge fields:\\

  There are 3 types of gauge fields.\\

  $A^a$ with a = 1,2,...,8 transform as (8, 1, 0)\\
  $W^k$ with k = 1,2,3     transform as (1, 3, 0)\\
  B                      transform as (1, 1, 0)\\

  In total, there are 12 (=8+3+1) gauge bosons in the SM.\\

  Please observe that the B boson transforms as a singlet
  under the SM gauge transformations, and it has zero hypercharge.
  We express this fact in various wqys which have exactly the
  same meaning:\\

   the B boson is a SM singlet.\\
   the B boson does not have a SM charge (quantum number).\\
   the B boson does not transform under the SM gauge group.\\
   the B boson is invariant under the SM gauge transformations.\\

  We will study the gauge transformation properties of the above
  12 gauge bosons in the next lectures.\\

  All the 12 gauge bosons transform as a spin 1 (vector) particles
  under the Lorentz transformations.  They are all massless vector
  bosons before the $SU(2)_L \times U(1)_Y$ local phase transformation
  symmetry (gauge symmetry) breaks down spontaneously into
  $U(1)_EM$ (the Electro-Magnetic gauge symmetry).\\

  The massless vector bosons have only two physical components
  (the so-called transversely polarized waves), and hence
  the description in terms of the 4-component vector bosons
  have two redundant degrees of freedom (the so-called longitudinal
  and scalar-like polarizations).  That physics does not depend
  on these two redundant degrees of freedom can be expressed as
  local phase transformation invariance, or the gauge symmetry.\\

  Higgs bosons:\\

  There is just one Higgs doublet in the SM:\\

  $\phi$ transforms as (1, 2, +1/2)

  The Higgs doublet transforms as scalar (singlet) under the
  Lorentz transformation.  The following expressions have
  exactly the same meaning:\\

   the Higgs boson is a scalar boson\\
   the Higgs boson transform as a singlet under Lorentz transformations\\
   the Higgs boson does not transform under Lorentz transformations\\
   the Higgs boson is invariant under Lorentz transformations\\

  It is often useful to introduce its charge conjugate field
  $\phi^c = i \sigma^2 \phi^*$ which transforms as (1, 2, -1/2)
  when we write down the SM Lagrangian in a compact form.\\

  Please observe that the Higgs field $(\phi^c)$ has identical
  SM quantum numbers as the left-handed leptons (Li with
  i = 1,2,3).  This fact may be a hint of new physics.\\

  Now, let me give the homework.\\

{\bf hw03-3}:

  The sextet Qi can be expressed as
  \begin{eqnarray}
  Qi = (uLi, dLi)^T =
  \begin{pmatrix}
    u_{Li} \\ d_{Li}
  \end{pmatrix} 
  \end{eqnarray}
  by showing its doublet-ness under $SU(2)_L$ explicit.
  Here T stands for the transpose.  I will often use the middle
  notation, (a,b)^T for column vectors, since it saves text lines.\\

  The so-called Gelmann-Nishijima relation,
\begin{eqnarray}
  Q = T^3 + Y
\end{eqnarray}
  gives the electric charge of $u_Li$ and $d_Li$.  Here $T^3$ is the
  diagonal member of the three SU(2) generators,
  \begin{eqnarray}
  T^3 = \sigma^3 /2
  \end{eqnarray}
  and Y=1/6 is the hypercharge of Qi's
  \begin{eqnarray}
  Y Qi = (1/6) Qi .
  \end{eqnarray}
  Mathematically, we may write
  \begin{eqnarray}
  Qi &=& (uLi,dLi)^T \\
     = u_{Li} (1,0)^T + d_{Li} (0,1)^T
  \end{eqnarray}
  Please obtain the electric charges of uLi and dLi's
  by using
  \begin{eqnarray}
  Q (1,0)^T &=& (T^3 + Y) (1,0)^T = ( 1/2 + Y) (1,0)^T\\
  Q (0,1)^T &=& (T^3 + Y) (0,1)^T = (-1/2 + Y) (1,0)^T
  \end{eqnarray}
hint: uLi can be written as $u_{Li} (1,0)^T$, i.e., the upper
  component of the SU(2) doublet, Qi.\\

  Likewise, the doublets Li can be written as
  \begin{eqnarray}
  L_i = (v_{Li}, l_{Li})^T .
  \end{eqnarray}
  Please obtain the electric charge of vLi and lLi by using
  \begin{eqnarray}
  Y Li = (-1/2) Li
  \end{eqnarray}
  and the Nishijima-Gellmann relation.\\

  *****
  Please observe that the two-vectors, Qi and Li, are NOT the
  eigen vectors of the 2x2 matrix operator Q.  Rather, their
  upper and lower components have different eigen values of the
  $U(1)_EM$ charge operator (generator) Q.  The electromagnetic
  interactions, or the $U(1)_EM$ gauge interactions break the
  $SU(2)_L$ symmetry explicitly.  The meaning of the word `explicitly'
  will be made clear in latter lectures.\\
  *****

  Under SU(2), the doublet fields transform as
  \begin{eqnarray}
  phi \to phi' &=& U(\theta_1, \theta_2, \theta_3) \phi \\
              &=& \exp{ i (T^1 \theta_1 + T^2 \theta_2 + T^3 \theta_3) } \phi
  \end{eqnarray}
  where $\theta_k$ (k=1,2,3) are real parameters that specify a
  particular SU(2) transformation, and
  $T^k = \sigma^k/2 (k=1,2,3)$
  are the SU(2) generators in the doublet (fundamental) representation. \\

  All the elements of the SU(2) transformations, the U matrices,
  are obtained by using its 3 generators and 3 real parameters.
  \begin{eqnarray}
  U(\theta_1, \theta_2, \theta_3)  &=&\exp{ i (T^1 \theta_1 + T^2 \theta_2 + T^3 \theta_3) }
  \end{eqnarray}
  Here $\sigma^k (k=1,2,3)$ are Pauli matrices:
  \begin{eqnarray}
  \sigma^1 &=& 
  \begin{pmatrix}
    0 & 1 \\
    1 & 0
  \end{pmatrix}\\
  \sigma^2 &=&
  \begin{pmatrix}
    0 & -i \\
    i & 0
  \end{pmatrix}\\ 
  \sigma^3 &=&
  \begin{pmatrix}
    1 & 0 \\
    0&-1
  \end{pmatrix}
  \end{eqnarray}


{\bf hw03-4}:

  Show 
  \begin{eqnarray}
    (\sigma^k)^2=1=
    \begin{pmatrix}
      1 & 0\\
      0 & 1
    \end{pmatrix}
  \end{eqnarray}
  for all k=1, 2, 3.

{\bf hw03-5}:

  Show
  \begin{eqnarray}
    (T^k)^\dagger = T^k   (k = 1,2,3)
  \end{eqnarray}
  and
  \begin{eqnarray}
  U(\theta_1, \theta_2, \theta_3)^\dagger U(\theta_1, \theta_2, \theta_3)
  = 1.
  \end{eqnarray}
hint: $e^A e^B = e^(A+B)$ if and only if $[A,B] = 0$.

  The SU(2) transformation matrix U is `unitary', $U^\dagger U = 1$,
  and its three generators, $T^k (k = 1,2,3)$ are Hermite matrices.

{\bf hw03-6}:

  Show the $SU(2)_L$ and $U(1)_Y$ transformation properties of
  $\phi^*$ .
  Since this homework is a little bit tough for some students,
  please let me guide you step by step.
  We may start from the $U(1)_Y$ transformation of phi:
  \begin{eqnarray}
  phi \to phi' = U(\theta_0) \phi
              = \exp{ i Y \theta_0 } \phi
  \end{eqnarray}
  with Y = +1/2 for $\phi$.  It then follows
  \begin{eqnarray}
  \phi^* \to \phi^*' &=& (\phi')^*\\
                  &=& U(\theta_0)^* \phi^*\\
                  &=& \exp{ -i Y \theta_0 } \phi^*\\
                  &=& \exp{ i (-Y) \theta_0 } \phi^*
  \end{eqnarray}
  From this, we can tell that if the hypercharge (the eigenvalue of
  the operator Y) of $\phi$ is +1/2, then the hypercharge of $\phi^*$ is
  -1/2.

  Very generally, the complex conjugate of any field has an opposite
  U(1) charge of the original field.
  In QFT (Quantum Field Theory), the complex conjugate fields
  annihilate and create particles with opposite charges, as
  we will learn later in my lectures.\\

  Now, let us go on studying the SU(2) transformation of $\phi^*$
\begin{eqnarray}
  \phi^* \to (\phi^*)'
         = (\phi')^*
         = ( U \phi )^*
         = U^* \phi^*
         = \exp{ -i (T^1* \theta_1 + T^2* \theta_2 + T^3* \theta_3) } \phi^*
\end{eqnarray}
  

  Here, I denote $U(\theta_1,\theta_2,\theta_3)$ simply as U for brevity.
  $T^k*$ is a shorthand for $(T^k)^*$ (k = 1,2,3).

  Since $U^*$ is not the same as U, the SU(2) transformation of $\phi^*$
  is different from that of phi.  We say that phi transform as 2,
  whereas $\phi^*$ transform as $2^*$.\\
\begin{eqnarray}
  &&2:    \phi   \to \phi'   = U   \phi\\
  &&2^*:  \phi^* \to \phi^*' = U^* \phi^*
\end{eqnarray}


  Please note that
\begin{eqnarray}
  T^1* = T^1, T^2* = -T^2, T^3* = T^3
\end{eqnarray}
  and hence
\begin{eqnarray}
  \phi^* \to \phi^*'
         = \exp{ iT^1(-\theta_1) +iT^2(\theta_2) +iT^3(-\theta_3) } \phi^*
\end{eqnarray}
  or
\begin{eqnarray}
  U(\theta_1,\theta_2,\theta_3)^* = U(-\theta_1,\theta_2,-\theta_3).
\end{eqnarray}

  $U^*$ and $U$ are clearly DIFFERENT from each others, unless both
  $\theta_1$ and $\theta_3$ vanish.  (They transform the same way along
  the $\theta_2$ direction, or for the rotation about the axis 2,
  which infact was used to define G-parity for pions).

  We express the answer of this homework (which I solved
  explicitly above, rather than giving its hints) as follows:\\

   $\phi$   transforms as (1, 2,    1/2)\\
   $\phi^*$ transforms as (1, $2^*$, -1/2)

  under $SU(3)_C$, $SU(2)_L$, $U(1)_Y$.

{\bf hw03-6}:

  Please show that the product
\begin{eqnarray}
  (\phi^*)^T \phi = \phi^\dagger \phi
\end{eqnarray}
   is invariant under SU(2).  This is a common property of all unitary
  transformations, since it follows from the unitarity condition
  of the transformation matrix,  $U^\dagger U = 1$ or
  $U^* = ( U^{-1} )^T$
  We refer to the above identity as; the complex representation
  of the unitary group transforms as the transpose of the inverse
  representation.

{\bf hw03-7}:

  Please show that
\begin{eqnarray}
  \phi^c = (i\sigma^2) \phi^*
\end{eqnarray}
  transforms as a doublet 2, not as an anti-doublet ($2^*$).
  In other words, phi and phi^c transform exactly the same way under
  $SU(2)_L$, even though they transform with opposite charge for
  $U(1)_Y$.

  This is significantly more difficult than the previous homework.
  Please follow the steps given below.
\begin{eqnarray}
  \phi^c \to (\phi^c)'
  = (\phi')^c
  = (i\sigma^2) (\phi')^*
  = (i\sigma^2) (U \phi)^*
  = (i\sigma^2) U^* \phi^*
\end{eqnarray}
   where we already obtained U^* in hw03-3.  Please show
\begin{eqnarray}
  (-i\sigma^2) (i\sigma^2) = 1
\end{eqnarray}
  and insert the above identity betweek $U^*$ and $\phi^*$,
  and observe
\begin{eqnarray}
  \phi^c \to (\phi^c)'
  = (\phi')^c
  = (i\sigma^2) (\phi')^*
  = (i\sigma^2) U^* \phi^*
  = (i\sigma^2) U^* (-i\sigma^2) (i\sigma_2) \phi^*
  = (i\sigma^2) U^* (-i\sigma^2) \phi^c
\end{eqnarray}
 
  Now the assignment is to show
\begin{eqnarray}
  (i\sigma^2) U^* (-i\sigma^2) = U
\end{eqnarray}
or equivalently
\begin{eqnarray}
   \sigma^2 U^* \sigma^2 = U
\end{eqnarray}
  for arbitrary $(\theta_1,\theta_2,\theta_3)$.

  If this is the first time that you show this, I suggest that
  you show the identity for infinitesimal transformations:
  \begin{eqnarray}
  U = \exp{ i(T^1 \theta_1 + T^2 \theta_2 + T^3 \theta_3) }
    = 1 + i(T^1 \theta_1 + T^2 \theta_2 + T^3 \theta_3) + higher~orders\\

  U^* = \exp{ -i (T^1* \theta_1 + T^2* \theta_2 + T^3* \theta_3) }
    = 1 - i(T^1* \theta_1 + T^2* \theta_2 + T^3* \theta_3) + higher~orders
  \end{eqnarray}

  You will find it convenient if you prove the following identity
  among the $\sigma$-matrices first:

  $(\sigma^2) (\sigma^k)^* (\sigma^2) = -(\sigma^k)$  for k=1,2,3.

  Please show the above identity many times until it becomes one
  of your favorite vocabulary.  For advanced students, I suggest
  that you show the identity for finite transformations.\\

  Let us summarize our findings here:\\

   $\phi$   transforms as (1, 2,    1/2)\\
   $\phi^*$ transforms as (1, $2^*$, -1/2)\\
   $phi^c$ transforms as (1, 2,   -1/2)\\

  Now that we showed that $\phi^c$ transforms as the same as $\phi$,
  under $SU(2)_L$, but has the same Y charge as L, $\phi^c$ and L
  transform exactly the same way under the SM gauge transformation.\\

Let me stop here, and I will review the usefulness of obtaining 
an SU(2) soublet $\phi^c$ from an SU(2) doublet $\phi$ in my next 
lecture.  Because the excercises with $\sigma$ matrices need 
patient excercises, please do this homework before my next 
lecture.

Best regards,\\

Kaoru

\end{document}

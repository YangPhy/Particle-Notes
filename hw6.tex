\documentclass[12pt]{article}
\usepackage{amsmath,graphicx,color,epsfig,physics}
\usepackage{float}
\usepackage{subfigure}
\usepackage{slashed}
\usepackage{color}
\usepackage{multirow}
\usepackage{feynmp}
\textheight=9.5in \voffset=-1.0in \textwidth=6.5in \hoffset=-0.5in
\parskip=0pt
\def\del{{\partial}}


\begin{document}

\begin{center}
{\large\bf HW6 for Advanced Particle Physics} \\

\end{center}

\vskip 0.2 in

Dear students:\\

  Today, I review a few fundamental properties of ``group'', which
  I found useful throughout my research in particle physics.\\

  Let me start from the definition of a group:\\

  For arbitrary elements a and b of a group,
  \begin{itemize}
      \item $a\times b = c$ is also an element of the group.
      \item $e$ is an element of the group such that $a \times e = e\times a = a$.
      \item for an arbitrary element $a$ there is an element $a^{-1}$.
      such that $a\times a^{-1} = a^{-1} \times a = e$.
      \item $(a\times b) \times c = a \times (b \times c)$.
  \end{itemize}

  The element $e$ is called the unity element (no transformation),
  and the element $a^{-1}$ is called the inverse element.

  Linear representation:  For all the groups I introduce in
  my lectures, we can find a matrix representation.  For instance,
  we can represent an abstract relation $a \times b = c$ in terms of
  $n \times n$ matrices $A$, $B$, $C$ such that $AB = C$.

  The matrix representation for $e$ (the identity or unity) is $I$,
  which is the diagonal matrix $I = {\rm diag} \{1,1,...,1 \}$.

  When a matrix $A$ is a representation of a group element,
  ${\rm det}{A} \neq 0$, in order for $A^{-1}$ to exist.
  For the $SU(n)$ group, ${\rm det}{A}=1$.

  If a matrix representation exisits, the last rule 4
  (associativity) always holds.

  In all matrix representations of a transformation group,
  the state which is transformed is expressed as a column
  vector.  Transformations are then expressed as matrix
  multiplication from the left-hand-side, and the transformed
  state is also a column vector.

{\bf hw06-1}:  Please show that if $A \times B = C$ is a linear (matrix) representation of $a\times b = c$, then
\begin{enumerate}
    \item $A \times B = C$ \label{it.matrixrep}
    \item $A^* \times B^* = C^*$ (complex representation) \label{it.matrixrepcom};
    \item $(A^{-1})^T \times (B^{-1})^T = (C^{-1})^T$ (inverse-tranpose)\label{it.matrixrepinvtran};
    \item $A^{-1})^\dagger \times (B^{-1})^\dagger = (C^{-1})^\dagger$
    (complex inverse-tranpose, or inverse-dagger). \label{it.matrixrepinvdag}
\end{enumerate}
   are all representations of the same operation $a\times b = c$.
  In general there are 4 linear representations as above.\\


  When the matrix is real ($A^* = A$), \ref{it.matrixrepcom}=\ref{it.matrixrep} and \ref{it.matrixrepinvdag}=\ref{it.matrixrepinvtran}, and
  hence there are just two representations, \ref{it.matrixrep} and \ref{it.matrixrepinvtran}.
  (example: rotation, Lorentz transformation of co-ordinates
  and four-vectors, etc)

  If the representation \ref{it.matrixrep} is called `covariant', then
  the representation \ref{it.matrixrepinvtran} may be called `contra-variant'.

  In case of the vector representation of the rotation
  group, covariant and contra-variant vectors are the
  same.  In this case, the norm
  $v^T v$
  is conserved, because $O^T O = O^{-1} = 1$.

  In the vector representation of the Lorentz group, if
  a column vector $a^\mu = (a^0, a^1, a^2, a^3)^T$
  transforms as 1 (covariant), then the column vector
  $a_\mu = (a_0, a_1, a_2, a_3)^T = (a^0, -a^1, -a^2, -a^3)^T$
  transforms as \ref{it.matrixrepinvtran} (contra-variant), and hence the product
  $a_\mu a^\mu = (a_0, a_1, a_2, a_3)(a^0, a^1, a^2, a^3)^T$
  is invariant under the transformation.

{\bf hw06-2}:  When a four-vector $p^\mu = (E, p_x, p_y, p_z)^T$ transform as
\begin{eqnarray}
    p^\mu = (E, p_x, p_y, p_z)^T \to (p')^\mu = (E', p_x', p_y', p_z')^T
\end{eqnarray}
   we can express the transformation by a matrix representation,
\begin{eqnarray}
    (E', p_x', p_y', p_z')^T = L (E, p_x, p_y, p_z)^T
\end{eqnarray}
    with a real $4\times 4$ matrix $L$.  In this case, the contravariant
  vector transform as
  \begin{eqnarray}
    p_\mu = (E, -p_x, -p_y, -p_z)^T \to p'_\mu = (E', -p_x', -p_y', -p_z')^T
  \end{eqnarray}
  where
  \begin{eqnarray}
    (E', -p_x', -p_y', -p_z')^T = L' (E, -p_x, -p_y, -p_z)^T
  \end{eqnarray}
   by using another real $4\times 4$ matrix $L'$.  Show,
\begin{eqnarray}
    L' = (L^{-1})^T
\end{eqnarray}
hint: Please use your knowledge that
\begin{eqnarray}
    p^\mu p_\mu = E^2 - p_x^2 - p_y^2 - p_z^2
\end{eqnarray}
  is invariant under Lorentz transformation.\\


  When the matrix is unitary ($A^\dagger = A^{-1}$),
  then \ref{it.matrixrepcom}=\ref{it.matrixrepinvtran}, and \ref{it.matrixrep}=\ref{it.matrixrepinvdag}.  Therefore, we usually consider
  only two representations, \ref{it.matrixrep} (covariant) and \ref{it.matrixrepcom} (complex).
  In other words, the complex representation (\ref{it.matrixrepcom}) is also
  the inverse-transpose representation (\ref{it.matrixrepinvtran}).  Therefore,
  the norm
  $(v^*)^T v = v^\dagger v$
  is conserved, because $U^\dagger = U^{-1} U = 1$.

  In $SU(3)_{color}$, the state which transofrms as covariant
  may be called quark, whereas the state which transforms
  as its complex representation may be called anti-quark.
\begin{eqnarray}
  &&q   \to q'   = U   q \\
  &&q^* \to  (q^*)' = U^* q^*
\end{eqnarray}

  For a Unitary group, because complex representation is the
  same as the inverse-transpose representation,
  $U^* = (U^{-1})^T$
  we can introduce covariant-contravariant representations
  for quark and anti-quark.

  Let us denote a quark state as
  $q = (q^1, q^1, q^3)^T$
  which transforms as
\begin{eqnarray}
  q \to q' = ((q')^1, (q')^2, (q')^3)^T = U (q^1, q^1, q^3)^T
\end{eqnarray}
   We can now denote an anti-quark state as
  $q^* = (q_1, q_2, q_3)^T$
  by using the lower-indices (contravariant).  It transforms as
\begin{eqnarray}
  q^* \to  (q')^* &=& (q'_1, q'_2, 1'_3)^T \\
  &=& U^* (q_1, q_2, q_3)^T\\
  &=& (U^{-1})^T (q_1, q_2, q_3)^T
\end{eqnarray}
 
  Please note that we use upper $1,2,3$ indices for the triplet
  or fundemental representation, and lower indices for its
  complex-conjugate or anti-triplet or $3^*$ representation.

  Let us now introduce a tensor representation of the $SU(3)$
  transformations.
  We express the components of the $3\times 3$ Unitary matrices
  $U$ and $U^*$ by using upper and lower indices.
\begin{eqnarray}
  U^i_j = (U)_{ij}\\
  U_i^j = (U^*)_{ij}
\end{eqnarray}
 
The transformations of $q^i$ and $q_i$ can be expressed as
\begin{eqnarray}
 && q^i \to (q')^i = U^i_j q^j~~({\rm quarks}= {\rm triplets}= 3),\\
 && q_i \to (q')_i = U_i^j q_j~~({\rm anti-quarks} = {\rm anti-triplets} = 3^*)
\end{eqnarray}

  The rule governing the upper and lower indices are very simple.
  Please observe that under the $SU(3)$ transformations, quarks are
  transformed into quarks, and hence both the initial state and
  the transformed state should have upper indices.  In exactly
  the same manner, anti-quarks are transformed into anti-quarks.
  Therefore, both the starting and transformed states should have
  lower indices.

  The transformation matrix (tensor) should have upper and
  lower indices, such that the repeated indices (the indices
  of original state and one of the U tensor) which are summed
  over should appear as a pair of up and down indices.

  In the tensor representation, the invariance of the norm
  is expressed simply as
\begin{eqnarray}
  q_i q^i \to  (q')_i (q')^i
  = (U_i^j q_j) (U^i_k q^k)
  =  U_i^j U^i_k  q_j q^k
  =  \delta ^j_k   q_j q^k
  =  q_k q^k
\end{eqnarray}
  Repeated pair of up and down indices should be summed over,
  and if no indices remain, the product is $SU(3)$ invariant.
  In this notation, the unitarity of the SU(3) transformation
  is expressed as
\begin{eqnarray}
  U_i^j U^i_k  = \delta^j_k
\end{eqnarray}
 

{\bf hw06-3}: Please confirm the above statements for the following
  three expressions which appear above:
\begin{itemize}
  \item $ q^i \to q'^i = U^i_j q^j$,
  \item $ q_i \to q'_i = U_i^j q_j$,
  \item $  q_i q^i \to q'_i q'^i =  U_i^j U^i_k  q_j q^k =  \delta^j_k  q_j q^k =  q_k q^k$.
\end{itemize}
 
  You may find the rules exactly the same as what we learned
  from Albert Einstein for the Lorentz transformation properties
  of arbitrary tensors.

  This representation is called tensor representation, and
  is useful (almost necessary) when studying the transformation
  properties of composite representations, such as multiple-quark
  states, like di-quarks and Baryons.\\


 {\bf  Now let us move on to $SU(2)$.}

  Although $2 \times 2$ matrix representation of $SU(2)$ is a special case of
  $n \times n$ matrix representation of $SU(n)$, it is convenient to use all
  4 representations in case of $SU(2)$.  When we adopt all 4
  representations of $SU(2)$, we call them spinor representations.
  It is an extention of the tensor representation I introduced
  above, where we use upper and lower indices to distinguish
  covariant and contra-variant indices.  We need 4 types of indices
  to distinguish the 4 representations.  They are usually expressed
  by upper and lower indices for the original (\ref{it.matrixrep}) and
  inverse-transpose (\ref{it.matrixrepinvtran}), whereas by dotted-upper and dotted-lower
  indices for complex (\ref{it.matrixrepcom}) and inverse-dagger (\ref{it.matrixrepinvdag}) representations.

  We have shown in the homework 04, that for any doublet
  $\phi$ of $SU(2)$, which transforms as a double (\ref{it.matrixrepcom}) representation,
  \begin{eqnarray}
    \phi \to \phi' = U \phi,
  \end{eqnarray}
  the state
  $\phi^c = (i\sigma^2) \phi^*$
  also transforms as a double (\ref{it.matrixrepcom})
  \begin{eqnarray}
  \phi^c \to (\phi^c)' = U \phi^c.
  \end{eqnarray}

  Therefore, we find that its complex conjugate
  $(\phi^c)^* = (i\sigma^2) \phi$
  transforms as $2^*$, which is a complex representation,
  but it is also an inverse-transpose because of unitarity,
  $U^* = (U^\dagger)^T = (U^{-1})^T$.
  We can hence regard
  $(\phi^c)^* = (i\sigma^2) \phi$
  as a contra-variant representation of the original
  representation, \ref{it.matrixrepcom}, of $SU(2)$. Therefore,
  $[(i\sigma^2) \phi]^T L = \phi^T (-i\sigma^2) L$
  is invariant under $SU(2)$.

{\bf hw06-4}: Show the $SU(2)$ transformation of $\phi$, $\phi^c$,
  $(\phi^c)^*$, $L$, and show the invariance of the two terms,
  $(\phi^c)^\dagger L$, $[(-i\sigma^2) \phi]^T L = \phi^T (-i\sigma^2) L$
  and confirm that they are in fact the same.

{\bf hw06-5}:  For $\phi = (\phi^+, \phi^0)^T$ and $L = (\nu_L, l_L)^T$,
  please express
  $(\phi^c)^\dagger L =  \phi^T (-i\sigma^2) L$
  in terms of the component fields, $\phi^+$, $\phi^0$,
  $\nu_L$ and $l_L$.

  The above exercises may be useful for you to learn that the same
  $SU(2)$ invariant combination can have several expressions.
  When an element of a group is labeled by continuous real numbers
  (like $\theta$), it is called a continuous group.
  For instance, for a real number $\theta$, we can define a complex number
  $U(\theta) = e^{i\theta}$.
  Then the standard complex algebra gives
 \begin{itemize}
   \item $U(a) \times U(b) = U(a+b)$
   \item $U(0) = 1$
   \item $U(a)^{-1} = U(-a)$
   \item $(U(a)\times U(b))\times U(c) = U(a)\times (U(b)\times U(c)) = U(a+b+c)$.
 \end{itemize}
  for arbitrary real numbers $a$, $b$, $c$.  This is a complex-number
  representation of the $U(1)$ group.

  In quantum mechanics, the state $\psi$ is represented by a complex
  function, and its phase can be changed as
\begin{eqnarray}
  \psi \to \psi' = U(\theta) \psi
\end{eqnarray}
  without changing the probability, $|\psi|^2$.  If the $U(1)$ transformation
  does not change physics, the particle number (charge) is conserved.

  Let us review Noether's theorem for a simple Lagrangian density:
\begin{eqnarray}
  {\cal L}  =  (\del^\mu \phi)^\dagger (\del_\mu \phi) -m^2 \phi^\dagger \phi
\end{eqnarray}
   which is clearly invariant under $U(1)$ transformation,
\begin{eqnarray}
  \phi \to \phi' = U \phi = e^{i\theta} \phi.
\end{eqnarray}

  The invariance of the Lagrangian can be expressed as
  \begin{eqnarray}
    {\cal L}(\phi, \del_\mu \phi) = {\cal L}(\phi', \del_\mu \phi')
  \end{eqnarray}
  or
  \begin{eqnarray}
    0 = \delta {\cal L} &=& {\cal L}(\phi', \del_\mu \phi') - {\cal L}(\phi, \del_\mu \phi)\\
  &=&  [\delta {\cal L} / \delta{\phi}] \delta\phi +  [\delta {\cal L} / \delta{\del_\mu \phi}] \delta{\del_\mu \phi}
  \end{eqnarray}
  
By using the equation of motion,
\begin{eqnarray}
  [\delta {\cal L} / \delta{\phi}]  =  \del_\mu [\delta {\cal L}/\delta{\del_\mu \phi}]
\end{eqnarray}
  we find
\begin{eqnarray}
  \delta L  =  (\del_\mu [\delta L/\delta{\del_\mu \phi}]) \delta\phi +  [\delta L / \delta{\del_\mu \phi}] (\del_\mu \delta\phi) = \del_\mu ([\delta L/\delta{\del_\mu \phi}] \delta\phi) =  \del_\mu j(x)^\mu = 0.
\end{eqnarray}
  We call
  \begin{eqnarray}
    j(x)^\mu =  [\delta L/\delta{\del_\mu \phi}] \delta\phi
  \end{eqnarray}
  as a current, and 
  \begin{eqnarray}
    \del_\mu j(x)^\mu = 0
  \end{eqnarray}
  as current conservation.  It can be written as
  \begin{eqnarray}
    \del_0 j(x)^0 = \sum_{k=1,2,3} \del_k j(x)^k
  \end{eqnarray}
   and its $d^3 x = dx dy dz$ integral gives
  \begin{eqnarray}
    \del_0 Q = \int d^3x \nabla \cdot {\vec j} (x) = {\rm surface~ integral} = 0
  \end{eqnarray}
    with $Q = \int d^3x j(x)^0$

{\bf hw06-6}:  Please follow the above derivation of Noether's theorem,
  and obtain $j^\mu(x)$ and $Q$ for our Lagrangian. Can you tell
  that $Q$ represents the total probability of our field $\phi(x)$ ?\\

  $SO(3)$ group has a real $3\times 3$ matrix representation.
  The $3 \times 3$ real matrix representation $O$ of $SO(3)$ satisfies
    ${\rm det}\{O\} = 1$, $O^* = O$, $O^T = O^{-1}$.

{\bf hw06-7}: Please show that an arbitrary element of the $SO(3)$
  group can be parametrized in terms of three real numbers $x$, $y$, $z$
\begin{eqnarray}
  O(x,y,z) = e^{ -i x J_x -i y J_y -i z J_z }
\end{eqnarray}

hint: $O^* = O$ gives $J_k^* = -J_k$, $O^T = O^{-1}$ gives $J_k^T = -J_k$.
  How many independent parameters of a complex $3 \times 3$ matrix survives?

{\bf hw06-8a}: Please obtain the generators $J_x$, $J_y$, $J_z$, as
  $3 \times 3$ matrices, such that
\begin{itemize}
  \item $O(x,0,0)$ gives a rotation by an angle x about the x axis,
  \item $O(0,y,0)$ gives a rotation by an angle y about the y axis,
  \item $O(0,0,z)$ gives a rotation by an angle z about the z axis.
\end{itemize}

hint: Please note that the $2\times 2$ $SO(2)$ generator we obtained in the
  previous homework can be regarded as a $2 \times 2$ component of the
  operator $J_z$ in the $(x,y)$ plane. Likewise, it can be regarded
  as $J_x$ in the $(y,z)$ plane, $J_y$ in the $(z,x)$ plane. Please respect
  the cyclic properties when we obtain $J_x$, $J_y$, $J_z$ as $3 \times 3$
  Hermetian anti-symmetric matrices.

{\bf hw06-8b}: Please show 
\begin{eqnarray}
  &&[ J_x, J_y ] = i J_z \\
  &&[ J_y, J_z ] = i J_x \\
  &&[ J_z, J_x ] = i J_y
\end{eqnarray}

                                          
{\bf hw06-8c}: Please obtain the normalization $T(A)$
\begin{eqnarray}
  Tr\{ J_k J_l \} = T(A) \delta_{kl}
\end{eqnarray}
   Here $A$ stands for Adjoint, since $SO(3)$, or the $3\times 3$ orthogonal matrix
  representation of $SU(2)$ is called the adjoint representation. In
  general, the generators of $SU(n)$ group in the adjoint representation
  is $(n^2-1) \times (n^2-1)$ matrix, and it is normalized as
\begin{eqnarray}
  Tr\{ T^a T^b \} = T(A) \delta^{ab}
\end{eqnarray}
  with $T(A) = n$.  Do we find $T(A) = 2$ from our $J_x$, $J_y$, $J_z$ ?

{\bf hw06-8d}:  Please obtain the matrices $O(x,0,0)$, $O(0,y,0)$, $O(0,0,z)$,
  by summing over all the powers of $x$, $y$, $z$. \\

  $SU(2)$ group also has $2\times 2$ complex matrix representation, which
  is called fundamental representation.  The nxn (unitary) matrix
  representation $U$ of $SU(n)$ satisfies
\begin{eqnarray}
  {\rm det}\{U\} = 1, U^\dagger = U^{-1}
\end{eqnarray}
 

{\bf hw06-9a}:  Please show that an arbitrary element of SU(2) can be
  parametrized by $3$ real numbers, $x$, $y$, $z$, as
\begin{eqnarray}
  U(x,y,z) = e^{ -i x J_x -i y J_y -i z J_z },
\end{eqnarray}
  where $J_k$ are now $2\times 2$ complex matrices that satisfy
\begin{eqnarray}
  Tr\{J_k\} = 0, J_k^\dagger = J_k  (k=1,2,3).
\end{eqnarray}
 
{\bf hw06-9b}:  Please show that $J_k = \sigma_k/2$, gives the same
  commutation relations as $SO(3)$:
  \begin{eqnarray}
    &&[ J_x, J_y ] = i J_z \\
    &&[ J_y, J_z ] = i J_x \\
    &&[ J_z, J_x ] = i J_y
  \end{eqnarray}

{\bf hw06-9c}:  Please obtain the matrices $U(x,0,0)$, $U(0,y,0)$
  and $U(0,0,z)$, by summing over all order of $x$, $y$, $z$.

  If an element of a continuous group can be parametrized
  in terms of real numbers $x_k$ as  $A(x_k) = e^{ i \sum_k x_k T_k }$
  and if $A(x_k)$ is a matrix representation of a group, then $A(x_k) \times A(x'_k) = A(x"_k)$
  should follow, because the product of two transformations
  should also be a transformation.

{\bf hw06-10}: Please show that the above identity follows if all the
  commutators of the generators are linear combinations of generators:
\begin{eqnarray}
  [ T_k, T_l ] = i \sum_m f^m_{kl} T_m
\end{eqnarray}
  This is called the algebra of a group, and the real constants
  $f^{klm} = f^m_{kl}$ are called structure constants.

{\bf hw06-11}: Please show
\begin{eqnarray}
  &&{\rm det}\{A\} = 1  \Rightarrow   Tr\{T_k\} = 0\\
  &&A^\dagger = A^{-1} \Rightarrow T_k^\dagger = T_k.
\end{eqnarray}

{\bf hw06-12}: Please show that $n^2-1$ independent elements of an arbitrary
  $n\times n$ complex matrix $M$ satisfy $M^\dagger=M$ and $Tr\{M\}$=0.  This counts
  the number of generators of $SU(n)$ transformations.\\

That's all for the homework PT1: hw06.\\

Best regards,\\

Kaoru

\end{document}

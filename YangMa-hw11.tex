\documentclass[11pt]{article}
\usepackage{amsmath,graphicx,color,epsfig,physics}
%\usepackage{pstricks}
\usepackage{float}
\usepackage{subfigure}
\usepackage{slashed}
\usepackage{color}
\usepackage{multirow}
\usepackage{feynmp}
\usepackage[top=1in, bottom=1in, left=1.2in, right=1.2in]{geometry}
\def\del{{\partial}}
\begin{document}
\title{Particle physics HW11}
\author{Yang Ma}

\maketitle

\section{ }
In matrix reprensation, we have the Lorectz transformation as
\begin{eqnarray}
    (b^\mu) \to ({b'}^\mu) &=&      L     (b^\mu) \\
    (a_\mu) \to ({a'}_\mu) &=& L' (a_\mu).
\end{eqnarray}
Now we write out
\begin{eqnarray}
    (a'_\mu)^T (b'^\mu) =  (a_\mu)^T L^T  L' (b^\mu)
\end{eqnarray}
and see the condition
\begin{eqnarray}
    L'=(L^{-1})^T,
\end{eqnarray}
helps to guarantee Lorentz invariance of $a_\mu b^\mu$.

\section{ }
With
\begin{eqnarray}
    M^{ij} = \epsilon_{ijk} J_k,  ~~ \omega_{ij} = \epsilon_{ijk} \theta_k, ~~ M^{0k} = K_k,  ~~              \omega_{0k} = y_k, 
  \end{eqnarray}
we can write
\begin{eqnarray}
    \frac{1}{2} \omega_{ab} M^{ab}
    &=&  \frac{1}{2} (\epsilon_{ijk}\epsilon^{ijl}\theta_k J_l + 2\omega_{0k} M^{0k})\\
    &=&  \frac{1}{2} (2\delta^l_k \theta_k J_l + 2\omega_{0k} M^{0k})\\
    &=& \theta_1 J_1  +\theta_2 J_2  +\theta_3 J_3  +y_1 K_1  +y_2  K_2  +y_3  K_3, 
\end{eqnarray}
    
\section{ }
With
\begin{eqnarray}
    \omega_{\mu\nu} = g_{\mu_\rho}g_{\nu \sigma} \omega^{\rho\sigma},
\end{eqnarray}
we see
\begin{eqnarray}
    \omega_{12} = g_{11}g_{22} \omega^{12}=\omega^{12}, ~~\omega_{03} = g_{00}g_{33} \omega^{03}=\omega^{03}.
\end{eqnarray}
With
\begin{eqnarray}
    \omega_{\mu\nu} = g_{\mu_\rho} \omega^\rho_{~\nu},
\end{eqnarray}
we see
\begin{eqnarray}
    &&\omega_{12} = g_{11} \omega^1_{~2}=-\omega^1_{~2}, ~~\omega_{21} = g_{22} \omega^2_{~1}=-\omega^2_{~1},\\
    &&\omega_{03} = g_{00} \omega^0_{~3}=\omega^0_{~3}, ~~~\omega_{30} = g_{33} \omega^3_{~0}=-\omega^3_{~0}.
\end{eqnarray}
Recall
\begin{eqnarray}
    \omega_{ab}=-\omega_{ba},
\end{eqnarray}
we finally have
\begin{eqnarray}
   && \omega_{12}  
    = -\omega_{21}
    =  \omega^{12}
    = -\omega^1_{~2}
    =  \omega^2_{~1} \\
    && \omega_{03}     
      = -\omega_{30}
      = -\omega^{03}
      =  \omega^0_{~3}
      =  \omega^3_{~0}
\end{eqnarray}

\section{ }
This problem overlaps with previous one.

\section{ }
\begin{eqnarray}
    [J_1,J_2]&=&
    \begin{pmatrix}
        0 & 0 & 0 & 0\\
        0 & 0 & 0 & 0 \\
        0 & 0 & 0 & -i \\
        0 & 0 & i & 0
    \end{pmatrix}
    \begin{pmatrix}
        0 & 0 & 0 & 0 \\
        0 & 0 & 0 & i \\
        0 & 0 & 0 & 0 \\
        0 & -i & 0 & 0 
    \end{pmatrix}
    -
    \begin{pmatrix}
        0 & 0 & 0 & 0 \\
        0 & 0 & 0 & i \\
        0 & 0 & 0 & 0 \\
        0 & -i & 0 & 0 
    \end{pmatrix}
    \begin{pmatrix}
        0 & 0 & 0 & 0\\
        0 & 0 & 0 & 0 \\
        0 & 0 & 0 & -i \\
        0 & 0 & i & 0
    \end{pmatrix}\\ 
    &=&
    \begin{pmatrix}
        0 & 0 & 0 & 0\\
        0 & 0 & 1 & 0\\
        0 &-1 & 0 & 0\\
        0 & 0 & 0 & 0
    \end{pmatrix}
    =i
    \begin{pmatrix}
        0 & 0 & 0 & 0 \\
        0 & 0 & -i & 0 \\
        0 & i & 0 & 0 \\
        0 & 0 & 0 & 0 
    \end{pmatrix}
      =iJ_3 
\end{eqnarray}
\begin{eqnarray}
    [J_2,J_3]&=&
    \begin{pmatrix}
        0 & 0 & 0 & 0 \\
        0 & 0 & 0 & i \\
        0 & 0 & 0 & 0 \\
        0 & -i & 0 & 0 
    \end{pmatrix}
    \begin{pmatrix}
        0 & 0 & 0 & 0 \\
        0 & 0 & -i & 0 \\
        0 & i & 0 & 0 \\
        0 & 0 & 0 & 0 
    \end{pmatrix}
    -
    \begin{pmatrix}
        0 & 0 & 0 & 0 \\
        0 & 0 & -i & 0 \\
        0 & i & 0 & 0 \\
        0 & 0 & 0 & 0 
    \end{pmatrix}
    \begin{pmatrix}
        0 & 0 & 0 & 0 \\
        0 & 0 & 0 & i \\
        0 & 0 & 0 & 0 \\
        0 & -i & 0 & 0 
    \end{pmatrix} \\ 
    &=& 
    \begin{pmatrix}
        0 & 0 & 0 & 0\\
        0 & 0 & 0 & 0 \\
        0 & 0 & 0 & 1 \\
        0 & 0 &-1 & 0
    \end{pmatrix}
    =i
    \begin{pmatrix}
        0 & 0 & 0 & 0\\
        0 & 0 & 0 & 0 \\
        0 & 0 & 0 & -i \\
        0 & 0 & i & 0
    \end{pmatrix}
    =iJ_1
\end{eqnarray}
\begin{eqnarray}
    [J_3,J_1]&=&
    \begin{pmatrix}
        0 & 0 & 0 & 0 \\
        0 & 0 & -i & 0 \\
        0 & i & 0 & 0 \\
        0 & 0 & 0 & 0 
    \end{pmatrix}
    \begin{pmatrix}
        0 & 0 & 0 & 0\\
        0 & 0 & 0 & 0 \\
        0 & 0 & 0 & -i \\
        0 & 0 & i & 0
    \end{pmatrix}
    -
    \begin{pmatrix}
        0 & 0 & 0 & 0\\
        0 & 0 & 0 & 0 \\
        0 & 0 & 0 & -i \\
        0 & 0 & i & 0
    \end{pmatrix}
    \begin{pmatrix}
        0 & 0 & 0 & 0 \\
        0 & 0 & -i & 0 \\
        0 & i & 0 & 0 \\
        0 & 0 & 0 & 0 
    \end{pmatrix}\\ 
    &=& 
    \begin{pmatrix}
        0 & 0 & 0 & 0 \\
        0 & 0 & 0 & -1 \\
        0 & 0 & 0 & 0 \\
        0 & 1 & 0 & 0 
    \end{pmatrix}
    =i
    \begin{pmatrix}
        0 & 0 & 0 & 0 \\
        0 & 0 & 0 & i \\
        0 & 0 & 0 & 0 \\
        0 & -i & 0 & 0 
    \end{pmatrix}
    =iJ_2
\end{eqnarray}
As as summary, we see 
\begin{eqnarray}
    [ J_j, J_k ] = i\epsilon_{jkl} J_l .
  \end{eqnarray}

\section{ }
\begin{eqnarray}
    &&[M_i,M_j]=[2J_i,2J_j]=4[J_i,J_j]=4i\epsilon_{ijk}J_k=2i\epsilon_{ijk}M_k \\
    &&[N_i,N_j]=[-J_i,J_j]=[J_i,J_j]=i\epsilon_{ijk}J_l=-i\epsilon_{ijk}N_k
\end{eqnarray}

\section{ }
With the commutation relations
\begin{eqnarray}
    &&[T^a,T^b]=if_{abl}T^l, \\
    &&[T^b,T^c]=if_{bcm}T^m, \\
    &&[T^c,T^a]=if_{can}T^n, 
\end{eqnarray}
we have
\begin{eqnarray}
  &&[T^a,[T^b,T^c]]= if_{bcm}[T^a,T^m]=-f_{bcm}f_{amp}T^p, \\
  &&[T^b,[T^c,T^a]]= if_{can}[T^b,T^n]=-f_{can}f_{bnq}T^q, \\
  &&[T^c,[T^a,T^b]]= if_{abl}[T^c,T^l]=-f_{abl}f_{clr}T^r.
\end{eqnarray}
Plug above relations into the Jacobi identity
\begin{eqnarray}
  [A,[B,C]] + [B,[C,A]] + [C,[A,B]] = 0,
\end{eqnarray}
we then see the
\begin{eqnarray}
    f_{bcm}f_{amp}T^p+f_{can}f_{bnq}T^q+f_{abl}f_{clr}T^r=0.
\end{eqnarray}
By choosing $p=q=r$, the above relation turns into
\begin{eqnarray}
    f_{bcm}f_{amp}+f_{can}f_{bnp}+f_{abl}f_{clp}=0.
\end{eqnarray}

Now let's use the adjoint reprensation
\begin{eqnarray}
    (T^a(A))_{bc} = (F^a)_{bc} = -i f_{abc},
\end{eqnarray}
to rewrite the commutation relation as
\begin{eqnarray}
    [T^a,T^b]_{lp}&=&(-if_{aln})(-if_{bnp})-(-if_{blm})(-if_{amp})=f_{blm}f_{amp}-f_{aln}f_{bnp} \\ 
    &=&(if_{abc})(-if_{clp})=f_{abc}f_{clp}=f_{abs}f_{slp}.
\end{eqnarray}
Choose $l=c$ in above equation, we have
\begin{eqnarray}
    f_{bcm}f_{amp}-f_{acn}f_{bnp}=f_{abs}f_{scp},
\end{eqnarray}
which is nothing but the one we derived from Jacobi identity if we change $s\to l$. Note we used $f_{acn}=-f_{can}$ and $f_{lcp}=-f_{clp}$.

\section{ }
\begin{eqnarray}
    [k_1,k_2]&=&
    \begin{pmatrix}
        0 & i & 0 & 0\\ 
        i & 0 & 0 & 0 \\
        0 & 0 & 0 & 0 \\
        0 & 0 & 0 & 0 
    \end{pmatrix}
    \begin{pmatrix}
        0 & 0 & i & 0\\ 
        0 & 0 & 0 & 0 \\
        i & 0 & 0 & 0 \\
        0 & 0 & 0 & 0
    \end{pmatrix}
    -
    \begin{pmatrix}
        0 & 0 & i & 0\\ 
        0 & 0 & 0 & 0 \\
        i & 0 & 0 & 0 \\
        0 & 0 & 0 & 0
    \end{pmatrix}
    \begin{pmatrix}
        0 & i & 0 & 0\\ 
        i & 0 & 0 & 0 \\
        0 & 0 & 0 & 0 \\
        0 & 0 & 0 & 0 
    \end{pmatrix}\\
    &=&
    \begin{pmatrix}
        0 & 0 & 0 & 0\\
        0 & 0 & -1 & 0\\
        0 & 1 & 0 & 0\\
        0 & 0 & 0 & 0
    \end{pmatrix}
    =-i
    \begin{pmatrix}
        0 & 0 & 0 & 0 \\
        0 & 0 & -i & 0 \\
        0 & i & 0 & 0 \\
        0 & 0 & 0 & 0 
    \end{pmatrix}
    =-iJ_3 
\end{eqnarray}
\begin{eqnarray}
    [k_2,k_3]&=&
    \begin{pmatrix}
        0 & 0 & i & 0\\ 
        0 & 0 & 0 & 0 \\
        i & 0 & 0 & 0 \\
        0 & 0 & 0 & 0
    \end{pmatrix}
    \begin{pmatrix}
        0 & 0 & 0 & i \\
        0 & 0 & 0 & 0 \\
        0 & 0 & 0 & 0 \\
        i & 0 & 0 & 0 
    \end{pmatrix}
    -
    \begin{pmatrix}
        0 & 0 & 0 & i \\
        0 & 0 & 0 & 0 \\
        0 & 0 & 0 & 0 \\
        i & 0 & 0 & 0 
    \end{pmatrix}
    \begin{pmatrix}
        0 & 0 & i & 0\\ 
        0 & 0 & 0 & 0 \\
        i & 0 & 0 & 0 \\
        0 & 0 & 0 & 0
    \end{pmatrix}\\
    &=& 
    \begin{pmatrix}
        0 & 0 & 0 & 0\\
        0 & 0 & 0 & 0 \\
        0 & 0 & 0 & -1 \\
        0 & 0 & 1 & 0
    \end{pmatrix}
    =-i
    \begin{pmatrix}
        0 & 0 & 0 & 0\\
        0 & 0 & 0 & 0 \\
        0 & 0 & 0 & -i \\
        0 & 0 & i & 0
    \end{pmatrix}
    =-iJ_1
\end{eqnarray}
\begin{eqnarray}
    [k_3,k_1]&=&
    \begin{pmatrix}
        0 & 0 & 0 & i \\
        0 & 0 & 0 & 0 \\
        0 & 0 & 0 & 0 \\
        i & 0 & 0 & 0 
    \end{pmatrix}
    \begin{pmatrix}
        0 & i & 0 & 0\\ 
        i & 0 & 0 & 0 \\
        0 & 0 & 0 & 0 \\
        0 & 0 & 0 & 0 
    \end{pmatrix}
    -
    \begin{pmatrix}
        0 & i & 0 & 0\\ 
        i & 0 & 0 & 0 \\
        0 & 0 & 0 & 0 \\
        0 & 0 & 0 & 0 
    \end{pmatrix}
    \begin{pmatrix}
        0 & 0 & 0 & i \\
        0 & 0 & 0 & 0 \\
        0 & 0 & 0 & 0 \\
        i & 0 & 0 & 0 
    \end{pmatrix} \\
    &=&
    \begin{pmatrix}
        0 & 0 & 0 & 0 \\
        0 & 0 & 0 & 1 \\
        0 & 0 & 0 & 0 \\
        0 & -1 & 0 & 0 
    \end{pmatrix}
    =-i
    \begin{pmatrix}
        0 & 0 & 0 & 0 \\
        0 & 0 & 0 & i \\
        0 & 0 & 0 & 0 \\
        0 & -i & 0 & 0 
    \end{pmatrix}
    =-iJ_2
\end{eqnarray}
\begin{eqnarray}
    [J_1,K_2]&=&
    \begin{pmatrix}
        0 & 0 & 0 & 0\\
        0 & 0 & 0 & 0 \\
        0 & 0 & 0 & -i \\
        0 & 0 & i & 0
    \end{pmatrix}
    \begin{pmatrix}
        0 & 0 & i & 0\\ 
        0 & 0 & 0 & 0 \\
        i & 0 & 0 & 0 \\
        0 & 0 & 0 & 0
    \end{pmatrix}
    -
    \begin{pmatrix}
        0 & 0 & i & 0\\ 
        0 & 0 & 0 & 0 \\
        i & 0 & 0 & 0 \\
        0 & 0 & 0 & 0
    \end{pmatrix}
    \begin{pmatrix}
        0 & 0 & 0 & 0\\
        0 & 0 & 0 & 0 \\
        0 & 0 & 0 & -i \\
        0 & 0 & i & 0
    \end{pmatrix} \\ 
    &=&
    \begin{pmatrix}
        0 & 0 & 0 & -1 \\
        0 & 0 & 0 & 0 \\
        0 & 0 & 0 & 0 \\
        -1 & 0 & 0 & 0 
    \end{pmatrix}
    =i
    \begin{pmatrix}
        0 & 0 & 0 & i \\
        0 & 0 & 0 & 0 \\
        0 & 0 & 0 & 0 \\
        i & 0 & 0 & 0 
    \end{pmatrix}
    =iK_3
\end{eqnarray}
\begin{eqnarray}
    [J_2,K_3]&=&
    \begin{pmatrix}
        0 & 0 & 0 & 0 \\
        0 & 0 & 0 & i \\
        0 & 0 & 0 & 0 \\
        0 & -i & 0 & 0 
    \end{pmatrix}
    \begin{pmatrix}
        0 & 0 & 0 & i \\
        0 & 0 & 0 & 0 \\
        0 & 0 & 0 & 0 \\
        i & 0 & 0 & 0 
    \end{pmatrix}
    -
    \begin{pmatrix}
        0 & 0 & 0 & i \\
        0 & 0 & 0 & 0 \\
        0 & 0 & 0 & 0 \\
        i & 0 & 0 & 0 
    \end{pmatrix}
    \begin{pmatrix}
        0 & 0 & 0 & 0 \\
        0 & 0 & 0 & i \\
        0 & 0 & 0 & 0 \\
        0 & -i & 0 & 0 
    \end{pmatrix} \\
    &=&
    \begin{pmatrix}
        0 & -1 & 0 & 0\\ 
        -1 & 0 & 0 & 0 \\
        0 & 0 & 0 & 0 \\
        0 & 0 & 0 & 0 
    \end{pmatrix}
    =i
    \begin{pmatrix}
        0 & i & 0 & 0\\ 
        i & 0 & 0 & 0 \\
        0 & 0 & 0 & 0 \\
        0 & 0 & 0 & 0 
    \end{pmatrix}
    =iK_1
\end{eqnarray}
\begin{eqnarray}
    [J_3,K_1]&=&
    \begin{pmatrix}
        0 & 0 & 0 & 0 \\
        0 & 0 & -i & 0 \\
        0 & i & 0 & 0 \\
        0 & 0 & 0 & 0 
    \end{pmatrix}
    \begin{pmatrix}
        0 & i & 0 & 0\\ 
        i & 0 & 0 & 0 \\
        0 & 0 & 0 & 0 \\
        0 & 0 & 0 & 0 
    \end{pmatrix}
    -
    \begin{pmatrix}
        0 & i & 0 & 0\\ 
        i & 0 & 0 & 0 \\
        0 & 0 & 0 & 0 \\
        0 & 0 & 0 & 0 
    \end{pmatrix}
    \begin{pmatrix}
        0 & 0 & 0 & 0 \\
        0 & 0 & -i & 0 \\
        0 & i & 0 & 0 \\
        0 & 0 & 0 & 0 
    \end{pmatrix} \\ 
    &=&
    \begin{pmatrix}
        0 & 0 & -1 & 0\\ 
        0 & 0 & 0 & 0 \\
        -1 & 0 & 0 & 0 \\
        0 & 0 & 0 & 0
    \end{pmatrix}
    =i
    \begin{pmatrix}
        0 & 0 & i & 0\\ 
        0 & 0 & 0 & 0 \\
        i & 0 & 0 & 0 \\
        0 & 0 & 0 & 0
    \end{pmatrix}
    =iK_2
\end{eqnarray}
We then summarize
\begin{eqnarray}
    &&[ K_l, K_m ] = -i\epsilon_{lmn} J_n, \\
    &&[ J_l, K_m ] =  i\epsilon_{lmn} K_n.
\end{eqnarray}

\section{ }
When $y_1=y_2=0$, $y_3=y$, we write
\begin{eqnarray}
    &&L(y_3 = y)
    = e^{ i K_3 y }= \exp
    {\begin{pmatrix}
        0 & 0 & 0 & -y \\
        0 & 0 & 0 & 0 \\
        0 & 0 & 0 & 0 \\
        -y & 0 & 0 & 0 
    \end{pmatrix}}\\
    &=&
    1+
    \begin{pmatrix}
        0 & 0 & 0 & -y \\
        0 & 0 & 0 & 0 \\
        0 & 0 & 0 & 0 \\
        -y & 0 & 0 & 0 
    \end{pmatrix}
    +\frac{1}{2!}\begin{pmatrix}
        0 & 0 & 0 & -y \\
        0 & 0 & 0 & 0 \\
        0 & 0 & 0 & 0 \\
        -y & 0 & 0 & 0 
    \end{pmatrix}^2
    +\frac{1}{3!}\begin{pmatrix}
        0 & 0 & 0 & -y \\
        0 & 0 & 0 & 0 \\
        0 & 0 & 0 & 0 \\
        -y & 0 & 0 & 0 
    \end{pmatrix}^3
    +\cdots \\
    &=&
    \begin{pmatrix}
        \cosh y & 0 & 0 & -\sinh y \\
        0 & 1 & 0 & 0 \\
        0 & 0 & 1 & 0 \\
        -\sinh y & 0 & 0 & \cosh y
    \end{pmatrix}
  \end{eqnarray}

\section{ }
Compare
\begin{eqnarray}
    \begin{pmatrix}
      t' \\ x' \\ y' \\ z'
    \end{pmatrix}
    = L(y_3=y)
    \begin{pmatrix}
      t \\ x \\ y \\ z
    \end{pmatrix}
    =
    \begin{pmatrix}
      t\cosh y -z\sinh y \\ x \\ y \\ z\cosh y -t\sinh y
    \end{pmatrix},
  \end{eqnarray}
with
\begin{eqnarray}
    \begin{pmatrix}
      t' \\ x' \\ y' \\ z'
    \end{pmatrix}
    =
    \begin{pmatrix}
     \gamma (t-\beta z) \\ x \\ y \\ \gamma (z-\beta t)
    \end{pmatrix},
\end{eqnarray}
so we have
\begin{eqnarray}
    \cosh y=\gamma, ~~~\sinh y = \gamma \beta.
\end{eqnarray}
The result also shows that $\beta -\tanh y $.

\section{ }
Since
\begin{eqnarray}
    [J_3,J_3]=[K_3,k_3]=0,
\end{eqnarray}
we can write
\begin{eqnarray}
    &&e^{ i\phi_a J_3 } e^{ i\phi_b J_3 }
   = e^{ i(\phi_a+\phi_b) J_3 },\\
    &&e^{ i a K_3 } e^{ i b K_3 }
  = e^{ i (a+b) K_3 }.
\end{eqnarray}
This would lead to
\begin{eqnarray}
    L(\theta_3=\phi_a) L(\theta_3=\phi_b)
     = e^{ i\phi_a J_3 } e^{ i\phi_b J_3 }
     = e^{ i(\phi_a+\phi_b) J_3 }
     = L(\theta_3 = \phi_a + \phi_b ),
  \end{eqnarray}
   and
  \begin{eqnarray}
    L(y_3 = a) L(y_3 = b)
    = e^{ i a K_3 } e^{ i b K_3 }
    = e^{ i (a+b) K_3 }
    = L(y_3 = a+b ) .
  \end{eqnarray}

  \section{ }
  Apply Lorentz transformation on
  \begin{eqnarray}
    p^\mu &=& 
    \begin{pmatrix}
      m \\ 0 \\ 0 \\ 0
    \end{pmatrix},
  \end{eqnarray}
one have
  \begin{eqnarray}
    {p'}^\mu = B_z(y) p^\mu =
    \begin{pmatrix}
        \cosh y & 0 & 0 & \sinh y \\
        0 & 1 & 0 & 0 \\
        0 & 0 & 1 & 0 \\
        \sinh y & 0 & 0 & \cosh y
    \end{pmatrix}
    \begin{pmatrix}
        m \\ 0 \\ 0 \\ 0
      \end{pmatrix}
    =
    \begin{pmatrix}
      m \cosh y \\ 0 \\ 0 \\ m \sinh y 
    \end{pmatrix}
    =
    \begin{pmatrix}
     E \\ 0 \\ 0 \\ p
    \end{pmatrix},
  \end{eqnarray}

\begin{eqnarray}
    {p{''}}^\mu = R_y(\theta)B_z(y) p^\mu =
    \begin{pmatrix}
        1 & 0 & 0 & 0 \\
        0 &\cos\theta & 0 &\sin\theta \\
        0 & 0 & 1 & 0\\
        0 &-\sin\theta & 0 & \cos\theta
    \end{pmatrix}
    \begin{pmatrix}
        E \\ 0 \\ 0 \\ p
    \end{pmatrix} 
    =
    \begin{pmatrix}
      E \\ p \sin\theta \\ 0 \\ p\cos\theta
    \end{pmatrix} 
\end{eqnarray}
\begin{eqnarray}
    {p{'''}}^\mu = R_z(\phi)R_y(\theta)B_z(y) p^\mu =
    \begin{pmatrix}
     1 & 0 & 0 & 0\\
     0 & \cos\phi & -\sin\phi & 0 \\
     0 & \sin\phi & \cos\phi & 0 \\
     0 & 0 & 0 & 1
    \end{pmatrix}
    \begin{pmatrix}
        E \\ p \sin\theta \\ 0 \\ p\cos\theta
      \end{pmatrix} 
      =
    \begin{pmatrix}
      E \\ p\sin\theta\cos\phi \\ p\sin\theta\sin\phi \\ p\cos\theta
    \end{pmatrix}
\end{eqnarray}

\end{document}
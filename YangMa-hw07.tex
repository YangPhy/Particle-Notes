\documentclass[11pt]{article}
\usepackage{amsmath,graphicx,color,epsfig,physics}
%\usepackage{pstricks}
\usepackage{float}
\usepackage{subfigure}
\usepackage{slashed}
\usepackage{color}
\usepackage{multirow}
\usepackage{feynmp}
\usepackage[top=1in, bottom=1in, left=1.2in, right=1.2in]{geometry}
\def\del{{\partial}}
\begin{document}
\title{Particle physics HW7}
\author{Yang Ma}

\maketitle

\section{ }
For an $SU(2)_L \times U(1)_Y$ rotation:
  \begin{eqnarray}
    \phi \to \phi' = U(\theta_2) U(\theta_3) U(\theta_0) \phi \label{eq.721ro}
  \end{eqnarray}
   where $U(\theta_2) = U(0,\theta_2,0,0)$, $U(\theta_3) = U(0,0,\theta_3,0)$, $U(\theta_0) = U(0,0,0,\theta_0)$, with the notation
\begin{eqnarray}
  U(\theta_1,\theta_2,\theta_3,\theta_0)
  = e^{ i \sum_k \theta_k T^k }
\end{eqnarray}
 where $T^k = \sigma^k/2 (k=1,2,3)$ and $T^0 = {\rm diag}\{1,1\}/2$, we can calculate it step by step.
 Now we take
 \begin{eqnarray}
    <\phi_u> &=& \frac{v}{\sqrt 2} \sin\theta e^{i\alpha}, \\
    <\phi_d> &=& \frac{v}{\sqrt 2} \cos\theta e^{i\beta} 
  \end{eqnarray}
     and 
  \begin{eqnarray}
    \theta_0 &=& -(\alpha+\beta), \\
    \theta_3 &=& -(\alpha-\beta), \\
    \theta_2 &=& -2\theta,
  \end{eqnarray}
  with $v>0$, and then have 
\begin{eqnarray}
    U(\theta_0)
    \begin{pmatrix}
      <\phi_u> \\ <\phi_d>
    \end{pmatrix}&=&
    \begin{pmatrix}
      e^{-i(\alpha+\beta)/2} & 0 \\0 &   e^{-i(\alpha+\beta)/2} 
    \end{pmatrix}
    \begin{pmatrix}
        \sin\theta v e^{i \alpha}/\sqrt 2 \\ \cos\theta v e^{i \beta}/\sqrt 2 
    \end{pmatrix} \\
      &=&
      \begin{pmatrix}
        v \sin\theta e^{i(\alpha-\beta)/2} /\sqrt 2 \\
        v \cos\theta e^{i(\beta-\alpha)/2} /\sqrt 2
      \end{pmatrix}
\end{eqnarray}

\begin{eqnarray}
    U(\theta_3)U(\theta_0)
    \begin{pmatrix}
      <\phi_u> \\ <\phi_d>
    \end{pmatrix}&=&
    \begin{pmatrix}
        e^{-i(\alpha-\beta)/2} & 0 \\0 &   e^{i(\alpha-\beta)/2} 
    \end{pmatrix}         
    \begin{pmatrix}
        v \sin\theta e^{i(\alpha-\beta)/2} /\sqrt 2 \\
        v \cos\theta e^{i(\beta-\alpha)/2} /\sqrt 2
    \end{pmatrix}\\
    &=&
    \begin{pmatrix}
        \frac{\sin\theta v}{\sqrt 2} \\
        \frac{\cos\theta v}{\sqrt 2}
    \end{pmatrix}
\end{eqnarray}

\begin{eqnarray}
    U(\theta_2) U(\theta_3)U(\theta_0)
    \begin{pmatrix}
      <\phi_u> \\ <\phi_d>
    \end{pmatrix}
    &=&
    \begin{pmatrix}
        \cos \theta & -\sin \theta \\ \sin \theta  & \cos \theta  
    \end{pmatrix}
    \begin{pmatrix}
        \frac{\sin\theta v}{\sqrt 2} \\
        \frac{\cos\theta v}{\sqrt 2}
    \end{pmatrix}\\
    &=&
    \begin{pmatrix}
      0 \\ \frac{v}{\sqrt 2}
    \end{pmatrix}.
\end{eqnarray}

\section{ }
For the Higgs potential
\begin{eqnarray}
  V(\phi) = \frac{\lambda}{4} (\phi^\dagger \phi)^2
  + \mu^2 (\phi^\dagger \phi),
\end{eqnarray}
we have
\begin{eqnarray}
  \frac{\del V(\phi)}{\del \phi} = \frac{\lambda}{2}(\phi^\dagger \phi) \phi^\dagger + \mu^2 \phi^\dagger = 0
\end{eqnarray}
when $\phi=<\phi>$. Then we see $<\psi>$ should satisfy
\begin{eqnarray}
  [\frac{\lambda}{2} (<\phi>^\dagger <\phi>) + \mu^2] <\phi>^\dagger = 0,
\end{eqnarray}
which implies
\begin{enumerate}
  \item If $\mu^2 > 0$, i.e. $\frac{\lambda}{2} (<\phi>^\dagger <\phi>) + \mu^2>0$, we have to let $<\phi>^\dagger = <\phi> = 0$.
  \item If $\mu^2 < 0$, then apart from the solution $<\phi> = 0$, we can let
  $\frac{\lambda}{2} (<\phi>^\dagger <\phi>) + \mu^2=0$, i.e. $<\phi>^\dagger <\phi> = -2\mu^2/\lambda$ be another solution.
\end{enumerate}

\section{ }
Since we know that
\begin{eqnarray}
  Y \psi = Y
  \begin{pmatrix}
    \psi_u \psi_d
  \end{pmatrix}
  = \frac{1}{2}
  \begin{pmatrix}
    \psi_u \psi_d
  \end{pmatrix},
\end{eqnarray}
and
\begin{eqnarray}
  T^3 
  \begin{pmatrix}
    0 \\ \frac{v}{\sqrt 2}
  \end{pmatrix}
  &=& \frac{1}{2}
  \begin{pmatrix}
    1 & 0\\0 &-1
  \end{pmatrix}
  \begin{pmatrix}
    0 \\ \frac{v}{\sqrt 2}
  \end{pmatrix} \\
  &=& - \frac{1}{2}
  \begin{pmatrix}
    0 \\ \frac{v}{\sqrt 2}
  \end{pmatrix},
\end{eqnarray}
we obtain
\begin{eqnarray}
  Q <\psi> = (T^3+Y) <\psi> = (\frac{1}{2}-\frac{1}{2})<\psi>=0.
\end{eqnarray}
As a result,
\begin{eqnarray}
  U_{EM}(\theta) <\psi> = e^{iQ\theta}<\psi> = <\psi>,
\end{eqnarray}
where we used $e^{A}\psi=a \psi$ if $a$ is the eigenvalue of operator $A$ acting on $\psi$.

\section{ }
By definition, it is straightforward to write out that
\begin{eqnarray}
  D_\mu <\phi>
  &=& D_\mu
  \begin{pmatrix}
    0 \\ \frac{v}{\sqrt 2}
  \end{pmatrix}
  =(\del_\mu + i g T^k W^k_\mu + i g' Y B_\mu)
  \begin{pmatrix}
    0 \\ \frac{v}{\sqrt 2}
  \end{pmatrix} \\ 
  &=&i( g T^k W^k_\mu + g' Y B_\mu)
  \begin{pmatrix}
    0 \\ \frac{v}{\sqrt 2},
  \end{pmatrix} 
\end{eqnarray}
where we used $\del_\mu <\psi> =0$ since $v$ is a constant.
Then we continue to write
\begin{eqnarray}
  D_\mu <\phi>
    &=&i[g(T^1W^1_\mu+T^2W^2_\mu) + (gT^3W^3_\mu+g'YB_\mu)]
  \begin{pmatrix}
    0 \\ \frac{v}{\sqrt 2}
  \end{pmatrix} \\
  &=&i[(g/\sqrt2)(T^+W^+_\mu+T^-W^-_\mu)+(1/2)(gW^3_\mu \sigma^3 +g'B_\mu)]
  \begin{pmatrix}
    0 \\ \frac{v}{\sqrt 2},
  \end{pmatrix}
\end{eqnarray}
by setting
\begin{eqnarray}
  T^+&=&T^1+iT^2, \\
  T^-&=&T^1-iT^2,\\
  W^+_\mu &=& (W^1_\mu - iW^2_\mu)/\sqrt2, \\ 
  W^-_\mu &=& (W^1_\mu + iW^2_\mu)/\sqrt2,
\end{eqnarray}
and writting $T^1 W^1_\mu + T^2 W^2_\mu$ in terms of $T^+$, $T^-$, $W^+_\mu$ and $W^-_\mu$,
\begin{eqnarray}
  T^1 W^1_\mu + T^2 W^2_\mu
  &=& (T^1+iT^2)(W^1_\mu-iW^2_\mu)/2 + (T^1-iT^2)(W^1_\mu+iW^2_\mu)/2 \nonumber \\
  &=& T^+(W^1_\mu-iW^2_\mu)/2 + T^-(W^1_\mu+iW^2_\mu)/2  \nonumber \\
  &=& (T^+ W^+_\mu)/\sqrt2 + (T^- W^-_\mu)/\sqrt2.
\end{eqnarray}
Finally, we have
\begin{eqnarray}
  D_\mu <\phi>
  &=&i[(g/\sqrt2)(T^+W^+_\mu+T^-W^-_\mu)+(1/2)(gW^3_\mu \sigma^3+g'B_\mu)]
  \begin{pmatrix}
    0 \\ \frac{v}{\sqrt 2}
  \end{pmatrix} \\
  &=&\frac{v}{\sqrt 2}
  \begin{pmatrix}
    \frac{g}{\sqrt 2} W^+_\mu \\ (-gW^3_\mu+g'B_\mu)/2
  \end{pmatrix} 
   =\frac{iv}{\sqrt 2} 
   \begin{pmatrix}
    \frac{g}{\sqrt 2} W^+_\mu \\ -(g_z/2) Z_\mu
  \end{pmatrix}\\
   &=& i
   \begin{pmatrix}
    \frac{gv}{2} W^+_\mu \\ -\frac{g_zv}{2\sqrt2} Z_\mu
   \end{pmatrix},
\end{eqnarray}
where
\begin{eqnarray}
  Z_\mu &=& (g W^3_\mu - g' B_\mu)/\sqrt{ g^2 + g'^2 } \\
  g_z &=& \sqrt{ g^2 + g'^2 }.
\end{eqnarray}

\section{ }
With above expression for $D_\mu <\psi>$, we write
\begin{eqnarray}
  {\cal L}_{Higgs}&=&(D_\mu <\psi>)^\dagger (D^\mu <\psi>) \\
  &=& (-i)
  \begin{pmatrix}
    \frac{gv}{2}W^-_\mu & -\frac{g_z v}{2 \sqrt 2}Z_\mu
  \end{pmatrix}
  (i)
  \begin{pmatrix}
    \frac{gv}{2}W^{+\mu} \\ -\frac{g_z v}{2 \sqrt 2}Z^\mu
  \end{pmatrix} \\
  &=&(\frac{gv}{2})^2 W^-_\mu W^{+\mu}+ (\frac{g_zv}{2})^2Z_\mu Z^\mu/2 \\
  &=&m_W^2 W^-_\mu W^{+ \mu} + m_Z^2 Z_\mu Z^\mu/2,
\end{eqnarray}
by introducing
\begin{eqnarray}
  m_W = g v/2 \\
  m_Z = g_z v/2.
\end{eqnarray}

\section{ }
First we write out $D_\mu$ as
\begin{eqnarray}
  D_\mu
  &=&\del_\mu +i g T^k W^k_\mu + i g' Y B_\mu
  =\del_\mu +ig(T^1W^1_\mu+T^2W^2_\mu+T^3W^3_\mu) +ig'YB_\mu \nonumber \\
  &=&\del_\mu +i(g/\sqrt2)(T^+W^+_\mu + T^-W^-_\mu)
            +igT^3 W^3_\mu +ig'Y  B_\mu.
\end{eqnarray}
Then we need to consider the Weinberg rotation
\begin{eqnarray}
  \begin{pmatrix}
    W^3_\mu \\  B_\mu
  \end{pmatrix}
  =
  \begin{pmatrix}
    \cos\theta_W & \sin\theta_W \\
    -\sin\theta_W & \cos\theta_W
  \end{pmatrix}
  \begin{pmatrix}
    Z_\mu \\ A_\mu
  \end{pmatrix},
\end{eqnarray}
and have $D_\mu$ as
\begin{eqnarray}
  D_\mu
   &=&\del_\mu +i(g/\sqrt2)(T^+W^+_\mu + T^-W^-_\mu)
            +igT^3 ( \cos\theta_W Z_\mu + \sin\theta_W A_\mu) \nonumber \\
           &&+ig'Y  (-\sin\theta_W Z_\mu + \cos\theta_W A_\mu)\nonumber \\
  &=&\del_\mu +i(g/\sqrt2)(T^+W^+_\mu + T^-W^-_\mu)
            +i(g\cos\theta_W T^3 -g'\sin\theta_W Y) Z_\mu\nonumber \\
            &&+i(g\sin\theta_W T^3 +g'\cos\theta_W Y) A_\mu.
\end{eqnarray}

Now we introduce
\begin{eqnarray}
  e &=& g\sin\theta_W = g'\cos\theta_W \\
  g_Z&=&g/\cos\theta_W= g'\sin\theta_W 
\end{eqnarray}
and hence have
\begin{eqnarray}
  D_\mu
  &=&\del_\mu +i(g/\sqrt2)(T^+W^+_\mu + T^-W^-_\mu)
            +ig_Z(\cos^2\theta_W T^3 -\sin^2\theta_W Y) Z_\mu
            \nonumber \\ &&+ie(T^3 + Y) A_\mu \nonumber \\
  &=&\del_\mu +i(g/\sqrt2)(T^+W^+_\mu + T^-W^-_\mu)
            +ig_Z(T^3 -\sin^2\theta_W Q) Z_\mu
            +ieQ A_\mu,
\end{eqnarray}
where we used 
\begin{eqnarray}
  Q = T^3 + Y,
\end{eqnarray}
in the last step.



\end{document}
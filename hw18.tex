\documentclass[12pt]{article}
\usepackage{amsmath,graphicx,color,epsfig,physics}
\usepackage{float}
\usepackage{subfigure}
\usepackage{slashed}
\usepackage{color}
\usepackage{multirow}
\usepackage{feynmp}
\textheight=9.5in \voffset=-1.0in \textwidth=6.5in \hoffset=-0.5in
\parskip=0pt
\def\del{{\partial}}
\def\dgr{\dagger}
\def\eps{\epsilon}
\def\lmd{\lambda}
\def\th{\theta}
\begin{document}

\begin{center}
{\large\bf HW18 for Advanced Particle Physics} \\

\end{center}

\vskip 0.2 in

Dear students,\\
This week I introduce the perturbative expansion of
the scattering matrix in quantum field theory (QFT), and try to
compute simple transition matrix elements in the Standard Model
that appear in the very first expansion.
\begin{eqnarray}
    \bra{f}S\ket{i}&=& S_{fi} = i T_{fi}=\bra{f}T e^{i\int d^4x {\cal L}_{int}(x)} \ket{i} \\ 
    &=& \bra{f}T \sum_{n=0}^{n=\infty}\frac{1}{n!} (i\int d^4x {\cal L}_{int}(x))^n \ket{i} \\
    &=& \bra{f}i\int d^4x {\cal L}_{int}(x) \ket{i} +h.o. \label{eq.18_1}
\end{eqnarray}
Since only one term appears, the T-ordering does nothing.

The SM contains many terms in ${\cal L}_{int}(x)$, which can be written
as transitions among
\begin{itemize}
    \item $V-f-f$: $W$,$Z$,$\gamma$, gluon and quarks and leptons
    \item $H-f-f$: Higgs boson and quarks and leptons
    \item $V-V-V$ and $V-V-V-V$: gauge boson self-interactions
    \item $H-V-V$ and $H-H-V-V$: Higgs and gauge bosons
    \item $H-H-H$ and $H-H-H-H$: Higgs boson self-interactions
\end{itemize}

{\bf hw18-1}: Can you identify all the terms in the SM Lagrangian that
contribute to the above interactions?

The above interactions have mass dimension 4 or 3 (H-V-V and H-H-H
coupling terms have mass dimension 3, while all the rest have mass
dimension 4), and they are renormalizable interactions.  You will
study the renormalization of high-frequency (high-energy, or
ultra-violet) quantum fluctuations in perturbative QFT in advanced
courses.  Here, it suffices to tell, that renormalizability of QFT
has been shown in theories of spin $0$, $1/2$, $1$ fields, whose interaction
terms have mass dimension $4$ or less, and when all spin $1$ particles
are gauge bosons. Please note that the SM satisfies all the above
conditions, except for the neutrino mass terms and the neutrino
interactions with the Higgs bosons: see {\bf hw18-2}.

Because the existence of graviton (quantum mechanics of gravity) is
almost certain, the condition of renormalizability in QFT may sound
too strong for you. The absence of renormalizability means that the
theory can only be an effective one, which may apply only at a given
energy scale (such as at energy scale much smaller than the gravity
scale, or the Planck mass, $10^{19}$ GeV). Such theory cannot have
predictive power beyond its validity region, and in addition,
whenever the quantum fluctuation contains sensitivity to high
momentum scale, we cannot calculate quantum corrections.

It is possible to define a non-renormalizable QFT at a given energy
scale, excluding even quantum fluctuations above the scale.  It is
then often possible to study the scale dependence of such effective
theory (examples include QCD radiative corrections to Fermi-theory
of weak interactions, such as the strengths and mixing among various
operators of order $G_F$ between the weak boson mass scale and the energy
scale of $B/D/K$ mesons, effective theory of pion interactions in models
with spontaneously broken chiral symmetry, called chiral perturbation
theories, etc). In my understanding, the renormalizable QFT differs
from those effective theories only by the condition that there is no
a priori frequency/energy upper bound on our ability to evaluate
quantum fluctuations. Within the validity range, radiative corrections
of effective theories are identical to those of renormalizable theories.

Because of this, I do not regard ``renormalizability'' as a guiding
principle for introducing new interactions.  This is why I prefer
dimension 5 operators to describe the neutrino mass in the SM.
If we introduce two or three new particles, called right-handed
neutrino's, we can give Dirac masses to neutrino's, and the theory
is renormalizable.

Historically, theory of massive spin 1 particles were believed to be
non-renormalizable.  The discovery of the renormalizability of
spontaneously broken gauge theory (1971 by t'Hooft) made SM a consistent
QFT that has powerful predictive powers, once all UV sensitive terms
(the terms that appear in our Lagrangian!) are fixed (renormalized)
by measurements.  This is made possible, because quantum fluctuations
satisfy the gauge symmetry of the theory, and hence the pattern of
UV sensitive fluctuations are limited to finite forms.  Once these
finite number of UV sensitive fluctuations are fixed by observations,
all the remaining quantum fluctuations give finite contributions
(quantum or radiative corrections) to arbitrary scattering matrix
elements, which can be tested experimentally.

Physics of quantum correction in QFT and that of its renormalizability
are hence the established basis of our understanding of the laws of
nature.  Our pursuit for physics beyond QFT can only start from its
deep understanding.  Let us start studying it.

In my SM Lagrangian, I introduced one term whose mass dimension is 5,
the so-called Majorana mass term for neutrinos:
\begin{eqnarray}
    {\cal L}_{\nu~mass}
    = -\frac{y^{\nu}_{ij}}{2\Lambda}
    ({\phi^c}^\dagger L) \cdot ({\phi^c}^\dagger L) \label{eq.18_3}
\end{eqnarray}
where you now know that $\cdot$ means the Lorentz invariant interaction
of two left-handed fermions,
\begin{eqnarray}
    \psi_L \cdot \psi_L
= {\psi_L^c}^\dagger \psi_L
= {\psi_L}^T (i\sigma^2) \psi_L\label{eq.18_4}
\end{eqnarray}
in the spinor space. You can write the above term as
\begin{eqnarray}
    {\cal L}_{\nu~mass}
= -\frac{y^{\nu}_{ij}}{2\Lambda}
(\phi^T i\sigma^2 L)^T i\sigma^2 (\phi^T i\sigma^2 L)\label{eq.18_5}
\end{eqnarray}
if you are sure that you know what each $i\sigma^2$ means.  (You should
distinguish a matrix in the gauge $SU(2)_L$ space and a matrix in the
spinor space that governs the Lorentz transformations).

In any case, this term has mass dimension $5$, which may be clear from
our introduction of an arbitrary mass scale $1/\Lambda$. It clearly
has interactions like 
\begin{eqnarray}
    H-f-f~ and ~H-H-f-f~ (f=l~ or ~\nu) \label{eq.18_6}
\end{eqnarray}
whose mass dimensions are $4$ and $5$, respectively, and hence our QFT
model is non-renormalizable. Since the neutrino mass is very small,
\begin{eqnarray}
    M(\nu)_{ij} = \frac{y^{\nu}_{ij} v^2}{2 \Lambda} \label{eq.18_7}
\end{eqnarray}
the interactions like Eq.\ref{eq.18_6} are negligibly small in the SM.

{\bf hw18-2}: Please express Eq.\ref{eq.18_3} or Eq.\ref{eq.18_5} in terms of component fields,
\begin{eqnarray}
    && L = (\nu_l, l_L)^T \label{eq.18_8a}\\
    && \phi = (0, (v+H)/\sqrt{2})^T \label{eq.18_8b}
\end{eqnarray}
and identify the neutrino mass term, $H-f-f$ interaction terms, and
$H-H-f-f$ interaction terms. Please distinguish between $f = \nu$ and $f=l$.
We should always remember that Fermi's theory of weak interactions
are 4-Fermi interactions,
$f-f-f-f$
whose mass dimension is $6$, and hence are non-renormalizable.  They have
common (the universality was found by Cabibbo) parameter $G_F$ (whose mass
dimension is $-2$, in order to compensate for the dimension 6 interactions
of $4$ fermions). It was originally written as
\begin{eqnarray}
    {\cal L}_{Fermi}
= G_F ({\overline \Psi} \gamma^\mu \Psi) ({\overline \Psi} \gamma_\mu \Psi) \label{eq.18_9}
\end{eqnarray}
assuming the vector current form of a pair of fermion, just like the
electromagnetic current that couple to the photon in QED.

We will find in the following lectures that the weak interactions
of the SM reduces to the Fermi theory at low energies (energies much
smaller than the W mass), with appropriate modifications (only
left-handed components of fermions have charged weak interactions):
\begin{eqnarray}
    \frac{G_F}{\sqrt{2}}=\left(\frac{1}{2}\right)^2 \left(\frac{g_W}{{\sqrt 2}m_W}\right)^2 =\frac{1}{8} \left(\frac{g_W}{m_W}\right)^2 =\frac{1}{8} \left(\frac{2}{v}\right)^2 =\frac{1}{2v^2} \label{eq.18_10}
\end{eqnarray}
We will learn the numerical factors above in the next homework.

Although Fermi's theory of weak interactions are non-renormalizable,
and hence we cannot predict what happens at high energies, it gives
accurate description of all the charged current weak interactions at low
energies (energies below the W mass scale), and the observed
universality of weak interactions (say, the universality between the
charged weak currents of hadrons and those of leptons was discovered
by Cabibbo in 1962) led us to discover the SM of the electroweak
gauge theory in 1967, even before the discovery of the weak bosons.

I therefore do not hesitate including non-renormalizable terms in the
SM Lagrangian.  We should test if they reproduce the data accurately
(most crucially, the neutrino-less double beta decay experiments
should demonstrate fermion number non-conservation).

By using the first order of the S-matrix expansion, we can compute all
the transition matrix elements, exactly the same way.  You can try out all of them, but I give just $4$ exercises
below, which are related to my questions in the very first two homeworks
(why the width/mass ratio of top, $Z$ and $W$ are similar?).
\begin{eqnarray}
    W^- \to l + {\overline \nu} \label{eq.18_11a}\\
    Z   \to  f + {\overline f} \label{eq.18_11b} \\
    H   \to  \tau + {\overline \tau} \label{eq.18_11c}\\
    t   \to  b + W^+ \label{eq.18_11d}
\end{eqnarray}
Last week, we examined (\ref{eq.18_11c}), and found the amplitudes in terms of
Dirac 4-spinor wave functions, $u(p_\tau,h_\tau)$ and $v(p_{\overline \tau},h_{\overline \tau})$.

This week, let us calculate the amplitudes for (\ref{eq.18_11a}), in the massless lepton limits. The wave functions are then very simple 2-spinors,
and we can calculate them easily.

\subsection{$W^- \to l + {\overline \nu}_l (l=e,\mu,\tau)$}
Please first recall the covariant derivative of the SM:
\begin{eqnarray}
    D_\mu = \del_\mu
    + i g_s           T^a A^a_\mu
    + i \frac{g_W}{\sqrt 2} [T^+ W^+_\mu  + T^- W^-_\mu]
    + i g_Z [ T^3 - Q \sin^2\theta_W ] Z_\mu
    + i e Q A_\mu \label{eq.18_12}
\end{eqnarray}
after the $SU(2)_L \times U(1)_Y$ breaks down to $U(1)_{EM}$.  Here $A^a_\mu$
($a=1$ to $8$) are gluons, $T^a$ are $3\times3$ generators of the $SU(3)_c$ group, $T^\pm = T^1 \pm iT^2$ and $T^3$ are $2\times 2$ generators of the $SU(2)_L$ group, $A_\mu$ is the photon field and $Q=T^3+Y$ is the EM charge.

For simplicity, let us take the EW gauge-boson contributions only.
The relevant interaction Lagrangian can be expressed as
\begin{eqnarray}
   && {\cal L}_{int}(x)
    = {\overline \Psi}(x)_{f'} i(D_\mu -\del_\mu) \gamma^\mu \Psi(x)_f \\ 
    &=& -\frac{g_W}{\sqrt 2} W^+_\mu(x) u_L(x)^\dgr \sigma_-^\mu d_L(x)
    -\frac{g_W}{\sqrt 2} W^-_\mu(x)  d_L(x)^\dgr \sigma_-^\mu u_L(x)
    \nonumber \\&&-g_Z Z_\mu(x) [(T^3_f-Q_f\sin^2\theta_w) f_L(x)^\dgr \sigma_-^\mu f_L(x)  + (-Q_f\sin^2\theta_w) f_R(x)^\dgr \sigma_+^\mu f_R(x) ]
    \nonumber \\ &&  -e A_\mu(x)  Q_f   [ f_L(x)^\dgr \sigma_-^\mu f_L(x)
                                  +f_R(x)^\dgr \sigma_+^\mu f_R(x) ]\\
    &=& V_\mu(x)     [  -g_L^{Vff'} f_L(x)^\dgr \sigma_-^\mu f'_L(x)
                    -g_R^{Vff'} f_R(x)^\dgr \sigma_+^\mu f'_R(x) ] \label{eq.18_13}
\end{eqnarray}
Here, I wrote first the definition of the interaction terms, their
explicit form for $W^\pm$, $Z$ and photon ($A$), and then a generic
expression which is convenient for calculating the transition
matrix elements (Feynman diagrams).

The abbreviations are:
% \begin{eqnarray}
%     e = gW \sw    = gW \sin\theta_W
%     = gZ \cw\sw = gZ \cos\theta_W \sin\theta_W
% \sw2 = \sin^2\theta_W
%     = e^2/(gW)^2
% \end{eqnarray}
% 14) 
\begin{eqnarray}
    e   = g_W \sin\theta_W = g_Z \cos\theta_W \sin\theta_W \label{eq.18_14}
\end{eqnarray}  
for the EW couplings, and
\begin{eqnarray}
    &&f_L(x) = \psi_L^f(x) \label{eq.18_15a} \\
    &&f_R(x) = \psi_R^f(x) \label{eq.18_15b}
\end{eqnarray}
for $f = u, d, \nu_e, e$, for the first generation,
or  $f = u,c,t, d,s,b, e,\mu,\tau, \nu_e,\nu_\mu,\nu_\tau$ (or $\nu_1,\nu_2,\nu_3$)
for 3 generations.

With this notation, all the EW couplings of the SM can be made into
a table (which you typically find in any matrix element calculators,
such as MadGraph):
\begin{eqnarray}
g_L^{Wud} = \frac{g_W}{\sqrt 2} V(CKM)_{ud},~~~
g_L^{Wnl} = \frac{g_W}{\sqrt 2} V(MNS)_{ln}^* \label{eq.18_16a}
\end{eqnarray}
for $W^+$,
\begin{eqnarray}
g_L^{Wdu} = \frac{g_W}{\sqrt 2} V(CKM)_{ud}^*,~~~
g_L^{Wln} = \frac{g_W}{\sqrt 2} V(MNS)_{ln} \label{eq.18_16b}
\end{eqnarray}
for $W^-$,
\begin{eqnarray}
    g_R^{Wf'f} = 0 \label{eq.18_16c}
\end{eqnarray}
for all ${Wf'f}$ combinations,
\begin{eqnarray} \label{eq.18_16d}
g_L^{Zff} &=& g_Z (T^3_f - Q_f \sin^2\theta_w ) \\ 
         &=& g_Z (  1/2            )~  for~ f=\nu_l ~(l=e,\mu,\tau)\nonumber \\
         &=& g_Z ( -1/2 +     \sin^2\theta_w )  ~for ~f=l~ (=e,\mu,\tau)\nonumber \\
         &=& g_Z (  1/2 - 2/3 \sin^2\theta_w )  ~for ~f=u~ (=u,c,t) \nonumber \\
         &=& g_Z ( -1/2 + 1/3 \sin^2\theta_w )  ~for ~f=d~ (=d,s,b) \nonumber
\end{eqnarray}
\begin{eqnarray}\label{eq.18_16e}
    g_R^{Zff} &=& g_Z (      - Q_f \sin^2\theta_w ) \\ 
            &=& 0                  ~     for ~f=\nu_l ~(l=e,\mu,\tau) \nonumber \\
            &=& g_Z (      +     \sin^2\theta_w )~  for ~f=l ~(l=e,\mu,\tau)\nonumber \\
            &=& g_Z (      - 2/3 \sin^2\theta_w ) ~ for ~f=u ~(u=u,c,t)\nonumber \\
            &=& g_Z (      + 1/3 \sin^2\theta_w ) ~ for ~f=d ~(d=d,s,b)\nonumber 
\end{eqnarray}
for Z,
\begin{eqnarray}\label{eq.18_16f}
g_L^{Aff} &=& g_R^{Aff} = e Q_f \\ 
                &=& 0         ~for~ f=\nu_l~ (l=e,\mu,\tau) \nonumber \\
                &=& e ( -1 )  ~for~ f=l ~(l=e,\mu,\tau) \nonumber \\
                &=& e ( 2/3)  ~for~ f=u ~(u=u,c,t) \nonumber \\
                &=& e (-1/3)  ~for~ f=d ~(d=d,s,b)\nonumber 
\end{eqnarray}
for the photon $A$.

{\bf hw18-3}: Derive Eq.\ref{eq.18_13}, and obtain all the EW couplings Eqs.(\ref{eq.18_16a},\ref{eq.18_16b},\ref{eq.18_16c},\ref{eq.18_16d},\ref{eq.18_16e},\ref{eq.18_16f}).

Although the above couplings Eqs.(\ref{eq.18_16a},\ref{eq.18_16b},\ref{eq.18_16c},\ref{eq.18_16d},\ref{eq.18_16e},\ref{eq.18_16f}) may look complicated for you, we should appreciate the fact that the
\begin{eqnarray}
    4(W^+,W^-,Z,A) \times 12(q+l) \times 12(q+l) \times 2(L,R) = 1,152
\end{eqnarray}
couplings of the EW couplings of the SM
\begin{eqnarray}
    g_L^{Vf'f} ~and~ g_R^{Vf'f} \label{eq.18_18}
\end{eqnarray}
are parametrized by just two couplings $g_W$ and $g_Y=g_W/\cos\theta_W$ ($SU(2)_L$ and $U(1)_Y$ gauge couplings) or equivalently by $e$ and $\sin^2\theta_W$
This universality of the couplings is the most important
consequence of the gauge invariance of the SM Lagrangian.
Please note that the two $3$ by $3$ unitary mixing matrices and the
fermion masses are the parameters of the Yukawa sector of the SM.

Throughout the 1970's, all the above couplings (except those including
the top quark) have been measured accurately, and the SM of the EW
theory was established by 1979.
The $W$ and $Z$ bosons were discovered in 1982 and 1983 by $p{\overline p}$ collider (SppbarS) experiments at CERN, and the precision measurements of the
weak boson masses and their couplings were performed in 1990's by
$e^+e^-$ colliders LEP (CERN) and SLC (SLAC), and by $p{\overline p}$ collider
Tevatron (FNAL). The validity of the EW gauge theory including
quantum (radiative) corrections was confirmed when the top quark
was discovered in 1995 at the Tevatron collider within the mass range
predicted by the theory. Once the top quark mass is determined
precisely, the electroweak theory predicted that the Higgs boson
mass should be below about 200 GeV or so, and it was discovered
at the right mass range in 2012 at the $pp$ collider LHC.

Let us now compute the amplitudes for the transitions:
\begin{eqnarray}
    W^+(q,\lambda) \to l^+(p_1,+1/2) \nu_l(p_2,-1/2)~(l=e,\mu,\tau)\label{eq.18_20}
\end{eqnarray}
where I introduced the four momenta:
\begin{eqnarray}
    q^\mu = p_1^\mu + p_2^\mu \label{eq.18_21}
\end{eqnarray}
for $W^+$, $l^+$, and $\nu_L$. I neglect the lepton and neutrino masses,
so they are purely left-handed (helicity of $l^+$ is $+1/2$, and that of
$\nu_l$ is $-1/2$), whereas W can have 3 polarizations, $\lambda = \pm1,0$.

The initial state is
\begin{eqnarray}
    \ket{i}=b^\dagger_{W^+}(q,\lambda)\ket{0} \label{eq.18_22}
\end{eqnarray}
where I chose $W^-$ as a particle, and $W^+$ as an anti-particle.
Let us take the $W^+$ boson rest-frame
\begin{eqnarray}
    q^\mu = (m_W, 0, 0, 0)\label{eq.18_23}
\end{eqnarray}
and its spin polarization along the $z$-direction
\begin{eqnarray}
    J_z W^+ = \lambda W^+ ~(\lambda = \pm1,0) \label{eq.18_24}
\end{eqnarray}
The final state is a pair of $l^+$ and $\nu_l$ in the same frame:
\begin{eqnarray}
    \ket{f}=a^\dagger_{\nu_l}(p_2,-1/2) b^\dagger_{l^+}(p_1,1/2) \ket{0} \label{eq.18_25}
\end{eqnarray}
where
\begin{eqnarray}
p_1 = E(1,  \sin\theta, 0,  \cos\theta),~~~
p_2 = E(1, -\sin\theta, 0, -\cos\theta)\label{eq.18_26}
\end{eqnarray}
from the energy momentum conservation, we can tell $E = m_W/2$
For simplicity, we ignore all the masses of fermions that appear in
W-boson decays.

{\bf hw18-4}: Explain in the massless limit, $\psi_L(x)$ has only helicity
$-1/2$ fermions, and helicity $+1/2$ anti-fermions.  Also, explain
why the two fermion four momenta in the W rest frame Eq.\ref{eq.18_23} can
be parametrized as Eq.\ref{eq.18_26}. Why we don't need to specify the
azimuthal angle $\phi$?

Let us first obtain the amplitude
\begin{eqnarray}
    &&iT[W^+(q,\lambda) \to l^+(p_1,+1/2) \nu_l(p_2,-1/2)] \nonumber \\ 
    &=&\bra{l^+(p_1,+1/2),\nu_l(p_2,-1/2)} S\ket{W^+(q,\lambda)} \\
    &=& \bra{l^+(p_1,+1/2),\nu_l(p_2,-1/2)} Te^{i\int d^4x {\cal L}_{int}(x)} \ket{W^+(q,\lambda)} \\
    &=& \bra{l^+(p_1,+1/2),\nu_l(p_2,-1/2)} i\int d^4x {\cal L}_{int}(x) \ket{W^+(q,\lambda)}+\cdots \\
    &=&\bra{l^+(p_1,+1/2),\nu_l(p_2,-1/2)} i\int d^4x
    (-g_L^{Wnl} W^+_\mu(x) \nu_L^\dagger \sigma_-^\mu l_L(x) \ket{W^+(q,\lambda)} + \cdots\\
    &=&(-ig_L^{Wnl}) \int d^4x \bra{0}b_{l^+}(p_1,1/2) a_{\nu_l}(p_1,-1/2)
    W^+_\mu(x) \nu_L^\dagger \sigma_-^\mu l_L(x) b^\dagger_{W^+}(q,\lambda) \ket{0}+\cdots \label{eq.18_28}
\end{eqnarray}
where I retained only the first non-vanishing contribution (the Born
approximation).  Please note that I wrote the amplitude as $iT_{fi}$,
following the $S$-matrix notation,
\begin{eqnarray}
    \bra{f}S\ket{i}=i\bra{f}T\ket{i}=i T_{fi}\label{eq.18_29}
\end{eqnarray}
The operators in ${\cal L}_{int}(x)$ of Eq.\ref{eq.18_28} are:
\begin{eqnarray}
    W^+_\mu(x) = W^-_\mu(x)^\dgr
= \sum_h \int \frac{d^3k}{2E}
( b_{W^+}(k,h)         \epsilon_\mu(k,h)   e^{-ikx}
+ a_{W^-}(k,h)^\dagger \epsilon_\mu(k,h)^* e^{ikx})_{k^0=E_k} \label{eq.18_30}
\end{eqnarray}
where the sum is over $h=+1,0,-1$, and $E_k=\sqrt{m_W^2+{\vec k}^2}$.

Likewise for fermions,
\begin{eqnarray}
    f_L(x)^\dgr
    = \sum_h \int \frac{d^3k_1}{2E_1}
    ( a_f(k_1,h_1)^\dgr u_L(k_1,h_1)^\dgr e^{+ik_1x}
    + b_f(k_1,h_1)      v_L(k_1,h_1)      e^{-ik_1x} )_{k_1^0=E_{k_1}} \label{eq.18_31}
\end{eqnarray}
\begin{eqnarray}
    f'_L(x)
= \sum_h \int \frac{d^3k_2}{2E_2}
( a_{f'}(k_2,h_2)    u_L(k_2,h_2) e^{-ik_2x}
+ b_{f'}(k_2,h_2)^\dgr v_K(k_2,h_2) e^{+ik_2x} )_{k_2^0=E_{k_2}}  \label{eq.18_32}
\end{eqnarray}
Now, we can evaluate the matrix elements Eq.\ref{eq.18_28} by inserting the free field solutions, Eqs.(\ref{eq.18_30},\ref{eq.18_31},\ref{eq.18_32}) into Eq.\ref{eq.18_28}. Please confirm that only the following terms survive after using the definition of the vacuum,
\begin{eqnarray}\label{eq.18_33}
&&a(any~particle,~any~momentum,~any~spin)\ket{0} = 0, \nonumber \\ 
&&b(any~particle,~any~momentum,~any~spin) \ket{0} = 0 
\end{eqnarray}
and an appropriate normalization, $\bra{0}\ket{0} = 1$, which depends on how we
normalize the creation/annihilation operators. I adopt the covariant
normalization
\begin{eqnarray}\label{eq.18_34}
&&[ a_{k,h}, a^\dagger_{k',h'} ] = 2E \delta^3(k - k') \delta_{hh'} \nonumber \\
&&[ b_{k,h}, b^\dagger_{k',h'} ] = 2E \delta^3(k - k') \delta_{hh'}
\end{eqnarray}
as explained in the previous lectures. In Eq.\ref{eq.18_34}, ``,'' should be a
commutator for bosons, and an anti-commutator for fermions.

In our example, we find just 3 non-vanishing (anti-)commutators:
\begin{eqnarray}
    &&[ b_{W^+}(k,h), b^\dagger_{W^+}(q,\lambda)]= (2\pi)^3 2E_k \delta^3(k-q) \delta_{h,\lmd}\label{eq.18_35a}\\ 
    &&\{ a_{\nu_l}(p_1,-1/2), a^\dagger_f(k_1,h_1) \}
    = (2\pi)^3 2E_{k_1} \delta^3(k_1-q) \delta_{h_1,-1/2} \delta_{f,\nu_l}\label{eq.18_35b} \\
    &=&  \{ b_{l^+}(p_1,1/2), b^\dagger_{f'}(k_2,h_2) \}
    = (2\pi)^3 2E_{k_2} \delta^3(k_2-p_1) \delta_{h_2,+1/2} \delta_{f',l^+}\label{eq.18_35d}
\end{eqnarray}
In Eqs.(\ref{eq.18_35b},\ref{eq.18_35d}), I denote $\{A,B\} = AB+BA$ for anti-commutators.

The three 3-dimensional momentum integrals, in terms of $k$ for $W^+$,
$k_1$ for $\nu_l$, $k_2$ for $l^+$, can be done instantly by the three
3-dimensional $\delta$ functions, and the factors of $(2\pi)^3 2E$ cancel.
After using our simple vacuum normalization $\bra{0}\ket{0}=1$,
we find that Eq.\ref{eq.18_28} reduces to
\begin{eqnarray}
 &&   iT[W^+(q,\lmd) \to l^+(p_1,+1/2) \nu_l(p_2,-1/2)] \nonumber \\
&=& (-ig_L^{Wnl}) \int d^4x
\bra{0} b_{l^+}(p_1,1/2) a_{\nu_l}(p1,-1/2)
    W^+_\mu(x) \nu_L(x) \sigma_-^\mu l_L(x)
                                b^\dagger_{W^+}(q,\lmd) \ket{0}\\ 
&=& (-ig_L^{Wnl}) \int d^4x e^{-ix(q-p_1-p_2)}
\eps_\mu(W^+,q,\lmd) u_L(p_2,-1/2)^\dgr \sigma_-^\mu v_L(p_1,+1/2) \label{eq.18_37}
\end{eqnarray}

{\bf hw18-5}: Please derive Eq.\ref{eq.18_37} by yourself, following the above steps.

Now, the four dimensional integral over the spacetime $x^\mu$ simply
gives the four-momentum conservation
\begin{eqnarray}
    \int d^4x e^{-ix(q-p_1-p_2)} = (2\pi)^4 \delta^4(q-p_1-p_2) \label{eq.18_38}
\end{eqnarray}
We define the reduced transition matrix elements after factorizing the
four-momentum conservation factor as
\begin{eqnarray}
    S_{fi}= i T_{fi} = i M_{fi} (2\pi)^4 \delta^4(p_f-p_i) \label{eq.18_39}
\end{eqnarray}
where $p_f$ and $p_i$ are sum over all the four momenta in the final and
initial states, respectively. Finally, the transition matrix elements
are expressed as
\begin{eqnarray}
  &&  i M_{fi}
= i M(W^+(q,\lmd) \to l^+(p_1,+1/2) \nu_l(p_2,-1/2))\\
&=& \eps_\mu(W^+,q,\lmd)
u_L(\nu_l,p_2,-1/2)^\dgr [ -ig_L^{Wnl} \sigma_-^\mu ] v_L(l^+,p_1,+1/2) \label{eq.18_40a}
\end{eqnarray}
This final form of the matrix elements have wave functions of the
external particles, $W^+$,$l^+$ and $\nu_l$, which sandwiches the factor
inside the parenthesis ``$[ ~]$'', which reads
\begin{eqnarray}
    -ig_L^{Wln} \sigma_-^\mu \label{eq.18_40b}
\end{eqnarray}
Feynman showed that in all the amplitudes at arbitrary orders, whenever
the ${\cal L}_{int}(x)$ term Eq.\ref{eq.18_13} appears, its effect can be represented simply by the above factor in Eq.\ref{eq.18_40b} .
Once we know the rule, we can immediately write down the $S$ matrix
elements as Eq.\ref{eq.18_40a}. Eq.\ref{eq.18_40b} is called the ``Feynman rule'' for the $Wln$ term inside ${\cal L}_{int}(x)$ in Eq.\ref{eq.18_13}.

Note that the $2\times 2$ $\sigma_\pm^\mu$ vectors can be expressed as
\begin{eqnarray}
    \sigma_\pm
=    \gamma^\mu P_\pm
=    \frac{\gamma^\mu (1 \pm \gamma_5)}{2} \label{eq.18_41a}
\end{eqnarray}
by using the Dirac four spinors in its both sides,
\begin{eqnarray}
    {\overline \Psi}(x) \gamma^\mu \frac{1 \pm \gamma_5}{2} \Psi(x)
=    \psi_\pm(x)^\dgr  \sigma_\pm^\mu  \psi_\pm(x) \label{eq.18_41b}
\end{eqnarray}
by denoting the two Weyl spinors as $\psi_L(x)=\psi_-(x)$ and
$\psi_R(x)=\psi_+(x)$.  The Feynman rules are usually expressed by
using the $\gamma$ matrix and the projector, $P_\pm$ in Eq.\ref{eq.18_41a}.

Let us now compute the amplitudes in Eqs.(\ref{eq.18_40a},\ref{eq.18_40b}): first remove the common factor of $i$, and we find for the matrix elements $M_{fi}$
\begin{eqnarray}
 &&   M(W^+(q,\lmd) \to l^+(p_1,+1/2) \nu_l(p_2,-1/2))\nonumber \\
&=& -g_L^{Wnl}
\eps_\mu(W^+,q,\lmd)
u_L(\nu_l,p_2,-1/2)^\dgr \sigma_-^\mu v_L(l^+,p_1,+1/2) \label{eq.18_42}
\end{eqnarray}

{\bf hw18-6}: Derive Eq.\ref{eq.18_42} by yourself, following my explanation.

Now, all we should do is to insert the wave functions of the free
fields we obtained previously.
The polarization vectors are:
\begin{eqnarray}
&&\eps^\mu(q,+1) = ( 0, -1, -i, 0 )/\sqrt2 \label{eq.18_43a} \\
&&\eps^\mu(q, 0) = ( 0,  0,  0, 1 ) \label{eq.18_43b} \\
&& \eps^\mu(q,-1) = ( 0, +1, -i, 0 )/\sqrt2 \label{eq.18_43c}
\end{eqnarray}
when we choose the $z$-axis along the $W^+$ polarization direction.

Please note that the $W^+$ boson produced at hadron colliders are
polarized strongly. For instance, if $W^+$ is produced by an
annihilation of an up quark with the momentum along the positive $z$
direction, and the anti-down quark with the momentum along the
negative $z$ direction, the produced $W^+$ is $100\%$ polarized along the
negative $z$-axis, because only left-handed up quark and right-handed
anti-down quark can annihilate to produce $W^+$.  Its wave function can
hence be approximated by Eq.\ref{eq.18_43c}.
When the up quark momentum is along the negative $z$ direction, the $W^+$
should have $100\%$ polarization along the positive $z$ direction, and
its wave function should be Eq.\ref{eq.18_43a}.

We cannot produce $W^+$ boson with $J_z=0$ (Eq.\ref{eq.18_43b}) eigenstate in massless quark-anti-quark annihilation in the SM.  Please note that an
un-polarized $W^+$ should be an equal mixture of the $3$ polarized states.
Therefore, we can tell that the W bosons produced at colliders are
always polarized strongly.  Even when the $W^+$ momentum along the z-axis
is zero, $p_z=0$, or its rapidity is zero, $y=(1/2)\ln((E+p_z)/(E-p_z))=0$,
where the above two processes occur with the same probability, the
produced $W$'s are purely transversely polarized.

Next, let us recall the massless fermion wave functions.  The neutrino
wave function should be
\begin{eqnarray}
    u_L(p,-1/2) =  \sqrt{2E}
    \begin{pmatrix}
        0 \\ 1
    \end{pmatrix} \label{eq.18_44}
\end{eqnarray}
if the momentum $p$ is along the positive $z$-axis
\begin{eqnarray}
    p = E(1, 0, 0, 1) \label{eq.18_45}
\end{eqnarray}
The $p_2$ momentum in Eq.\ref{eq.18_26} is obtained from Eq.\ref{eq.18_45} by $-(\pi-\theta)$ rotation about the $y$-axis (or equivalently, $+(\pi+\theta)$ rotation):
\begin{eqnarray}
    R_y(-(\pi-\theta)) p = E(1, -\sin\theta, 0, -\cos\theta) = p_2 \label{eq.18_46}
\end{eqnarray}

{\bf hw18-7}: Show Eq.\ref{eq.18_46}, and also $R_y(-(\pi-\theta))p = R_y(\pi+\theta) p$.
Therefore,
\begin{eqnarray}
    u_L(p_2,-1/2)
    &=& \sqrt{2E} R_y(-(\pi-\theta)) 
    \begin{pmatrix}
    0 \\ 1
    \end{pmatrix}
    =\sqrt{2E} e^{-i\sigma^2/2(-(\pi-\theta))}
    \begin{pmatrix}
        0 \\ 1
    \end{pmatrix} \\ 
    &=&\sqrt{2E} e^{-i
    \begin{pmatrix}
        0 & -i \\
        i & 0
    \end{pmatrix}
    (-(\pi-\theta))/2 }
    \begin{pmatrix}
        0 \\ 1
    \end{pmatrix} \\
    &=& \sqrt{2E}
    \begin{pmatrix}
        \cos(\pi/2-\th/2) & \sin(\pi/2-\th/2) \\
        -\sin(\pi/2-\th/2) &  \cos(\pi/2-\th/2)
    \end{pmatrix}
    \begin{pmatrix}
        0 \\ 1
    \end{pmatrix} \\
    &=& \sqrt{2E} 
    \begin{pmatrix}
        \cos(\theta/2) \\ \sin(\theta/2)
    \end{pmatrix} \label{eq.18_47}
\end{eqnarray}

{\bf hw18-8}: Obtain the spinor wave function Eq.\ref{eq.18_47} by yourself.  Repeat the calculation by using $R_y(\pi+\theta)$ instead of $R_y(-(\pi-\theta))$. Observe that the sign of the spinor wave function depends on the way
we make rotation.

Finally, we need the $l^+$ wave function $v_L(p_1,+1/2)$
Again, when the momentum is along the positive $z$-axis Eq.\ref{eq.18_45},
\begin{eqnarray}
    v_L(p,+1/2) &=& u_R(p,+1/2)^c = i\sigma^2 u_R(p,+1/2)^* \\
    &=& 
    \begin{pmatrix}
        0 & 1 \\
        -1 & 0
    \end{pmatrix}
    \sqrt{2E}
    \begin{pmatrix}
        1 \\ 0
    \end{pmatrix}
    =\sqrt{2E}
    \begin{pmatrix}
        0 \\ -1
    \end{pmatrix} \label{eq.18_49}
\end{eqnarray}
The momentum $p_1$ in Eq.\ref{eq.18_49} is obtained from $p$ in Eq.\ref{eq.18_45} as
\begin{eqnarray}
    R_y(\theta) p = p_1 = E(1,  \sin\theta, 0,  \cos\theta) \label{eq.18_50}
\end{eqnarray}
and hence
\begin{eqnarray}
    v_L(p_1,+1/2) &=& R_y(\theta) v_L(p,+1/2) = \sqrt{2E} e^{-i\sigma^2/2 \theta}
    \begin{pmatrix}
        0 \\ -1
    \end{pmatrix}\\
    &=& \sqrt{2E}  e^{-i
    \begin{pmatrix}
        0 & -i \\
        i & 0
    \end{pmatrix}
    \theta/2}
    \begin{pmatrix}
        0 \\ -1
    \end{pmatrix}\\
    &=&\sqrt{2E}
    \begin{pmatrix}
        \cos(\th/2) & -\sin(\th/2)\\
        \sin(\th/2) &  \cos(\th/2)
    \end{pmatrix}
    \begin{pmatrix}
        0 \\ -1
    \end{pmatrix}\\
    &=& \sqrt{2E}
    \begin{pmatrix}
        \sin(\th/2) \\ -\cos(\th/2)
    \end{pmatrix}\label{eq.18_51}
\end{eqnarray}

{\bf hw18-9}: Please obtain the spinor wave function Eq.\ref{eq.18_51}, by yourself.

Now, by inserting Eq.\ref{eq.18_51} and Eq.\ref{eq.18_47} into Eq.\ref{eq.18_42}, we find
\begin{eqnarray}
    &&M(W^+(q,\lmd) \to l^+(p_1,+1/2) \nu_l(p_2,-1/2)) \\ 
    &=& -g_L^{Wnl} \eps_\mu(W^+,q,\lmd) u_L(\nu_l,p_2,-1/2)^\dagger \sigma_-^\mu v_L(l^+,p_1,+1/2) \\
    &=& -g_L^{Wln} (2E) \eps_\mu(W^+,q,\lmd) 
    \begin{pmatrix}
        \cos(\th/2)&\sin(\th/2)
    \end{pmatrix}
    \sigma_-^\mu
    \begin{pmatrix}
        \sin(\th/2)\\ -\cos(\th/2)
    \end{pmatrix}\\
    &=& -g_L^{Wln} (2E)
    \eps_\mu(W^+,q,\lmd)
    ( 0, \cos^2(\th/2)-\sin^2(\th/2), -i, -2\sin(\th/2)\cos(\th/2) ) \\
    &=&-g_L^{Wln} (2E)
    \eps_\mu(W^+,q,\lmd)
    ( 0, \cos(\theta), -i, -\sin(\theta) ) \label{eq.18_52}
\end{eqnarray}

{\bf hw18-10}: Obtain Eq.\ref{eq.18_52}. And then obtain all the $3$ amplitudes for $\lmd= +1, -1, 0$, by inserting the wave function Eqs.(\ref{eq.18_43a},\ref{eq.18_43b},\ref{eq.18_43c}) into Eq.\ref{eq.18_52}.

We can instantly obtain the amplitudes by simply inserting the $W^+$
polarization vectors of Eqs.(\ref{eq.18_43a},\ref{eq.18_43b},\ref{eq.18_43c}) into Eq.\ref{eq.18_52}.

We achieved the first calculation of the transition amplitudes.
Please make, however, the following very important observation:
\begin{eqnarray}
    &&{\overline u}(\nu_l,p_2,-1/2) \gamma^\mu P_L v(l^+,p_1,+1/2) 
    =u_L(\nu_l,p_2,-1/2)^\dagger \sigma_-^\mu v_L(l^+,p_1,+1/2) \\ 
    &=& 2E ( 0, \cos(\theta), -i, -\sin(\theta) ) = 2E( 0, \cos(\theta), +i, -\sin(\theta) )^* \\
    &=& -(2E)\sqrt{2} ( 0, -\cos(\theta), -i, \sin(\theta) )^*/\sqrt{2}\\
    &=&-(2E)\sqrt{2} R_y(\theta) \eps^\mu(J_z=+1)^* \label{eq.18_53}
\end{eqnarray}

{\bf hw18-11}: Obtain the next-to-the-bottom line of Eq.\ref{eq.18_53} from the bottom line. This is simply a review of polarization vector and the vector
representation of the rotation matrix.

The current of $l^+ \nu_l$ pair is proportional to the complex conjugate
of the vector boson polarization vector whose polarization is along
the $l^+$ momentum direction in the $W^+$ rest frame. This is a consequence
of angular momentum conservation, since $l^+$ is purely right-handed and
$\nu_l$ is purely left-handed, the total angular momentum is $+1$ along the
$l^+$ momentum direction.

We therefore found that the $W^+ \to l^+ \nu_l$ decay amplitudes are
proportional to the products of $W^+$ polarization vectors, which is
called the $d$-function:
\begin{eqnarray}
    \eps_\mu(J_z=\lmd) R_y(\theta) \eps^\mu(J_z=+1)^*
= - d^{J=1}_{\lmd,+1}(\theta) \label{eq.18_54}
\end{eqnarray}
The minus sign is introduced for the four-vector polarization vectors
to give the normalization of the $d$-functions to be $1$, when there is
no rotation.

In general, if the final state polarization is $\lmd'$, we denote
\begin{eqnarray}
    d^{J=1}_{\lmd,\lmd'}(\theta)
= - \eps_\mu(J_z=\lmd) R_y(\theta) \eps^\mu(J_z=\lmd')^*\label{eq.18_55}
\end{eqnarray}
Here, $J=1$ stands for the spin $1$ particle. $d$-functions can be defined
for a massive particle with arbitrary spin $J$ (in non-relativistic
quantum mechanics), where $J_z=\lmd$ in the initial state, and $J_z=\lmd'$
in the final state can take any values between $J$ and $-J$. You can find
a table of $d$-functions in RPP, under the title Clebsh-Gordan
coefficients and $d$-functions.

In case of $J=1$, we find
\begin{eqnarray}\label{eq.18_56}
    &&d^{J=1}_{\pm1,\pm1} =  (1+\cos\theta)/2,~~
    d^{J=1}_{\pm1,\mp1} =  (1-\cos\theta)/2,~~
    d^{J=1}_{\pm1,   0} =  \mp \sin\theta/\sqrt2,~~\nonumber \\
    &&d^{J=1}_{   0,\pm1} =  \pm \sin\theta/\sqrt2,~~
    d^{J=1}_{   0,   0} =  \cos\theta
\end{eqnarray}

{\bf hw18-12}: Please confirm Eq.\ref{eq.18_56} by using our polarization
vectors. When the $d$-function is $1$, we find that the two polarization
vectors overlap perfectly, while when the $d$-function is zero, we say
that they (the initial and final polarization vectors) are orthogonal.

All the d-functions give the same value when their squared is integrated
over $\cos\theta$.  In case of $J=1$, we find
\begin{eqnarray}
    \int_{-1}^{+1} d\cos\theta |d^{J=1}_{\lmd,\lmd'}|^2 = 2/3 \label{eq.18_57}
\end{eqnarray}

{\bf hw18-13}: Confirm Eq.\ref{eq.18_57} explicitly for all $d$-functions Eq.\ref{eq.18_56}.

By using the $d$-function, the $W^+$ decay amplitudes Eq.\ref{eq.18_52} can be expressed compactly as
\begin{eqnarray}
    &&M(W^+(q,\lmd) \to l^+(p_1,+1/2) \nu_l(p_2,-1/2)) \\
    &=& -g_L^{Wnl} \eps_\mu(W^+,q,\lmd) u_L(\nu_l,p_2,-1/2)^\dagger \sigma_-^\mu v_L(l^+,p_1,+1/2) \\
    &=& -g_L^{Wln} (2E) \eps_\mu(W^+,q,\lmd) ( 0, \cos\theta, -i, -\sin\theta ) \\
    &=& g_L^{Wln} (2E) \sqrt{2} \eps_\mu(J_z=\lmd) R_y(\theta) \eps^\mu(J_z=+1)^* \\
    &=&g_L^{Wln} (2E) \sqrt{2} d^{J=1}_{\lmd,+1}(\theta) \label{eq.18_58}
\end{eqnarray}

{\bf hw18-14}: Please compare the results of {\bf hw18-10} with the above expression Eq.\ref{eq.18_58} which uses the $d$-functions.

Now, the physical meaning of the transition amplitudes, their explicit
angular dependences are very clear, which simply shows the overlap of
the initial $W^+$ polarization wave function (for $\lmd = +1, 0, -1$,
respectively) and the final $W^+$ polarization wave function, which
should have $\lmd' = +1$ along the $l^+$ momentum direction, since $l^+$ has
$+1/2$ helicity along $l^+$ momentum direction, and $\nu_l$ has $-1/2$ helicity
in the opposite direction in the $W^+$ rest frame (in the SM).

Let us stop our hw18 here.\\

Best regards,\\

Kaoru

\end{document}
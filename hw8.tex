\documentclass[12pt]{article}
\usepackage{amsmath,graphicx,color,epsfig,physics}
\usepackage{float}
\usepackage{subfigure}
\usepackage{slashed}
\usepackage{color}
\usepackage{multirow}
\usepackage{feynmp}
\textheight=9.5in \voffset=-1.0in \textwidth=6.5in \hoffset=-0.5in
\parskip=0pt
\def\del{{\partial}}


\begin{document}

\begin{center}
{\large\bf HW8 for Advanced Particle Physics} \\

\end{center}

\vskip 0.2 in

Dear students,

This week, I introduce

\begin{enumerate}
  \item ${\cal L}_{Higgs}$ gives W and Z mass terms, the Higgs boson kinetic
  terms and the interactions among W, Z, Higgs.  They are
  given all in the ``Unitary gauge''.
  \item The Higgs potential in ${\cal L}_{Higgs}$ has $SO(4)$, or $SU(2)_L \times SU(2)_R$
  symmetry, which breaks down spontaneously to $SO(3)$, or $SU(2)_{custodial}$.
  This ``accidental'' symmetry gives the tree-level relation
  \begin{eqnarray}
    \rho =\frac{G_{NC}}{G_{CC}}=\frac{g_Z^2/m_Z^2}{g_W^2/m_W^2 }=\left (\frac{m_W}{m_Z\cos\theta_W} \right )^2=1
  \end{eqnarray}
  This tree-level relation is broken by radiative corrections, because
  the $SU(2)_{custodial}$ symmetry is not respected by the Yukawa interactions
  (mainly by the heavy top quark mass) and by the Hyper-charge gauge
  interactions, $U(1)_Y$.  Experiments verify the symmetry including the
  radiative corrections.  Therefore, the existence of the custodial
  $SU(2)$ symmetry in the Higgs potential is the most important knowledge
  we have for the otherwise completely mysterious physics of the
  spontaneous breakdown of $SU(2)_L \times U(1)_Y$ gauge symmetry.

  We can show that the $SU(2)_{custodial}$ symmetry exists in
  general if only $SU(2)_L$ doublet Higgs bosons have v.e.v.'s.
  Existence of non-doublet vacuum expectation values (such as
  triplet, quartet, etc) break the symmetry in general
  (excluding subtle cancellation among non-doublet v.e.v.'s).

  Existence of the $SU(2)_L$ doublet and $Y=\pm 1/2$ Higgs boson is the
  necessary condition for the SM to give masses to quarks and leptons.
  That they are the only Higgs bosons which give masses to the weak
  bosons (W and Z) is an important clue for us to explore the origin
  of the Higgs bosons and their interactions.

  \item I explain how to obtain quark and lepton masses from ${\cal L}_{Yukawa}$,
  again in the Unitary gauge for the minimum SM Higgs boson.

  We need two Unitary matrices each to diagonalize the up
  quark, down quark, and charged lepton mass matrices
  ($3 \times  3$ matrices in the 3 generation space).

  For neutrinos, the two Unitary matrices should be a transpose of one
  another, if their masses are Majorana type.  In this case, the diagonal
  elements are in general complex, and the two relative phases (among
  the three phases) have observable consequences in lepton number
  violating processes, such as in neutrino-less double beta decay
  ($0\nu\beta\beta$).

  In the SM, all quark/leptons with the same charge, i.e., u's($2/3$),
  d's($-1/3$), l's($-1$), and nu's($0$), have exactly the same $SU(3)_C$
  $\times SU(2)_L \times U(1)_Y$ quantum numbers.  Because of this,
  all the electrically neutral gauge interactions (all gluons, Z and
  $\gamma$) are diagonal in the mass-eigenstate basis, at least in the
  tree level.  This is called diagonal neutral currents, or the
  absence/smallness of the flavor changing neutral currents (FCNC).
  Therefore, FCNC's in the SM are proportional to radiative corrections,
  and they can be sensitive to new physics beyond the SM.

  In the charged current (CC), one combination of the Unitary matrices
  which diagonalize up and down (neutrino and charged lepton) mass
  matrices survive, and they give inter-generation transitions in
  the mass-eigenstate bases.  The Unitary matrix for the quark sector
  CC is called CKM (Cabibbo-Kobayashi-Maskawa) and that for the lepton
  sector is sometimes called MNS (Maki-Nakagawa-Sakata) or PMNS,
  including Pontecorvo.

  We find that the $3\times 3$ Unitary matrix for quarks has 3 mixing angles
  and one CP phase.  The $3\times 3$ Unitary matrix for leptons also has
  3 mixing angle and one CP phase for transitions which conserve
  the lepton number (such as the neutrino flavor oscillations),
  but can also have two additional phases if all the 3 neutrino
  masses are Majorana type.

  \item Higgs interactions in the minimum SM are very simple, since all the
  masses of weak-bosons, quarks and leptons are proportional to one
  and only one vacuum expectation value $v$, (Majorana neutrino mass is
  proportional to $v^2$), and the Higgs interactions are obtained simply
  by replacing $v$ by $v+H$ in the unitary gauge.

  The Higgs boson mass and the Higgs boson self interactions, ($HHH$ and
  $HHHH$ interactions) have slightly different relationship, because two
  terms (with positive $\lambda$ and negative $\mu^2$) in the Higgs potential
  contribute to give the vacuum expectation value and the Higgs boson
  mass at the same time.

  All these couplings of the Higgs boson should be measured, since they
  can be sensitive to the existence of more particles (such as another
  Higgs doublet, which preserves the custodial $SU(2)$ symmetry, and can
  preserve the flavor diagonal couplings if only one of the Higgs
  doublet gives masses to fermions of the same charge).
\end{enumerate}

{\bf Now let me give my homework 08 (and 09) on the above subjects}:

  Let me repeat the SM Lagrangian as
  \begin{eqnarray}
    {\cal L}_{SM} = {\cal L}_{gauge} + {\cal L}_{fermion} + {\cal L}_{Higgs} + {\cal L}_{Yukawa} \label{eq.8LSM}
  \end{eqnarray}
   and I write ${\cal L}_{Higgs}$ in terms of the Higgs bosons whose
  3-2-1 quantum numbers are $(1,2,+1/2)$
  \begin{eqnarray}
    \phi = ( \phi^+, \phi^0 )^T
  \end{eqnarray}
  the Higgs doublet. Its gauge invariant Lagrangian is
  \begin{eqnarray}
    {\cal L}_{Higgs} = (D_\mu \phi)^\dagger (D^\mu \phi) - V(\phi)
  \end{eqnarray}
   with
  \begin{eqnarray}
    V(\phi) = \lambda/4 (\phi^\dagger \phi)^2
                  + \mu^2 (\phi^\dagger \phi)
  \end{eqnarray}
  The potential has a minimum when $\lambda > 0$
  and the potential should have zero slope at the minimum:
  \begin{eqnarray}
    \frac{dV(\phi)}{d\phi}|_{\phi=<\phi>}=0
  \end{eqnarray}
  which reads
  \begin{eqnarray}
    (\frac{\lambda}{2} <\phi>^\dagger <\phi> + \mu^2) <\phi>^\dagger = 0. \label{eq.8pde}
  \end{eqnarray}
  When $\mu^2 > 0$, there is only one minimum at
  $<\phi> = 0$
  which is a symmetric vacuum. The Higgs boson is complex and
  its mass is $\mu$, and all $W^1$, $W^2$, $W^3$ gauge bosons as well as
  B bosons are massless.  We cannot distinguish up quark from
  down quark, neutrino from charged lepton, and in fact even
  leptons are confined due to unbroken $SU(2)_L$ gauge interactions.
  The hyper-change gauge boson is like a photon, but it couples to
  hypercharge, rather than the electric charge.  The world looks
  very different from the world which we live.

  If $\mu^2 < 0$
  then Eq.\ref{eq.8pde} gives two solutions, one is $<\phi> = 0$, and the other is
  \begin{eqnarray}
    <\phi>^\dagger <\phi> = -2\mu^2/\lambda = v^2/2 \label{eq.82so}
  \end{eqnarray}
  The solution $<\phi> = 0$ actually gives a local maximum of the potential,
  and the solution Eq.\ref{eq.82so} gives the true minimum (vacuum in QFT).
  You have shown this in the last homework. The minimum point
  (vacuum) can be at any location as long as Eq.\ref{eq.82so} is satisfied.
  We can for instance parametrize it as
  \begin{eqnarray}
    <\psi>=
    \begin{pmatrix}
      \frac{v}{\sqrt2} \sin\theta e^{i\alpha} \\
      \frac{v}{\sqrt2} \cos\theta e^{i\beta}
    \end{pmatrix}
  \end{eqnarray}
  with arbitrary real angles $\theta$, $\alpha$, $\beta$.  In the previous
  homework, we showed that a combination of two $SU(2)_L$ rotations
  and one $U(1)_Y$ rotation can bring the minimum to
  \begin{eqnarray}
    <\phi> = 
    \begin{pmatrix}
      0 \\ v/\sqrt2
    \end{pmatrix} \label{eq.8phib}
  \end{eqnarray}

  In other words, no matter where the minimum resides in the doublet
  field $\phi$, we can always define the new $T^3$ axis such that the
  v.e.v. has the form Eq.\ref{eq.8phib}, and hence is has zero electromagnetic charge:
  \begin{eqnarray}
    Q <\phi> = (T^3 + Y) <\phi> = (-1/2 + 1/2) <\phi> = 0. \label{eq.8Q0}
  \end{eqnarray}

  This is the reason why the electromagnetic $U(1)$ symmetry, generated
  by the charge operator $Q$ survives after the symmetry breakdown.
  Theoretically, the SM vacuum is invariant under $U(1)_{EM}$
 \begin{eqnarray}
  <\phi> \to <\phi>' = e^{iQ\theta} <\phi> = <\phi>
 \end{eqnarray}
  because of Eq.\ref{eq.8Q0}. The vacuum (Eq.\ref{eq.8phib}) is not invariant under both $SU(2)_L$ and $U(1)_Y$ symmetries. Since all the interactions in
  the Lagrangian are symmetric under $SU(2)_L \times U(1)_Y$ symmetry,
  we express the symmetry breaking due to the vacuum as spontaneous
  breakdown of the symmetry.  Since the vacuum is symmetric under
  $U(1)$ transformation along one combination
 \begin{eqnarray}
  Q = T^3 + Y
 \end{eqnarray}
  of the 4 generators of the original $SU(2)_L \times U(1)_Y$ symmetry,
  we say that the $SU(2)_L \times U(1)_Y$ symmetry is spontaneously
  broken down to $U(1)_{EM}$ symmetry by the Higgs boson v.e.v. (Eq.\ref{eq.8phib}).

  Now, let us show that the above symmetry breaking pattern of the
  Higgs doublet model preserves a global symmetry, which was found
  and named $SU(2)_{custodial}$ by t'Hooft.

{\bf hw08-1}:

  We can express a Higgs doublet field in terms 4 real fields
  ($\phi_k$, $k=1,2,3,4$):
\begin{eqnarray}
  \phi &=& ( \phi^+, \phi^0 )^T  \label{eq.8phi1}\\
  \phi^+ &=& ( \phi_1 + i\phi_2 )/\sqrt2 \label{eq.8phi2}\\
  \phi^0 &=& ( \phi_3 + i\phi_4 )/\sqrt2 \label{eq.8phi3}
\end{eqnarray}
  Please write the Higgs potential in terms of the 4 real bosons
\begin{eqnarray}
  V(\phi) = V(\phi_1,\phi_2,\phi_3,\phi_4) \label{eq.8vphi}
\end{eqnarray}

{\bf hw08-2}:
  Show that the Higgs potential Eq.\ref{eq.8vphi} is invariant under the global
  $SO(4)$ transformation
  \begin{eqnarray}
    \begin{pmatrix}
      \phi_1 \\ \phi_2 \\ \phi_3 \\ \phi_4
    \end{pmatrix}
    \to
    \begin{pmatrix}
      \phi'_1 \\ \phi'_2 \\ \phi'_3 \\ \phi'_4
    \end{pmatrix}
    =O
    \begin{pmatrix}
      \phi_1 \\ \phi_2 \\ \phi_3 \\ \phi_4
    \end{pmatrix}
  \end{eqnarray}
  where $O^T O = 1$.  Please observe that if $\mu^2>0$, then all the
  4 real bosons have the same mass, $\mu$.
  Now, let us examine what happens to this global symmetry,
  when $\mu^2 < 0$.

{\bf hw08-3}:
  Please show that the potential is minimized when
  \begin{eqnarray}
    <\phi_1>^2 + <\phi_2>^2 + <\phi_3>^2 + <\phi_4>^2
    = v^2
    = -4\mu^2/\lambda \label{eq.8v2}
  \end{eqnarray}
  It is now clear that the minimum point (vacuum) can be an arbitrary
  location on the surface of the 4-dimensional sphere of the radius v,
  in Eq.\ref{eq.8v2}.
  Without loosing generality, we can choose
  \begin{eqnarray}
    (<\phi_1>,<\phi_2>,<\phi_3>,<\phi_4>) = (0,0,v,0) \label{eq.84rbv}
  \end{eqnarray}

{\bf hw08-4}:
  Let us parametrize the 4 real bosons on the vacuum (Eq.\ref{eq.84rbv}) as
  \begin{eqnarray}
    \phi_1 = \pi_1,~~\phi_2 = \pi_2,~~\phi_3 = v + h,~~\phi_4 = \pi_3 \label{eq.84rbv4}
  \end{eqnarray}
  Show that the 4 real fields $(h, \pi_1, \pi_2, \pi_3)$ have zero v.e.v.:
  \begin{eqnarray}
    <\pi_1> = <\pi_2> = <\pi_3> = <h> = 0
  \end{eqnarray}
  Please rewrite the Higgs potential in terms of $\pi_k (k=1,2,3)$,
  and $h$, and observe that the 3 $\pi_k$'s are massless, and
  the field $h$ has the mass
  \begin{eqnarray}
    m(\pi_k) &=& 0,~~~   (k=1,2,3) \\
    m(h)     &=& \sqrt{\lambda} v/2
  \end{eqnarray}
  Please observe that the Higgs potential after the symmetry
  breakdown, (Eq.\ref{eq.84rbv}), still has the global $SO(3)$ symmetry:
  \begin{eqnarray}
    (\pi_1,\pi_2,\pi_3)^T \to O (\pi_1,\pi_2,\pi_3)^T
  \end{eqnarray}
   where $O$ is a $3\times 3$ orthogonal matrix, $O^T O = 1$.

  This $SO(3)$ symmetry is the custodial $SU(2)$ symmetry.
  Under this global $SO(3)$ or $SU(2)$ symmetry, the $3$ massless
  Goldstone bosons $(\pi_1, \pi_2, \pi_3)$ transform as a triplet,
  and hence they have the same mass (zero-mass) and have the
  same interactions (with the Higgs boson, h).

  Because the 3 Goldstone bosons transform as a triplet under this
  global $SO(3)$ symmetry, the three $SU(2)$ weak bosons, $W^1$, $W^2$, $W^3$,
  acquire the same mass, $m_W$, after the Higgs mechanism.
\begin{eqnarray}
  m(W^1) = m(W^2) = m(W^3) = m(W^\pm) \label{eq.8mw}
\end{eqnarray}

  This global $SO(3)$ symmetry is respected by the $SU(2)_L$ gauge
  interactions, but not by the $U(1)_Y$ gauge interactions
  (this will be proven below), and the $W^3$ mixes with B to make
  the Z bosons.  Because the $W^3-B$ mixing is determined by their
  weak coupling strengths, the following rule follows:
\begin{eqnarray}
  \frac{m_W}{m_Z}=\frac{m(W^3)}{m_Z}=\frac{g}{g_Z}=\frac{g}{\sqrt{g^2+g'2}} =\cos\theta_W \label{eq.8wBmix1}
\end{eqnarray}
  or, according to Veltman,
\begin{eqnarray}
  \rho = \frac{G_{NC}}{G_{CC}}=\frac{g_Z^2/m_Z^2}{g_W^2/m_W^2}=\left( \frac{m_W}{m_Z\cos\theta_W}\right)^2=1 \label{eq.8wBmix2}
\end{eqnarray}
  That all the three $W^k$ masses are the same (Eq.\ref{eq.8mw}) because of the global $SO(3)$ symmetry is the basis of this famous relation Eq.\ref{eq.8wBmix1} or Eq.\ref{eq.8wBmix2} of the MSM.  Although the above real field basis is the easiest way to observe that the Higgs potential keeps the global $SO(3)$ symmetry after the spontaneous symmetry breakdown, it is not easy to see how this symmetry is broken by the other interactions in the SM.
  An elegant method was found by t'Hooft, and he named the residual
  global symmetry of the Higgs potential, the custodial $SU(2)$ symmetry.
  Let us now follow him.

  First, please recall the charge conjugate doublet
  \begin{eqnarray}
    \phi^c = i \sigma^2 \phi^* =
    \begin{pmatrix}
      {\phi^0}^* \\ -\phi^-
    \end{pmatrix}
  \end{eqnarray}
  We already know that $\phi^c$ has exactly the same $SU(2)_L$
  transformations as $\phi$, but that it has opposite hypercharge
  \begin{eqnarray}
    Y(\phi^c) = -Y(\phi) = -1/2
  \end{eqnarray}
  and hence the $U(1)_Y$ interactions distinguish $\phi$ and $\phi^c$.
  t'Hooft recognized that the MSM Higgs potential can be written as
  \begin{eqnarray}
    V(\phi)
    = V(\phi^c)
    = \frac{\lambda}{4} \frac{[\phi^\dagger \phi + (\phi^c)^\dagger \phi^c]^2}{4}
         +\frac{\mu^2 [\phi^\dagger \phi + (\phi^c)^\dagger \phi^c]}{2} \label{eq8.V2}
  \end{eqnarray}

{\bf hw08-5}: Show Eq.\ref{eq8.V2}. If you remember showing this in one of previous homework, you can skip this homework. He introduced a quartet field,
\begin{eqnarray}
  \Phi = 
  \begin{pmatrix}
    \phi^c&\phi
  \end{pmatrix} \label{eq.8quarfild}
\end{eqnarray}
  and introduced Global $SU(2)$ transformation, which we often denote
  as $SU(2)_R$ (the reason will become clear below):
  \begin{eqnarray}
    \Phi \to \Phi' = \Phi U_R^\dagger \label{eq.8su2r}
  \end{eqnarray}
  Please note that the above transformation is essentially the $SU(2)$
  rotation of $\phi$ and $\phi^c$.

{\bf hw08-6}: Show that Eq.\ref{eq.8su2r} is the same as
\begin{eqnarray}
  \Phi^T \to \Phi'^T = U_R^* \Phi^T
\end{eqnarray}
 where
\begin{eqnarray}
  \Phi^T = 
  \begin{pmatrix}
    \phi^c \\  \phi
  \end{pmatrix}
\end{eqnarray}
  Therefore, the ``doublet'', $( \phi^c, \phi )^T$ transform as
  $2^*$ of $SU(2)_R$.

{\bf hw08-7}:
  Please show
\begin{eqnarray}
  Tr\{ \Phi^\dagger \Phi \}
    = \phi^\dagger \phi + (\phi^c)^\dagger \phi^c
    = 2 \phi^\dagger \phi
\end{eqnarray}
   and hence we can express $V(\phi)$ as
\begin{eqnarray}
  V(\phi) = \lambda/2 ( tr\{ \Phi^\dagger \Phi \}/2 )^2
                   -\mu^2 ( tr{ \Phi^\dagger \Phi }/2 )
\end{eqnarray}
  The Higgs potential is clearly invariant under the joint transformation
\begin{eqnarray}
  \Phi \to \Phi' = (U_L) \Phi (U_R)^\dagger \label{eq.8joitran}
\end{eqnarray}
 where $U_L$ is the $SU(2)_L$ gauge transformation, and
  $U_R$ is the global $SU(2)$ transformation.

{\bf hw08-8}: Prove the above statement.
  Now the quartet field acquires the v.e.v. at the potential
  minimum, which occurs in our basis as
\begin{eqnarray}
  < \Phi >=
  \begin{pmatrix}
    <\phi^c> & <\phi> 
  \end{pmatrix}
  =\frac{v}{\sqrt 2}
  \begin{pmatrix}
    1&0 \\0 &1
  \end{pmatrix}
\end{eqnarray}
  Since the v.e.v. is a unity matrix, the vacuum is invariant under
  the global $SU(2)$ symmetry when the phases of $SU(2)_L$ and $SU(2)_R$
  are exactly the same:
\begin{eqnarray}
  <\Phi> \to <\Phi>'
             = (U_{custodial}) <\Phi> (U_{custodial})^\dagger
             = <\Phi> \label{eq.8ptp}
\end{eqnarray}
  where transformation
  \begin{eqnarray}
    U_{custodial} = U_L = U_R = e^{ i T^a \theta^a }
  \end{eqnarray}
  gives the ``diagonal subgroup'' of $SU(2)_L \times SU(2)_R$.

{\bf hw08-9}: Show the invariance Eq.\ref{eq.8ptp}.
  In this description, the MSM Higgs potential has the global
  $SU(2)_L \times SU(2)_R$ symmetry, which breaks down to the global
  $SU(2)_{custodial}$ symmetry.  This corresponds to the breaking of
  $SO(4)$ down to $SO(3)$ when we parametrize the Higgs doublet field
  in terms of the 4 real fields.

  Let us show how the 3 massless states (the Goldstone bosons) appear
  in this formalism.  Please refer to our $\pi^k$ ($k=1,2,3$) notation
  in Eq.\ref{eq.84rbv4}.  Inserting Eq.\ref{eq.84rbv4} to Eq.(\ref{eq.8phi1},\ref{eq.8phi2},\ref{eq.8phi3}), we find
\begin{eqnarray}
  \phi= 
  \begin{pmatrix}
    (\phi_1 + i\phi_2)/\sqrt{2} \\
    (\phi_3 + i\phi_4)/\sqrt{2} 
  \end{pmatrix}
  =
  \begin{pmatrix}
    (\pi_1  + i\pi_2)/\sqrt{2} \\
    (v + h  + i\pi_3)/\sqrt{2}
  \end{pmatrix}
\end{eqnarray}
\begin{eqnarray}
  \phi^c =
  \begin{pmatrix}
    (v + h - i\pi_3)/\sqrt{2} \\
    -(\pi_1 - i\pi_2)/\sqrt{2} 
  \end{pmatrix}
\end{eqnarray}
\begin{eqnarray}
  \Phi =
  \begin{pmatrix}
    \phi^c & \phi
  \end{pmatrix}
  =
  \begin{pmatrix}
    ( v + h -i\pi_3)/\sqrt{2} & (\pi_1 + i\pi_2)/\sqrt{2} \\
    (-\pi_1 +i\pi_2)/\sqrt{2} & ( v + h +i\pi_3)/\sqrt{2}
  \end{pmatrix} \label{eq.8Phimx}
\end{eqnarray}

{\bf hw08-10}: Show that Eq.\ref{eq.8Phimx} can be expressed as
\begin{eqnarray}
  \Phi = \frac{v+h}{\sqrt{2}} + \frac{i}{\sqrt 2} \sigma^k \pi'_k \label{eq.8Phiex}
\end{eqnarray}
 where
\begin{eqnarray}
  \pi'_1 = \pi_2,~~\pi'_2 = \pi_1,~~\pi'_3 = -\pi_3
\end{eqnarray}
  By inserting Eq.\ref{eq.8Phiex} into the Higgs potential term $Tr\{ \Phi^\dagger \Phi \}$
  we can tell that the three real bosons, $\pi'_k$ $(k=1,2,3)$ are all
  massless (Goldstone bosons), and they transform as a triplet of
  $SU(2)_{custodial}$. Can you show this?

  You will learn that the result should be the same as in our $SO(4)$
  formulation. I think that it is useful for you to learn that the
  same symmetry, $SO(3)$ and $SU(2)$, for $\pi_k$ or $\pi'_k$, can appear
  quite differently in the two formulations.

  The $SU(2)$ language of t'Hooft is more convenient since we can clearly
  tell that $U(1)_Y$ breaks the custodial $SU(2)$ symmetry, simply because
  $\phi$ and $\phi^c$ have opposite hyper charges.  We will learn later
  that the custodial $SU(2)$ symmetry is also broken by the Yukawa
  interactions, when $m_t \neq m_b$.

  In perturbative QFT, relations among masses and couplings (or
  even in scattering amplitudes) that are satisfied at the tree
  (classical) level are broken by radiative (quantum) corrections
  if the relations are not protected by a symmetry of all the
  interactions.  In our example, the $m_w/m_z = g/g_z$ relationship
  Eq.\ref{eq.8wBmix1} is a consequence of the custodial $SU(2)$ symmetry in the Higgs potential. Because the symmetry is broken by the hyper-charge
  gauge interactions and also by the Yukawa interactions which give
  $m_t \neq m_b$, the relation Eq.\ref{eq.8wBmix2} receive radiative corrections which vanish when $g'/g \to 0$ ($\sin^2\theta_W \to 0$), and $(m_t-m_b) \to 0$.
  In other words, the corrections are proportional to the factors
  like $\sin^2\theta_W$ and ($m_t-m_b$). That we can sometimes tell how
  the quantum corrections should behave in terms of the model
  parameters is very important when we study phenomenological
  consequences of QFT models, such as the SM and its extentions.

Please let me stop hw08 here.\\

Best wishes,\\

Kaoru

\end{document}

\documentclass[11pt]{article}
\usepackage{amsmath,graphicx,color,epsfig,physics}
%\usepackage{pstricks}
\usepackage{float}
\usepackage{subfigure}
\usepackage{slashed}
\usepackage{color}
\usepackage{multirow}
\usepackage{feynmp}
\usepackage[top=1in, bottom=1in, left=1.2in, right=1.2in]{geometry}
\def\del{{\partial}}
\begin{document}
\title{Particle physics HW12}
\author{Yang Ma}

\maketitle

\section{ }
With
\begin{eqnarray}
    &&A_k = (J_k + i K_k)/2, \\
    &&B_k = (J_k - i K_k)/2, 
\end{eqnarray}
and the combination relations
\begin{eqnarray}
    &&[ J_i, J_j ] =  i \epsilon_{ijk} J_k, \\
    &&[ J_i, K_j ] =  i \epsilon_{ijk} K_k, \\
    && [ K_i, K_j ] = -i \epsilon_{ijk} J_k, 
\end{eqnarray}
we see
\begin{eqnarray}
    [ A_i, A_j ] &=& \frac{1}{4} [J_i + i K_i,J_j + i K_j] \\
    &=&\frac{1}{4}([J_i,J_j]+i[J_i,K_j]+i[K_i,J_j]-[K_i,K_j])\\
    &=&\frac{1}{4}(2i \epsilon_{ijk} J_k-2\epsilon_{ijk} K_k) \\
    &=&\frac{i}{2} \epsilon_{ijk} (J_k + i K_k) = i \epsilon_{ijk} A_k,
\end{eqnarray}
\begin{eqnarray}
    [ B_i, B_j ] &=& \frac{1}{4} [J_i - i K_i,J_j - i K_j] \\
    &=&\frac{1}{4}\{[J_i,J_j]-i[J_i,K_j]-i[K_i,J_j]-[K_i,K_j]\} \\
    &=&\frac{1}{4}(2i \epsilon_{ijk} J_k+2\epsilon_{ijk} K_k) \\
    &=&\frac{i}{2} \epsilon_{ijk} (J_k - i K_k) = i \epsilon_{ijk} B_k,
\end{eqnarray}
\begin{eqnarray}
    [ A_i, B_j ] &=& \frac{1}{4} [J_i + i K_i,J_j - i K_j] \\
    &=&\frac{1}{4}([J_i,J_j]-i[J_i,K_j]+i[K_i,J_j]+[K_i,K_j]) \\
    &=&0
\end{eqnarray}

\section{ }
By pluging in
\begin{eqnarray}
    && J_k = (A_k + B_k) \\
    && K_k = (A_k - B_k)/i
\end{eqnarray}
to the Lorentz transformation operator, we have
\begin{eqnarray}
    &&L(\theta_1, \theta_2, \theta_3, \eta_1, \eta_2, \eta_3)
  =  e^{-i[ J_1\theta_1 +J_2\theta_2 +J_3\theta_3 +K_1\eta_1 +K_2\eta_2 +K_3\eta_3 ]} \\
  &=&e^{\sum_k(A_k+B_k)\theta_k -i\sum_k(A_k-B_k)\eta_k} \\
  &=& e^{-i[ A_1(\theta_1-i\eta_1) +A_2(\theta_2-i\eta_2) +A_3(\theta_3-i\eta_3)
          +B_1(\theta_1+i\eta_1) +B_2(\theta_2+i\eta_2) +B_3(\theta_3+i\eta_3) ]} 
\end{eqnarray}

\section{ }
Do the boost along $z$-axis
\begin{eqnarray}
   && L_A(0,0,0,0,0,\eta_3) u_L
    e^{ A_3       (-\eta_3) } u_L
    = e^{ \frac{\sigma^3}{2} (-\eta_3) }u_L \\
    &=&
    \begin{pmatrix}
      e^{-\eta_3/2} & 0 \\
      0 & e^{\eta_3/2}
    \end{pmatrix} 
    \sqrt{2E}
    \begin{pmatrix}
        0 \\ 1
    \end{pmatrix}
    =\sqrt{2E e^{\eta_3}}
    \begin{pmatrix}
        0 \\ 1
    \end{pmatrix}
    =\sqrt{2E'}
    \begin{pmatrix}
        0 \\ 1
    \end{pmatrix},
\end{eqnarray}
we see that
\begin{eqnarray}
    E' = e^{\eta_3} E = (\cosh\eta_3 + \sinh\eta_3) E = \gamma(1+\beta) E 
\end{eqnarray}

\section{ }
With
\begin{eqnarray}
    p^\mu = (E, 0, 0, E)^T 
\end{eqnarray}
we have
\begin{eqnarray}
    h=\frac{\vec{J} \cdot {\vec p}}{|\vec{p}|}=\frac{\sigma^3}{2}.
\end{eqnarray}
Apply the helicity operator to $u_L$, we have
\begin{eqnarray}
    h u_L=\frac{\sigma^3}{2} u_L.
\end{eqnarray}

\section{ }
For
\begin{eqnarray}
    p^\mu = (E, 0, 0, E)^T,
\end{eqnarray}
we have known the corresponding wavefunction
\begin{eqnarray}
    u_L=\sqrt{2E} (0,1)^T.
\end{eqnarray}
Under Lorentz transformation, the 4 momentum turns into
\begin{eqnarray}
    p^\mu \to {p'}^\mu
    = E' (1, \sin\theta\cos\phi, \sin\theta\sin\phi, \cos\theta)^T, 
\end{eqnarray}
and we now need to obtain its wavefunction.
\begin{eqnarray}
    B_z(\eta)u_L=e^{-\frac{\sigma^3}{2} \eta}=
    \begin{pmatrix}
      e^{-\eta/2} & 0 \\
      0 & e^{\eta/2}
    \end{pmatrix}
    \sqrt{2E}
    \begin{pmatrix}
        0 \\ 1
    \end{pmatrix}
    =\sqrt{2E'}
    \begin{pmatrix}
        0 \\ 1
    \end{pmatrix}
\end{eqnarray}
\begin{eqnarray}
   R_y(\theta)B_z(\eta)u_L&=& e^{-\frac{i \sigma^2}{2} \theta}\sqrt{2E'}
   \begin{pmatrix}
    0 \\ 1
    \end{pmatrix}\\
    &=& \sqrt{2E'}
    \begin{pmatrix}
        \cos \frac{\theta}{2} & -\sin \frac{\theta}{2} \\
        \sin \frac{\theta}{2} & \cos \frac{\theta}{2}
    \end{pmatrix}
    \begin{pmatrix}
        0 \\ 1
    \end{pmatrix}\\
    &=&\sqrt{2E'}
    \begin{pmatrix}
        -\sin \frac{\theta}{2} \\ \cos \frac{\theta}{2}
    \end{pmatrix}
\end{eqnarray}
\begin{eqnarray}
    R_z(\phi)  R_y(\theta)B_z(\eta)u_L &=& e^{-\frac{i \sigma^3}{2} \theta}\sqrt{2E'}
    \begin{pmatrix}
    -\sin \frac{\theta}{2} \\ \cos \frac{\theta}{2}
    \end{pmatrix}\\
    &=& \sqrt{2E'}
    \begin{pmatrix}
        -e^{-\frac{i \phi}{2}}\sin \frac{\theta}{2} \\ e^{\frac{i \phi}{2}}\cos \frac{\theta}{2}
    \end{pmatrix}
\end{eqnarray}

\section{ }
With
\begin{eqnarray}
    &&\frac{\vec p}{|\vec p|}
    = (\sin\theta\cos\phi, \sin\theta\sin\phi, \cos\theta)^T, \\ 
    &&\vec{J} = (\frac{\sigma^1}{2}, \frac{\sigma^2}{2}, \frac{\sigma^3}{2})^T ,
\end{eqnarray}
we have the helicity
\begin{eqnarray}
    h&=&\frac{\vec{J} \cdot {\vec p}}{|\vec{p}|}=\frac{\sigma^1}{2}\sin\theta\cos\phi +\frac{\sigma^2}{2}\sin\theta\sin\phi +\frac{\sigma^3}{2} \cos\theta, \\
    &=& \frac{1}{2}
    \begin{pmatrix}
        \cos\theta & \sin\theta e^{-i\phi} \\ 
        \sin\theta e^{i\phi} & -\cos\theta
    \end{pmatrix},
\end{eqnarray}
and then can have
\begin{eqnarray}
    h u_L&=&\frac{1}{2} 
    \begin{pmatrix}
        \cos\theta & \sin\theta e^{-i\phi} \\ 
        \sin\theta e^{i\phi} & -\cos\theta
    \end{pmatrix}
    \sqrt{2E'}
    \begin{pmatrix}
        -e^{-\frac{i \phi}{2}}\sin \frac{\theta}{2} \\ e^{\frac{i \phi}{2}}\cos \frac{\theta}{2}
    \end{pmatrix} \\
    &=& \frac{\sqrt{2E'}}{2}
    \begin{pmatrix}
        -e^{-\frac{i \phi}{2}}(\sin\frac{\theta}{2}\cos\theta- \cos\frac{\theta}{2}\sin\theta)\\ -e^{\frac{i \phi}{2}}(\sin\frac{\theta}{2}\sin\theta+\cos\frac{\theta}{2}\cos\theta)
    \end{pmatrix}\\ 
    &=& \frac{\sqrt{2E'}}{2}
    \begin{pmatrix}
        e^{-\frac{i \phi}{2}}\sin \frac{\theta}{2} \\ -e^{\frac{i \phi}{2}}\cos \frac{\theta}{2}
    \end{pmatrix}
    =-\frac{1}{2}u_L
\end{eqnarray}

\section{ }
For
\begin{eqnarray}
   && \sigma_-^\mu = (1, -\sigma_1, -\sigma_2, -\sigma_3) \\
   &&p^\mu
  = |\vec {p}| (1, \sin\theta\cos\phi, \sin\theta\sin\phi, \cos\theta)^T
\end{eqnarray}
we can write out
\begin{eqnarray}
    p_\mu\sigma_-^\mu = |{\vec p}| (1+ \sigma^1 \sin\theta\cos\phi +\sigma^2\sin\theta\sin\phi + \sigma^3 \cos\theta).
\end{eqnarray}
Applying $p_\mu\sigma_-^\mu$ to $u_L$, and note
\begin{eqnarray}
    \begin{pmatrix}
        \cos\theta & \sin\theta e^{-i\phi} \\ 
        \sin\theta e^{i\phi} & -\cos\theta
    \end{pmatrix}
    u_L=-u_L,
\end{eqnarray}
we have
\begin{eqnarray}
    p_\mu\sigma_-^\mu u_L = |{\vec p}| \left[ 1+
    \begin{pmatrix}
        \cos\theta & \sin\theta e^{-i\phi} \\ 
        \sin\theta e^{i\phi} & -\cos\theta
    \end{pmatrix}
        \right] u_L =(1-1)u_L=0.
\end{eqnarray}

\section{ }
\begin{eqnarray}
    hu_L^c=\frac{1}{2} 
    \begin{pmatrix}
        1 & 0 \\
        0 & -1
    \end{pmatrix}
    \sqrt{2E}
    \begin{pmatrix}
        1 \\ 0 
    \end{pmatrix}
    = \frac{\sqrt{2E}}{2}
        \begin{pmatrix}
            1 \\ 0 
        \end{pmatrix}
    =\frac{1}{2}u_L
\end{eqnarray}

\section{ }
\begin{eqnarray}
    && L_A(0,0,0,0,0,\eta_3) u_L^c
     e^{ A_3       (-\eta_3) } u_L
     = e^{ \frac{\sigma^3}{2} (-\eta_3) }u_L \\
     &=&
     \begin{pmatrix}
       e^{-\eta_3/2} & 0 \\
       0 & e^{\eta_3/2}
     \end{pmatrix} 
     \sqrt{2E}
     \begin{pmatrix}
         1 \\ 0
     \end{pmatrix}
     =\sqrt{2E e^{-\eta_3}}
     \begin{pmatrix}
         1 \\ 0
     \end{pmatrix}
     =\sqrt{2E'}
     \begin{pmatrix}
         1 \\ 0
     \end{pmatrix}
 \end{eqnarray}

 \section{ }
 \begin{eqnarray}
    && L_B(0,0,0,0,0,\eta_3) u_L^c
     e^{ B_3       (\eta_3) } u_L
     = e^{ \frac{\sigma^3}{2} \eta_3 }u_L \\
     &=&
     \begin{pmatrix}
       e^{\eta_3/2} & 0 \\
       0 & e^{-\eta_3/2}
     \end{pmatrix} 
     \sqrt{2E}
     \begin{pmatrix}
         1 \\ 0
     \end{pmatrix}
     =\sqrt{2E e^{\eta_3}}
     \begin{pmatrix}
         1 \\ 0
     \end{pmatrix}
     =\sqrt{2E'}
     \begin{pmatrix}
         1 \\ 0
     \end{pmatrix}
 \end{eqnarray}


\section{ }
For
\begin{eqnarray}
    p^\mu = (E, 0, 0, E)^T,
\end{eqnarray}
we have known the corresponding wavefunction
\begin{eqnarray}
    u_L^c=\sqrt{2E} (1,0)^T.
\end{eqnarray}
For a generic 4 momentum turns into
\begin{eqnarray}
    {p'}^\mu
    = E' (1, \sin\theta\cos\phi, \sin\theta\sin\phi, \cos\theta)^T, 
\end{eqnarray}
we can obtain its wavefunction using Lorentz transformation on $u_L^c$
\begin{eqnarray}
    B_z(\eta)u_L^c=e^{-\frac{\sigma^3}{2} \eta}=
    \begin{pmatrix}
      e^{\eta/2} & 0 \\
      0 & e^{-\eta/2}
    \end{pmatrix}
    \sqrt{2E}
    \begin{pmatrix}
        1 \\ 0
    \end{pmatrix}
    =\sqrt{2E'}
    \begin{pmatrix}
        1 \\ 0
    \end{pmatrix}
\end{eqnarray}
\begin{eqnarray}
   R_y(\theta)B_z(\eta)u_L^c&=& e^{-\frac{i \sigma^2}{2} \theta}\sqrt{2E'}
   \begin{pmatrix}
    1 \\ 0
    \end{pmatrix}\\
    &=& \sqrt{2E'}
    \begin{pmatrix}
        \cos \frac{\theta}{2} & -\sin \frac{\theta}{2} \\
        \sin \frac{\theta}{2} & \cos \frac{\theta}{2}
    \end{pmatrix}
    \begin{pmatrix}
        1 \\ 0
    \end{pmatrix}\\
    &=&\sqrt{2E'}
    \begin{pmatrix}
        \cos\frac{\theta}{2} \\ \sin\frac{\theta}{2}
    \end{pmatrix}
\end{eqnarray}
\begin{eqnarray}
    R_z(\phi)  R_y(\theta)B_z(\eta)u_L^c &=& e^{-\frac{i \sigma^3}{2} \theta}\sqrt{2E'}
    \begin{pmatrix}
        \cos\frac{\theta}{2} \\ \sin\frac{\theta}{2}
    \end{pmatrix}\\
    &=& \sqrt{2E'}
    \begin{pmatrix}
        -e^{\frac{i \phi}{2}}\cos\frac{\theta}{2} \\ e^{\frac{i \phi}{2}}\sin\frac{\theta}{2}
    \end{pmatrix}
\end{eqnarray}

\section{ }
\begin{eqnarray}
    h u_L^c&=&\frac{1}{2} 
    \begin{pmatrix}
        \cos\theta & \sin\theta e^{-i\phi} \\ 
        \sin\theta e^{i\phi} & -\cos\theta
    \end{pmatrix}
    \sqrt{2E'}
    \begin{pmatrix}
        -e^{\frac{i \phi}{2}}\cos\frac{\theta}{2} \\ e^{\frac{i \phi}{2}}\sin\frac{\theta}{2}
    \end{pmatrix} \\
    &=& \frac{\sqrt{2E'}}{2}
    \begin{pmatrix}
        e^{-\frac{i \phi}{2}}(\sin\frac{\theta}{2}\sin\theta+\cos\frac{\theta}{2}\cos\theta)\\ -e^{\frac{i \phi}{2}}(\sin\frac{\theta}{2}\cos\theta- \cos\frac{\theta}{2}\sin\theta)
    \end{pmatrix}\\ 
    &=& \frac{\sqrt{2E'}}{2}
    \begin{pmatrix}
        e^{-\frac{i \phi}{2}}\cos \frac{\theta}{2} \\ -e^{\frac{i \phi}{2}}\sin \frac{\theta}{2}
    \end{pmatrix}
    =-\frac{1}{2}u_L^c
\end{eqnarray}

For
\begin{eqnarray}
   \sigma_+^\mu = (1, \sigma_1, \sigma_2, \sigma_3)
\end{eqnarray}
we write
\begin{eqnarray}
    p_\mu\sigma_+^\mu = |{\vec p}| (1- \sigma^1 \sin\theta\cos\phi -\sigma^2\sin\theta\sin\phi - \sigma^3 \cos\theta).
\end{eqnarray}
Applying $p_\mu\sigma_+^\mu$ to $u_L^c$, and note
\begin{eqnarray}
    \begin{pmatrix}
        \cos\theta & \sin\theta e^{-i\phi} \\ 
        \sin\theta e^{i\phi} & -\cos\theta
    \end{pmatrix}
    u_L^c=-u_L^c,
\end{eqnarray}
we have
\begin{eqnarray}
    p_\mu\sigma_+^\mu u_L^c = |{\vec p}| \left[ 1+
    \begin{pmatrix}
        \cos\theta & \sin\theta e^{-i\phi} \\ 
        \sin\theta e^{i\phi} & -\cos\theta
    \end{pmatrix}
        \right] u_L^c =(1-1)u_L^c=0.
\end{eqnarray}

\section{ }
Consider Lorentz transformation
\begin{eqnarray}
    L(\theta_1, \theta_2, \theta_3, \eta_1, \eta_2, \eta_3)
   = L_A(\theta_1, \theta_2, \theta_3, \eta_1, \eta_2, \eta_3)
   \times L_B(\theta_1, \theta_2, \theta_3, \eta_1, \eta_2, \eta_3),
\end{eqnarray}
the rotation parts ($\theta_1,\theta_2,\theta_3$) are exactly same for $L_A$ and $L_B$, while the boost part ($\eta_1,\eta_2,\eta_3$) are real and opposite
between $L_A$ and $L_B$ (so it vanish). This transformation can be rewrite as
\begin{eqnarray}
    R_z(2\phi)R_y(2\theta),
\end{eqnarray}
and it leads
\begin{eqnarray}
    && p^\mu=(E,0,0,E)^T \to {p'}^\mu =E (1, \sin2\theta\cos2\phi, \sin2\theta\sin2\phi, \cos2\theta)^T \\
    && u_L=(0,1)^T\to u_L= \sqrt{2E}
      \begin{pmatrix}
        -e^{-i \phi}\sin \theta \\ e^{i \phi}\cos \theta.
    \end{pmatrix}
\end{eqnarray}
Also we can write the helicity
\begin{eqnarray}
    \frac{1}{2} 
    \begin{pmatrix}
        \cos2\theta & \sin2\theta e^{-2i\phi} \\ 
        \sin2\theta e^{2i\phi} & -\cos2\theta
    \end{pmatrix},
\end{eqnarray} 
which is actually in the same form with what we have done in {\bf hw12-6} with $\theta \to 2\theta$, $\phi \to 2 \phi$, so it is obvious that
\begin{eqnarray}
    h u_L=-\frac{1}{2}u_L
\end{eqnarray}


























\end{document}
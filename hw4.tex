\documentclass[12pt]{article}
\usepackage{amsmath}
\textheight=9.5in \voffset=-1.0in \textwidth=6.5in \hoffset=-0.5in
\parskip=0pt
\begin{document}

\begin{center}
{\large\bf HW4 for Advanced Particle Physics} \\

\end{center}

\vskip 0.2 in

Dear students:\\

  Please let me repeat the rule for grading your reports.
  As I explained to you at the beginning of the last lecture,
  I will read and grade the report which reached my address
  (kaoru.hagiwara@kek.jp) first.  All the other reports will
  be read and graded by our grader, in the order of their
  receipt.  Therefore those reports which reach me at a
  later time may not be read.  Although this may be disappointing
  to you, I would like to encourage you to do all the homework
  and submit your report to me, since it will help you strongly
  in the future, when you start doing research in particle physics.
  All your reports will be kept in my file, even if they are not
  read.  This rule applies from hw03.  I will read and give comments
  to all your first two reports, those for hw01 and hw02, and
  return to you after grading.\\


  Today, I summarized the SU(3)xSU(2)xU(1) invariance of
  the SM Lagrangian with matter fields (quarks and leptons)
  and the Higgs boson (excluding the derivatives and gauge bosons).\\

  We do not consider the space-time transformation of each
  field, and simply assume that the Lagrangian terms as
  written in the SM are Lorentz invariant.  We will study
  the Lorentz transformations in the latter part of my
  lectures.\\

  We treat the $SU(3)\times SU(2)\times U(1)$ symmetry transformations
  as global (space-time independent) transformations in
  this homework. Space-time dependent transformations
  (gauge transformations) are introduced next week.\\

  The transformations are called ``global'' when the angular
  variables, $\theta$'s
\begin{itemize}
    \item $\theta^a$ with $a=1~{\rm to}~8$ for SU(3)
    \item $\theta^i$ with $i=1,2,3$  for SU(2)
    \item $\theta$                for U(1)
\end{itemize}
  are the same everywhere in the spacetime.

  Let us review step by step the invariance of the
  SM Lagrangian (density):
\begin{eqnarray}
    L_{SM}   =   L_{gauge} + L_{fermion} + L_{Higgs} + L_{Yukawa}
\end{eqnarray}

  Let us set asside $L_{gauge}$ untill my next lecture,
  since it appears only when the $SU(3)\times SU(2)\times U(1)$
  phase transformations are local (when all the above 12
  angular variables are space-time dependent).
  $L_{fermion}$ looks like the following (I drop the
  generation index (1,2,3) in this homework, since
  the symmetry properties are identical for all
  the three generations):
\begin{eqnarray}
    L_{fermion}
   = Q^\dagger  iD_\mu \sigma_-^\mu Q
   + u_R^\dagger iD_\mu \sigma_+^\mu u_R
   + d_R^\dagger iD_\mu \sigma_+^\mu d_R
   + L^\dagger  iD_\mu \sigma_-^\mu L
   + l_R^\dagger iD_\mu \sigma_+^\mu l_R
\end{eqnarray}
   
  Let us for the moment pretend that we know that
  the above terms are Lorentz invariant and Hermetian
  (which will be studied a few lectures later), and
  just use the ``definition'' of the covariant derivative,
\begin{eqnarray}
    D_\mu \phi \to (D_\mu \phi)' = U (D_\mu \phi),
\end{eqnarray}
  when the field $\phi$ transforms as,
 $\phi \to \phi' = U \phi$
  In other words, $(D^\mu \phi)$ transform exactly
  the same way as $\phi$ iteself.  This is called
  ``covariant''.

  In the above Lagrangian terms, since $\sigma_\pm^\mu$ are
  just numbers in the guage quantum number space (they
  are matrices in the spinor space, which will be studied
  in later lectures when we consider Lorentz transformations),
  we can write
\begin{eqnarray}
    &&Q \to Q' = UQ \\
    &&iD_\mu \sigma_-^\mu Q \to U (iD_\mu \sigma_-^\mu Q)
  \end{eqnarray}
   and similarly for all the other terms.

  Now, we note that all the above 5 terms in
  $L_{fermion}$ have exactly the same form:
\begin{eqnarray}
    \phi^\dagger \phi
\end{eqnarray}
  for $\phi$ = Q, uR, dR, L, lR.  If $\phi$ transforms as
\begin{eqnarray}
    \phi \to \phi' = U \phi
\end{eqnarray}
   then $\phi^*$ transform as
\begin{eqnarray}
    \phi^* \to \phi'^* = U^* \phi^*
\end{eqnarray}
   and hence
   \begin{eqnarray}
    \phi^\dagger = (\phi-*)^T   \to
    ({\phi'}^*)^T = ((U \phi)^*)^T
    =
    (U \phi)^\dagger
    =
    \phi^\dagger U^\dagger
   \end{eqnarray}
  

  It is then clear that
\begin{eqnarray}
    \phi^\dagger \phi \to {\phi'}^\dagger \phi'
                    \to \phi^\dagger U^\dagger U \phi
                    = \phi^\dagger \phi
\end{eqnarray}
  for all Unitary transformations, that satisfy
\begin{eqnarray}
    U^\dagger U = 1.
\end{eqnarray}
   including $SU(3)_C$, $SU(2)_L$, $U(1)_Y$ and their arbitrary
  combinations.

{\bf hw04-1}:

  The sextet Q has two indices (other than the irrelevant
  generation index which we ignore, and the spinor indices
  which can be fixed if we stay in a fixed Lorentz frame).
  Let us denote those indieces as $i=1,2,3$ for $SU(3)_C$,
  and $k=1,2$ for $SU(2)_L$.  Let us write them explicitly:
  \begin{eqnarray}
    Q_{ik}
  \end{eqnarray}
  

  When we make $SU(3)_C$ transformation, the k index doesn't
  change, and
  $( Q_{1k}, Q_{2k}, Q_{3k} )^T$
  transform as a triplet.  With this notation, please
  show the $SU(3)_C$ transformation of Q
\begin{eqnarray}
    Q \to Q' = U Q
\end{eqnarray}
   with explicit components, i.e., from $Q_{ik}$ to $Q'_{jl}$.

  hint: SU(3) transformation matrix U is a $3 \times 3$ matrix.

  hint2: Matrix multiplication rule is
  \begin{eqnarray}
    ( {\rm Matrix} * {\rm vector} )_m = \sum_n ({\rm Matrix})_{mn} ({\rm vector})_n
  \end{eqnarray}
  when ``vector'' is a column vector.

{\bf hw04-2}:

  Let us examine the $SU(2)_L$ transformation of Q.
  Under $SU(2)_L$, the second $k$ index changes but not
  the first $i$ index.  So, the relevant vector is
  a two vector (k=1 and k=2):
  $( Q_{i1}, Q_{i2} )^T$
  I often write it as
  $( Q_{i1}, Q_{i2} )^T  = ( u_{Li}, d_{Li} )^T$

  Here, the name uL for k=1 components of Q, and
  dL for k=2 componets of Q, which are both color
  triplets with i=1,2,3 indices.  Now, Under SU(2)
  transformation, Q transforms as
  $Q_{ik} \to Q'_{il} = U_{lk} Q_{ik}$
 (Summation over repeated indices, k, which can be written as
  $\sum_{k=1}^{2}$, is understood.)

  Let us examine a specific SU(2) transformation,
\begin{eqnarray}
    U(\theta^1,\theta^2,\theta^3) = U(0,\pi,0),
\end{eqnarray}
    that is a rotation about the 2 axis by $\pi$.
  Please obtain the transformation matrix
  $U(0,\pi,0)$
  and show what happens to our $Q_{ik}$ under
  this transformation.  Please show your results
  by using the $(uL, dL)^T$ notation as above.

{\bf hw04-3}:

  Under $U(1)_Y$, all 6 components of Q transform
  exactly the same way with $Y=1/6$:
\begin{eqnarray}
    Q_{ik} \to Q'_{ik} = U Q_{ik}
                   = e^{iY\theta} Q_{ik}
                   = e^{i(1/6)\theta} Q_{ik}
\end{eqnarray}
    This can be expressed as
\begin{eqnarray}
    Q \to Q' = e^{i(1/6)\theta} Q
\end{eqnarray}
    for brevity.  Now, let us consider a special
  $U(1)_Y$ transformation with $\theta=3\pi$.
  Please show the transformation of all 5 fermions:
\begin{eqnarray}
    &&  Q \to Q'  = U_Y(3\pi) = ...\\
    && u_R \to u_R' = U_Y(3\pi) = ...\\
    && d_R \to d_R' = U_Y(3\pi) = ...\\
    && L  \to L'  = U_Y(3\pi) = ...\\
    && l_R \to l_R' = U_Y(3\pi) = ...
\end{eqnarray}
 Now, let us go on to the Higgs Lagrangian,
\begin{eqnarray}
    L_{\rm Higgs}
  = (D^\mu \phi)^\dagger (D_\mu \phi) - V(\phi),
\end{eqnarray}
  where $V(\phi)$ is the Higgs potential
\begin{eqnarray}
    V(\phi)
  =
  \lambda/4 (\phi^\dagger \phi)^2 + \mu^2 (\phi^\dagger \phi)
\end{eqnarray}
  

{\bf hw04-4}:

  Show that $\phi^\dagger \phi$ is invariant under
  $SU(3)_C$, $SU(2)_L$, and $U(1)_Y$.

{\bf hw04-5}:

  Prove the following identity:
\begin{eqnarray}
    \phi^\dagger \phi = (\phi^c)^\dagger (\phi^c)
\end{eqnarray}
   Once the above identity is proven, the Higgs potential $V(\phi)$
  can now be written in terms of $\phi$ and $\phi^c$, through
  the combination
  \begin{eqnarray}
    \phi^\dagger \phi
    =
    (1/2)[(\phi^\dagger \phi)+(\phi^c)^\dagger (\phi^c)]
  \end{eqnarray}
  This fact plays an extremely important role in the SM,
  when compared to the data.

{\bf hw04-6}:

  Once the Higgs potential of the SM is expressed as above,
  show that the potential is invariant under a new global
  SU(2) transformation, which transform
$ ( \phi, \phi^c )^T$
  to
  $( \phi', {\phi^c}' )^T
  =
  U ( \phi, \phi^c )^T$
  This symmetry of the SM Higgs potential was identified
  first by t'Hooft, whose meaning will be clear when I explain
  the spontaneous breaking of the $SU(2)_L \times U(1)_Y$
  symmetry in a later lecture.  This symmetry of the SM
  Higgs potential has very important phenomenological
  consequences, which have been proven to be true
  by experiments.

Now, let us move on to the Yukawa term:
\begin{eqnarray}
    L_{Yukawa}
    = {y_u Q^\dagger u_R \phi^c + h.c.}
    + {y_d Q^\dagger d_R \phi   + h.c.}
    + {y_l L^\dagger l_R \phi   + h.c.}
\end{eqnarray}
 

{\bf hw04-7}:

  Show the $SU(3)_C$ invariance of each term.

{\bf hw04-8}:

  Show the $SU(2)_L$ invariance of each term.

{\bf hw04-9}:

  Show the $U(1)_Y$ invariance of each term.

{\bf hw04-10}:

  Assume that the couplings, $y_u$, $y_d$, $y_l$ are all
  complex numbers, please write down the hermetian
  conjugate terms ( $+ h.c.$ terms) explicitly.\\

  Finally, neutrinos.  Because the origin of the
  neutrino masses are not yet known, let me give
  my personal version of the SM Lagrangian which
  does NOT introduce a 6'th multiplet (or the 16'th
  fermion) of each generation, the right-handed neutrino,
  which is usually written as $\nu_R$.  Instead, we
  introduces higher dimensional interactions into
  the SM Lagrangian.  (The meaning of interactions
  whose mass dimension is greater than 4 can be
  understood clearly only after you learn quantization
  of our classical Lagrangian, how quantum corrections
  are calculated, and how the corrected results can be
  compared with data.  The key words for those matters
  are `renormalization' and effective field theories.
  Those subjects were covered in the advanced course
  of my KEK lectures.)\\

  It is written as follows:
\begin{eqnarray}
    L_{neutrino}
  =
  y_\nu ((\phi^c)^\dagger L) \cdot ((\phi^c)^\dagger L)/2\Lambda
  + h.c.,
\end{eqnarray}
    where $\Lambda$ is a large mass as compared to the SM scale,
  $v =$ 256 GeV.  The neutrino masses are inversely proportional
  to this scale $\Lambda$.  The simbol ``$\cdot$'' is a secret
  product that contracts the fermionic indices (of two L's)
  to make the above term invariant under the Lorentz
  transformation, which will be explained carefully in my
  later lectures.

{\bf hw04-11}:

  Show that the term
 $ (\phi^c)^\dagger L$
  is invariant under $SU(3)_C$, $SU(2)_L$, and $U(1)_Y$.

{\bf hw04-12}

  Please write down $(\phi^c)^\dagger L$, in terms of the
  component fields, by using
\begin{eqnarray}
    L &=& ( \nu_L, l_L)^T\\
    \phi &=& ( \phi^+, \phi^0 )^T
\end{eqnarray}
  Please note that both $\phi^+$ and $\phi^0$ are complex fields,
  and we may write
\begin{eqnarray}
    (\phi^+)^* &=& \phi^-\\
  (\phi^0)^* &=& \phi^0*
\end{eqnarray}
  
{\bf hw04-13}

  Please write down all the terms which appear in $L_{neutrino}$
  in terms of the component fields, by ignoring the $\cdot$
  simbol which makes the two fermion product Lorentz invariant.
  In other words, please write down the following expressions
\begin{eqnarray}
    L_{neutrino}
  =
  y_\nu [ (\phi^c)^\dagger L) ]^2/2\Lambda + h.c.
\end{eqnarray}
    in terms of the component fields.  Can you observe the term
  which is proportional to
  $(\nu_L)^2$ ?
  It gives the neutrino Mayorana mass when the Higgs boson
  aquires the vacuum expectation value.

That's all for hw04.\\

Best regards,\\

Kaoru




\end{document}

\documentclass[12pt]{article}
\usepackage{amsmath,graphicx,color,epsfig,physics}
\usepackage{float}
\usepackage{subfigure}
\usepackage{slashed}
\usepackage{color}
\usepackage{multirow}
\usepackage{feynmp}
\textheight=9.5in \voffset=-1.0in \textwidth=6.5in \hoffset=-0.5in
\parskip=0pt
\def\del{{\partial}}


\begin{document}

\begin{center}
{\large\bf HW9 for Advanced Particle Physics} \\

\end{center}

\vskip 0.2 in

Dear students,\\

Today, I explained quark and lepton masses and their mixings.
The following points are important:

\begin{enumerate}
  \item Diagonalization of quark and lepton mass matrices give two Unitary
  transformations (one each for Left-handed and Right-handed 
components) between the quark and lepton mass eigenstates and the states which
  appear in the Lagrangian before the symmetry breakdown (current 
states).

 \item In charged currents (that couple to $W^+$ and $W^-$ bosons), there
 appear Unitary mixing between different generations of quarks and
 leptons.  They are called $V_{CKM}$ and $V_{MNS}$.

 \item In neutral currents (that couple to $Z$ and $\gamma$), there do not
 appear couplings between different generations (absence of FCNC).
 This is because all the fermions (of different generations) with
 the same electric charge has exactly the same representations under
 $SU(2)_L \times U(1)_Y$.  This was first observed by GIM in 1971, 4 years
 after Weinberg presented his ``model for leptons'' in 1967. \label{item.9C}

 \item The SM Higgs boson does not have couplings between quarks and
 leptons of different generations (absence of FCNC).  This is because
 the same Unitary matrices that rotates the current states to the
 mass eigenstates rotates the Higgs-fermion-fermion couplings. \label{item.9D}

 The rules \ref{item.9C} and \ref{item.9D} of the SM (absence of FCNC in $Z$ and Higgs boson
 couplings) give a powerful tool to look for physics beyond the SM.
 The rule \ref{item.9D} will be discussed in the next lecture/hw10.

\end{enumerate}
  {\bf Now, let us go ahead studying the Yukawa interactions}:

  \begin{eqnarray}
    {\cal L}_{Yukawa} =&& \sum_{i,j} y^u_ij Q_i^\dagger \phi^c uR_j
                          + y^d_ij Q_i^\dagger \phi   dR_j
                          \nonumber \\&&+ y^l_ij L_i^\dagger \phi   lR_j
                          + y^n_ij (L_i*\phi)*(L_j*\phi)/2\Lambda
                          + h.c.
  \end{eqnarray}
   where the summation are over generations $i,j = 1,2,3$.
  In the last term I introduced the $*$ products,
\begin{eqnarray}
  L_i*\phi = ({n_L}_i, {l_L}_i) i\sigma^2 (\phi^+, \phi^0)^T
               = ({n_L}_i)(\phi^0) - ({l_L}_i)(\phi^+)
\end{eqnarray}
  among the weak $SU(2)$ doublets, and
\begin{eqnarray}
  L_i*L_j = 
  \begin{pmatrix}
    {L_i}_1 & {L_i}_2
  \end{pmatrix}
  i\sigma^2 
  \begin{pmatrix}
    {L_j}_1 \\{L_j}_2
  \end{pmatrix}
  = {L_i}_1 {L_j}_2 - {L_i}_2 {L_j}_1 \label{eq.9Lij}
\end{eqnarray}
  among the two left-handed spinor doublets (Weyl spinors).  Note that
  the last expression of Eq.\ref{eq.9Lij} does not vanish because $L$'s are
  anti-commuting (fermionic) operators.  That Eq.\ref{eq.9Lij} (as well as all the other mass terms and the $h-f-f$ couplings) is invariant under the
  Lorentz transformation will be proven in my lectures in the next 
week.

  In the Unitary gauge we can denote
  \begin{eqnarray}
    \phi&=&
    \begin{pmatrix}
      0 \\ (v+h)/\sqrt2
    \end{pmatrix} \\
    \phi^c&=&
    \begin{pmatrix}
      (v+h)/\sqrt2  \\0 
    \end{pmatrix} 
  \end{eqnarray}

  The mass terms are then
\begin{eqnarray}
  {\cal L}_{Yukawa}
    &=& \sum_{i,j} y^u_{ij} (v/\sqrt2) {u_L}_i^\dagger {u_R}_j
               + y^d_{ij} (v/\sqrt2) {d_L}_i^\dagger {d_R}_j
               + y^l_{ij} (v/\sqrt2) {l_L}_i^\dagger {l_R}_j
               \nonumber \\ &&+ y^n_{ij} (v^2/4\Lambda) {n_L}_i*{n_L}_j
               + h.c. \\
    &=& \sum_{i,j} -M^u_{ij} {u_L}_i^\dagger {u_R}_j
                 -M^d_{ij} {d_L}_i^\dagger {d_R}_j
                 -M^l_{ij} {l_L}_i^\dagger {l_R}_j
                 -M^n_{ij} {n_L}_i*{n_L}_j/2
                 + h.c.
\end{eqnarray}
  where I introduce the mass matrices $M^u$, $M^d$, $M^l$, $M^\nu$.
  $M^u$, $M^d$, $M^l$ are generic complex 3 by 3 matrices, whereas
  $M^\nu$ is a symmetric complex 3 by 3 matrix.

{\bf hw09-1}: Can you tell why $M^n$ is a symmetric matrix?
  They can be diagonalized by using 3 by 3 unitary matrices:
\begin{eqnarray}
  M^u &=& (U^u_L)   {\rm diag}\{m_u, m_c,m_t\} (U^u_R)^\dagger \\
  M^d &=& (U^d_L)   {\rm diag}\{m_d, m_s,  m_b  \} (U^d_R)^\dagger \\
  M^l &=& (U^l_L)   {\rm diag}\{m_e, m_\mu, m_\tau\} (U^l_R)^\dagger \\
  M^\nu &=& (U^\nu_L)^* {\rm diag}\{m_1, m_2,  m_3  \} (U^\nu_L)^\dagger
\end{eqnarray}
  The two unitary matrices to diagonalize the general matrix such as
  $M^u$, $M^d$, $M^l$, are obtained as the unitary matrix $(U_R)$ which
  diagonalizes the Hermetian matrix $M^\dagger M$ and the unitary matrix
  $(U_L)$ which diagonalizes $M M^\dagger$.  Since the eigenvalues of the
  Hermetian matrices, $M^\dagger M$ and $M M^\dagger$ are common and
  non-negative, we can take their square root as the quark and charged
  lepton masses.

{\bf hw09-2}: Show that the unitary matrix $U^u_L$ diagonalizes the 
Hermetian matrix ${M^u}^\dagger M^u$, and the unitary matrix $U^u_R$ 
diagonalizes $(M^u) (M^u)^\dagger$.

  The unitary matrix which diagonalizes the symmetric complex matrix 
for the neutrino Majorana masses can be obtained easily by solving the
  standard eigen value equations, but the eigen values are in general
  complex.  Let us assume that the 3 eigenvalues are found to be
\begin{eqnarray}
  \lambda_k = m_k e^{i \alpha_k}   (k=1,2,3)
\end{eqnarray}
  by using the magnitude $m_k = | \lambda_k |$ and the phase, $\alpha_k
  = {\rm arg}\{ \lambda_k \}$. We can then express the diagonal matrix as
 \begin{eqnarray}
  {\rm diag}\{ \lambda_1, \lambda_2, \lambda_3 \}
  = P {\rm diag}\{ m_1,        m_2,        m_3        \} P,
 \end{eqnarray}
  where P is the diagonal phase matrix,
\begin{eqnarray}
  P =  {\rm diag}\{ e^{i\alpha_1/2}, e^{i\alpha_2/2}, e^{i\alpha_3/2}\}
\end{eqnarray}
  If we denote the unitary matrix which we obtained from the three
  eigen vectors as ${U^\nu_L}'$,
\begin{eqnarray}
  M^\nu &=& {{U^\nu_L}'}^*  {\rm diag}\{\lambda_1, \lambda_2, \lambda_3\}  {{U^\nu_L}'}^\dagger \\
  &=& {{U^\nu_L}'}^* P {\rm diag}\{m_1, m_2, m_3\} P {{U^\nu_L}'}^\dagger \\
  &=& {U^\nu_L}^*     {\rm diag}\{m_1, m_2, m_3\}  {U^\nu_L}^\dagger
\end{eqnarray}
  and hence (please note the ordering of ${U^\nu_L}'$ and $P$)
  \begin{eqnarray}
    U^\nu_L = {U^\nu_L}' P^* \label{eq.9unm}
  \end{eqnarray}

{\bf hw09-3}: Show that the unitary matrix $U^\nu_L$ (Eq.\ref{eq.9unm}) diagonalize the (Majorana) neutrino mass matrix, $M^\nu$.
  Finally, we can express all the current fermions in terms of
  mass eigenstates by using the 7 Unitary matrices:
\begin{eqnarray}
  ({u_L}_1, {u_L}_2, {u_L}_3)^T &=& U^u_L (u_L, c_L, t_L)^T \label{eq.9unim1}\\
  ({u_R}_1, {u_R}_2, {u_R}_3)^T &=& U^u_R (u_R, c_R, t_R)^T \label{eq.9unim2}\\
  ({d_L}_1, {d_L}_2, {d_L}_3)^T &=& U^d_L (d_L, s_L, b_L)^T \label{eq.9unim3}\\
  ({d_R}_1, {d_R}_2, {d_R}_3)^T &=& U^d_R (d_R, s_R, b_R)^T  \label{eq.9unim4}\\
  ({l_L}_1, {l_L}_2, {l_L}_3)^T &=& U^l_L (e_L, \mu_L, \tau_L)^T \label{eq.9unim5}\\
  ({l_R}_1, {l_R}_2, {l_R}_3)^T &=& U^l_R (e_R, \mu_R, \tau_R)^T \label{eq.9unim6}\\
  ({\nu_L}_1, {\nu_L}_2, {\nu_L}_3)^T &=& U^n_L ({\nu_1}_L, {\nu_2}_L, {\nu_3}_L)^T \label{eq.9unim7}
\end{eqnarray}
  By inserting the above expressions into the SM Lagrangian before the
  symmetry breakdown, we can obtain all the gauge and Higgs interactions of quark and lepton mass eigenstates.
\begin{eqnarray}
  (u, c, t),~~ (d, s, b), ~~ (e, \mu, \tau),~~ (\nu_1, \nu_2, \nu_3)
\end{eqnarray}
  Please first note that all the neutral gauge boson ($\gamma$, $Z$, 
gluons!) interactions are diagonal in the mass-eigenstates, simply because 
the same unitary matrix appears always as a pair like
\begin{eqnarray}
  (U^x_L)^\dagger (U^x_L) &=& 1   ~~~for~ x=u,d,l,\nu \\
  (U^x_R)^\dagger (U^x_R) &=& 1   ~~~for~ x=u,d,l
\end{eqnarray}

{\bf hw09-4}: Show the $\gamma (A_\mu)$, $Z_\mu$, and gluon ($A^a_\mu$) 
interactions of $x=u_L$, $u_R$, $d_L$, $d_R$, $l_L$, $l_R$, $\nu_L$, in terms of the mass eigenstates fermions.

hint: In the original Lagrangian, they are
\begin{eqnarray}
  L_{fermions} =  Q_i^\dagger iD_\mu \sigma_-^\mu Q_i
  + {u_R}_i^\dagger iD_\mu \sigma_+^\mu {u_R}_i
  + {d_R}_i^\dagger iD_\mu \sigma_+^\mu {d_R}_i
  +  L_i^\dagger iD_\mu \sigma_-^\mu L_i
  + {l_R}_i^\dagger iD_\mu \sigma_+^\mu {l_R}_i
\end{eqnarray}
  where $i=1,2,3$ are generation indices. The covariant derivative 
reads
\begin{eqnarray}
  iD_\mu = i\del_\mu -g_Z (T^3-Qx_w) Z_\mu -e Q A_\mu -g_S T^a A^a_\mu
\end{eqnarray}
  with $x_w=\sin^2\theta_W$, for $Z$, $\gamma$ and gluons. Please note that
\begin{eqnarray}
  T^3-Qx_w &=&  1/2-( 2/3)x_w ~for~ {u_L}_i (i=1,2,3) \\
  T^3-Qx_w &=&     -( 2/3)x_w ~for~{u_R}_i (i=1,2,3) \\
  T^3-Qx_w &=& -1/2-(-1/3)x_w ~for~ {d_L}_i (i=1,2,3) \\
  T^3-Qx_w &=&     -(-1/3)x_w ~for~ {d_R}_i (i=1,2,3) \\
  Q &=&  2/3 ~for~ {u_L}_i ~and~ {u_R}_i (i=1,2,3) \\
  Q &=& -1/3 ~for~ {d_L}_i ~and~ {d_R}_i (i=1,2,3) \\
  Q &=& -1   ~for~ {l_L}_i ~and~ {l_R}_i (i=1,2,3)
\end{eqnarray}

  The necessary condition to ensure this diagonality (the absence of 
FCNC) is to have the unitary relations between the current states and the
  mass eigenstates Eq.(\ref{eq.9unim1},\ref{eq.9unim2},\ref{eq.9unim3},\ref{eq.9unim4},\ref{eq.9unim5},\ref{eq.9unim6},\ref{eq.9unim7}).

{\bf hw09-5}: When the charm quark was not known, most physicists thought
  that there are only 3 quarks (Gell-Mann's $SU(3)$).  Two down
  (charge $-1/3$) quarks, $d$ and $s$, and one charge $2/3$ quark, $u$.
  The Lagrangian (which Weinberg should have tried in 1967) should have been something like
\begin{eqnarray}
  L =  Q^\dagger  iD_\mu \sigma_-^\mu Q
  + {s_L'}^\dagger iD_\mu \sigma_-^\mu s_L'
  + u_R^\dagger  iD_\mu \sigma_+^\mu u_R
  + d_R^\dagger  iD_\mu \sigma_+^\mu d_R
  + s_R^\dagger  iD_\mu \sigma_+^\mu s_R
\end{eqnarray}
  where $Q = (u_L, d_L')^T$ with $d_L' = \cos\theta_C d_L + \sin\theta_C s_L$
  which was found by Cabibbo from the charged current analysis (Cabibbo
  Universality between the leptonic and hadronic weak interactions was
  the first hint of the Universality of weak interactions, which motivated
  Weinberg and others to examine the Gauge theory of weak interactions).
  $s_L'$ should be its orthogonal combination,
\begin{eqnarray}
  s_L' = -\sin\theta_C d_L + \cos\theta_C s_L
\end{eqnarray}
  Here all quarks without prime ($u_L$,$d_L$,$s_L$,$u_R$,$d_R$,$s_R$) are in the mass eigenstates, where as $d_L'$ and $s_L'$ are the current states.  Please insert the covariant derivative for the neutral currents,
\begin{eqnarray}
  iD_\mu = i\del_\mu -g_Z (T^3-Qx_w) Z_\mu -e Q A_\mu -g_S T^a A^a_\mu
\end{eqnarray}
 and obtain the interactions among all the quarks. You may find the
  $Z$ boson coupling between $d$ and $s$ quarks, proportional to 
$\sin\theta_C$, which disturbed Weinberg, because FCNC between $d$ and $s$ quarks are known to be extremely small (determined from $K^0-{\overline K^0}$ mixing matrix elements, which give extremely small mass differences between the
  two mass eigenstates, $K_L$ and $K_S$).

{\bf hw09-6}:  Consult RPP in the PDG webpage, look for meson summary 
table, and obtain $m(K^+)$, $m(K_L)$, $m(K_S)$, and $R=(m(K_L)-m(K_S))/(m(K_L)+m(K_S))$. Please compare the observed ratio $R$ with a naive estimate
\begin{eqnarray}
  R = ((g_Z  \sin\theta_C)^2/m_Z^2)^2 (m_K)^4
\end{eqnarray}
   Please note at this stage that the current state $Q=(u_L, d_L')^T$ is
  a doublet under $SU(2)_L$ but the current state $s_L'$ is a singlet.
  In other words, among the two quarks of charge $-e/3$, $d_L'$ and $s_L'$,
  one is a doublet, and the other is a singlet under $SU(2)_L$.
  This violates the no-FCNC condition.  GIM introduced a new quark,
  $c_L$ and $c_R$ in the current states, and made sL's into a down component
  a new doublet,
\begin{eqnarray}
  Q' = (c_L, s_L')^T.
\end{eqnarray}
 
  In this way, GIM introduced a 2nd up quark, $c$, and made the $2\times 2$
  version of the unitary transformations Eq.(\ref{eq.9unim1},\ref{eq.9unim2},\ref{eq.9unim3},\ref{eq.9unim4}). Once this is done, the neutral currents are automatically flavor diagonal (there is no $d-s-Z$ couplings in the Lagrangian).

  FCNC can still appear in QFT via quantum/radiative corrections.
  Benjamin Lee and Mary Gaillard used the quantization rules obtained
  by tHooft in 1971 to perform the calculation in the GIM model (with
  a massive c quark).  From the one-loop contribution to the $K_0-{\overline K_0}$ transition amplitudes, they estimated the charm mass to be around
  2 GeV in 1972.  The charmonium particle $J/\psi$ was discovered in 
1974.

  Let us now compute the charged current interactions in the SM by
  using the covariant derivative expression:
\begin{eqnarray}
  iD_\mu &=& i\del_\mu - g_s T^a A^a_\mu
                        - g   T^k W^k_\mu
                        - g'  Y   B_\mu \nonumber \\
             &=&i\del_\mu - g_s T^a A^a_\mu
                        - g/\sqrt2 ( T^+ W^+_\mu + T^- W^-_\mu )
                        \nonumber \\ &&- g_Z (T^3 -Q\sin^2\theta_W) Z_\mu
                        - e Q A_\mu,
\end{eqnarray}
  where $e = g\sin \theta _W = g_Z\sin\theta _W \cos\theta _W = g'\cos\theta _W$, and $A_\mu$ denotes the photon. Only $W^+_\mu$ and $W^-_\mu$ mediate charged currents in the SM.  From the Lagrangian term
\begin{eqnarray}
 {\cal L} = Q^\dagger iD_\mu \sigma_-^\mu Q
  + L^\dagger iD_\mu \sigma_-^\mu L,
\end{eqnarray}
    we pick up the charged current parts for quarks:
\begin{eqnarray}
  {\cal L} _{CC}
  &=& 
  \begin{pmatrix}
    u_L^\dagger & d_L^\dagger
  \end{pmatrix} 
  \frac{-g}{\sqrt2} \sigma_-^\mu
  \begin{pmatrix}
    0 & W^+_\mu\\
    W^-_\mu & 0
  \end{pmatrix}
  \begin{pmatrix}
    u_L \\ d_L
  \end{pmatrix}\\ 
    &=& \frac{-g}{\sqrt2} ( u_L^\dagger \sigma_-^\mu W^+_\mu d_L
              + d_L^\dagger \sigma_-^\mu W^-_\mu u_L )
\end{eqnarray}

Now, let us insert the mass-eigenstates by using (49a-b):
\begin{eqnarray}
  {\cal L} _{CC} &=& \frac{-g}{\sqrt2}
  \begin{pmatrix}
    u_L^\dagger & c_L^\dagger & t_L^\dagger
  \end{pmatrix}
  {U^u_L}^\dagger \sigma_-^\mu W^+_\mu U^d_L 
  \begin{pmatrix}
    d_L \\ s_L \\ b_L 
  \end{pmatrix}
  + h.c. \\
  &=& \frac{-g}{\sqrt2} 
  \begin{pmatrix}
    u_L^\dagger & c_L^\dagger & t_L^\dagger
  \end{pmatrix}
\sigma_-^\mu W^+_\mu V 
\begin{pmatrix}
  d_L \\ s_L \\ b_L 
\end{pmatrix}
  + h.c., \label{eq.9LCC}
\end{eqnarray}
where
\begin{eqnarray}
  V = V_{CKM} = {U^u_L}^\dagger (U^d_L) \label{eq.9CKM}
\end{eqnarray}


{\bf hw09-7}: Show Eq.\ref{eq.9LCC} with Eq.\ref{eq.9CKM}.

  Let us examine here what Kobayashi and Maskawa found in 1973.
  The mixing matrix V is a unitary $2 \times 2$ matrix in the 4-quark model of
  GIM.  Let us write it down.  From Eq.\ref{eq.9LCC}, we find
\begin{eqnarray}
  {\cal L}_{CC}
  =
  \frac{-g}{\sqrt2}  W^+_\mu 
  \begin{pmatrix}
    u_L^\dagger & c_L^\dagger
  \end{pmatrix}
  \sigma_-^\mu V 
  \begin{pmatrix}
    d_L \\s_L
  \end{pmatrix}+ h.c.
\end{eqnarray}
  Let us count the number of independent physical parameters (the
  parameter whose values affect observable consequences of the model)
  inside the $2 \times 2$ unitary matrix $V$.
  Since $V$ is a $2 \times 2$ matrix of complex numbers, it has $2\times 2\times 2=8$ real numbers.
  The unitarity condition $V^\dagger V=1$ gives 4 constraints.

{\bf hw09-8}:  Why $V^\dagger V = 1$ gives 4 constraints among real numbers?
  4 real d.o.f. (degree of freedom) remain.  In $2\times 2$ unitary matrix,
  there is only 1 rotation angle (which is the Cabibbo angle 
$\theta_C$). The remaining 3 real degree of freedom should hence be phases.  We may hence write
\begin{eqnarray}
   V=
   \begin{pmatrix}
    e^{i \phi_{uL}} & e^{i\phi_{cL}}
   \end{pmatrix}
   \begin{pmatrix}
    \cos\theta_C & \sin\theta_C \\
    -\sin\theta_C & \cos\theta_C 
   \end{pmatrix}
   \begin{pmatrix}
    e^{-i \phi_{dL}} \\ e^{-i \phi_{sL}}
   \end{pmatrix}
\end{eqnarray}
  with 4 phases. Among the 4 phases, one overall phase, common to all
  of them ($\phi_{uL}=\phi_{cL}=\phi_{dL}=\phi_{sL}$) does not have observable
  consequences, because physics is invariant under its magnitude due
  to quark number conservation.

{\bf hw09-9}: Please show that the common phase among all the quark fields
  do not change $V$, and in addition, does not change any other terms in
  the whole SM Lagrangian (quark number conservation).

  We remove this overall phase, and keep the three relative phases:
\begin{eqnarray}
  V
  =
  \begin{pmatrix}
    1 & e^{i\phi_{cL}}
  \end{pmatrix}
  \begin{pmatrix}
    \cos\theta_C & \sin\theta_C \\
    -\sin\theta_C & \cos\theta_C 
   \end{pmatrix}
   \begin{pmatrix}
    e^{-i \phi_{dL}} \\ e^{-i \phi_{sL}}
   \end{pmatrix}
\end{eqnarray}
  
  Now, we have 1 angle and 3 phases, which agree with the d.o.f. 
counting. Please note that we can absorb all the 3 phases by shifting the $c_L$, $d_L$, $s_L$ quark fields as follows:
\begin{eqnarray}
  &&c_L \to  e^{i\phi_{cL}} c_L'\\
  &&d_L \to  e^{i\phi_{dL}} d_L'\\
  &&s_L \to  e^{i\phi_{sL}} s_L'
\end{eqnarray}
 
{\bf hw09-10}: Please show that not only the CC interactions but all the 
other gauge interactions of the SM Lagrangian remain the same when 
expressed in terms of $c_L'$, $d_L'$, $s_L'$.

  Let us now examine the Yukawa sector.  The Lagrangian after
  diagonalization reads
\begin{eqnarray}
  {\cal L}
  =
  m_u u_L^\dagger u_R + m_c c_L^\dagger c_R + m_d dL^\dagger d_R + m_s sL^\dagger s_R + h.c.
\end{eqnarray} 

  In terms of the new fields, $c_L'$, $d_L'$, $s_L'$, it reads
\begin{eqnarray}
  {\cal L} = m_u u_L^\dagger u_R
  + m_c e^{-i\phi_cL} c_L'^\dagger c_R
  + m_d e^{-i\phi_dL} d_L'^\dagger d_R
  + m_s e^{-i\phi_sL} s_L'^\dagger s_R
  + h.c.
\end{eqnarray}
  which looks different, but we notice that the gauge interactions
  of the SM do not change by shifting the phases of $q_R$'s,
\begin{eqnarray}
  c_R \to e^{i\phi_cL} c_R' \\
  d_R \to  e^{i\phi_dL} d_R' \\
  s_R \to  e^{i\phi_sL} s_R'
\end{eqnarray}

  Once we rewrite all the interactions in terms of $c_L'$, $c_R'$, $d_L'$, $d_R'$, $s_L'$, $s_R'$, all the 3 phase disappear.  Therefore, (4-1) phases can be absorbed into re-phasing of the quark fields. None of the 3 phases
  can affect physical observables. Therefore, the weak interactions 
of the 4-quark model is CP conserving (no phase in the interactions).

{\bf hw09-11}: Repeat the same countings for 6-quark model, for $3 \times 3$ Unitary matrix $V$. Can you tell that in addition to 3 angles (of 
3-dimensional rotations), 1 phase survive as a physical phase, whose magnitude
  affects physical observables?  This is what Kobayashi and Masukawa
  showed in 1973.

  Now, let us go on to the leptons:
\begin{eqnarray}
  {\cal L}_{CC} &=&  \frac{-g}{\sqrt2}
  \begin{pmatrix}
    e_L^\dagger &\mu_L^\dagger & \tau_L^\dagger
  \end{pmatrix}
  {U^l_L}^\dagger \sigma_-^\mu W^-_\mu U^n_L
  \begin{pmatrix}
    {\nu_1}_L \\ {\nu_2}_L \\ {\nu_3}_L
  \end{pmatrix}
    + h.c. \\ 
  &=& \frac{-g}{\sqrt2} 
  \begin{pmatrix}
    e_L^\dagger & \mu_L^\dagger & \tau_L^\dagger
  \end{pmatrix}
  \sigma_-^\mu W^-_\mu V + h.c.
\end{eqnarray}
  where
\begin{eqnarray}
  V = V_{MNS} = (U^l_L)^\dagger (U^n_L) \label{eq.9MNS}
\end{eqnarray}


{\bf hw09-12}: Show Eq.\ref{eq.9MNS}.

  Please note that $V_{CKM}$ is defined as a mixing for the down quarks,
\begin{eqnarray}
   \begin{pmatrix}
    d'_L \\ s'_L \\ b'_L
   \end{pmatrix}
   = V_{CKM}
   \begin{pmatrix}
    d_L \\ s_L \\ b_L
   \end{pmatrix}
\end{eqnarray}
 whereas $V_{MNS}$ is defined for a mixing for the neutrinos:
\begin{eqnarray}
  \begin{pmatrix}
    {\nu_e}_L \\ {\nu_\mu}_L \\ {\nu_\tau}_L
   \end{pmatrix} 
   =V_{MNS}
   \begin{pmatrix}
    {\nu_1}_L\\ {\nu_2}_L \\ {\nu_3}_L
   \end{pmatrix}. \label{eq.9numix}
\end{eqnarray}
   This has the historical origin, where
\begin{eqnarray}
  d'_L = V_{11} d_L + V_{12} s_L + V_{13} b_L
          = \cos\theta_C d_L + \sin\theta_C s_L + \cdots 
\end{eqnarray}
  is the Cabibbo $d'$ combination which reproduced the universality of 
the weak charged current strengths, for the quarks.  For the leptons,
  untill the neutrino oscillation was discovered in 1998 by
  Super-Kamiokande (2015 Nobel Prize for Kajita), only the current
  states, in the left-hand-side of Eq.\ref{eq.9numix} were known. It was then
  natural to express the known neutrino current states in terms of
  the mass-eigenstates.

That's all for hw09.\\

Best regards,\\

Kaoru

\end{document}

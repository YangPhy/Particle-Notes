\documentclass[12pt]{article}
\usepackage{amsmath,graphicx,color,epsfig,physics}
\usepackage{float}
\usepackage{subfigure}
\usepackage{slashed}
\usepackage{color}
\usepackage{multirow}
\usepackage{feynmp}
\textheight=9.5in \voffset=-1.0in \textwidth=6.5in \hoffset=-0.5in
\parskip=0pt
\def\del{{\partial}}


\begin{document}

\begin{center}
{\large\bf HW13 for Advanced Particle Physics} \\

\end{center}

\vskip 0.2 in

Dear students, \\

  This week, I introduce the Dirac fermion and its equation of motion
  the Dirac equation.  We reconstruct Dirac equations by combining two
  Weyl fermions, $\psi_L$ transforming under $S_L$, and $\psi_R$ transforming
  under $S_R$,
\begin{eqnarray} 
    \psi_L \to \psi_L' = S_L \psi_L \label{eq.13_1a}  \\
    \psi_R \to  \psi_R' = S_R \psi_R \label{eq.13_1b}
\end{eqnarray}
  where
\begin{eqnarray}
    S_L(\theta_k,\eta_k)
  = e^{\sum_k \sigma^k (-i\theta_k -\eta_k)/2}\label{eq.13_2a} \\ 
  S_R(\theta_k,\eta_k)
  = e^{\sum_k \sigma^k (-i\theta_k +\eta_k)/2} \label{eq.13_2b}
\end{eqnarray}

{\bf hw13-1}: Please confirm that the operators in Eq.\ref{eq.13_2a} and Eq.\ref{eq.13_2b} are the
  Lorentz transformations $L_A$ and $L_B$, respectively, which we
  introduced in the last lectures. Because
\begin{eqnarray}
  (S_L)^\dagger (S_R) = (S_R)^\dagger (S_L) = 1 \label{eq.13_2}
\end{eqnarray}

{\bf hw13-2}: Show Eq.\ref{eq.13_2}, or just explain by words if it is obvious to you.

  We can form an invariant mass term (the coupling between two Weyl
  fermions, $\psi_L$ and $\psi_R$)
\begin{eqnarray}
  {\cal L}_{Dirac~mass}
  = -m (\psi_L^\dagger \psi_R + \psi_R^\dagger \psi_L) \label{eq.13_3}
\end{eqnarray}

  Because $\psi_L$ and $\psi_R$ in Eq.(\ref{eq.13_1a}, \ref{eq.13_1b}) are intrdouced as independent fields,
  we can assign them the same charge for any phase transformations,
  including the $U(1)_{EM}$ transformation:
\begin{eqnarray}
  &&\psi_L \to \psi_L' = e^{iQ\theta} \psi_L  \label{eq.13_4a} \\
  &&\psi_R \to  \psi_R' = e^{iQ\theta} \psi_R \label{eq.13_4b}
\end{eqnarray}
  Here, we assign the same charge, $Q$, for both $\psi_L$ and $\psi_R$ fields.

{\bf hw13-3}: Show that the mass term in Eq.\ref{eq.13_3} is invariant under the $U(1)_{EM}$
  phase transformations Eq.(\ref{eq.13_4a}, \ref{eq.13_4b}).
  In case of Majorana mass term,
\begin{eqnarray}
  {\cal L}_{Majorana~mass}
  = -\frac{m}{2} ({\psi_L^c}^\dagger \psi_L + \psi_L^\dagger \psi_L^c) \label{eq.13_5}
\end{eqnarray}
  with
\begin{eqnarray}
  \psi_L^c = (i\sigma_2) \psi_L^* \label{eq.13_6}
\end{eqnarray}
  although the mass term in Eq.\ref{eq.13_5} is Lorentz invariant because of
\begin{eqnarray}
  &&\psi_L   \to \psi_L'   = S_L \psi_L  \label{eq.13_7a} \\
  && \psi_L^c \to {\psi_L^c}' = S_R \psi_L^c \label{eq.13_7b}
\end{eqnarray}
 
{\bf hw13-3b}: Although we showed Eq.\ref{eq.13_7b} before, if you cannot reproduce the proof
  in your mind, please prove it again.
  It cannot be made invariant under the phase transformation.
\begin{eqnarray}
  && \psi_L   \to \psi_L'   = e^{ iQ\theta} \psi_L  \label{eq.13_8a} \\
  && \psi_L^c \to {\psi_L^c}' = e^{-iQ\theta} \psi_L^c \label{eq.13_8b}
\end{eqnarray}
  simply because $\psi_L^c$ is proportional to $\psi_L^*$.

  In case of neutrino's (which has zero electromagnetic charge), the
  non-conserved ``charge'' may be called ``neutrino number'', or the
  ``lepton number'', or even the ``fermion number''.

{\bf hw13-4}: Please show that the first term in the Majorana mass term in Eq.\ref{eq.13_5}
  has the lepton number $2$, while the latter term has the lepton number
  $-2$, when we choose $Q$=Lepton number in Eq.(\ref{eq.13_8a}, \ref{eq.13_8b})

  Now, let us come back to the Dirac fermion.  The free Lagrangian of
  the Dirac fermion system (a system of $\psi_L$ and $\psi_R$ with the
  same charge) is written as
\begin{eqnarray}
  {\cal L}_{Dirac} = {\cal L}_{kin} + {\cal L}_{Dirac~mass} \label{eq.13_9}
\end{eqnarray}
  with
\begin{eqnarray}
  {\cal L}_{kin}
  = \psi_L^\dagger i\del_\mu \sigma_-^\mu \psi_L
  + \psi_R^\dagger i\del_\mu \sigma_+^\mu \psi_R  \label{eq.13_10}
\end{eqnarray}
  and Eq.\ref{eq.13_3} for ${\cal L}_{Dirac mass}$.  The e.o.m. from Eq.\ref{eq.13_9} reads
\begin{eqnarray}
  &&i\del_\mu \sigma_-^\mu \psi_L = m \psi_R   \label{eq.13_11a} \\
  && i\del_\mu \sigma_+^\mu \psi_R = m \psi_L \label{eq.13_11b}
\end{eqnarray}
  
{\bf hw13-5}: Derive the Dirac equations Eq.(\ref{eq.13_11a}, \ref{eq.13_11b}) from ${\cal L}_{Dirac}$.

{\bf hw13-6}: Show that the kinetic term ${\cal L}_{kin}$ is also invariant
  under the phase transformation Eq.(\ref{eq.13_4a}, \ref{eq.13_4b}).

  The only remaining task is to show that the kinetic term, ${\cal L}_{kin}$ Eq.\ref{eq.13_10}
  is Lorentz invariant.  In other words, we should yet to show that
\begin{eqnarray}
  && \psi_L^\dagger \sigma_-^\mu \psi_L = J_L^\mu \label{eq.13_6a} \\
  && \psi_R^\dagger \sigma_+^\mu \psi_R = J_R^\mu \label{eq.13_6b}
\end{eqnarray}
  transform as four-vectors. We show this by using infinitesimal
  transformations.

  Let us first remember the (infinitesimal) Lorentz tranformation of
  vectors.
\begin{eqnarray}
  V^\mu \to V'^\mu = L(\theta_k,\eta_k) V^\mu \label{eq.13_7}
\end{eqnarray}
  by using the $4 \times 4$ matrix representation:
\begin{eqnarray}
  L(\theta_k,\eta_k)
  &=& e^{-i( J_1 \theta_1 + J_2 \theta_2 + J_3 \theta_3
          +K_1 \eta_1   + K_2 \eta_2   + K_3 \eta_3  )} \\
  &=& 1  -i( J_1 \theta_1 + J_2 \theta_2 + J_3 \theta_3
          +K_1 \eta_1   + K_2 \eta_2   + K_3 \eta_3 ) + h.o. \label{eq.13_8}
\end{eqnarray}
  (note the minus sign before the generators, since we are transforming
  the object ``vector'' or ``current'', rather than the coordinates).

{\bf hw13-7}: Please obtain the transformation Eq.\ref{eq.13_8} as a $4 \times 4$ matrix up to
  the first order (in $\theta_k$ and $\eta_k$), by using the $4 \times 4$ matrix
  representations of all the $6$ generators.  Please give your answer
  as Eq.\ref{eq.13_9_2}:
\begin{eqnarray}
  L=
  \begin{pmatrix}
    1 & \eta_1 & \cdots & \cdots \\ 
    \cdots & 1 &-\theta_3 & \cdots \\
    \cdots & \cdots & 1 & \cdots \\
    \cdots & \cdots & \cdots & 1
  \end{pmatrix} \label{eq.13_9_2}
\end{eqnarray}
  Our task is hence simply to show that $J_L^\mu$ and $J_R^\mu$ in Eq.(\ref{eq.13_6a}, \ref{eq.13_6b}), 
  respectively, transform exactly the same way as Eq.\ref{eq.13_9_2}.

  Let us start with $J_L^\mu$
\begin{eqnarray}
  J_L^\mu
  &=& \psi_L^\dagger \sigma_-^\mu \psi_L
  \to J_L'^\mu
  = \psi_L'^\dagger \sigma_-^\mu \psi_L'
  = (S_L \psi_L)^\dagger \sigma_-^\mu (S_L \psi_L)
  = \psi_L^\dagger (S_L)^\dagger \sigma_-^\mu (S_L) \psi_L \label{eq.13_10_2}
\end{eqnarray}
  Therefore, all we should study is the transformation
\begin{eqnarray}
  \sigma_-^\mu \to (S_L)^\dagger \sigma_-^\mu (S_L) \label{eq.13_11}
\end{eqnarray}
  in the infinitesimal transformation
\begin{eqnarray}
  S_L(\theta_k,\eta_k)
  &=& e^{ \sum_k \sigma^k (-i\theta_k -\eta_k)/2}
  = 1 + \sum_k \sigma^k (-i\theta_k -\eta_k)/2 + h.o. \label{eq.13_12a} \\
  S_L^\dagger(\theta_k,\eta_k)
  &=& e^{ \sum_k \sigma^k (+i\theta_k -\eta_k)/2}
  = 1 + \sum_k \sigma^k (+i\theta_k -\eta_k)/2 + h.o. \label{eq.13_12b}
\end{eqnarray}
  By inserting Eq.(\ref{eq.13_12a},\ref{eq.13_12b}) into Eq.\ref{eq.13_11}, we find
\begin{eqnarray}
  \sigma_-^\mu
    \to {\sigma_-^\mu}'
    &=& (S_L)^\dagger \sigma_-^\mu (S_L) \\
    &=& (1 + \sum_k \sigma^k (+i\theta_k -\eta_k)/2 )
      \sigma_-^\mu
      (1 + \sum_k \sigma^k (-i\theta_k -\eta_k)/2 ) \\ 
    &=& \sigma_-^\mu
     +\sum_k \frac{i\theta_k}{2} [\sigma^k,\sigma_-^\mu]
     -\sum_k \frac{\eta_k}{2}  \{ \sigma^k,\sigma_-^\mu \} + h.l. \label{eq.13_13}
\end{eqnarray}
  where $[A,B] = AB-BA$, $\{A,B\}=AB+BA$.

{\bf hw13-8}: Please calculate the commutator and anti-commutator in Eq.\ref{eq.13_13} for $\mu=0, 1, 2, 3$, separately, by using the properties of the
  $\sigma$ matrices.  Show that the result agrees exactly with Eq.\ref{eq.13_9_2}.

{\bf hw13-9}: Repeat the above proof for $J_R^\mu$, by noting that the only
  difference between $J_L^\mu$ and $J_R^\mu$ are the difference between
  $\sigma_-^\mu$ and $\sigma_+^\mu$, i.e., they are the same for $\mu=0$,
  but the sign is opposite for $\mu=1,2,3$, whereas ($S_L$) and ($S_R$) differ
  only in the sign of $\eta_k$.  Show that the result agree exactly
  with Eq.\ref{eq.13_9_2} again.

{\bf hw13-10}: After proving (in {\bf hw13-8,9}) that the combinations Eq.(\ref{eq.13_6a},\ref{eq.13_6b}) transform as four vectors, confirm that none of the combinations below
\begin{itemize}
  \item $\psi_L^\dagger \sigma_+^\mu \psi_L$,
  \item $\psi_R^\dagger \sigma_-^\mu \psi_R$,
  \item $\psi_L^\dagger \sigma_+^\mu \psi_R$ and its hermetian conjugate
  \item $\psi_L^\dagger \sigma_-^\mu \psi_R$ and its hermetian conjugate
\end{itemize}
  transforms as a vector. Please don't waste your time by explicitly
  calculating the transformations of the above objects, but try to
  explain the reason by words.  I think that it is possible to explain.

  This homework, {\bf hw13-8} and {\bf hw13-9}, completes the proof that the Dirac
  Lagrangian Eq.\ref{eq.13_10}+Eq.\ref{eq.13_3}
\begin{eqnarray}
  {\cal L}_{Dirac}
  &=& {\cal L}_{kin} + {\cal L}_{Dirac~mass} \\
  &=& \psi_L^\dagger i\del_\mu \sigma_-^\mu \psi_L 
  + \psi_R^\dagger i\del_\mu \sigma_+^\mu \psi_R
  -m (\psi_L^\dagger \psi_R + \psi_R^\dagger \psi_L) \label{eq.13_15}
\end{eqnarray}
  is invariant under the Lorentz transformation, from which we find
  the Lorentz covariant equation of motion,
\begin{eqnarray}
  &&i\del^\mu \sigma_-^\mu \psi_L = m \psi_R \label{eq.13_16a}\\
  && i\del^\mu \sigma_+^\mu \psi_R = m \psi_L\label{eq.13_16b}
\end{eqnarray}
the Dirac equation.  We reproduced what Dirac found in 1928!

It is not difficult now to show that the Dirac equations, Eq.(\ref{eq.13_16a}, \ref{eq.13_16b}), are Lorentz covariant, i.e., the left-hand-side and the
  right-hand-side of the equations transform the same way under an
  arbitrary Lorentz transformation.

  We showed in {\bf hw13-8,9} that the following combinations are Lorentz
  invariant:
\begin{eqnarray}
&& (\psi_L)^\dagger V_\mu \sigma_-^\mu (\psi_L) \label{eq.13_17a}\\ 
&&(\psi_R)^\dagger V_\mu \sigma_+^\mu (\psi_R)  \label{eq.13_17b}
\end{eqnarray}
  For instance, the Lorentz invariance of Eq.\ref{eq.13_17a} can be expressed as
\begin{eqnarray}
  (\psi_L)^\dagger V_\mu  \sigma_-^\mu  (\psi_L)
  \to (\psi_L')^\dagger V_\mu' \sigma_-^\mu (\psi_L')
  = (\psi_L)^\dagger (S_L)^\dagger V_\mu' \sigma_-^\mu (S_L) (\psi_L)
  = (\psi_L)^\dagger V_\mu  \sigma_-^\mu   (\psi_L) \label{eq.13_18}
\end{eqnarray}
  The above transformation rule can be read as the product
\begin{eqnarray}
  (\psi_L)^\dagger B \label{eq.13_19}
\end{eqnarray}
  is Lorentz invariant, when we regard the combination
\begin{eqnarray}
  B = \{V_\mu  \sigma_-^\mu  (\psi_L)\} \label{eq.13_20}
\end{eqnarray}
  as a two-spinor (complex two component column vector).  The invariance
  Eq.\ref{eq.13_18} can now be expressed as
\begin{eqnarray}
  (\psi_L)^\dagger B \to (\psi_L')^\dagger  B'
    = (\psi_L)^\dagger (S_L)^\dagger B'
    = (\psi_L)^\dagger B \label{eq.13_21}
\end{eqnarray}
  This tells that $B$ should transform as
\begin{eqnarray}
  B \to B' = \{(S_L)^\dagger\}^{-1} B = (S_R) B \label{eq.13_22}
\end{eqnarray}
  because $(S_L)^\dagger (S_R) = 1$ (Eq.\ref{eq.13_2}). Now we showed that both the left-hand-side and the right-hand-side of the Dirac equation Eq.\ref{eq.13_16a} transform by $(S_R)$.

{\bf hw13-11}: Show that the l.h.s. of the Dirac Eq.\ref{eq.13_16b} is trsnformed by $(S_L)$, just like $\psi_L$ in the r.h.s. of the equation.

  This completes the proof of the Lorentz invariance of the Dirac
  Lagrangian and the covariance of the Dirac equations.  When I was a
  student, I had hard time following the proof given in the textbooks.

  We could do this so easily simply because we learned from his successors
  like Weyl and Majorana, who found very simple (two component)
  representations of the Lorentz group, which allows us to describe all
  Lorentz transformations in the $2 \times 2$ matrix basis.

  Please appreciate the fact that the Dirac Lagrangian Eq.\ref{eq.13_15} and hence the Dirac equation Eq.(\ref{eq.13_16a}, \ref{eq.13_16b}) describe a system of two Weyl fermions, one transforming under $S_L$, and the other transforming under $S_R$, which are coupled by the invariant mass term (Eq.\ref{eq.13_3}).

  Because this is exactly how we understand the nature now in the SM
  of particle physics, I believe that this is the right way to understand
  Dirac fermions and the Dirac equations.

  Let us show here how the Dirac four-spinor is related to our two-spinor
  in the so-called Weyl or Chiral representation of the $\gamma$ matrices.

  Let us first introduce a Dirac four-spinor fermion field:
\begin{eqnarray}
  \Psi(x) = ( \psi_L(x), \psi_R(x) )^T = 
  \begin{pmatrix}
    \psi_L(x) \\ \psi_R(x)
  \end{pmatrix} \label{eq.13_23}
\end{eqnarray}
  where I use $\Psi$ for 4-spinor, $\psi_L$ and $\psi_R$ for 2-spinors.  The
  Dirac Lagrangian is written as
\begin{eqnarray}
  {\cal L}_{Dirac} = {\overline \Psi} i\del \gamma \Psi - m {\overline \Psi} \Psi \label{eq.13_24}
\end{eqnarray}
  where I introduce a notation for Lorentz invariant contraction
\begin{eqnarray}
  \del \gamma = \del_\mu \gamma^\mu \label{eq.13_25}
\end{eqnarray}
  with
\begin{eqnarray}
  \gamma^\mu =
  \begin{pmatrix}
    0 & \sigma_+^\mu \\
    \sigma_-^\mu & 0
  \end{pmatrix} \label{eq.13_26}
\end{eqnarray}
  and
\begin{eqnarray}
  {\overline \Psi} &=& \Psi^\dagger \gamma_0
              = ( \psi_L^\dagger, \psi_R^\dagger )
              \begin{pmatrix}
                0 & 1 \\
                1 & 0
              \end{pmatrix}  \\
        &=& ( \psi_R^\dagger, \psi_L^\dagger ) \label{eq.13_27}
\end{eqnarray}
  is called Dirac conjugate.

{\bf hw13-12}: Please show that the Lagrangian in Eq.\ref{eq.13_24} is exactly the same as ${\cal L}_{Dirac}$ in Eq.\ref{eq.13_15} which we studied above.  Please observe the importance of Dirac conjugate Eq.\ref{eq.13_27}, which exchange the ordering of $\psi_L^\dagger$ and $\psi_R^\dagger$. We already learned that this replacement ensures the Lorentz invariance of both terms in
  ${\cal L}_{Dirac}$ (Eq.\ref{eq.13_24}) expressed in terms of the 4-spinor Eq.\ref{eq.13_23}.

  The equation of motion (the Dirac equation) is then
\begin{eqnarray}
  ( i\del \gamma - m ) \Psi(x) = 0 \label{eq.13_28}
\end{eqnarray}

{\bf hw13-13}: Please show that the Dirac equation Eq.\ref{eq.13_28} is the same as the Dirac equation Eq.(\ref{eq.13_16a}, \ref{eq.13_16b}), we obtained above.

  We can show by multiplying $i\del \gamma + m $ from the left on Eq.\ref{eq.13_28}
\begin{eqnarray}
  [ i\del \gamma + m ] [ i\del \gamma - m ] \Psi(x)
  = [ \del_\mu \del^\mu + m^2 ] \Psi(x) = 0. \label{eq.13_29}
\end{eqnarray}
  That is, all four components of $\Psi(x)$ satisfy the Klein-Gordon eq.

{\bf hw13-14}: Show that the necessary condition for Eq.\ref{eq.13_29} to hold is the
  anti-commutation relation among $\gamma^\mu$:
\begin{eqnarray}
  \{ \gamma^\mu, \gamma^\nu \} = 2 g^{\mu\nu}. \label{eq.13_30}
\end{eqnarray}

{\bf hw13-15}: Show that the anti-commutator Eq.\ref{eq.13_30} holds for our representation
  Eq.\ref{eq.13_26} of the $\gamma$ matrices (called Chiral or Weyl representation).

  You should prove the following two identities:
\begin{eqnarray}
  && \sigma_+^a \sigma_-^b + \sigma_+^b \sigma_-^a = 2 g^{ab} \label{eq.13_31a}\\
  && \sigma_-^a \sigma_+^b + \sigma_-^b \sigma_+^a = 2 g^{ab} \label{eq.13_31b}
\end{eqnarray}
 
{\bf hw13-16}: Please show
\begin{eqnarray}
  {\gamma^0}^\dagger =  \gamma^0 , ~~~
      {\gamma^i}^\dagger = -\gamma^i (i=1,2,3) \label{eq.13_32}
\end{eqnarray}
  in Chiral representation Eq.\ref{eq.13_26}.  $\gamma^0$ is Hermetian and $\gamma^i$ are
  anti-Hermetian.

  Let us introduce the $\gamma^5 = \gamma_5$ matrix as follows.
\begin{eqnarray}
  \gamma^5 = i \gamma^0 \gamma^1 \gamma^2 \gamma^3 \label{eq.13_33}
\end{eqnarray}


{\bf hw13-17}: Please show
\begin{eqnarray}
  && \{ \gamma^5, \gamma^\mu \} = 0~~   for~\mu=0,1,2,3 \label{eq.13_34a} \\
  &&             (\gamma^5)^2 = 1 \label{eq.13_34b}
\end{eqnarray}
  hold for arbitrary representations of $\gamma$ matrices which satisfy the
  anti-commutation relations Eq.\ref{eq.13_30}.

In the Chiral Representation Eq.\ref{eq.13_26}, $\gamma^5$ is diagonal
\begin{eqnarray}
  \gamma^5=
  \begin{pmatrix}
    -1 & 0\\ 
    0 & 1
  \end{pmatrix} \label{eq.13_35}
\end{eqnarray}
  where $1$ is unit $2 \times 2$ matrix. Let us now introduce chiarl projectors:
\begin{eqnarray}
 &&P_L = ( 1 - \gamma^5 )/2 = 
 \begin{pmatrix}
   1 & 0 \\ 
   0 & 0
 \end{pmatrix} \label{eq.13_36a} \\
 && P_R = ( 1 + \gamma^5 )/2 =
 \begin{pmatrix}
   0 & 0 \\
   0 & 1
 \end{pmatrix} \label{eq.13_36b}
\end{eqnarray}
 
{\bf hw13-18}: Derive Eq.\ref{eq.13_35}, and observe the projective properties Eq.(\ref{eq.13_37a}, \ref{eq.13_37b}, \ref{eq.13_37c}, \ref{eq.13_37d}):
\begin{eqnarray}
  &&(P_L)^2 = P_L \label{eq.13_37a}\\
  &&(P_R)^2 = P_R,\label{eq.13_37b} \\
  &&(P_L)(P_R) = 0, \label{eq.13_37c} \\
  &&P_L + P_R = 1 \label{eq.13_37d}
\end{eqnarray}
  The property Eq.(\ref{eq.13_37a},\ref{eq.13_37b}) is called projective, Eq.\ref{eq.13_37c} tells that the two
  projections are orthogonal, Eq.\ref{eq.13_37d} tells that the two projections are
  complete (no more projections).

  By using the completeness condition Eq.\ref{eq.13_37d}, we can express all 4-spinors
  $\Psi$ as a summation over two states (projections)
\begin{eqnarray}
  &&\Psi = (P_L + P_R) \Psi = \Psi_L + \Psi_R \label{eq.13_38a} \\ 
  && P_L \Psi = \Psi_L = 
  \begin{pmatrix}
    \psi_L \\ 0
  \end{pmatrix} \label{eq.13_38b} \\ 
  &&P_R \Psi = \Psi_R =
  \begin{pmatrix}
    0 \\ \psi_R
  \end{pmatrix}\label{eq.13_38c}
\end{eqnarray}
 Here 0's in Eq.(\ref{eq.13_38b}, \ref{eq.13_38c}) are 2-component column vectors $(0,0)^T$.

  In this notation, $\Psi_L$ and $\Psi_R$ are still $4$-spinors, containing
  one Weyl $2$-spinors $\psi_L$ and $\psi_R$, respectively, together with
  $2$-component nul spinors, $(0,0)^T$.  Because two-spinors $\psi_L$ and
  $\psi_R$ transform differently under the Lorentz transformation, as
  we showed in Eq.(\ref{eq.13_1a},\ref{eq.13_1b}), upper and lower halves of a Dirac four-spinor
  transform differently under Lorentz transformation.  This is the
  reason why it was hard for me (and many students) to understand the
  property of the Dirac fermions.

  When I studied Dirac fermions in 1985 (6 years after I got my PhD),
  chiral representation was treated as just one of many representations
  of the Dirac fields and $\gamma$ matrices.  I chose it simply because
  it is the only representation which makes massless neutrino wave
  function simple.  We can now tell that only in the chiral representation
  the origin of the Dirac fermions in the SM is manifest.  It is only
  in the chiral representation that the Lorentz transformation is
  factorized into upper and lower two-spinor components.  The SM tells
  us that those two components were independent (massless spin $1/2$)
  particles before the symmetry breakdown.

That's all for hw13.

Best regards,

Kaoru


\end{document}

\documentclass[12pt]{article}
\usepackage{amsmath,graphicx,color,epsfig,physics}
\usepackage{float}
\usepackage{subfigure}
\usepackage{slashed}
\usepackage{color}
\usepackage{multirow}
\usepackage{feynmp}
\textheight=9.5in \voffset=-1.0in \textwidth=6.5in \hoffset=-0.5in
\parskip=0pt
\def\del{{\partial}}
\def\dgr{\dagger}
\def\eps{\epsilon}
\def\lmd{\lambda}
\def\th{\theta}
\begin{document}

\begin{center}
{\large\bf HW21 for Advanced Particle Physics} \\
  
\end{center}

\vskip 0.2 in

Dear students,

  Following the last homework, I plan to cover the following subject
  in this homework.

  A: \mu \to \numu e \nuebar

  where the 3 body phase space is evaluated carefully, and
  we compare its life-time with the prediction of the SM.

  As a supplement, I mention the two processes:

  A-1: \tau \to \nutau e \nuebar
  A-2: n    \to p      e \nuebar

  briefly.  A-1 shows the scaling law

  \Gamma(\tau \to \nutau e \nuebar)   (m_\tau)^5
  --------------------------------- = ----------
  \Gamma(\mu  \to \numu  e \nuebar)    (m_\mu)^5

  A-2 is to show the importance of phase space suppression
  due to the very small mass difference

  m(n) - m(p) = 1.29MeV

  which is not much larger than m(e)=0.51MeV.

  =======================================================
  A: mu -> numu e nuebar (continued from eq.(40) in hw20)
  =======================================================

  40) M_{fi} = -g_W^2/(2m_W^2)
                uL(e:p1)^\dagger   \sigma_-^\mu vL(nue:p2)
                uL(num:p3)^\dagger \sigma_-_\mu uL(mu:p,h)

  In order to evaluate the above amplitudes, we should first fix the
  final state momenta,

  41) p = p1 + p2 + p3

  in the muon rest frame,

  42) p^\mu = (m_\mu, 0, 0, 0)

  Let us start with the definition of n-body phase space

  43) dPSn(p=k1+k2+k3+...+kn) = (2pi)^4 \delta^4(p-k1-k2-k3-...-kn)
                                \Pi_{j=1}^{j=n} [d^3 kj/2Ej/(2pi)^3]

  The 3-body phase space (n=3) reads

  44) dPS3(p=k1+k2+k3) = (2pi)^4 \delta^4(p-k1-k2-k3)
                                \Pi_{j=1}^{j=3} [d^3 kj/2Ej/(2pi)^3]

  Let us first show that the 3-body phase space can be expressed as a
  convolution of two 2-body phase space.  The following identity

  45) 1 = \Int d^4 k23 \delta^4(k23 -k2 -k3)
          \Int ds \delta((k23)^2-s)
        = \Int d^3 k23 \delta^4(k23 -k2 -k3)
          \Int d(k23^0)
          \Int ds \delta((k23-0)^2-(E23)^2)
        = \Int d^3 k23 \delta^4(k23 -k2 -k3)
          \Int ds
          \Int d(k23^0) \delta((k23-0)^2-(E23)^2)
        = \Int d^3 k23 \delta^4(k23 -k2 -k3)
          \Int ds /(2E23)
        = \Int ds
          \Int d^3 k23/(2E23)
          \delta^4(k23 -k2 -k3)
        = \Int ds/(2pi)
          \Int d^3 k23/(2E23)/(2pi)^3
          (2pi)^4 \delta^4(k23 -k2 -k3)

  is useful.  In the above, E23 is the energy

  46) E23 = \sqrt{ s + (k23^1)^2 + (k23^2)^2 + (k23^3)^2 }

  of the k23=k2+k3 system when its invariant mass-squared is s.

hw21-1: Verify eqs.(45) and derive (48), (50).

  By inserting (45) into the definition of PS3 (44) we find

  47) dPS3(p=k1+k2+k3)
  = (2pi)^4 \delta^4(p-k1-k23) d^3 k23/(2E23)/(2pi)^3 d^3 k1/(2E1)/(2pi)^3
    d(s23)/(2pi)
    (2pi)^4 \delta^4(k23-k2-k3) d^3 k2/(2E2)/(2pi)^3 d^3 k3/(2E3)/(2pi)^3

  which can be expressed as

  48) dPS3(p=k1+k2+k3) = dPS2(p=k1+k23) d(s23)/(2pi) dPS2(k23=k2+k3)

  in terms of the 2-body phase space dPS2 factors.

  Here, I denote s=s23 as the invariant mass-squared of the k2+k3 system,
  so that we can easily generalize eq.(48) to the n-body phase space

  49) dPSn(p=k1+k2+k3+...+kn)
  = dPS2(p=k1+k_{23...n}) ds_{23...n}/(2pi)
    dPS_{n-1}(k_{23...n}=k2+k3+...+kn)
  = dPS2(p=k1+k_{23...n}) ds_{23...n}/(2pi)
    dPS2(k_{23...n}=k2+k_{3...n}) ds_{3...n}/(2pi)
    ...
    dPS2(k_{n-1,n}=k_{n-1}+kn)

  By using the above formula, we can obtain the phase space of arbitrary
  number of particles by using the 2-body phase space.  The integration
  region of the invariant masses is simply

  50) (m2+m3)^2 < s23 < (\sqrt{s}-m1)^2

  for the 3-body phase space case.  Generalization to the n-body phase
  space case is straightforward:

  51) (m2+m3+...+mn)^2 < s_{23...n} < (\sqrt{s}-m1)^2
         (m3+...+mn)^2 < s_{3...n}  < (\sqrt{s_{23....n}}-m2)^2
                         ...
        (m_{n-1}+mn)^2 < s_{n-1,n} < (\sqrt{s_{n-2,n-1,n}-m_{n-2})^2

  If all the particles are massless, the integration region is simply

  52) 0 < s_{n-1,n} < s_{n-2,n-1,n} < s_{n-3,...n}
                                    < ... < s_{23...n} < s

  Please note the virtual mass ordering of the n-body phase space.
  The first MC parton shower (Pythia by Torbjoern Shoestrand) made
  use of this virtual mass ordering of the phase space.

  The total n-body phase space of the massless particles is simply

  53) PSn = (1/8pi) [(1/8pi)(1/2pi)]^{n-2}
            \Int_0^s ds_{23...n} (1-s_{23...n}/s)
            \Int_0^s_{23...n} ds_{3...n} (1-s_{3...n}/s_{23...n})
            ...
            \Int_0^s_{n-2,n-1,n} ds_{n-1,n} (1-s_{n-1,n}/s_{n-2,n-1,n})
          = (1/8pi) (1/16pi^2)^{n-2}
            \Int_0^s ds_{23...n} (1-s_{23...n}/s)
            ...
            \Int_0^s_{n-3...n} ds_{n-2...n} (1-s_{n-2...n}/s_{n-3...n})
            s_{n-2,n-1,n} (1/2)
          = (1/8pi) (1/16pi^2)^{n-2}
            \Int_0^s ds_{23...n} (1-s_{23...n}/s)
            ...
            \Int_0^s_{n-4...n} ds_{n-3...n} (1-s_{n-3...n}/s_{n-4...n})
            (s_{n-3...n})^2 \Int_0^1 dx x(1-x) (1/2)
          = (1/8pi) (1/16pi^2)^{n-2}
            \Int_0^s ds_{23...n} (1-s_{23...n}/s)
            ...
            \Int_0^s_{n-5...n} ds_{n-4...n} (1-s_{n-4...n}/s_{n-5...n})
            (s_{n-4...n})^3 \Int_0^1 dx x^2(1-x) (1/6) (1/2)
          = (1/8pi) (1/16pi^2)^{n-2} s^{n-2}
            x 1/(1*2) * 1/(2*3) * 1/(3*4) *** 1/((n-2)*(n-1))

  54) PS2 = 1/8pi
      PS3 = 1/8pi (1/4pi)^2 s   (1/2)
      PS4 = 1/8pi (1/4pi)^4 s^2 (1/2) (1/6)
      PS5 = 1/8pi (1/4pi)^6 s^3 (1/2) (1/6) (1/12)
      PS6 = 1/8pi (1/4pi)^8 s^4 (1/2) (1/6) (1/12) (1/20)

  55) PSn = 1/8pi (s/16pi^2)^{n-2} 1/[(n-1)!(n-2)!]  (n>=2)

hw21-2: Show (54), if possible (55).

  I calculated the total n-body massless phase space (55) just for fun.
  I have no idea if it is OK, even though it is trivial to test it
  against numerical calculation.  Would you please check it ?
  Since the above phase space for only massless particles is the
  largest possible one, the overall factor of

  56)  (2pi) (1/16pi^2)^{n-1}/(n-1)!/(n-2)!  (n>=1)

  shows the suppression factor for multiple body phase space.

  In our convention of using Lorentz invariant S-matrix elements
  and the Lorentz invariant phase space volume, the transition
  matrix elements for 1 or 2 to n particles have mass dimension
  of s^{-(n-2)/2}, and hence the factor

  57) \Int |M(1 or 2 > n particles)|^2 dPSn

  is proportional to

  58) s^{-(n-2)} PSn

  canceling the s^{(n-2)} factor in PSn (56).  The coefficient of
  the phase space factor (58)=(56) is proportional to the square
  of the product of all the couplings that give rise to the transition
  amplitudes.

  Whenever you evaluate cross sections (2 to n) or decay widths (1 to n),
  with whatever method, you should compare your results with the above
  general estimate.  I often found error in my calculation this way.

  In any case, please note an additional factor of 1/16pi^2 from the
  phase space whenever we add one more particle in the final state.
  It is just like the loop factor for virtual corrections.  Even if
  the coupling factor g^2 is large, say g^2=1, the n-body rate is small
  since it is proportional to (g^2/8pi) * (g^2/16pi^2)^{n-2}.

  Before closing exercises for the phase space, let us give the 1-body
  phase space (n=1), which is important for the 1-body production
  processes like W/Z/Higgs production in hadron collisions.  From the
  definition of the n-body phase space (43), we can write

  59) dPS1(p=k1) = (2pi)^4 \delta^4(p-k1) [d^3 k1/2E1/(2pi)^3]

  where E1=\sqrt{(k1)^2+m^2)}.  By introducing k1^0, we can re-write
  the 3-dimensional phase space by the 4-dimensional one with the
  on-shell condition:

  60) [d^3 k1/2E1] = d^4 k1 \delta(k1^0-E1)/(2E1)
                   = d^4 k1 \delta((k1^0)^2-(E1)^2) \Theta(k1^0)
                   = d^4 k1 \delta((k1)^2-m^2) \Theta(k1^0)

  In the last expression, (k1)^2=(k1*k1)=k1^\mu k1_\mu is the invariant
  4-momentum square.  By inserting (60) back into (59), we find

  61) dPS1(p=k1) = (2pi) \delta( p^2 - m^2 )

hw21-3: Derive (61).

  In other words, the 1-body phase space is a delta function which picks
  up the initial 4-momentum when its invariant squared agrees with the
  invariant mass squared of the produced particle.

  If a particle X is produced in pp collisions by a collision

  62) a + b > X

  and if the momentum of a parton a and b are:

  63) p_a = (\sqrt{s}/2) (xa, 0, 0,  xa)
      p_b = (\sqrt{s}/2) (xb, 0, 0, -xb)

  in the pp collision rest frame (the lab frame of the symmetric colliders
  like the LHC) then, the total 4-momentum of the initial state is

  64) p = p_a + p_b

  and the final 4-momentum is pX.  The 1-body phase space is then

  65) dPS1(p_a+p_b=pX) = (2pi) \delta( (p_a+p_b)^2 - mX^2 )
                       = (2pi) \delta( s*xa*xb - mX^2 )

  and the 4-momentum of X is

  66) pX = p_a + p_b = (\sqrt{s}/2) (xa+xb, 0, 0, xa-xb)

hw21-4:  Show (65) and (66).

                          *****

  Let us come back to the muon decay, and evaluate the amplitudes (40)

  40) M_{fi} = -g_W^2/(2m_W^2)
                uL(e:p1)^\dagger   \sigma_-^\mu vL(nue:p2)
                uL(num:p3)^\dagger \sigma_-_\mu uL(mu:p,h)
  67)        = -g_W^2/(2m_W^2)
                ubar(e:p1)   \gamma_-^\mu P_L v(nue:p2)
                ubar(num:p3) \gamma_-_\mu P_L u(mu:p,h)

  In the second line above (67), I rewrite the amplitudes in terms of
  Dirac four spinors, \gamma matrices, and projector P_L=(1-\gamma_5)/2.

  We can compare the above SM amplitudes for the muon decay with the
  the amplitudes of the Fermi operator

  68) H = G_F/\sqrt{2}
      * \overline{e}(x)     \gamma^\alpha (1-\gamma_5) \nue(x)
      * \overline{\numu}(x) \gamma_\alpha (1-\gamma_5) \mu(x)

  and find

  69) G_F/\sqrt{2} = (1/4) (g_W/\rt2)^2/mW^2 = g_W^2/(8m_W^2)

hw21-5: Show (69) by yourself, and calculate G_F from the observed
  W boson mass and couplings:

  70) mW = 80GeV, g_W^2 = 4pi \alpha_W = 4pi \alpha/\sin^2\theta_W

  with \alpha = 1/128, \sin^2\theta_W = 0.233.  How well do we
  reproduce the observed value of G_F?

  The factor of 1/4 comes from the two chiral projectors (1-\gamma_5)/2
  for the lepton doublets L, and the V-A currents used to define the
  Fermi operator.  The factor of additional 1/\sqrt{2} is mysterious,
  and has a historical origin.  When Fermi introduced the operator,
  he assumed that the interactions are vector-vector type, like QED,
  conserving Parity.

  71) H_old = G_F
            * \overline{e}(x)     \gamma^\alpha \nue(x)
            * \overline{\numu}(x) \gamma_\alpha \mu(x)

  The magnitude of G_F was then determined by comparing the muon life
  time calculated with the above Hamiltonian (71).  After the parity
  violation was discovered (in polarized nucleon beta decay distribution)
  and after the V-A currents are established, the Fermi operator was
  modified to the form (68).  With the modification from (71) to (68),
  where the currents are changed from V*V to (V-A)*(V-A), and the
  G_F is replaced by G_F/\sqrt{2}, the relationship between the muon
  decay lifetime and the Fermi (or muon decay) constant G_F remains
  the same.  Physicists could continue using the same value of G_F,
  which is determined solely from the muon life time, even after the
  theory changed from V*V to (V-A)*(V-A).

  The proof goes as follows: We can express the V-A current product as

  72) (V-A)*(V-A) = V*V + V*A + A*V + A*A

  the contribution from (Parity violating) V*A and A*V terms vanish
  when we integrate over the whole phase space to obtain the muon
  life time.  When we neglect the electron mass, we can show that
  A*A term gives exactly the same contribution to the total integral
  of the V*V term.  This factor of two appears in the total decay rate,
  which is proportional to the square of the amplitudes, and hence to
  G_F^2.  Therefore, we obtain the same muon life time by multiplying
  1/\sqrt{2} in the amplitude.

  Now, let us compute the muon lifetime in the SM by using
  the amplitudes (40)=(67).

  Although it is easy to compute the total decay rate (or the muon
  lifetime) starting from the above amplitudes, I would like to
  introduce you to a well-known (and useful) technique called Fierz
  transformation.  It makes the computation of the electron
  energy/angular distribution with respect to the muon polarization
  direction (the Parity violation !) very easy.

  The basic idea behind Fierz transformation is to exchange the fermions
  which couple to currents.  In our example, the amplitude (40)=(67)
  obtained from the SM Feynman rule is a product of two charged currents,

  73) mu-numu current and e-nue current

  After Fierz transformation, we can express the same amplitude as
  a product of two neutral currents,

  74) mu-e current and numu-nue current

  This is very convenient since the muon polarization information is
  then directly transferred to the electron momentum, and the final
  numu and nuebar made from the numu-nue current can be integrated out
  without loss of information, since there is no way that individual
  momentum of the two missing neutrinos are measured.

                         *****

  Let me start from the Fierz identity for the SU(n) generators
  in the fundamental representation:

  75) \Sum_a (T^a)_ij (T^a)_kl
    = T_F {\delta_il \delta_kl -(1/n) \delta_ij \delta_kl}

  where the sum is over all the generators, from a=1 to a=n^2-1.
  T_F=1/2 is our standard normalization for the generators in the
  fundamental (n) representation of SU(n).

hw21-6: Prove (75).

  hint 1: multiplying both sides by \delta_{il} \delta_{jk} and sum over
  all the indices makes \Sum_a Tr(T^a T^a) = T_F (n^2-1).
  hint 2: multiplying (T^b)_{jk} and sum over j,k should give
  C_F T^b_{il}, with C_F = T_F (n^2-1)/n.

  Applying (75) to SU(2), we find (by noting T_F = 2 for \sigma^i's)

  76) \Sum_a (\sigma^a)_ij (\sigma^a)_kl
      = 2 {\delta_il \delta_kj -(1/2) \delta_ij \delta_kl}
      = 2 \delta_il \delta_kj - \delta_ij \delta_kl

  Eq.(76) can be re-expressed as

  77) (\sigma_-^\mu)_ij (\sigma_-_\mu)_kl
  = \delta_ij \delta_kl - \Sum_a \sigma^a_ij \sigma^a_kl
  = \delta_ij \delta_kl - 2\delta_il \delta_kj + \delta_ij \delta_kl
  = 2 {\delta_ij \delta_kl - \delta_il \delta_kj}

  Likewise, we find

  78) (\sigma_-^\mu)_il (\sigma_-_\mu)_kj
  = \delta_il \delta_kj - \Sum_a \sigma^a_il \sigma^a_kj
  = \delta_il \delta_kj - 2\delta_ij \delta_kl + \delta_il \delta_kj
  = 2 {\delta_il \delta_kj - \delta_ij \delta_kl}

  From (77) and (78), we find

  79) (\sigma_-^\mu)_ij (\sigma_-_\mu)_kl
  = - (\sigma_-^\mu)_il (\sigma_-_\mu)_kj

hw21-7: Derive (79) by yourself.

  Therefore, the product of two V-A currents is identical (up to the
  overall sign) to that of two swapped V-A currents.  By inserting (79)
  into the amplitude (40)=(67), we find

  80) M
  = -g_W^2/(2m_W^2) uL(e:p1)^\dagger   \sigma_-^\mu vL(nue:p2)
                    uL(num:p3)^\dagger \sigma_-_\mu uL(mu:p,h)
  =  g_W^2/(2m_W^2) uL(e:p1)^\dagger   \sigma_-^\mu uL(mu:p,h)
                    uL(num:p3)^\dagger \sigma_-_\mu vL(nue:p2)

  In the latter expression above, the first current produces a
  left-handed electron from a polarized massive muon, and the second
  current produces a pair of numu and nuebar which are unobservable.

  When performing QED radiative corrections to the muon decay, the above
  Fierz transformation is very useful, since the photon is attached
  only to the first current including the loop correction.  Such
  computation was first done by Kinoshita and Sirlin, which eventually
  led Toichiro Kinoshita to propose the Kinoshita's theorem, telling
  that observables which sum over final states that have the same
  energy (energy degenerate states) are free from the mass singularity,
  such as log(Energy/m_electron).  The theorem helped us develop
  perturbative QCD for inclusive observables which are free from
  the mass singularity such as log(Energy/m_quark) or
  log(Energy/m_gluon).

  Although we can evaluate (80) in the muon rest frame right away,
  there is another technique which helps us integrating out the
  un-observed numu-nuebar currents.

  In order to do this, we parametrize the 3-body phase space as

  81) dPS3 = dPS2(p=p1+p23) d(p23)^2/(2pi) dPS2(p23=p2+p3)

  Please note here that the four momentum p23=p2+p3 is not that of
  the propagating W (q=p-p3=p1+p2).  This simplification works only
  when we neglect the W propagator correction, which is proportional
  to q^2/m_W^2 \sim 10^{-6}.

  The differential decay width is now written as

  82) d\Gamma(mu(h) > e numu nuebar)
  = 1/2M |M(mu(p,h) > e(p1,L) + numu(p3,L) + nuebar(p2,R))|^2 dPS3
  = 1/2M (g^2/(2mW^2))^2
         | uL(p1)^\dagger \sigma_-^\mu uL(p,h)
           uL(p3)^\dagger \sigma_-_\mu vL(p2) |^2  dPS3
  = (g^4/8mW^4)/M
   uL(k1)^\dgr \sigma_-^\mu uL(p,h) (uL(p1)^\dgr \sigma_-^\nu uL(p,h))^*
   uL(k3)^\dgr \sigma_-_\mu vL(p2)  (uL(p3)^\dgr \sigma_-_\nu vL(p2))~*
    dPS3
  = (g^4/8mW^4)/M
   uL(k1)^\dgr \sigma_-^\mu uL(p,h) (uL(p1)^\dgr \sigma_-^\nu uL(p,h))^*
   uL(k3)^\dgr \sigma_-_\mu vL(p2)  (uL(p3)^\dgr \sigma_-_\nu vL(p2))~*
    dPS2(p=p1+p23) d(p23)^2/(2pi) dPS2(p23=p2+p3)

hw21-8: Derive (82) by yourself.

  At this stage, we note that the numu-nuebar current component of
  the decay rate depends only on their 4-momenta, p2 and p3, and
  their contribution can be integrated out for a given (p23)^2.

  Let us make the phase space integration of the term

  83)
  uL(p3)^\dgr \sigma_-_\mu vL(p2)  (uL(p3)^\dgr \sigma_-_\nu vL(p2))~*
  dPS2(p23=p2+p3)

  in the rest frame of p23, by denoting (p23)^2=m^2

  84) p23 = (m, 0, 0, 0)
      p2  = (m/2)(1,  \sinth\cosphi,  \sinth\sinphi,  \sinth)
      p3  = (m/2)(1, -\sinth\cosphi, -\sinth\sinphi, -\sinth)

  85) dPS2(p23=p2+p3) = (1/8pi) d\costh/2 d\phi/2pi

  The numu and nuebar wave functions are:

  86) uL(p3) = Rz(phi) Ry(-pi+theta) \rt{m} (0,1)^T
      vL(p2) = Rz(phi)  Ry(theta)    \rt{m} (0,-1)^T

hw21-9: Compute (86), and obtain the currents (87)

  87a)  uL(p3)^\dgr \sigma_-_\mu vL(p2)
  87b) (uL(p3)^\dgr \sigma_-_\nu vL(p2))~*

hw21-10: Observe that the angular integration over the tensor (83)
  vanishes for all the off-diagonal elements, and the diagonal
  elements give (88) [tell me if 1/12pi factor is OK or not]

  88) \Int { uL(p3)^\dgr \sigma_-_\mu vL(p2)
            (uL(p3)^\dgr \sigma_-_\nu vL(p2))~* } dPS2(p23=p2+p3)
   = m^2/(12pi) \diag{ 0, 1, 1, 1 }
   = m^2/(12pi) { -g^\mu\nu + p23^\mu p23^\nu/(p23)^2 }

  Note that the bottom line in (88) agrees with the second line in the
  p23 rest frame.  The covariant expression in the bottom line is valid
  in an arbitrary Lorentz frame since the current products transform as
  a tensor when integrated over the invariant internal phase space.

  We insert (88) into (82), and find

  89) d\Gamma(mu(h) > e numu nuebar)
  = 1/2M |M(mu(p,h) > e(p1,L) + numu(p3,L) + nuebar(p2,R))|^2 dPS3
  = (g^4/8mW^4)/M
   uL(p1)^\dgr \sigma_-^\mu uL(p,h) (uL(p1)^\dgr \sigma_-^\nu uL(p,h))^*
   uL(p3)^\dgr \sigma_-_\mu vL(p2)  (uL(p3)^\dgr \sigma_-_\nu vL(p2))~*
    dPS2(p=p1+p23) d(p23)^2/(2pi) dPS2(p23=p2+p3)
  = (g^4/8mW^4)/M
   uL(p1)^\dgr \sigma_-^\mu uL(p,h) (uL(p1)^\dgr \sigma_-^\nu uL(p,h))^*
   m^2/(12pi) { -g^\mu\nu + p23^\mu p23^\nu/m^2 }
   dm^2/(2pi) dPS2(p=p1+p23)

  The invariant mass squared m^2=(p23)^2 can be expressed as

  90a) m^2 = p23^2 = (p-p1)^2 = M^2 -2p*p1 = M^2 (1-x)
  90b) x = 2p*p1/M^2 = 2E_e/M = 1-m^2/M^2

  by using the electron energy E_e in the muon rest frame.  By noting

  91a) dm^2 = M^2 dx
  91b) dPS2(p=p1+p23) = 1/8pi (1-(p23)^2/p^2) d\costh/2
                      = x/16pi d\costh

  we arrive at the expression

  92) d\Gamma / dx d\costh
  = (g^4/8mW^4)/M m^2/(12pi) M^2/(2pi) x/16pi
   uL(p1)^\dgr \sigma_-^\mu uL(p,h) (uL(p1)^\dgr \sigma_-^\nu uL(p,h))^*
   { -g^\mu\nu + p23^\mu p23^\nu/m^2 }
  = (g^4/8mW^4) M^3/(384pi^3) x(1-x)
   uL(p1)^\dgr \sigma_-^\mu uL(p,h) (uL(p1)^\dgr \sigma_-^\nu uL(p,h))^*
   { -g^\mu\nu + p23^\mu p23^\nu/m^2 }
  = (G_F)^2 M^3/(96pi^3) x(1-x)
   uL(p1)^\dgr \sigma_-^\mu uL(p,h) (uL(p1)^\dgr \sigma_-^\nu uL(p,h))^*
   { -g^\mu\nu + p23^\mu p23^\nu/m^2 }

  where in the last part I used

  93) G_F^2 = (g^2/8mW^2*\rt{2})^2 = g^4/32mW^2

hw21-11: Derive (92) by inserting (88) into (82).

  Now the problem is reduced to compute the current

  94) uL(p1)^\dgr \sigma_-^\mu uL(p,h)

  in the muon rest frame, when the muon is polarized along the
  z-axis (h=J_z=1/2),

  95) uL(p,h=J_z=1/2) = \sqrt{M} (1,0)^T

  and the electron momentum is

  96) p1^\mu = (M/2) x(1, \sin\theta, 0, \cos\theta)

  and its wave function reads

  97) uL(k1) = Ry(\theta) \sqrt{Mx} (0,1)^T

  Since the muon spin polarization axis (z-axis) is the symmetry axis of
  the problem, I ignore (integrated out) the trivial azimuthal angle,
  by setting \Int d\phi/2pi = 1.

  Note also

  98) p23^\mu = p^\mu - p1^\mu
              = M/2 (2-x, -x\sin\theta, 0, -x\cos\theta)

hw21-12: Calculate the electron wave function (97), and evaluate
  the current (94), and obtain a compact expression for the
  energy-angular distribution of the electron in the rest frame
  of the polarized muon (99):

  99)    d\Gamma
      --------------
      dx d\cos\theta

hw21-13: Observe Parity violating asymmetry in the electron angular
  distribution.  Does the electron prefer to be emitted along the muon
  spin direction (\cos\theta>0) or in the backward direction
  (\cos\theta<0) ?

  This type of Parity violating asymmetry was predicted by T.D.Lee and
  C.N.Yang in 1956 (as a possible solution of the K_L-K_S puzzle), and
  was observed in the electron angular distribution in polarized nucleon
  decay a year later by C.S.Wu.  Because the solution of the K_L-K_S
  puzzle is NOT Parity violation but that they differ by their CP
  parity (with small violation, discovered in 1964), we may regard
  that it was a successful prediction based on the wrong assumption.

hw21-14:  Integrate out the electron energy and angles, and obtain
  the total decay width of the muon, or the muon lifetime.  How well
  do we agree with the observed value of the muon lifetime?  Is the
  agreement or disagreement consistent with our approximation?

                         *****

  Here, I give two comments.

  First, about the ratio of the \mu life time and the \tau lifetime.

  From the above calculation, we learned that

  100) 2m\Gamma = |M|^2 PS3

  where m=m_\mu.  The amplitudes scale as

  101) M = G_F * m^2

  since there are 4 external fermions, and all their energies are
  scaled by \sqrt{m_\mu}.  3-body phase space is proportional to

  102) PS3 = m^2

  By combining (100)-(102), we observe

  103) \Gamma = G_F^2 m^5

  ignoring the numerical factors.  This scaling low should work for
  \tau decays, since \tau mass is still negligibly small as compared
  to the W mass.  We therefore expect the following:

  104) \Gamma(\tau \to e \nuebar \nutau)     m_\tau
       --------------------------------- = [ ------ ]^5
       \Gamma(\mu  \to e \nuebar \numu )      m_\mu

  whose numerical value should be (1.78/0.106)^5=1.33*10^6.

  Please check this naive prediction against the data.  The partial
  width can be obtained by multiplying the total width by the branching
  fraction:

  105) \Gamma(\tau \to e \nuebar \nutau)
     = \Gamma_\tau * B(\tau \to e \nuebar \nutau)

hw21-15: Please obtain \Gamma(\tau \to e \nuebar \nutau) from the PDG
  table, and compare it with our prediction (104).  How well do we
  reproduce the data?  Is it consistent with our approximations?

                         *****

  Another problem is why the neutron lifetime is so extremely long.  We
  find that the weak transition amplitudes for the process

  106) n(p) \to p(p1) + e(p2) + nuebar(p3)

  is not too much different from the amplitudes for the muon decay.
  In fact, since the overall mass scale of the neutron decay is the
  neutron mass, which is about 9 times larger than the muon mass.
  We could have predicted that the neutron decay width is 9^5 = 6*10^4
  times larger than the muon decay width, by using the (m_n/m_\mu)^5
  rule that worked for the tau lepton width.  The neutron width is
  actually 10^{-10} times the muon width.  What is the origin of the
  10^{-15} suppression?

  Part of this huge suppression factor comes from the phase space,
  because of the near degeneracy of the neutron and proton masses:

  107) m(n) - m(p) = 1.29 MeV

  which is only 2.5 times the electron mass

  108) a = m(e)/(m(n) - m(p)) = 0.51/1.29 = 0.40

  Since

  109) M = m(n) = 940MeV

  the 3-body phase space is

  110) M^2 1/(256\pi^3) = 1.26*10^{-4) M^2

  if all the decay particles are massless.  In case of the muon decay,
  (110) is an excellent approximation, where M = m_\mu.  In case of
  the neutron decay (106), the phase space suppression is extremely
  strong.  Let us estimate it by using the exact formula

  111) dPS3
  = dPS2(p=p1+p23) d(m23)^2/(2pi) dPS2(p23=p2+p3)
  = 1/(8pi) 2p1^*/M d(m23)^2/(2pi) 1/(8pi) (1-me^2/(m23)^2)
  = 1/(128pi^3) d(m23)^2 2p1^*/M (1-me^2/(m23)^2)

  The integration region of (m23) is

  112) M-m(p) > m23 > me

  I introduce an integration variable

  113) x = m23/(M-m(p))

  and the integration range is

  114) 1 > x > a = 0.40  (see (98))

  115) d(m23)^2 = (M-m(p))^2 dx^2 = 2(M-m(p))^2 x dx

  116) 2p1^*/M
  = 1/M^2 * 2Mp1^*
  = 1/M^2 *[ (M+mp+m23)(M+mp-m23)(M-mp+m23)(m-mp-m23) ]^{1/2}
  = 1/M^2 *(M+mp)*(M-mp)*[(1+x)(1-x)]^{1/2}
  = 2(M-mp)/M * (1-x^2)^{1/2}

  Inserting (115) and (116) into (111), I find

  117) dPS3
  = 1/(128pi^3) d(m23)^2 2p1^*/M (1-me^2/(m23)^2)
  = 1/(128pi^3) 2(M-mp)^2 xdx 2(M-mp)/M (1-x^2)^{1/2} (1-a/x)
  = 1/(32pi^3) (M-mp)^3/M dx (x-a) \sqrt{1-x^2}

  So, all I need to do is the integral

  118) \Int_a^1 dx (x-a) \sqrt{1-x^2}
     = (1/2) \Int_{a^2}^1 d(x^2) \sqrt{1-x^2}
          -a \Int_a^1 dx \sqrt{1-x^2}
     = (1/2) (2/3) (1-y)^{3/2}|^{a^2}_1
       -a \Int_0^{\arccos(a) d\theta \sin^2\theta
     = (1/2) (2/3) (1-a^2)^{3/2}
       -a \Int_0^{\arccos(a)} d\theta (1-\cos(2\theta))/2
     = (1/3) (1-a^2)^{3/2}
      -(a/2) {\arccos(a) - (1/2)\sin(2\theta)|^{arccos(a)}_0
     = (1/3) (1-a^2)^{3/2}
      -(a/2) {\arccos(a) - (1/2)\sin(2\arccos(a))}

  I'm not sure if my integral (118) is right.  I guess that its numerical
  value should be somewhere around (1-a)*0.2 \sim 0.1

  Inserting this into (117) gives

  119) PS3 = 1/(320pi^3) (M-mp)^3/M

  Compared to the massless 3 body phase space (110) the suppression
  factor is

  120) PS3/PS3(massless decay products)
     = ((M-mp)/M)^3
     = (1.29/940)^3
     = 3*10^{-9}

hw21-16: Follow the above calculation of the 3 body phase space,
  and check numerically if my estimate above (120) is fine.

  This is still not enough to explain the smallness of the neutron width.
  We should estimate the matrix elements.  I note that both the neutron
  and proton wave functions are large, proportional to \sqrt{M}, since
  they are non-relativistic.  On the other hand the electron and
  neutrino wave functions are small because of the smallness of their
  energy, bounded by (mn-mp).  The amplitude is hence suppressed by
  this factor, as compared to the massless fermion case.  The width
  should hence be suppressed by its squared.  Now, I find

  121) \Gamma(n to penu)
  = \Gamma(mp=0) * ((mn-mp)/mn)^2 * PS3/PS3(mp=0)
  = \Gamma(mp=0) * (1.3/950)^2 * 3*10^{-9}
  = \Gamma(mp=0) * 2*10^{-6}   * 3*10^{-9}
  = \Gamma(mp=0) * 6*10^{-15}

  Therefore, my naive prediction is

  122) \Gamma(n to penu)
  = \Gamma(mu to enunu) * (m_n/m_mu)^5 * 6*10^{-15}
  = \Gamma(mu to enunu) * 5*10^5       * 6*10^{-15}
  = \Gamma(mu to enunu) * 3*10^{-9}

  and hence

  123) \tau(n to penu) = 0.3 * 10^9 \tau(mu to enunu)
                       = 0.3 * 10^9 * 2*10^{-6} s
                       = 600 s
                       = 10 min

  Well, it's not that bad.  Please follow my estimates above, and you
  may obtain more accurate comparison.

  I do this type of rough order-of-magnitude estimates whenever it
  is relevant.  As you may learn from reading my memo above, I don't
  need a computer to make such estimates.  Very often, I find that
  my estimates are quite well, and whenever they disagree significantly,
  I learn something new in physics.  Fun in physics sometimes lies in
  such details.

  If you could follow and enjoy my discussion above, you really learned
  well from my lectures.

That's all for this long homework 21.

Best regards,

Kaoru



\end{document}
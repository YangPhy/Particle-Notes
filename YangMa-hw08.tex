\documentclass[11pt]{article}
\usepackage{amsmath,graphicx,color,epsfig,physics}
%\usepackage{pstricks}
\usepackage{float}
\usepackage{subfigure}
\usepackage{slashed}
\usepackage{color}
\usepackage{multirow}
\usepackage{feynmp}
\usepackage[top=1in, bottom=1in, left=1.2in, right=1.2in]{geometry}
\def\del{{\partial}}
\begin{document}
\title{Particle physics HW8}
\author{Yang Ma}

\maketitle

\section{ }
In terms 4 real fields, we write
  ($\phi_k$, $k=1,2,3,4$):
\begin{eqnarray}
  \phi &=& ( \phi^+, \phi^0 )^T,  \\
  \phi^+ &=& ( \phi_1 + i\phi_2 )/\sqrt2, \\
  \phi^0 &=& ( \phi_3 + i\phi_4 )/\sqrt2, 
\end{eqnarray}
so
\begin{eqnarray}
    \phi^\dagger \phi &=&
    \begin{pmatrix}
        \phi^- & {\phi^0}^*
    \end{pmatrix}
    \begin{pmatrix}
        \phi^+ \\ \phi^0
    \end{pmatrix}
    =\phi^+ \phi^- + \phi^0 {\phi^0}^*\\
    &=&\frac{1}{2}[( \phi_1 + i\phi_2 ) ( \phi_1 - i\phi_2 )+( \phi_3 + i\phi_4 )( \phi_3 - i\phi_4 )]\\
    &=&\frac{1}{2} ( \phi_1^2 +\phi_2^2 +\phi_3^2 +\phi_4^2).
\end{eqnarray}
Now we have
\begin{eqnarray}
    V(\phi)&=&\frac{\lambda}{4}(\phi^\dagger \phi)^2+\mu^2 \phi^\dagger \phi \\
    &=&\frac{\lambda^2}{16}(\phi_1^2+\phi_2^2+\phi_3^2+\phi_4^2)^2+\frac{\mu^2}{2}(\phi_1^2+\phi_2^2+\phi_3^2+\phi_4^2).
\end{eqnarray}

\section{ }
\begin{itemize}
    \item We see that $V(\phi)$ is a function of $\phi^\dagger \phi$, so we need only to show that $\phi^\dagger \phi$ is invariant under global
    $SO(4)$ transformation. Since $O^\dagger O=1$, we have
    \begin{eqnarray}
        \phi^\dagger \phi \to \phi^\dagger O^\dagger O \phi = \phi^\dagger \phi,
    \end{eqnarray}
    which leads to the invariance of $V(\phi)$.
    \item Recall the Higgs potential
    \begin{eqnarray}
        V(\phi)=\frac{\lambda^2}{16}(\phi_1^2+\phi_2^2+\phi_3^2+\phi_4^2)^2+\frac{\mu^2}{2}(\phi_1^2+\phi_2^2+\phi_3^2+\phi_4^2),
    \end{eqnarray} 
    and compare with the Klein-Gordon Lagrangian
    \begin{eqnarray}
        {\cal L}= \frac{1}{2} (\del_\mu \phi)(\del^\mu \phi) - \frac{1}{2}\phi,
    \end{eqnarray}
    we see that $\frac{\mu^2}{2} \phi^2_k$ terms provides the mass term for $\phi_k$, ($k=1,2,3,4$) and $m=\mu^2$ for all 4 bosons. 
\end{itemize}

\section{ }
\begin{eqnarray}
    \frac{dV}{d\phi_k}|_{\phi_j=<\phi_j>}&=&
    (\frac{\lambda^2}{4}\sum_{i=1}^4 <\phi_i>^2+\mu^2) <\phi_k>=0 
  \end{eqnarray}
If $\mu^2<0$, we see there are two solutions, $<\phi_k>=0$ or
\begin{eqnarray}
    (<\phi_1>,<\phi_2>,<\phi_3>,<\phi_4>) = (0,0,v,0).
\end{eqnarray}
Plug these two solutions back to the potential, we see that the second one minimizes the potential.

\section{ }
\begin{itemize}
    \item With 
    \begin{eqnarray}
        \phi_1 = \pi_1,~~\phi_2 = \pi_2,~~\phi_3 = v + h,~~\phi_4 = \pi_3,
    \end{eqnarray}
    we have
    \begin{eqnarray}
        <\pi_1> &=& <\phi_1> = 0 ,~~<\pi_2>= <\phi_2> = 0,\\
       <h> &=& <\phi_3> - v =0 ,~~< \pi_3>=<\phi_4> =0.
    \end{eqnarray}
    \item 
    \begin{eqnarray}
        V(\phi)&=&\frac{\lambda^2}{16}(\phi_1^2+\phi_2^2+\phi_3^2+\phi_4^2)^2+\frac{\mu^2}{2}(\phi_1^2+\phi_2^2+\phi_3^2+\phi_4^2) \\
        &=&\frac{\lambda}{16}(\sum_{i=1}^3 \pi_i^2+(v+h)^2)^2+\frac{\mu^2}{2} (\sum_{i=1}^3 \pi_i^2+(v+h)^2)
    \end{eqnarray} 
    Compare with the Klein-Gordon Lagrangian, we see the mass term of $\pi_k (k=1,2,3)$ is 
    \begin{eqnarray}
        \sqrt{\frac{\lambda v^2}{8}+\frac{\mu^2}{2}}=0,
    \end{eqnarray}
    and that for $h$ is
    \begin{eqnarray}
        \sqrt{\frac{\lambda v^2}{4}+\frac{\mu^2}{2}}=\sqrt{\lambda} v/2
    \end{eqnarray}
    \item Write
    \begin{eqnarray}
        <\psi>=
        \begin{pmatrix}
            \pi_1\\\pi_2\\\pi_3\\v
        \end{pmatrix}
        =
        \begin{pmatrix}
            0\\0\\0\\v
        \end{pmatrix}
    \end{eqnarray},
    then we see
    \begin{eqnarray}
        \begin{pmatrix}
            0\\0\\0
        \end{pmatrix}
        \to
        O
        \begin{pmatrix}
            0\\0\\0\\v
        \end{pmatrix}
        =        
        \begin{pmatrix}
            0\\0\\0\\v
        \end{pmatrix}.
    \end{eqnarray}
\end{itemize}
    
\section{ }
Note we have shown $\phi^\dagger \phi={\phi^c}^\dagger \phi$ beforn and it is obvious that the Higgs potential can be written as  \begin{eqnarray}
    V(\phi)
    = V(\phi^c)
    = \frac{\lambda}{4} \frac{[\phi^\dagger \phi + (\phi^c)^\dagger \phi^c]^2}{4}
         +\frac{\mu^2 [\phi^\dagger \phi + (\phi^c)^\dagger \phi^c]}{2}.
  \end{eqnarray}

\section{ }
Take the transverse of
\begin{eqnarray}
    \Phi \to \Phi' = \Phi U_R^\dagger,
  \end{eqnarray}
then we have
\begin{eqnarray}
    \Phi^T \to \Phi'^T = (\Phi U_R^\dagger)^T={ U_R^\dagger}^T \Phi^T= U_R^* \Phi^T,
\end{eqnarray}
  where
  \begin{eqnarray}
    \Phi^T = 
    \begin{pmatrix}
      \phi^c \\  \phi
    \end{pmatrix}.
  \end{eqnarray}

\section{ }
\begin{eqnarray}
    Tr\{ \Phi^\dagger \Phi \}
      = Tr
          \begin{pmatrix}
              {\phi^c}^\dagger \phi^c  & {\phi^c}^\dagger \phi \\ \phi^\dagger \phi^c & \phi^\dagger \phi
          \end{pmatrix}
      =\phi^\dagger \phi + {\phi^c}^\dagger \phi^c
      = 2 \phi^\dagger \phi
\end{eqnarray}

\section{ }
With $U_L^\dagger U_L=1$ and $U_R^\dagger U_R=1$, we have
\begin{eqnarray}
    Tr\{\phi^\dagger \phi \} \to     Tr\{U_R \phi^\dagger U_L^\dagger U_L \phi U_R^\dagger \}=  Tr\{ U_R^\dagger U_R \phi^\dagger \phi  \}=Tr\{ \phi^\dagger \phi  \}
\end{eqnarray}

\section{ }
\begin{eqnarray}
    <\Phi> &\to& <\Phi>'
               = (U_{custodial})
               \frac{v}{\sqrt 2}
                \begin{pmatrix}
                    1&0 \\0 &1
                \end{pmatrix}
    (U_{custodial})^\dagger \\
               &=&\frac{v}{\sqrt 2}
               (U_{custodial})(U_{custodial})^\dagger \\
               &=& \frac{v}{\sqrt 2}
               \begin{pmatrix}
                   1&0 \\0 &1
               \end{pmatrix}=<\Phi>
  \end{eqnarray}

\section{ }
\begin{eqnarray}
    \Phi &=&
    \begin{pmatrix}
      \phi^c & \phi
    \end{pmatrix}
    =
    \begin{pmatrix}
      ( v + h -i\pi_3)/\sqrt{2} & (\pi_1 + i\pi_2)/\sqrt{2} \\
      (-\pi_1 +i\pi_2)/\sqrt{2} & ( v + h +i\pi_3)/\sqrt{2}
    \end{pmatrix}\\
    &=&\frac{v+h}{\sqrt{2}} +\frac{i}{\sqrt 2}(\pi_1 \sigma^1+\pi_2\sigma^2+\pi_3\sigma^3 \\
    &=& \frac{v+h}{\sqrt{2}} + \frac{i}{\sqrt 2} \sigma^k \pi'_k
  \end{eqnarray}



\end{document}
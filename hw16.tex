\documentclass[12pt]{article}
\usepackage{amsmath,graphicx,color,epsfig,physics}
\usepackage{float}
\usepackage{subfigure}
\usepackage{slashed}
\usepackage{color}
\usepackage{multirow}
\usepackage{feynmp}
\textheight=9.5in \voffset=-1.0in \textwidth=6.5in \hoffset=-0.5in
\parskip=0pt
\def\del{{\partial}}
\def\dgr{\dagger}
\def\eps{\epsilon}

\begin{document}

\begin{center}
{\large\bf HW16 for Advanced Particle Physics} \\

\end{center}

\vskip 0.2 in

Dear students,\\
In this lecture, I introduce vector boson wave functions, both massless
(photon and gluons) and massive ($W$ and $Z$).  At the end of the lecture,
I add two comments on the fermion wave functions, concerning the
helicity conservation for massless (Weyl) fermions and the normalization
of $u$ and $v$ spinors.

Let me start with our gauge field Lagrangian
\begin{eqnarray}
  {\cal L}_{gauge}
= -\frac{1}{4} F_{\mu\nu} F^{\mu\nu}
= -\frac{1}{2} (\del_\mu A_\nu) (\del^\mu A^\nu -\del^\nu A^\mu) \label{eq.16_1}
\end{eqnarray}
which is common for all three gauge fields ($8$ gluons, $3$ W's of $SU(2)_L$,
and $B$ for $U(1)_Y$), when we drop the interaction terms for $SU(3)$
and $SU(2)$ gauge bosons (the $3$ and $4$ point self-interactions).

After the symmetry breakdown, $SU(2)$ $W$ and $U(1)$ $B$ gauge bosons are mixed,
and then the photon (together with the $8$ gluons) has the Lagrangian of
the form Eq.\ref{eq.16_1}. In case of $Z$ boson, the mass term appears
\begin{eqnarray}
  {\cal L}_{Z}
  = -\frac{1}{4} Z_{\mu\nu} Z^{\mu\nu} +\frac{1}{2} m_Z^2 Z_\mu Z^\mu
  = -\frac{1}{2} (\del_\mu Z_\nu) (\del^\mu Z^\nu -\del^\nu Z^\mu)
    +\frac{1}{2} m_Z^2 Z_\mu Z^\mu \label{eq.16_2}
\end{eqnarray}
For the $W$ boson, the neutral $W^1$ and $W^2$ bosons have exactly the same mass, and hence they are combined to make a complex field $W^+$ and $W^-$,
whose Lagrangian is
\begin{eqnarray}
  {\cal L}_{W}
= -\frac{1}{2} W^+_{\mu\nu} W^{- \mu\nu} + m_W^2 W^+_\mu W^{- \mu}
= - (\del_\mu W^+_\nu) (\del^\mu W^{-\nu} -\del^\nu W^{-\mu}) + m_W^2 W^+_\mu W^{-\mu} \label{eq.16_3}
\end{eqnarray}

{\bf hw16-1}: Please confirm the above statements, obtain the first lines (checking the overall normalization and the signs of each term), and derive the second lines of Eqs.(\ref{eq.16_1}, \ref{eq.16_2}, \ref{eq.16_3}).

Now, let us try to obtain the equation of motions (E.O.M.) for all the above gauge bosons, starting from the above Lagrangian. Let us start
with the photon (and the gluons) in Eq.\ref{eq.16_1}:
\begin{eqnarray}
  {\cal L}_{gauge}
&=& -\frac{1}{4} F_{\mu\nu} F^{\mu\nu}
= -\frac{1}{2} (\del_\mu A_\nu) (\del^\mu A^\nu -\del^\nu A^\mu) \\
&=& +\frac{1}{2} A_\nu (\del_\mu \del^\mu A^\nu -\del^\mu \del^\nu A_\mu)
  + (total~derivative~ term~ \del_\mu (\cdots))\\
&=& +\frac{1}{2} A_\nu (g^{\mu\nu} \del_a \del^a -\del^\mu \del^\nu) A_\mu
  + (total~derivative ~term ~\del_\mu (\cdots))\\
&=& +\frac{1}{2} A_\mu (g^{\mu\nu} \del_a \del^a -\del^\mu \del^\nu) A_\nu
  + (total ~derivative ~term ~\del_\mu (\cdots)) \label{eq.16_4}
\end{eqnarray}
The above Lagrangian is written (up to the total derivative) in terms of $A_\mu$, but in such a way that it does not contain $\del_\nu A_\mu$.
Therefore, the E.O.M. is simply
\begin{eqnarray}
  \frac{\delta {\cal L}}{\delta(A_\mu)} -\frac{\del_\nu \delta {\cal L}}{\delta(\del_\nu A_\mu)}
=  \frac{\delta {\cal L}}{\delta(A_\mu)}
= (g^{\mu\nu} \del_a \del^a -\del^\mu \del^\nu) A_\nu
= 0. \label{eq.16_5}
\end{eqnarray}

{\bf hw16-2}: Show Eq.\ref{eq.16_5}. You may obtain the E.O.M. directly from the first
expression in ${\cal L}_{gauge}$, and confirm that it gives the same E.O.M.
The technique I used above is quite helpful.

Although the above E.O.M. looks fine (Lorentz covariant), but has a
problem since
\begin{eqnarray}
  \del_\mu (g^{\mu\nu} \del_a \del^a -\del^\mu \del^\nu) = 0\label{eq.16_6}
\end{eqnarray}
If we consider the E.O.M. Eq.\ref{eq.16_5} as a matrix equation, where the
differential operator may be expressed as a $4\times 4$ matrix, and the
$A_\nu$ field as a column $4$ vector, Eq.\ref{eq.16_6} tells that the $4$ column vectors in the $4\times 4$ matrix are linearly dependent, and hence its
determinant is zero. We cannot obtain a solution of such equation.

This is a consequence of gauge invariance, which tells that only two
components among the $4$ components of $A^\nu$ are independent.

{\bf hw16-3}: Please confirm the above statement by writing $A_\nu(x)$ as
\begin{eqnarray}
  A_\nu(x) = \epsilon_\nu(q) e^{-iqx} \label{eq.16_7}
\end{eqnarray}
with
\begin{eqnarray}
  q^\mu = (E,0,0,E). \label{eq.16_8}
\end{eqnarray}

Although we haven't yet shown that Eq.\ref{eq.16_7} is a solution of the E.O.M. of $A_\nu(x)$, by taking it as a Fourier transform, we can transform the
$4\times 4$ matrix of derivative operators to the $4\times 4$ matrix of four momentum
components. Please obtain the momentum space equation for
$\epsilon_\nu(q)$, and express it as a $4\times 4$ matrix times a column $4\times 1$
vector. Can you calculate the determinant?

Please note that when we use the solution of the free field equations
to study the impacts of interactions, such as
\begin{eqnarray}
  (~ Differential~ Operator~ ) ~A(x) = J(x) \label{eq.16_9}
\end{eqnarray}
where the free field E.O.M. is the $J(x)=0$ limit,
\begin{eqnarray}
  (~ Differential~ Operator~ )  A^{(0)}(x) = 0, \label{eq.16_10}
\end{eqnarray}
we often use a solution $G(x,x')$ which satisfies
\begin{eqnarray}
  (~ Differential~ Operator~ ) G(x,x') = \delta^3(x-x') \label{eq.16_11}
\end{eqnarray}
in terms of which, $A(x)$ can be expressed as
\begin{eqnarray}
  A(x) = A^{(0)}(x) + \int d^3x' G(x,x') J(x'). \label{eq.16_12}
\end{eqnarray}
in perturbation theory. This is called Green's function method,
which plays an essential role in perturbative QFT (where the
Green's functions are called ``propagators'').

{\bf hw16-4}:  Show that Eq.\ref{eq.16_12} gives a solution for Eq.\ref{eq.16_9} in the linear (first or Born) approximation in $J(x)$, and hence it is useful if the correction, $A(x)-A^{(0)}$, is small as compared to the
free field approximation, $A^{(0)}$.  Can you tell how we can
obtain higher order corrections?

Summing up, if the differential operator describing the E.O.M. for
a free field has zero determinant, we cannot obtain the free field
solution, and in addition, we cannot obtain the Green's function.

This all happens because of the gauge invariance, which tells that
not all components of our gauge fields, $A^\mu(x)$, are independent.

For massless gauge bosons, only two among the four components are
physical, while for massive gauge bosons, only three among the four
are physical (on its mass shell: when $q^2 = m_V^2$).

When we include interactions in perturbation theories, we should find
Green's functions (propagators) for all the free fields, but this is
possible only if the operator which appears in the free field E.O.M.
is invertible, i.e., only when their determinants are non-zero.

The above two statements, the first one for the degrees of freedom
of physical (on-shell, or $E = \sqrt{ {\vec p} \cdot {\vec p} + m^2 }$) states,
and the second one for the existence of Green's functions, are
related, and hence both are expressed as the necessity of
`gauge fixing', but they are not the same.  This is because the
conditions for the non-zero determinant is weaker than the condition
for the physical states.  Well, a bit confusing?

Please keep what I said above in your mind as something what should
be achieved in a consistent theory of perturbative QFT, and you will
learn more precise treatments in advanced courses (whose key words
are gauge fixing, quantization of Abelian and non-Abelian gauge
theories, quantum or radiative corrections to scattering amplitudes).

Now, let us come back to our free field E.O.M. for massless gauge
fields, Eq.\ref{eq.16_5}, which I copy here
\begin{eqnarray}
  (g_{ab} \del_c \del^c -\del_a \del_b) A^b(x) = 0. \label{eq.16_13a}
\end{eqnarray}
Because only two among the $4$ components of $A^b$ are physical for
massless vector bosons, we can impose two additional constraints.
Typically, we impose the following two conditions:
\begin{eqnarray}
  && \del_a A^a(x) = 0 \label{eq.16_13b} \\
  && n_a A^a(x) = 0 \label{eq.16_13c}
\end{eqnarray}

Actually, only one of Eq.\ref{eq.16_13b} or Eq.\ref{eq.16_13c} suffices to make the determinant non-zero, and hence they are separately used as a different gauge fixing condition: Eq.\ref{eq.16_13b} is called covariant gauge fixing, while Eq.\ref{eq.16_13c} is called
\begin{itemize}
  \item $n^a = (1,0,0,0)$: temporal gauge, Coulomb gauge
  \item $n^2 = n_a n^a = 0$: light-cone gauge
  \item Other general $n^a$: axial gauge
\end{itemize}
Let us first show that the covariant gauge fixing constraint Eq.\ref{eq.16_13b}
makes the E.O.M. Eq.\ref{eq.16_13a} very simple:
\begin{eqnarray}
  \del_a \del^a A^b(x) = 0. \label{eq.16_15}
\end{eqnarray}
All $A^b(x)$ components should now satisfy the Klein-Gordon equation.

{\bf hw16-5}:  Show that Eq.\ref{eq.16_13a} and Eq.\ref{eq.16_13b} give Eq.\ref{eq.16_15}.

Because the above Klein-Gordon operator can be expressed as
\begin{eqnarray}
  \del_a \del^a A^b(x) = (\del_a \del^a) g^{ab} A_b(x) = 0 \label{eq.16_16}
\end{eqnarray}
it is certainly invertible as a $4\times 4$ matrix.

{\bf hw16-6}: By using the Fourier transform Eq.\ref{eq.16_7}, obtain the E.O.M. in the momentum space as a $4\times 4$ matrix, and find its inverse.

Therefore, we can now have a free field expansion:
\begin{eqnarray}
  A^a(x) = \sum_h \int \frac{d^3p}{2E(2\pi)^3}
  [a(p,h) e^{-ipx} \epsilon^a(p,h) + a(p,h)^\dgr e^{ipx} \epsilon^a(p,h)^* ] \label{eq.16_17}
\end{eqnarray}
just like for fermions. The Lorentz covariant part of the wave
functions, $\epsilon^a(p,h)$, are called ``polarization vectors''.
Let us find E.O.M. for the polarization vectors.

First, the covariant gauge fixing condition Eq.\ref{eq.16_13b} gives
\begin{eqnarray}
  p_\mu \epsilon^\mu(p,h) = 0. \label{eq.16_18}
\end{eqnarray}

{\bf hw16-7}: Show Eq.\ref{eq.16_18}.

Second, since there are only two physical degrees of freedom for
massless vector (or for any massless particles of non-zero spin),
we should impose additional constraint Eq.\ref{eq.16_13c} to fix $\epsilon(p,h)$,
\begin{eqnarray}
  n_\mu \epsilon^\mu(p,h) = 0. \label{eq.16_19}
\end{eqnarray}

The polarization vector has various expressions depending on our
choice of the $n^\mu$ vector.

The only necessary condition that the constraints Eq.\ref{eq.16_18} and Eq.\ref{eq.16_19} determine $\epsilon(p,h)$ unique is that $p^\mu$ and $n^\mu$ are not collinear:
\begin{eqnarray} 
  n\cdot p = n_a p^a \neq 0 \label{eq.16_20}
\end{eqnarray}

Since physics does not depend on our choice of $n^\mu$ vector, and
since $n^\mu$ dependence appears only after the covariant gauge
fixing Eq.\ref{eq.16_13a}, which is sufficient to make the determinant of the
differential operator non-zero, this arbitrariness of choosing
$n^\mu$ is sometimes called the residual symmetry after gauge ``fixing''.

In case of non-Abelian gauge theory, this residual symmetry after
gauge fixing was identified elegantly by Becchi-Rouet-Stora, and
is called BRS symmetry.  I could follow tHooft's proof of quantization
and renormalization of non-Abelian gauge theory only after BRS.

Let us start with massive gauge boson, since it is easier to
understand the meaning of our polarization vector, $\epsilon^a(p,h)$,
in the rest frame of the vector boson by using our quantum mechanics
knowledge.

Let us use the W boson Lagrangian Eq.\ref{eq.16_3} for our exercise,
\begin{eqnarray}
  {\cal L}_{W}
= -\frac{1}{2} W^+_{\mu\nu} W^{- \mu\nu} + m_W^2 W^+_\mu W^{- \mu}
= - (\del_\mu W^+_\nu) (\del^\mu W^{-\nu} -\del^\nu W^{-\mu})
  + m_W^2 W^+_\mu W^{-\mu} \label{eq.16_21}
\end{eqnarray}

The $Z$ boson case is obtained simply by replacing the mass from $m_W$
to $m_Z$, and by replacing $b(p,h)^\dgr$ by $a(p,h)^\dgr$ in the following
free field expansion.

By using the same technique (by re-writing the Lagrangian Eq.\ref{eq.16_21} not
to have $\del^\mu W^+_\nu$, except for the total derivative term), we
obtain the E.O.M. for $W^-_\mu(x)$
\begin{eqnarray}
  (g^{ab} [\del_a \del^a+m_W^2] -\del^a \del^b) W^-_b(x) = 0. \label{eq.16_22}
\end{eqnarray}
Just the same as in Eq.\ref{eq.16_5}, where the Klein-Gordon operator now has the mass term. Now, again, let us impose the covariant gauge fixing
condition
\begin{eqnarray}
  \del^b W^-_b(x) = 0. \label{eq.16_23}
\end{eqnarray}
We find, that all components of $W^-_b(x)$ fields $(b=0,1,2,3)$ satisfy
the Klein-Gordon equation:
\begin{eqnarray}
  [ \del_a \del^a + m_W^2 ] W^-_b(x) = 0. \label{eq.16_24}
\end{eqnarray}

{\bf hw16-8}: Derive  Eq.\ref{eq.16_22}, and then Eq.\ref{eq.16_24}.

The free field expansion hence becomes,
\begin{eqnarray}
  W^-_a(x)
= \sum_h \int \frac{d^3p}{2E(2\pi)^3}
[ a(p,h) e^{-ipx} \epsilon_a(p,h) + b(p,h)^\dgr e^{ipx} \epsilon_a(p,h)^* ] \label{eq.16_25}
\end{eqnarray}
where the polarization vector $\epsilon^\mu(p,h)$ satisfy
\begin{eqnarray}
  p_\mu \epsilon^\mu(p,h) = 0. \label{eq.16_26}
\end{eqnarray}

{\bf hw16-9}: Show Eq.\ref{eq.16_26}.

Let us study the vector boson polarization vector, in the rest frame
of a massive vector boson
\begin{eqnarray}
  p^\mu = (m, 0, 0, 0)^T \label{eq.16_27}
\end{eqnarray}
In this frame, $h=+1, 0, -1$ is the spin component along the $z$-axis:
\begin{eqnarray}
  J_z \epsilon^\mu(p,h) = h \epsilon^\mu(p,h) \label{eq.16_28}
\end{eqnarray}

By using the matrix representation of $J_3$ which we obtained in the
past homeworks, we can show that the following vectors satisfy the
eigen-value equation Eq.\ref{eq.16_28}:
\begin{eqnarray} 
  && \eps^\mu(p,h=+1) = (0, -1, -i, 0)^T / \sqrt{2} \label{eq.16_29a} \\
  && \eps^\mu(p,h= 0) = (0,  0,  0, 1)^T \label{eq.16_29b} \\
  && \eps^\mu(p,h=-1) = (0,  1, -i, 0)^T / \sqrt{2} \label{eq.16_29c}
\end{eqnarray}

{\bf hw16-10}: Show that Eq.(\ref{eq.16_29a},\ref{eq.16_29b},\ref{eq.16_29c}) satisfy Eq.\ref{eq.16_28}.

The above polarization vectors satisfy
\begin{eqnarray}
&& \eps^\mu(p,h) p_\mu = 0 \label{eq.16_30a} \\
&& \eps^\mu(p,h) \eps_\mu(p,h')^* = -\delta_{hh'} \label{eq.16_30b}
\end{eqnarray}

{\bf hw16-11}: Show Eq.(\ref{eq.16_30a},\ref{eq.16_30b}). Note the minus sign in the normalizations Eq.\ref{eq.16_30b}.

The polarization vectors Eq.\ref{eq.16_29a} and Eq.\ref{eq.16_29c} is obtained from Eq.\ref{eq.16_29a} as follows:
\begin{eqnarray}
&& J_- \eps^\mu(p,+1) = \sqrt{2} \eps^\mu(p, 0) \label{eq.16_31a} \\
&& J_- \eps^\mu(p, 0) = \sqrt{2} \eps^\mu(p,-1) \label{eq.16_31b} \\
&& J_- \eps^\mu(p,-1) = 0 \label{eq.16_31c}
\end{eqnarray}
We can also show
\begin{eqnarray}
  J_+ \eps^\mu(p,+1) = 0, \label{eq.16_32}
\end{eqnarray}
and hence the $3$ polarization vectors transform as a triplet (vector)
under rotations.

{\bf hw16-12}: Show Eq.(\ref{eq.16_31a},\ref{eq.16_31b},\ref{eq.16_31c}) and Eq.\ref{eq.16_32}.

hint: Step up and down operators
\begin{eqnarray}
  J_\pm = J_x \pm i J_y \label{eq.16_33}
\end{eqnarray}
can be expressed as $4\times 4$ matrix by using the $4\times 4$ matrix representation of $J_x$ and $J_y$.

A somewhat strange looking sign convention of Eqs.(\ref{eq.16_29a},\ref{eq.16_29b},\ref{eq.16_29c}) was chosen by
Jacob and Wick, because of the simplicity of Eqs.(\ref{eq.16_31a},\ref{eq.16_31b},\ref{eq.16_31c}), and is called
Jacob-Wick phase convention.
The set of equations Eqs.(\ref{eq.16_31a},\ref{eq.16_31b},\ref{eq.16_31c}) tells that the three $4$-vectors Eqs.(\ref{eq.16_29a},\ref{eq.16_29b},\ref{eq.16_29c}) transform as a triplet under rotation in the rest frame of a massive vector boson.

{\bf hw16-13}: Please obtain
\begin{eqnarray}
&& \eps^\mu(p',h)   = B_z(y)   \eps^\mu(p,h) \label{eq.16_34a} \\
&& \eps^\mu(p{''},h)  = R_y(\theta) \eps^\mu(p',h) \label{eq.16_34b} \\
&& \eps^\mu(p{'''},h) = R_z(\phi)   \eps^\mu(p{''},h) \label{eq.16_34c} 
\end{eqnarray}
by using the matrix representations of $B_z$, $R_y$, $R_z$ we found in the
past homeworks. The last expressions Eq.\ref{eq.16_34c} for $h=+1,0,-1$ are the
general form of the vector boson polarization vector in the frame
where the vector boson $4$-momentum is
\begin{eqnarray}
  && p{'''}^\mu = (E,p\sin\theta\cos\phi,p\sin\theta\sin\phi,p\cos\theta)\label{eq.16_35a} \\
  && E = m\cosh y, p = m\sinh y \label{eq.16_35b}
\end{eqnarray}
When the rapidity is positive $(y>0)$, the polarization index $h=+1,0,-1$
is called the helicity, the eigen value of the operator
\begin{eqnarray}
  h=\frac{ {\vec J}\cdot {\vec p}}{|{\vec p}|} \label{eq.16_36}
\end{eqnarray}
Note my notation for $3$-vectors:
\begin{eqnarray}
&& {\vec J} = (J_x, J_y, J_z)^T \label{eq.16_37a} \\
&&{\vec p} = (p_x, p_y, p_z)^T \label{eq.16_37b} \\
&&{\vec a} \cdot {\vec b} = a_x b_x + a_y b_y + a_z c_z \label{eq.16_37c} \\
&& |{\vec a}| = \sqrt{ {\vec a}\cdot {\vec a} } \label{eq.16_37d}
\end{eqnarray}
Please note that the polarization vector of $h=0$ state and the $4$-vector
$p^\mu$ becomes very similar at high energies ($E \gg m$, or $y \gg 1$)
\begin{eqnarray}
&&p^\mu
= (E, p\sin\theta\cos\phi, p\sin\theta\sin\phi, p\cos\theta) \label{eq.16_38a} \\
&&\eps^\mu(p,h=0)
= \frac{1}{m}(p, E\sin\theta\cos\phi, E\sin\theta\sin\phi, E\cos\theta)  \label{eq.16_38b}
\end{eqnarray}
and therefore,
\begin{eqnarray}
  \eps^\mu(p,h=0) - \frac{p^\mu}{m}
&=& -\frac{E-p}{m} (1, -\sin\theta\cos\phi, -\sin\theta\sin\phi, -\cos\theta) \\
&=& -\frac{m}{E+p} (1, -\sin\theta\cos\phi, -\sin\theta\sin\phi, -\cos\theta) \\
&=& -e^{-y}  (1, -\sin\theta\cos\phi, -\sin\theta\sin\phi, -\cos\theta) \label{eq.16_39}
\end{eqnarray}

{\bf hw16-14}:  Show Eq.\ref{eq.16_39}.

This relation is the key in the understanding of the renormalizability
of spontaneously broken gauge theories with a massive vector boson,
and it gives rise to the Goldstone-boson equivalence theorem, which
tells that the helicity zero mode of the vector boson behaves as
the Goldstone boson at very high energies ($E \gg m$, or $e^y \gg 1$).

{\bf hw16-15}: Show that the polarization sum of the spin $1$ vector boson
wave functions gives
\begin{eqnarray}
  \sum_{h=\pm1,0} \eps^\mu(p,h) \eps^\nu(p,h)^*
= [ -g^{\mu\nu} + \frac{p^\mu p^\nu}{m^2} ] \label{eq.16_40}
\end{eqnarray}
in general frame Eqs.(\ref{eq.16_35a},\ref{eq.16_35b}). In the rest frame Eq.\ref{eq.16_27}, it is simply
\begin{eqnarray}
  \sum_{h=\pm1,0} \eps^\mu(p,h) \eps^\nu(p,h)^*
= [ -g^{\mu\nu} + \frac{p^\mu p^\nu}{m^2} ]=
\begin{pmatrix}
  0 & 0 & 0 & 0 \\
  0 & 1 & 0 & 0 \\
  0 & 0 & 1 & 0 \\
  0 & 0 & 0 & 1
\end{pmatrix} \label{eq.16_41}
\end{eqnarray}
whose trace ($=3$) counts the degree of freedom of massive vector boson.

Now, we are ready to study the polarization vectors of massless gauge
bosons.

We first notice that we cannot take the $mass \to 0$ limit in the massive
vector boson polarization vectors, since the helicity zero ($h=0$,
longitudinally polarized) wave function diverges as $E/m$ in this limit.

This contrasts sharply with the spin $1/2$ fermion wave function, for
which the massless limit was taken smoothly from the massive wave
functions (spinors). The reason for the difference is that in case
of spin $1/2$ fermions, the degree of freedom does not change when we
take the massless limit (the spin $1/2$ fermions have just two components,
helicity $+1/2$ and $-1/2$ components, whether it is massive or massless),
whereas in case of the vector bosons (or any higher spin particles),
the massive state has $3$ helicity states ($2s+1$ in case of spin $s$
particle) while the massless states have only $2$ ($h = +s$ or $-s$).

We need to have constraints which excludes the longitudinally polarized
state from the physical state of massless spin $1$ (and higher spin)
particles. In case of massless spin $1$ particles, we need two
constraints, as I explained above.  Let me copy them here:
\begin{eqnarray}
&& p_\mu \eps^\mu(p,h) = 0 \label{eq.16_42a}\\ 
&& n_\mu \eps^\mu(p,h) = 0 \label{eq.16_42b}
\end{eqnarray}

The first condition comes from the covariant gauge fixing, the same as
in the massive gauge boson, which projects out the scalar component
of the $4$-component vector, the term proportional to
\begin{eqnarray}
&& \eps^\mu(p,s=0) = p^\mu/m~  (if~ m \neq 0) \label{eq.16_43a}\\ 
&& \eps^\mu(p,s=0) = p^\mu/E ~ (if~ m = 0   ) \label{eq.16_43b}
\end{eqnarray}

The condition Eq.\ref{eq.16_42b} kills the remaining degree of freedom for
massless vector boson. The residual gauge fixing vector $n^\mu$
is arbitrary, as long as it is not parallel to $p^\mu$
\begin{eqnarray}
  n^\mu p_\mu = (n\cdot p) \neq 0. \label{eq.16_44}
\end{eqnarray}
When $n^2 = 0$ (light-like), $n^\mu$ is called the light-cone vector,
and the gauge Eq.\ref{eq.16_42b} is called the light-cone gauge.

{\bf hw16-16}: Show in the light cone gauge, the polarization sum can be
expressed as
\begin{eqnarray}
  \sum_{h=\pm 1} \eps^\mu(p,h) \eps^\nu(p,h)^*
    = -g^{\mu\nu} + (n^\mu p^\nu + n^\nu p^\mu)/(n\cdot p) \label{eq.16_45}
\end{eqnarray}
where $n\cdot p = n^\mu p_\mu \neq 0$.
My favorite light-cone gauge is to choose $n^\mu$ along the $p^\mu$
direction with the opposite $3$-momentum direction,
\begin{eqnarray}
  n^\mu = (|p|, -{\vec p}), ~p^\mu = (|p|, {\vec p}) \label{eq.16_46}
\end{eqnarray}
It is clear that $n^2 = p^2 = 0$, and $n\cdot p = 2|p|^2 > 0$ if $|p| > 0$.
(We do not consider a massless particle with zero momentum, since
there is no way that we can observe them as a photon.)

{\bf hw16-17}: Show that when we take ${\vec p}$ along the positive $z$-direction,
\begin{eqnarray}
  p^\mu = (|p|, 0, 0, |p|) \label{eq.16_47}
\end{eqnarray}
the polarization sum in this gauge gives
\begin{eqnarray}
  \sum_{h=\pm1} \eps^\mu(p,h) \eps^\nu(p,h)^*
    = -g^{\mu\nu} + (n^\mu p^\nu + n^\nu p^\mu)/(n\cdot p) =
    \begin{pmatrix}
      0 & 0 & 0 & 0 \\
      0 & 1 & 0 & 0 \\
      0 & 0 & 1 & 0 \\
      0 & 0 & 0 & 0
    \end{pmatrix} \label{eq.16_48}
\end{eqnarray}
The trace is $2$, counting the degree of freedom of massless vector boson.

This is the gauge I adopted when I made my computer code HELAS, which
has been used in MadGraph. The reason is simply that we can use the
three-momentum of the vector boson to fix the wave function, which is
most convenient when we write purely numerical codes.

For specific processes, we often find very simple analytic expressions
for scattering amplitudes by choosing the gauge vector, $n^\mu$.

{\bf hw16-18}: Show the $4\times 4$ matrix representation of the polarization
sum Eq.\ref{eq.16_48} when $p=(E,0,0,E)$ and $n=(1,\sin\theta,0,\cos\theta)$.

Now, let us examine the helicity conservation for our massless vector
boson wave functions. For simplicity, let us take the momentum along
the positive $z$ axis,
\begin{eqnarray}
  p^\mu = (E, 0, 0, E) \label{eq.16_49}
\end{eqnarray}
and our wave functions in the $n^\mu = (E, 0, 0, -E)$ gauge are
\begin{eqnarray}
  \eps^\mu(p,h) = (1/\sqrt{2}) (0, -h, -i, 0)^T \label{eq.16_50}
\end{eqnarray}
for the helicity $h=\pm 1$.

{\bf hw16-19}: Calculate
\begin{eqnarray}
  && A \eps^\mu(p,h) \label{eq.16_51a}\\
  && B \eps^\mu(p,h) \label{eq.16_51b}
\end{eqnarray}
for $A = J_x + K_y$ and $B = J_y - K_x$, and observe that they
are not vanishing (our wave functions are NOT an eigen
vectors of A and B with zero eigen values). This disagrees
with Weinberg's conjecture I explained in my lecture.

{\bf hw16-20}: Please show Eq.\ref{eq.16_52} and Eq.\ref{eq.16_53}:
\begin{eqnarray}
  A^2 \eps^\mu(p,h) = 0,~~~B^2 \eps^\mu(p,h) = 0 \label{eq.16_52}
\end{eqnarray}
\begin{eqnarray}
  p^\mu A \eps^\mu(p,h) = 0,~~~ p^\mu B \eps^\mu(p,h) = 0 \label{eq.16_53}
\end{eqnarray}
Eq.\ref{eq.16_52} tells that our wave functions give zero eigen value for $W^2$. Eq.\ref{eq.16_53} tells that although our $\eps^\mu(p,h)$ are not the eigen vector of $A$ and $B$, but the transformed state can be removed by using the covariant gauge fixing condition Eq.\ref{eq.16_42a}. Either (or both?) of the above may suffice to show the proportionality
\begin{eqnarray}
  W^a \eps^\mu(p,h) = h p^a \eps(p,h), \label{eq.16_54}
\end{eqnarray}
or the conservation of helicity under Poincare transformations.
This is my understanding.  Please let me know if you learn something
on this issue.

This completes my homework for vector boson polarization vectors.
Before starting QFT, let us examine the helicity conservation for
massless fermions:

{\bf hw16-21}: Obtain $A = J_x + K_y$ and $B = J_y - K_x$ for $S_L$ and $S_R$
representations, for $\psi_L$ and $\psi_R$. Let us denote them
$A_L$ and $B_L$ for $\psi_L$, and $A_R$ and $B_R$ for $\psi_R$.

{\bf hw16-22}: Show
\begin{eqnarray}
  && A_L u_L(p,-1/2) = B_L u_L(p,-1/2) = 0 \label{eq.16_55a} \\
  && A_R u_R(p,-1/2) = B_R u_R(p,-1/2) = 0 \label{eq.16_55b}
\end{eqnarray}
for our solutions of free massless fermion wave functions.

Following Weinberg's discussion, the wave functions are the
eigenfunctions of the operators $A$ and $B$ with zero eigenvalues,
and hence the Pauli-Lubanski operator is proportional to the
massless four momentum vector, leading to the Poincare invariance
of their ratio, the helicity.

Although we obtained the helicity conservation for massless (Weyl)
fermions as their E.O.M., this is a more general statement which
explains why such E.O.M. is obtained from Lorentz invariant Lagrangian.

I noticed that I forgot to study the normalization of fermion wave
functions, which correspond to the polarization sum Eq.\ref{eq.16_40} or Eq.\ref{eq.16_45} for vector bosons. Let me give it here.

First, our fermion wave functions are normalized as follows:
\begin{eqnarray}
  &&\sum_{h=\pm1/2} u(p,h) {\overline u}(p,h)
    = m + p^\mu \gamma_\mu \label{eq.16_56a}\\
  &&\sum_{h=\pm1/2} v(p,h) {\overline v}(p,h)
    = -(m - p^\mu \gamma_\mu) \label{eq.16_56b}
\end{eqnarray}
in the $4$-component notation (upper two transforms as $\psi_L$,
the lower two transform as $\psi_R$), and
\begin{eqnarray}
  {\overline \Psi} = \Psi^\dagger \gamma^0
             = (\psi_R^\dagger, \psi_L^\dagger) \label{eq.16_57}
\end{eqnarray}
is the Dirac conjugate. You can easily prove Eqs.(\ref{eq.16_56a},\ref{eq.16_56b}) by using
the general expressions for
\begin{eqnarray}
  u(p,h) =  
  \begin{pmatrix}
    u_L(p,h) \\ u_R(p,h)
  \end{pmatrix},~~~
  v(p,h) = 
  \begin{pmatrix}
    v_L(p,h) \\ v_R(p,h)
  \end{pmatrix} \label{eq.16_58}
\end{eqnarray}
you obtained in the past homework. You may do so, but I give
just two simple cases as this homework.

{\bf hw16-23}: Show Eq.(\ref{eq.16_56a},\ref{eq.16_56b}) for a massive fermion in the rest frame, $p = (m, 0, 0, 0)$.

{\bf hw16-24}: Show Eq.(\ref{eq.16_56a},\ref{eq.16_56b}) for a massless fermion in the frame $p = (E, 0, 0, E)$. Please show that there is no mixing
between the $\psi_L$ and $\psi_R$ components, as in Eq.(\ref{eq.16_60a},\ref{eq.16_60b}).
\begin{eqnarray}
  \sum_{h=\pm1/2} u(p,h) {\overline u}(p,h) &=&
  \begin{pmatrix}
    0 & u_L(p,-\frac{1}{2}) u_L^\dgr(p,-\frac{1}{2}) \\
    u_R(p,+\frac{1}{2}) u_R^\dgr(p,+\frac{1}{2}) & 0
  \end{pmatrix}\nonumber \\
  &=&
  \begin{pmatrix}
    0 & p\cdot \sigma_+\\
    p\cdot \sigma_- & 0
  \end{pmatrix} \label{eq.16_60a} \\
  \sum_{h=\pm1/2} v(p,\lambda) {\overline v}(p,h)&=&
  \begin{pmatrix}
    0 & v_L(p,+\frac{1}{2}) v_L^\dgr(p,+\frac{1}{2}) \\
    v_R(p,-\frac{1}{2}) v_R^\dgr(p,-\frac{1}{2}) & 0 
  \end{pmatrix} \nonumber \\
  &=&
  \begin{pmatrix}
    0 & -p\cdot \sigma_+\\
    -p\cdot \sigma_- & 0
  \end{pmatrix} \label{eq.16_60b}
\end{eqnarray}
That's all for hw16, and we are ready to start calculating transition 
amplitudes in QFT.\\

Best regards,\\

Kaoru

\end{document}
\documentclass[12pt]{article}
\usepackage{amsmath,graphicx,color,epsfig,physics}
\usepackage{float}
\usepackage{subfigure}
\usepackage{slashed}
\usepackage{color}
\usepackage{multirow}
\usepackage{feynmp}
\textheight=9.5in \voffset=-1.0in \textwidth=6.5in \hoffset=-0.5in
\parskip=0pt
\def\del{{\partial}}
\def\dgr{\dagger}
\def\eps{\epsilon}
\def\lmd{\lambda}
\def\th{\theta}
\begin{document}

\begin{center}
{\large\bf HW23 for Advanced Particle Physics} \\
  
\end{center}

\vskip 0.2 in

Dear students,\\

Today, we obtain e+e- to f fbar amplitudes in the second order of
  S-matrix expansion, with the Z-boson propagator.  I explain how the
  cross section and the invariant flux factor are defined together with
  our invariant phase space factor for our normalization of one-particle
  state.  I also explain Schwinger-Dyson summation of one-particle
  irreducible (1PI) two-point functions in the full (connected) Z-boson
  propagator, and show how Unitarity determines the Z-boson decay width.

  ==================================================
  E: e+ e- -> f fbar (continued from eq.(53) of hw22
  ==================================================

  I retain a few useful formula from hw22 below.

  1) D_\mu = \del_\mu
            + i g_W/\sqrt{2} [T^+ W^+_\mu  + T^- W^-_\mu]
            + i g_Z [ T^3 - Q \sin^2\theta_W ] Z_\mu
            + i e Q A_\mu

  2) L_\int = \Psibar i(D_\mu-\del_\mu) \gamma^\mu \Psi
            = -gZ Z_\mu \Psibar \gamma^\mu (T^3 - Q \sw2) \Psi + ...

  3a) u_L     gZ (T^3 - Q\sw2) = gZ( 1/2 - 2/3 \sw2) = gZuL
  3b) u_R     gZ (T^3 - Q\sw2) = gZ( 0   - 2/3 \sw2) = gZuR
  3c) d_L     gZ (T^3 - Q\sw2) = gZ(-1/2 + 1/3 \sw2) = gZdL
  3d) d_R     gZ (T^3 - Q\sw2) = gZ( 0   + 1/3 \sw2) = gZdR
  3e) v_L     gZ (T^3 - Q\sw2) = gZ( 1/2           ) = gZvL
  3f) l_L     gZ (T^3 - Q\sw2) = gZ(-1/2 +     \sw2) = gZlL
  3g) l_R     gZ (T^3 - Q\sw2) = gZ( 0   +     \sw2) = gZlR

  21a) \eps^\mu(P,+1) = ( 0, -1, -i, 0 )/\rt2
  21b) \eps^\mu(P, 0) = ( 0,  0,  0, 1 )
  21c) \eps^\mu(P,-1) = ( 0, +1, -i, 0 )/\rt2

  22a) uL(p,L) = \sqrt{2E} [ 0 ]
                           [ 1 ]
  22b) uR(p,R) = \sqrt{2E} [ 1 ]
                           [ 0 ]
  22c) vL(p,R) = uR(p,R)^c =  i\sigma^2 uR(p,R)^* = \rt{2E} [  0 ]
                                                            [ -1 ]
  22d) vR(p,L) = uL(p,L)^c = -i\sigma^2 uL(p,L)^* = \rt{2E} [ -1 ]
                                                            [  0 ]

  25a) uL(k1,L)^\dagger \sigma_-^\mu vL(k2,R)
     = mZ[-s(th/2),c(th/2)] [1,-\sigma^1,-\sigma^2,-\sigma^3] [c(th/2)]
                                                              [s(th/2)]
     = mZ[ 0, -c(th),  -i,  s(th) ]

  25b) uR(k1,R)^\dagger \sigma_+^\mu vR(k2,L)
     = mZ[c(th/2),s(th/2)] [1, \sigma^1, \sigma^2, \sigma^3] [ s(th/2)]
                                                             [-c(th/2)]
     = mZ[ 0, -c(th),  i,  s(th) ]

  26a) M(Z > fL fLbar)
  =(-gZfL) \uL(k1,L)^\dagger \sigma_-^\mu \vL(k2,R) \eps_\mu(P,\lambda)
  = -gZfL mZ [0,-c(th),-i,s(th)]*[0,-1,-i, 0]/\rt2  for \lambda=+1
  = -gZfL mZ [0,-c(th),-i,s(th)]*[0, 0, 0, 1]/\rt2  for \lambda= 0
  = -gZfL mZ [0,-c(th),-i,s(th)]*[0,+1,-i, 0]/\rt2  for \lambda=-1
  = -gZfL mZ (0 -c(th) +1)/\rt2   for \lambda=+1
  = -gZfL mZ (0 -0  -0 -s(th))    for \lambda= 0
  = -gZfL mZ (0 +c(th) +1)/\rt2   for \lambda=-1

  26b) M(Z > fR fRbar)
  =(-gZfR) \uR(k1,R)^\dagger \sigma_+^\mu \vR(k2,L) \eps_\mu(P,\lambda)
  = -gZfR mZ [0,-c(th),+i,s(th)]*[0,-1,-i, 0]/\rt2  for \lambda=+1
  = -gZfR mZ [0,-c(th),+i,s(th)]*[0, 0, 0, 1]       for \lambda= 0
  = -gZfR mZ [0,-c(th),+i,s(th)]*[0,+1,-i, 0]/\rt2  for \lambda=-1
  = -gZfL mZ (0 -c(th) -1)/\rt2   for \lambda=+1
  = -gZfL mZ (0 -0  -0 -s(th))    for \lambda= 0
  = -gZfL mZ (0 +c(th) -1)/\rt2   for \lambda=-1

  27a) \Gamma(Z > fL+fLbar)
  = 1/2mZ \Int |M(Z>fL+fLbar;\lambda,\theta)|^2 d\PS_2(P=k1+k2)

  27b) \Gamma(Z > fR+fRbar)
  = 1/2mZ \Int |M(Z>fR+fRbar;\lambda,\theta)|^2 d\PS_2(P=k1+k2)

  28) d\PS_2(P=k1+k2)
  = (2pi)^4 \delta^4(P-k1-k2) \d^3k1/(2E1)(2pi)^3 d^3k2/(2E2)(2pi)^3

  33) dPS_2 = (1/8pi) dcos(th)/2 d(phi)/2pi

  39) -1 < \cos\theta < 1, 0 < \phi < 2pi

  40) PS_2 = 1/8pi 2p^*/M -> 1/8pi (for m1=m2=0).

  41) d\Gamma = 1/2M |M(\lambda,\theta)|^2 (1/8pi) dcos(th)/2

  42) d\Gamma      1    |M(\lambda,\theta)|^2
      -------- = -----  ---------------------
      dcos(th)    16pi          2M

  44a) \Gamma(Z > uL + uLbar)  (u=u,c;   sum over 3 colors)
  44b) \Gamma(Z > uR + uRbar)  (u=u,c;   sum over 3 colors)
  44c) \Gamma(Z > dL + dLbar)  (d=d,s,b; sum over 3 colors)
  44d) \Gamma(Z > dR + dRbar)  (d=d,s,b; sum over 3 colors)
  44e) \Gamma(Z > lL + lLbar)  (l=e,\mu,\tau)
  44f) \Gamma(Z > lR + lRbar)  (l=e,\mu,\tau)
  44g) \Gamma(Z > vL + vLbar)  (v=v_e,v_\mu,v_\tau or v=v_1,v_2,v_3)

  45a) \Gamma(Z > u+ubar) = \Gamma(Z>uL+uLbar) + \Gamm(Z>uR+uRbar)
  45b) \Gamma(Z > d+dbar) = \Gamma(Z>dL+dLbar) + \Gamm(Z>dR+dRbar)
  45c) \Gamma(Z > l+lbar) = \Gamma(Z>lL+lLbar) + \Gamm(Z>lR+lRbar)
  45d) \Gamma(Z > v+vbar) = \Gamma(Z>vL+vLbar)

  46) \Gamma_Z = 2*\Gamma(Z>uubar)
                +3*\Gamma(Z>ddbar)
                +3*\Gamma(Z>llbar)
                +3*\Gamma(Z>nunubar)

              \Gamma(\cos(th)>0) - \Gamma(\cos(th)<0)
  52) A_FB =  ---------------------------------------
              \Gamma(\cos(th)>0) + \Gamma(\cos(th)<0)

  53a) \Gamma(cos(th)>0) = \Int_0^1  dcos(th) d\Gamma/dcos(th)
  53b) \Gamma(cos(th)<0) = \Int_-1^0 dcos(th) d\Gamma/dcos(th)

                              *****

  Let me finally show how to compute the differential cross section of
  a 2 to 2 process

  54) e-(k1,h1) + e+(k2,h2) -> f(k3,h3) + fbar(k4,h4)

  in the covariant perturbation theory.  Here, kj and hj for j=1,2,3,4
  are four-momenta and helicities.  We take the massless fermion limit,
  m1=m2=m3=m4=0 for simplicity.  The square-root of

  55) s = (k1+k2)^2 = (k3+k4)^2,

  \sqrt{s} gives the total energy of the process in the colliding e+e-
  rest frame (the laboratory frame of the symmetric collider experiments
  like LEP/SLC).  The four momenta in this frame are parametrized as

  56) k1 = (E,      0, 0,      E)
      k2 = (E,      0, 0,     -E)
      k3 = (E,  E\sth, 0,  E\cth)
      k4 = (E, -E\sth, 0, -E\cth)

  where \sth=\sin\theta, \cth=\cos\theta.  E=\sqrt{s}/2.

  Cross section for a 2 particle to n particle scattering events

  57a) a + b > c + d + ...

  are calculated in general as

  57b)       1
  d\sigma = ----  \Sum-ave  |M(a+b > c+d+...)|^2  d\PS(c+d+...)
             4F

  where the flux factor is defined in a Lorentz invariant manner in
  terms of the four-momenta (k1 and k2) of the incoming two particles
  of masses m_a and m_b

  58) 4F = 4 \sqrt{ (p_a*p_b)^2 - m_a^2 m_b^2 }

  Unlike the case of the particle decay width, whose normalization can
  be obtained directly from the unitarity of S-matrix elements, we need
  several steps to arrive at the normalization of the cross section
  formula (57b).  Since there should be many good textbooks explaining
  this, please study them carefully to convince yourself, since my
  explanation below is very sloppy.

  I studied it when I was a student, but at a much later time, I
  introduced the Lorentz invariant commutators between creation and
  annihilation operators, with the factor (2E) multiplying the
  \delta^3(pvec-pvec') factor, so that I can obtain the `invariant'
  matrix elements directly from the S-matrix.  It is this process,
  which allowed us to express the final state phase space, d^3p/(2pi)^3,
  as d^3p/2E(2pi)^3, for each particle.  Since the phase space should
  be V*d^3p/(2pi)^3, in the fixed frame with volume V that is used by
  Fermi to obtain his Golden Rule (59) below,

  59) |(2pi)^4 \delta^4(p_i - p_f)|^2
    = \Int d^4x e^{-i(p_i-p_f)*x} (2pi)^4 \delta^4(p_i - p_f)
    = \Int d^4x e^{-i 0*x}        (2pi)^4 \delta^4(p_i - p_f)
    = \Int d^4x                   (2pi)^4 \delta^4(p_i - p_f)
    = T * V                       (2pi)^4 \delta^4(p_i - p_f)

  the relative difference between our Lorentz invariant normalization of
  one-particle state and that of Fermi is \sqrt{2E*V} for each particle.
  This factor gives

  60) (1/\sqrt{2E*V})^2 V*d^3p/(2pi)^3 => d^3p/2E(2pi)^3

  for all the final state particles, thereby making our Lorentz
  invariant phase space factor by combining with the four-momentum
  conservation in the r.h.s. of eq.(59).  But for the initial state,
  this relative normalization difference gives

  61) |(2pi)^4 \delta^4(p_i - p_f) M_{fi}|^2
  = T*V (1/\sqrt{2E_a*V})^2 (1/\sqrt{2E_b*V})^2 |M_{fi}|^2 dPhi_{n_f}
  = T/(V*2E_a*2E_b})                            |M_{fi}|^2 dPhi_{n_f}

  where n_f is the number of final particles and dPhi_{n_f} is our
  invariant phase space which I introduced in my lectures.  The cross
  section is then defined as

  62)      # of transitions a+b > 1 + 2 + ... + n_f per unit time
  \sigma = ------------------------------------------------------
           # of incoming particles  per unit square per unit time
             |<1+2+...+n_f| S |a+b>|^2 / T
         = ---------------------------------
                        v_{ab}/V
            T/(V*2E_a*2E_b}) |M_{fi}|^2 dPhi_{n_f} / T
         = --------------------------------------------
                        v_{ab}/V
            |M_{fi}|^2 dPhi_{n_f}
         = -----------------------
             (2E_a)(2E_b)*v_{ab}

hw23-1:  Although I hesitate to ask you to follow my heuristic arguments
  above, please follow the steps from (59) to (62), perhaps consulting
  a good textbook.  Although I haven't had more rigorous derivation of
  the cross section formula than what I wrote above, this level of
  understanding was still useful for my own research.

  From the above argument, the invariant `flux' factor is defined as

  63) 4F = (2E_a)*(2E_b)*v_{ab}

  where v_{ab} is the relative velocity between the particles a and b.
  The appearance of the factors (2E_a) and (2E_b) is a consequence
  of our relativistic normalization, d^3p/(2E), of all the one-particle
  states.  The invariant expression (58) should hence be regarded as
  the Lorentz invariant expression of the relative velocity:

  64) v_{ab} = \sqrt{ (p_a*p_b)^2 - m_a^2 m_b^2 } / (E_a*E_b)

hw23-2:  Show that (64) gives the velocity of the particle a in the
  particle b rest frame, and vice versa.  Please also show that, in the
  a+b collision rest frame, it becomes

  65) v_{ab} = 2\beta   (\beta = p_a/E_a = p_b/E_b)

  when m_a=m_b.  Although we're sure we cannot call 2\beta as the
  `relative velocity' between two incoming particles with the same mass
  and the same momentum, it sounds fine in the non-relativistic limit.
  That it is not right in the relativistic limit is clear because
  2\beta can exceed 1.  I think that we can make sense of relative
  `velocity' only in the rest frame of one of the two particles.
  The massless limit,

  66) v_{ab} = 2, or 4F = (2E)(2E)2 = 2s   (s=4E^2)

  is obtained as a smooth limit of (64).  Although I think that there
  should be a way to define massless particle scattering from the first
  principle, I do not know how to do it.

                                 *****

  The symbol \Sum-ave in the definition of the cross section (57b) stands
  for summation over all the final states with the same four-momenta
  (which may differ by helicities, color degree of freedom, etc) and
  average over the initial states (helicities if the colliding particle
  has spin, color degree of freedom if the colliding quark and gluon
  distributions are defined as the sum over all colors, which is the
  case for all commonly used PDF's.

  In our case, if the initial e+e- beams are un-polarized, and if we do
  not measure the polarization of the final fermions (for f=\tau, it is
  possible to measure its polarization through Parity violating \tau
  decays),

  67) \Sum-ave = (1/2)(1/2) \Sum_{h1,h2,h3,h4,i3,i4}

  where (1/2) \Sum_{h1} is the average of e- helicities,
  (1/2) \Sum_{h2} is the average of e+ helicities,
  \Sum_{h3} is the sum over f helicities,
  \Sum_{h4} is the sum over fbar helicities,
  \Sum_{i3} is the sum over colors (i3=red,green,blue) if
  f is a quark,
  \Sum_{i4} is the sum over colors if fbar is an anti-quark.

  Please note that in the massless limit, the chirality conservation of
  the gauge interactions give the selection rule:

  68a) h2=-h1, h4=-h3

  and the quark-quark-Z vertex has the color factor of

  68b) \delta_{i3,i4}

  the contraction of 3 (quark) and 3^* (anti-quark) indices.

  The amplitudes M(e- e+ > f fbar) thus depend on

  69) M(e- e+ > f fbar) = M(s,\cth;h1,h3,i3)

  in the colliding beam rest frame.  Now the initial and the
  final states are

  70a) |e- e+>  = a^\dagger_{k1,h1} b^\dagger_{k2,-h1} |0>
  70b) |f fbar> = a^\dagger_{k3,h3} b^\dagger_{k4,-h3} |0>

  and we have the first (tree-level) contribution from the
  2'nd order expansion of the S=Te^{i\Int d^4x L_int(x)}:

  71) iM = <f fbar| T e^{i\Int d^4x L_int} |e- e+>
         = <f fbar| T e^{i\Int d^4x (L_eeZ + L_ffZ)} |e- e+>
         = <f fbar| T (1/2!){i\Int d^4x (L_eeZ + L_ffZ)}^2 |e- e+>
         = <f fbar| T {i\Int d^4x L_ffZ}{i\Int d^4y L_eeZ} |e- e+>

hw23-3:  Please confirm the cancellation of 1/2! under the T-ordering,
  and the appearance of the Z-boson propagator

  72) <0| T Z^\mu(x) Z^\nu(y) |0>

  whose Fourier transformation gives (by Feynman)

  73) i(-g^\mu\nu + q^\mu q^\nu/mZ^2)/[q^2-mZ^2+i0]

  by using the polarization vectors in the Z boson rest frame
  q=(mZ,0,0,0) with s=q^2=mZ^2:

  74) \Sum_{Jz=h=+1,0,-1} \epsilon^\mu(q,h) \epsilon^*\nu(q,h)
    = (-g^\mu\nu + q^\mu q^\nu/mZ^2 + i0)

hw23-4: Show that the amplitudes after removing the overall 4-momentum
  conservation factor reads

  75) M(h1,h3) = [ubar(k3,h3) \gamma_\mu v(k4,-h3)] (gZfh3)
                 (-g^\mu\nu + q^\mu q^\nu)/[q^2-mZ^2+i0]
                 [vbar(k2,-h1) \gamma_\nu u(k1,h1)] (gZeh1)

  When f-fbar is a quark-antiquark pair, there is an additional
  dependence on their color degree of freedom,

  76) \delta_{i3,i4}

  whose squared sum

  77) \Sum_{i3,i4} |\delta_{i3,i4}|^2 = 3

  is the color factor for q-qbar production from a color-singlet.  Now,
  we recall that the first term, Z > f fbar current has already been
  calculated n eq.(26):

  78a) h3=-1/2(=L): [ubar(k3,h3) \gamma^\mu v(k4,-h3)] (gZfh3)
                   =(gZfL) [uL^\dagger(k3) \sigma_-^\mu vL(k4) ]
                   =(gZfL) mZ [ 0, -c(th),  -i,  s(th) ]

  78b) h3=+1/2(=R): [ubar(k3,h3) \gamma^\mu v(k4,-h3)] (gZfh3)
                   =(gZfR) [uR^\dagger(k3) \sigma_+^\mu vR(k4)
                   =(gZfR) mZ [ 0, -c(th),  i,  s(th) ]

  All we need is hence to compute the initial current, which
  is much easier:

  79a) h1=-1/2(=L): [vbar(k2,-h1) \gamma^\mu u(k1,h1)] (gZeh1)
                  =(gZeL) [vL^\dagger(k2) \sigma_-^\mu uL(k1) ]
                  =(gZeL) mZ [ 0,  1,  -i,  0 ]

  79b) h1=+1/2(=R): [vbar(k2,-h1) \gamma^\mu u(k1,h1)] (gZeh1)
                  =(gZeR) [vR^\dagger(k2) \sigma_+^\mu uR(k1) ]
                  =(gZeR) mZ [ 0,  1,  i,  0 ]

hw23-5: Confirm (79a,b), and check their overall signs.

  As I already explained for the Z > f fbar amplitudes,
  I could write the above expression from the angular
  momentum conservation:

  80a) [vL^\dagger(k2) \sigma_-^\mu uL(k1)] = cnst \eps^\mu(q,Jz=-1)

  80b) [vR^\dagger(k2) \sigma_+^\mu uR(k1)] = cnst \eps^\mu(q,Jz=+1)

hw23-6: Obtain the constant coefficients for (80a) and (80b).

hw23-7: Please confirm the current conservation conditions for the
  massless fermions (81) by using Dirac equations:

  81) q_\mu vL^\dagger \sigma_-^\mu uL(k1)
    = q_\mu vR^\dagger \sigma_+^\mu uR(k1)
    = 0

  The non-vanishing terms are proportional to the fermion mass,
  and hence are extremely small for the e+e- annihilation currents.

  If we denote the amplitude (75) as

  82)  M(h1,h3) = J_\mu(h3)
                  (-g^\mu\nu + q^\mu q^\nu/mZ^2)/[q^2-mZ^2+i0]
                  J_\nu(h1)

  the term q^\mu q^\nu/mZ^2 does not contribute (this is true even if
  mf is very heavy, such as, e+ e- > t tbar).  Hence

  83) M(h1,h3) = - J_\mu(h3) J^\mu(h1) / [q^2-mZ^2+i0]

  Now, the amplitude diverges at q^2=mZ^2, which is a violation of
  unitarity.  The unitarization of the amplitude is done as follows
  in perturbative QFT.  We introduce two types of two-point functions
  (the Z boson propagator), 1PI (one-particle irreducible) and full
  (connected).  In the one-loop order, the 1PI two point function is
  expressed as

  84) \Pi^\mu\nu(q) = (-g^\mu\nu + q^\mu q^\nu/q^2) \Pi_T(q)
                                 +(q^\mu q^\nu/q^2) \Pi_L(q)

  If we denote the tensor in front of \Pi_T as PT^\mu\nu
  and that in front of \Pi_L as PL^\mu\nu, the tensors PT
  and PL make a set of projectors (just like PL and PR for
  fermions make the chiral projectors).

hw23-8: Show the projective properties of the two tensors:

  85) PT + PL = -g^\mu\nu
        PT*PL = 0
       (PT)^2 = -PT
       (PL)^2 =  PL

  PT is a projection on to Transverse/vector/spin-1
  PL is a projection on to Longitudinal/scalar/spin-0
  components of vector-bosons in QFT.  On the mass-shell (q^2=mZ^2),
  only the spin-1 component survives.  Away from the mass shell
  (off-shell), the virtual vector bosons have both spin-1 and spin-0
  components (in the Unitary gauge).  For instance, the pseudo-scalar
  meson decays like

  86) \pi^- > \mu \numubar, K^- > \mu \nubar

  are mediated by the W-boson propagator, through its spin-0 components
  at far off-shell, q^2 = m_\pi^2 << mW^2.

  Now, the full (connected) propagator can also be decomposed also into
  the spin-1 (vector/transverse) and spin-0 (scalar/longitudinal)
  components:

  87) G^\mu\nu(q) = (-g^\mu\nu + q^\mu q^\nu/q^2) G_T(q)
                               +(q^\mu q^\nu/q^2) G_L(q)

  We can now study the relationship between the 1PI propagator
  \Pi_T(q) and the full/connected propagator G_T(q) for the spin-1
  component, since \Pi_L or G_L do not contribute to e+e-
  annihilation.  In the tree-level

  88a) i G_T(q)   = i/(q^2-m^2+i0)   (m=mZ)
  88b) i \Pi_T(q) = 0

  In the one-loop level, we find

  89a) iG_T(q) = i/(q^2-m^2)
                +i/(q^2-m^2) [i\Pi_T(q) i/(q^2-m^2)]
                +i/(q^2-m^2) [i\Pi_T(q) i/(q^2-m^2)]^2
                +i/(q^2-m^2) [i\Pi_T(q) i/(q^2-m^2)]^3
                + ...

  89b) G_T(q) = 1/(q^2-m^2) {1 + [-\Pi_T(q)/(q^2-m^2)]
                               + [-\Pi_T(q)/(q^2-m^2)]^2 + ...}
              = 1/(q^2-m^2) {1 - [-\Pi_T(q)/(q^2-m^2)]}^{-1}
              = 1/(q^2-m^2) {1 + \Pi_T(q)/(q^2-m^2)}^{-1}

  89c) [G_T(q)]^{-1} = {1 + \Pi_T(q)/(q^2-m^2)} (q^2-m^2)
                     = q^2 - m^2 + \Pi_T(q)

hw23-9: Follow the steps in (89) by yourself.  It is called Schwinger-
  Dyson summation, named, respectively, after the theorists who pioneered
  perturbative QFT and who formulated it elegantly.

  The above derivation is useful when there appear more than one
  propagators in one channel (just like the neutral currents in the SM,
  where both photon and Z propagators contribute, and they mix in the
  one-loop order).  In such cases, the 1PI propagators \Pi_T,L become
  a matrix, and the ordering of the multiplication become important.

  In perturbative QFT, we define the pole of the full propagator as our
  starting point (the `free' on-shell particle).  The condition is

  90) [G_T(q)]^{-1} = 0 at q^2 = m_pole^2 = M^2 - iMG

  In general, the pole is complex, and we parametrize its real part as
  M^2 (the pole mass squared) and its imaginary part as iMG, where G is
  the full width (which we calculated above for Z).

  The pole condition (90) for the one-loop result (89c) reads:

  91) (M^2-iMG) - m^2 + \Pi_T(M^2-iMG) = 0

  Neglecting higher order terms, we find

  92a) M^2 - m^2 + Re\Pi_T(M^2) = 0
  92b)     -iMG + iIm\Pi_T(M^2) = 0

  from the real and imaginary parts of the identity (91).  (92a) is called
  the mass renormalization factor, and (92b) gives the normalization of
  the full width G=\Gamma:

  93) G = \Gamma = 1/M \Im\Pi_T(M^2)

hw23-10: Derive (91)-(93) from the definition (90) pf the pole.

  Now, please remember the unitarity relation (the optical theorem)

  94) 2 Im T_ii = \Sum_f |T_fi|^2

  By noting that the 1PI Z boson propagator is simply

  95) \Pi_T = <Z| T |Z> = T_ZZ

  we find

  96) G = \Gamma = 1/M Im\Pi_T(q)
                 = 1/M Im{ T_ZZ }
                 = 1/M 1/2 \Sum_f |T(Z>f)|^2
                 = 1/2M \Sum_f |T(Z>f)|^2

  The origin of the normalization factor 1/2M in the definition of the
  decay width is now clear (this is the only way that I can prove the
  validity of this normalization, which is independent of our
  `normalization' of our one-particle state).  In eq.(94), \Sum_f contains
  not only all the channels but also the phase space integrals of each
  channel.

  When you learn how to compute the one-loop amplitudes, one of the very
  first exercise that you do is to show that the imaginary part of the
  1PI propagator gives the mass times the full width.  When I worked on
  precision electroweak physics, I used the positivity of my 1PI
  propagator functions as a test of the overall sign (the imaginary part
  should be positive, if we define everything consistently).  My detailed
  instructions on the sign and the overall phase of each amplitudes are
  based on those experiences of mine.

  Finally the helicity amplitudes (99) can be evaluated on the Z-boson
  mass shell:

  97) M(h1,h3) = - J_\mu(h3) J^\mu(h1) / [q^2-M^2+iMG]  at q^2=M^2
               = i J_\mu(h3) J^\mu(h1) / MG

  with M = m_Z (observed) and G = \Gamma_Z.  Please note that the tree
  level (lowest order) amplitudes are pure-imaginary on top of the
  resonance, simply because 1/(q^2-M^2+iMG) = -i/MG at q^2=M^2.

  In order to obtain the results which satisfy the unitarity, one should
  use the \Gamma_Z which is obtained by using the same coupling constants
  and approximations for computing the helicity amplitudes (97).  If we
  use the observed value of \Gamma_Z, we will find inconsistencies, since
  the observed \Gamma_Z is typically a few % larger than our lowest order
  result, mainly because of the large positive QCD corrections to
  the Z > qqbar widths.

hw23-11: Obtain the overall factors (A,B,C,D) for

  98a) M(L,L) = A M/G gZeL gZfL (1+\cth)/2
  98b) M(L,R) = B M/G gZeL gZfR (1-\cth)/2
  98c) M(R,L) = C M/G gZeR gZfL (1-\cth)/2
  98d) M(R,R) = D M/G gZeR gZfR (1+\cth)/2

  From eq.(97), the factors A to D are all pure imaginary.  All the
  angular dependences follow from the angular momentum conservation for
  the s-channel spin-1 particle exchange, by the d-function

  99)
  d^{J=1}_{2h1,2h3}(\th) = - \eps^\mu(Jz=2h1) \eps_\mu(Jz'=2h3)^*

  which measures the overlap between the initial polarization vector with
  Jz=2h1 along the z-axis, and the final polarization vector with Jz'=2h3
  along the z'-axis, where the opening angle between the two axis is
  \theta.  Please note

  When h1=+h3 (LL and RR), the overlap is perfect at \theta=0,
  when h1=-h3 (LR and RL), the overlap is perfect at \theta=\pi.

hw23-12: Obtain the differential cross sections

  100) d\sigma
  = 1/2s (1/2)^2 \Sum_{h1,h3} |M(h1,h3)|^2 1/8\pi d\cth/2

  for un-polarized e+e- beams for general f by using the
  couplings gZfL and gZfR.  Please note

  101)
  d\sigma/d\cth = 1/(128\pi M^2) \Sum_{h1,h3} |M(h1,h3)|^2

  Next, please show the FB (forward-backward) asymmetries

  102)   \Int_0^1 d\cth d\sig/d\cth - \Int_{-1}^0 d\cth d\sig/d\cth
  A_FB = ----------------------------------------------------------
         \Int_{-1}^1 d\cth d\sig/d\cth

  for the process e- e+ > f fbar are expressed in terms of the
  Zff couplings, gZfL and gZfR.

hw23-13: Please express A_FB for f=e and f=b (f=d) as a function of
  \sw2 = \sin^2\theta_W.

  Please confirm that A_FB for f=b is more sensitive to \sw2
  than A_FB for f=e, by comparing the magnitudes of

  103) d A_FB / d\sw2 at \sw2 = 0.23

  for the two processes.

Although we can play a lot with the amplitudes which we calculated, 
let me stop my hw23 here.  If ILC or CEPC/FCC_ee will be build, they 
will perform precision Z boson physics on Z pole, probably after 
obtaining enough information on the Higgs boson at \sqrt{s}=250GeV.

Best regards,

Kaoru




\end{document}
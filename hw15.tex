\documentclass[12pt]{article}
\usepackage{amsmath,graphicx,color,epsfig,physics}
\usepackage{float}
\usepackage{subfigure}
\usepackage{slashed}
\usepackage{color}
\usepackage{multirow}
\usepackage{feynmp}
\textheight=9.5in \voffset=-1.0in \textwidth=6.5in \hoffset=-0.5in
\parskip=0pt
\def\del{{\partial}}
\def\dgr{\dagger}

\begin{document}

\begin{center}
{\large\bf HW15 for Advanced Particle Physics} \\

\end{center}

\vskip 0.2 in

Dear students,

  This homework hw15 is rather long, since I retained exercises for
  Lorentz transformation and charge conjugations of 4-spinors.
  Please submit your report by next next Monday, Mar/12, since
  I will be out of town between Mar/02(Fri) and Mar/12(Mon).


PT1-2018: hw15 (2018.02.27)

  Last week, we learned that Dirac fermion is a combination of two Weyl
  fermions, $\psi_L$ and $\psi_R$, which couple with the Dirac mass term.
  Because of this Lorentz invariant coupling mass term, the two Weyl
  fermions transform together covariant under the Lorentz transformation.
  Dirac fermion hence has four complex components.  We showed the Lorentz
  transformation of $\psi_L$ and $\psi_R$, and showed the Lorentz invariance
  of the Dirac Lagrangian.

  I introduced QFT in which all fields are interpreted as operators
  which create and annihilate particles and anti-particles, and obtained
  free-field expansions for scalars and fermions.  We then obtained
  explicit forms the spinors, $u_L(p,h)$ and $u_R(p,h)$, and their charge
  conjugates, $v_R(p,h)$ and $v_L(p,h)$, respectively.

  This week, I explain charge conjugation transformation carefully,
  since it is very important for understanding the interactions of
  fermions.

  At the end of my lengthy introduction of spin $1/2$ particles, I review
  all what we learned about Dirac/Majorana fermions by using the
  four-spinor notation.

  I then introduce Poincare algebra that dictates the space-time symmetry
  of QFT, and derive its most important consequences, the conservation
  of the invariant mass and the spin of a massive particle, and that of
  the helicity of massless particles.

  At the end of my lectures this week, I introduce free field expansions
  for the vector bosons.

  Let me re-introduce charge conjugation for spin $1/2$ particles, starting
  from Weyl fermions, because all the SM fermions are Weyl fermions before
  the symmetry breaking.  Let me write the SM Lagrangian for fermions as
 \begin{eqnarray}
  {\cal L}_{fermion} = (f_L)^\dagger iD_\mu \sigma_-^\mu (f_L)
  + (f_R)^\dgr iD_\mu \sigma_+^\mu (f_R) \label{eq.15_59}
 \end{eqnarray}
  where $f_L$ can be $Q_i$ or $L_i$, while $f_R$ can be ${u_R}_i$, ${d_R}_i$, ${l_R}_i$, with $i=1,2,3$ for family replication indices. Last week, we learned that ${\cal L}_{Dirac}$ is invariant under Charge conjugation.  The essence of the proof, however, is simply that the Lagrangian is a singlet (a point) in the spinor space, and therefore it is the same as its transpose.

  Well, our SM Lagrangian Eq.\ref{eq.15_59} is also a singlet in the spinor space, and hence its transpose is the same as itself. What do we learn?
\begin{eqnarray}
  {\cal L}_{fermion} &=& ( {\cal L}_{fermion} )^T
  =   [ (f_L)^\dagger iD_\mu \sigma_-^\mu (f_L) ]^T
    + [ (f_R)^\dagger iD_\mu \sigma_+^\mu (f_R) ]^T \nonumber \\
  &=& - (f_L)^T [\sigma_-^\mu]^T  [iD_\mu]^T [(f_L)^\dagger]^T
    - (f_R)^T [\sigma_+^\mu]^T  [iD_\mu]^T [(f_R)^\dagger]^T \nonumber \\
  &=& - (f_L)^T [\sigma_-^\mu]^T  [iD_\mu]^T  (f_L)^*
    - (f_R)^T [\sigma_+^\mu]^T  [iD_\mu]^T  (f_R)^* \label{eq.15_60}
\end{eqnarray}
  where
\begin{eqnarray}
  [iD_\mu]^T
  =  i\overleftarrow{\del}_\mu -g(T^a)^T A^a_\mu
  = -i\del_\mu             -g(T^a)^T A^a_\mu \label{eq.15_61}
\end{eqnarray}
  In Eq.\ref{eq.15_61}, $\overleftarrow{\del}$ means that the derivative operates in its left-hand-side (i.e., on $(f_L)^T$ and $(f_R)^T)$. When $f_L$ or $f_R$ are color-triplets (quarks) or weak-doublets ($Q$ or $L$), generators associated with gauge bosons should also be transposed.

  By inserting $i\sigma^2$ (note $ (\sigma^2)^2 = (i\sigma^2) (-i\sigma^2)=1$)
  in both sides of $[\sigma_\pm^\mu]^T$ and by using Eq.\ref{eq.15_55},
\begin{eqnarray}
  (\sigma^2) [\sigma_\pm^\mu]^T (\sigma^2) = \sigma_\mp^\mu \label{eq.15_55}
\end{eqnarray}
  and the ``definition'' of ``charge conjugate'' operations in Eq.({\ref{eq.15_30a},\ref{eq.15_30b}):
\begin{eqnarray}
  && (f_L)^c = -i\sigma^2 (f_L)^* \label{eq.15_30a} \\
  && (f_R)^c =  i\sigma^2 (f_R)^* \label{eq.15_30b}
\end{eqnarray}
  the Lagrangian (Eq.\ref{eq.15_60}) can be re-written as
\begin{eqnarray}
  {\cal L}_{fermion} &=& ( {\cal L}_{fermion} )^T
  = - (f_L)^T [\sigma_-^\mu]^T  [iD_\mu]^T  (f_L)^*
    - (f_R)^T [\sigma_-^\mu]^T  [iD_\mu]^T  (f_R)^* \nonumber\\
  &=&   ((f_L)^c)^\dagger \sigma_+^\mu [-iD_\mu]^T (f_L)^c
    + ((f_R)^c)^\dagger \sigma_-^\mu [-iD_\mu]^T (f_R)^c\label{eq.15_63}
\end{eqnarray}

{\bf hw15-1}: Derive Eq.\ref{eq.15_63} from Eq.\ref{eq.15_59}.

  By inserting Eq.\ref{eq.15_61} into Eq.\ref{eq.15_63} we find
\begin{eqnarray}
  {\cal L}_{fermion} &=& \sigma_+^\mu [-iD_\mu]^T (f_L)^c
  + ((f_R)^c)^\dagger \sigma_-^\mu [-iD_\mu]^T (f_R)^c \nonumber \\
  &=& ((f_L)^c)^\dagger \sigma_+^\mu [i\del_\mu +g(T^a)^TA^a_\mu] (f_L)^c
  + ((f_R)^c)^\dagger \sigma_-^\mu [i\del_\mu +g(T^a)^TA^a_\mu] (f_R)^c \label{eq.15_64}
\end{eqnarray}
  Let us compare Eq.\ref{eq.15_64} with the original Lagrangian Eq.\ref{eq.15_592}:
\begin{eqnarray}
  {\cal L}_{fermion}
  = (f_L)^\dagger \sigma_-^\mu [i\del_\mu -g T^a A^a_\mu] (f_L)
  + (f_R)^\dagger \sigma_+^\mu [i\del_\mu -g T^a A^a_\mu] (f_R)\label{eq.15_592}
\end{eqnarray}
  We find that the SM ${\cal L}_{fermion}$ is invariant under the transformation:
\begin{eqnarray}
   &&f_L \to f_L' = (f_L)^c \label{eq.15_65a}\\
    &&f_R \to f_R' = (f_R)^c \label{eq.15_65b}\\
   &&\sigma_\pm^\mu T^a gA^a_\mu
    \to (\sigma_\pm^\mu T^a gA^a_\mu)'
    =  \sigma_\mp^\mu (T^a)^T (-gA^a_\mu)
    =  \sigma_\mp^\mu (T^a)^* (-gA^a_\mu) \label{eq.15_65c}
\end{eqnarray}

  The transformation Eq.(\ref{eq.15_65a},\ref{eq.15_65b},\ref{eq.15_65c}) is called CP transformation, since the transformed states $f_L^c$ and $f_R^c$ have opposite helicities from the original states, $f_L$ and $f_R$, respectively.  The exchange of chiral sigma-vectors, $\sigma_-^\mu$ and $\sigma_+^\mu$, is mandatory in order to keep the Lorentz invariance.

  The Yukawa interactions ${\cal L}_{Yukawa}$ are also invariant under CP, if
  the couplings are all real. The CP transformation Eq.(\ref{eq.15_65a},\ref{eq.15_65b}) gives
\begin{eqnarray}
  &&(f_L)^\dagger (f_R)      + (f_R)^\dagger (f_L)
   \to ( (f_L)^\dagger (f_R)     + (f_R)^\dagger (f_L) )'
   =   (f_L^c)^\dagger (f_R^c) + (f_R^c)^\dagger (f_L^c) \nonumber \\
   &=& ( (f_L^c)^\dagger (f_R^c) + (f_R^c)^\dagger (f_L^c) )^T 
   = - (f_R^c)^T (f_L^c)^*  - (f_L^c)^T (f_R^c)^* 
   =   (f_R)^\dagger (f_L)     + (f_L)^\dagger (f_R)  \label{eq.15_66}
\end{eqnarray}

{\bf hw15-2}: Show Eq.\ref{eq.15_66}.

hint: $((f_R)^c)^T (f_L^c)^* = (i\sigma^2 f_R^*)^T (-i\sigma^2 fL^*)^*$

  In the absence of non-removable complex phases in ${\cal L}_{Yukawa}$, the
  CP transformation Eq.(\ref{eq.15_65a},\ref{eq.15_65b},\ref{eq.15_65c}) is a symmetry of ${\cal L}_{SM}$.

  In QED and QCD, the interactions conserve Parity.  For instance,
  \begin{eqnarray}
    &&{\cal L}_{QCD}
    = {\cal L}_{gauge} + f_L^\dagger iD_\mu \sigma_-^\mu f_L
                + f_R^\dagger iD_\mu \sigma_+^\mu f_R
                - m ( f_L^\dagger f_R + f_R^\dagger f_L ) \nonumber \\
    &=& {\cal L}_{gauge} + ( f_L^\dagger iD_\mu \sigma_-^\mu f_L )^T
                + ( f_R^\dagger iD_\mu \sigma_+^\mu f_R )^T
                - m ( f_L^\dagger f_R + f_R^\dagger f_L )^T \nonumber \\
    &=& {\cal L}_{gauge} + {f_L^c}^\dagger (-iD_\mu)^T \sigma_+^\mu f_L^c
                + {f_R^c}^\dagger (-iD_\mu)^T \sigma_-^\mu f_R^c
                + m ( {f_R^c}^T {f_L^c}^*  + {f_L^c}^T {f_R^c} ) \label{eq.15_67}
\end{eqnarray}
   By using Eq.\ref{eq.15_61} for $[D_\mu]^T$ and by using Eq.\ref{eq.15_66}, we can tell that ${\cal L}_{QCD}$ is invariant under the C transformation
\begin{eqnarray}
  && (f_L) \to (f_L)' = f_R^c \label{eq.15_68a} \\
  && (f_R) \to (f_R)' = f_L^c \label{eq.15_68b} \\
  && g T^a A^a_\mu \to (g T^a A^a_\mu)' = -g (T^a)^* A^a_\mu \label{eq.15_68c} 
\end{eqnarray}
  
{\bf hw15-3}: Show that Eq.\ref{eq.15_67} is invariant under C operation Eq.(\ref{eq.15_68a},\ref{eq.15_68b},\ref{eq.15_68c}).

  Please note that the C operation Eq.(\ref{eq.15_68a},\ref{eq.15_68b},\ref{eq.15_68c}) does not change the Lorentz transformation property of the original fields, $f_L$ and $f_R$, and the transformed field, $f_R^c$ and $f_L^c$, respectively.

  I feel that the CP transformation Eq.(\ref{eq.15_65a},\ref{eq.15_65b},\ref{eq.15_65c}) is an appropriate definition of fermion to anti-fermion (matter to anti-matter) transformations.  In other words, anti-particle of left-handed (helicity $-1/2$) fermion is
  right-handed (helicity $+1/2$). As explained before, the exchange of
  $\sigma_\pm^\mu$ simply reflects this exchange of helicities between a
  fermion and an anti-fermion. In this picture, Parity inversion is not
  a symmetry under matter anti-matter exchange transformations. Only
  when the interactions of $f_L$ and those of $f_R$ are identical (left-right
  symmetric), Parity is conserved. This is what Dirac arranged by giving
  exactly the same charge to $f_L$ and $f_R$. CP invariance plus P invariance
  gives C invariance of QED and QCD (QCD on perturbative vacuum).

  Before moving on, please let me summarize how Lorentz transformations
  and Charge conjugations are treated in the Dirac four-spinor formalism.

  First, the free field expansions %Eq.(\ref{eq.14_24a},\ref{eq.14_24b})
  become one equation
\begin{eqnarray}
  \Psi(x)
  = \sum_h \int \frac{d^3p}{(2E)(2\pi)^3}(a(p,h) e^{-ipx} u(p,h) + b^\dagger(p,h) e^{ipx} v(p,h)) \label{eq.15_69}
\end{eqnarray}
  with
\begin{eqnarray}
  u(p,h) =
  \begin{pmatrix}
    u_L(p,h) \\ u_R(p,h)
  \end{pmatrix}
  ,~~~~v(p,h) =
  \begin{pmatrix}
    v_L(p,h) \\ v_R(p,h)
  \end{pmatrix} \label{eq.15_70}
\end{eqnarray}

{\bf hw15-4}: Show that the Dirac equations %Eq.\ref{eq.13_28}
gives
\begin{eqnarray}
  && (  p\gamma -m ) u(p,h) = 0 \label{eq.15_71a} \\
  && ( -p\gamma -m)  v(p,h) = 0 \label{eq.15_71b}
\end{eqnarray}
  and that Eq.(\ref{eq.15_71a},\ref{eq.15_71b}) reproduces Dirac equations %Eq.(\ref{eq.13_28},\ref{eq.13_29}) 
  Very compact!

  Now let us study the Lorentz transformation of $\Psi(x)$. All the $6$
  generators of the Lorentz transformation is expressed as
\begin{eqnarray}
  \Sigma^{\mu\nu} = \frac{i}{4} [ \gamma^\mu, \gamma^\nu ] \label{eq.15_72}
\end{eqnarray}
  by using the $\gamma$ matrices. They may not look like generators of
  rotations and boosts, but you may notice the anti-symmetric property
\begin{eqnarray}
  \Sigma^{ab} = -\Sigma^{ba}, \label{eq.15_73}
\end{eqnarray}
  and hence there are just 6 independent generators. They satisfy
\begin{eqnarray}
  [S^{ab}, S^{cd}] = i( g^{ad} S^{bc} + g^{bc} S^{ad} - g^{ac} S^{bd} - g^{bd} S^{ac}) \label{eq.15_74}
\end{eqnarray}
  where I used a shorthand, $S^{ab} = \Sigma^{ab}$, etc. Since the right-hand
  side of Eq.\ref{eq.15_74} is a linear sum of all the generators, the transformation generated by them
\begin{eqnarray}
  S(\omega_{ab}) = e^{ -\frac{i}{2} \Sigma^{ab} \omega_{ab} } \label{eq.15_75}
\end{eqnarray}
  forms a group, for real $\omega _{ab}=-\omega_{ba}$.

{\bf hw15-5}: Show the commutators in Eq.\ref{eq.15_74} by using Clifford algebra,
\begin{eqnarray}
  \{ \gamma^\mu, \gamma^\nu \} = 2g^{\mu\nu} \label{eq.15_76}
\end{eqnarray}

{\bf hw15-6}: Show that the commutation relations Eq.\ref{eq.15_74} is nothing but the commutation relations among the Lorentz transformation generators,
with the identification:
\begin{eqnarray}
  && S^{ij} = \epsilon_{ijk} J_k \label{eq.15_77a}\\
  && S^{0k} = K_k \label{eq.15_77b}
\end{eqnarray}

 That $\Sigma^{\mu\nu}$ in Eq.\ref{eq.15_72} satisfies the Lorentz algebra 
 %in Eq.\ref{eq.15_74} 
 suggests that there is a set of $4$ component field (4-spinor) which transforms covariant under Lorentz transformation.  This is what Dirac discovered.

  In our chiral representation, the $4\times 4$ matrix representation of the
  Lorentz transformations splits into two $2\times 2$ matrix transformations.

{\bf hw15-7}: Starting from the Chiral representation of $\gamma$ matrices,
%Eq.\ref{eq.13_26}, 
show that Eq.\ref{eq.15_72} splits into two block diagonal pieces
\begin{eqnarray}
  S^{ab}=
  \begin{pmatrix}
    S_L^{ab} & 0 \\
    0 & S_R^{ab}
  \end{pmatrix} \label{eq.15_78}
\end{eqnarray}
  where $S_L$ and $S_R$ are $2\times 2$ matrices,
\begin{eqnarray}
 && S_L^{ab} = \frac{i}{4} ( \sigma_+^a \sigma_-^b - \sigma_+^b \sigma_-^a )\label{eq.15_79a} \\
 && S_R^{ab} = \frac{i}{4} ( \sigma_-^a \sigma_+^b - \sigma_-^b \sigma_+^a )\label{eq.15_79b}
\end{eqnarray}
  Therefore, the upper 2-spinors $\psi_L$ and the lower 2-spinors $\psi_R$, transform independently as
\begin{eqnarray}
  && \psi_L \to \psi_L' = e^{ -\frac{i}{2} \omega_{ab} S_L^{ab} } \psi_L \label{eq.15_80a} \\ 
  && \psi_R \to \psi_R' = e^{ -\frac{i}{2} \omega_{ab} S_R^{ab} } \psi_R \label{eq.15_80b}
\end{eqnarray}
  
{\bf hw15-8}: Obtain Eqs.(\ref{eq.15_81a},\ref{eq.15_81b},\ref{eq.15_81c},\ref{eq.15_81d},\ref{eq.15_81e},\ref{eq.15_81f}) from Eqs.(\ref{eq.15_80a},\ref{eq.15_80b}):
\begin{eqnarray}
  &&S^{23} = J_1 =
  \begin{pmatrix}
    \sigma^1/2  &  0  \\ 
    0   &    \sigma^1/2
  \end{pmatrix}\label{eq.15_81a} \\
  &&S^{31} = J_2 =
  \begin{pmatrix}
    \sigma^2/2  &  0  \\ 
    0   &    \sigma^2/2
  \end{pmatrix}\label{eq.15_81b} \\
  &&S^{12} = J_3 =
  \begin{pmatrix}
    \sigma^3/2  &  0  \\ 
    0   &    \sigma^3/2
  \end{pmatrix}\label{eq.15_81c} \\
  &&S^{01} = K_1 =
  \begin{pmatrix}
    -i \sigma^1/2  &  0  \\ 
    0   &    i\sigma^1/2
  \end{pmatrix}\label{eq.15_81d} \\
  &&S^{02} = K_2 =
  \begin{pmatrix}
    -i\sigma^2/2  &  0  \\ 
    0   &    i\sigma^2/2
  \end{pmatrix}\label{eq.15_81e} \\
  &&S^{03} = K_3 =
  \begin{pmatrix}
    -i\sigma^3/2  &  0  \\ 
    0   &    i\sigma^3/2
  \end{pmatrix}\label{eq.15_81f}
\end{eqnarray}
  as we already know.

  Let us now re-introduce the charge conjugation in the 4-spinor notation.
  Please note that I introduced charge conjugation as a symmetry of
  free Dirac Lagrangian, which is expressed in the 4-spinor notation as
\begin{eqnarray}
  \Psi =
  \begin{pmatrix}
    \psi_L \\ \psi_R
  \end{pmatrix}
  \to \Psi^c = 
  \begin{pmatrix}
    \psi_R^c \\ \psi_L^c
  \end{pmatrix} \label{eq.15_82}
\end{eqnarray}
 
  This time, let us start with the Lagrangian with gauge bosons
\begin{eqnarray}
  {\cal L} &=& {\overline \Psi} (i\del_\mu \gamma^\mu - m) \Psi \\
        &=& {\overline \Psi} ([i\del_\mu - g T^a A^a_\mu] \gamma^\mu - m) \Psi \label{eq.15_83}
\end{eqnarray}

  Here again, I introduce the non-Abelian generator $T^a$, so that it is
  useful for QCD. For QED, $gT^a A^a_\mu$ should be replaced by $eQ A_\mu$.
  Because Eq.\ref{eq.15_83} is a singlet in the 4-spinor space, and because Fermi operators anti-commute,
\begin{eqnarray}
  {\cal L}&=&  {\cal L}^T
  =  ({\overline \Psi} ([i\del_\mu - g T^a A^a_\mu] \gamma^\mu - m) \Psi)^T \\
  &=& -(\Psi)^T (\gamma^\mu)^T ([i\del_\mu]^T -g(T^a)^T A^a_\mu) ({\overline \Psi})^T
    +m (\Psi)^T  ({\overline \Psi})^T \\
  &=& (\Psi)^T ([i\del_\mu +g(T^a)^TA^a_\mu](\gamma^\mu)^T +m) ({\overline \Psi})^T \label{eq.15_84}
\end{eqnarray}
  In the above, I inroduced a notation, $[\del_\mu]^T$ for the derivative
  operator which act on its left-hand-side. Hence
\begin{eqnarray}
  [\del_\mu]^T = -\del_\mu \label{eq.15_85}
\end{eqnarray}

{\bf hw15-9}: Derive Eq.\ref{eq.15_84} from Eq.\ref{eq.15_83}.

  Now we introduce a $4\times 4$ matrix C which gives
\begin{eqnarray}
  C (\gamma^\mu)^T C^{-1} = -\gamma^\mu \label{eq.15_86}
\end{eqnarray}
  and we define $\Psi^c$ as
\begin{eqnarray}
  \Psi^c = C ({\overline \Psi})^T \label{eq.15_87}
\end{eqnarray}
 then the Lagrangian in Eq.\ref{eq.15_84} is expressed as
\begin{eqnarray}
  {\cal L} &=& {\cal L}^T
  = (\Psi)^T C^{-1} C
  ([i\del_\mu +g(T^a)^TA^a_\mu](\gamma^\mu)^T +m) C^{-1} C ({\overline \Psi})^T \\
  &=& (\Psi)^T C^{-1} ([i\del_\mu +g(T^a)^TA^a_\mu](-\gamma^\mu) +m)\Psi^c, \label{eq.15_88}
\end{eqnarray}
  Now, if our $4 \times 4$ matrix C, which satisfies Eq.\ref{eq.15_86} and Eq.\ref{eq.15_87}, gives
\begin{eqnarray}
  (\Psi)^T C^{-1} = -\overline{\Psi}^c \label{eq.15_89}
\end{eqnarray}
  we can express Eq.\ref{eq.15_88} as
\begin{eqnarray}
  {\cal L} = {\cal L}^T
  = \overline{\Psi}^c ([i\del_\mu +g(T^a)^TA^a_\mu] \gamma^\mu -m)\Psi^c \label{eq.15_90}
\end{eqnarray}
  By comparing Eq.\ref{eq.15_83} and Eq.\ref{eq.15_90}, we find that ${\cal L}$ is invariant under the joint transformations (Charge conjugate transformations)
\begin{eqnarray}
  && \Psi  \to \Psi^c = C {\overline \Psi}^T \label{eq.15_91a}\\
  && g T^a A^a_\mu \to -g (T^a)^T A^a_\mu = -g (T^a)^* A^a_\mu\label{eq.15_91b}
\end{eqnarray}
  
{\bf hw15-10}: Show that Eq.\ref{eq.15_89} gives Eq.\ref{eq.15_90}.

  Now let us find the $4\times 4$ matrix C which satisfy Eq.\ref{eq.15_86}, Eq.\ref{eq.15_87}, Eq.\ref{eq.15_89}.
  We first note that Eq.\ref{eq.15_87}+Eq.\ref{eq.15_89} implies
\begin{eqnarray}
  {\overline \Psi}
  &=& (\Psi^c)^\dagger \gamma^0
  = (C {\overline \Psi}^T)^\dagger \gamma^0
  = (C (\Psi^\dagger \gamma^0)^T)^\dagger \gamma^0
  = (C (\gamma^0)^T (\Psi^\dagger)^T)^\dagger \gamma^0\nonumber \\
  &=& (C (\gamma^0)^T \Psi^*)^\dagger \gamma^0
  = (\Psi)^T (\gamma^0)^* C^\dagger \gamma^0
  = (\Psi)^T (-C^{-1}) \label{eq.15_92}
\end{eqnarray}
  where the last step is our requirement Eq.\ref{eq.15_89}. In the chiral
  representation, $(\gamma^0)^* = \gamma^0$, and Eq.\ref{eq.15_92} gives
\begin{eqnarray}
  \gamma^0 C^\dgr \gamma^0 = -C^{-1}\label{eq.15_93}
\end{eqnarray}
  Hence
\begin{eqnarray}
  && C \gamma^0 C^\dagger \gamma^0 = -1 \label{eq.15_94a}\\
  && C \gamma^0 C^\dagger = -\gamma^0 \label{eq.15_94b}
\end{eqnarray}
  By comparing Eq.\ref{eq.15_94b} with Eq.\ref{eq.15_86}, because $(\gamma^0)^T = \gamma^0$, we find
\begin{eqnarray}
  C^\dgr = C^{-1},~or~C C^\dgr = 1. \label{eq.15_95}
\end{eqnarray}

  Therefore, if we find $C$ that satisfies Eq.\ref{eq.15_86} and Eq.\ref{eq.15_95} in the chiral representation of $\gamma$ matrix, we have the $C$ conjugation transformation Eq.(\ref{eq.15_91a},\ref{eq.15_91b}).

  Because we already know the explicit form of Eq.(\ref{eq.15_91a},\ref{eq.15_91b}),
\begin{eqnarray}
  \Psi^c = C ( {\overline \Psi})^T
  = C (\psi_R^\dgr, \psi_L^\dgr)^T
  = C
  \begin{pmatrix}
  \psi_R^* \\ \psi_L^*
  \end{pmatrix}
  =
  \begin{pmatrix}
  i\sigma^2 \psi_R^* \\ -i\sigma^2 \psi_L^*
  \end{pmatrix}\label{eq.15_96}
\end{eqnarray}
  we find
\begin{eqnarray}
  C=
  \begin{pmatrix}
    i\sigma^2   &    0  \\
    0   & -i\sigma^2 
  \end{pmatrix}
  = i \gamma^2 \gamma^0 \label{eq.15_97}
\end{eqnarray}

{\bf hw15-11}: Show Eq.\ref{eq.15_97} satisfies  Eq.\ref{eq.15_86}  and  Eq.\ref{eq.15_95} .

  Although I'm not sure how important for you to be fluent in the
  language of 4-spinors, I gave the above exercises in order for
  you to be able to understand what are done in old textbooks.

  We will find that Dirac 4-spinors are very convenient in calculating
  transition amplitudes because it gives Dirac propagator which is a
  $4\times 4$ matrix that automatically conserves fermion numbers.

  Now, we are ready to introduce Poincare transformations which includes
  both the Lorentz transformations and translations of the 4 space-time
  coordinates.  Because the Lorentz transformations have 6 generators,
  and there are 4 generators for the translation of the 4 coordinates,
  Poincare transformations have 10 generators.  We can show that these
  10 generators form a closed algebra (Poincare algebra), and hence the
  Poincare transformations form a group (Poincare group). Invariance
  of physics under the Poincare transformations dictates the basis of
  our theory, Quantum Field Theory, which has been proven to
  describe all particle phenomena observed so far.

  First, please confirm that the coordinate representation of the four
  momentum operator
\begin{eqnarray}
  P^\mu = i \del^\mu \label{eq.15_1}
\end{eqnarray}
  gives
\begin{eqnarray}
  E   =   i\frac{d}{dt}, ~~~
  P_x = - i \frac{d}{dx},~~~
  P_y = - i \frac{d}{dy},~~~
  P_z = - i \frac{d}{dz} \label{eq.15_2}
\end{eqnarray}
  in a covariant manner.  Here,
\begin{eqnarray}
  &&P^\mu = (P^0, P^1, P^2, P^3) = (E, P_x, Py, P_z) \label{eq.15_3a} \\
  &&x^\mu = (x^0, x^1, x^2, x^3) = (t,   x,   y,   z) \label{eq.15_3b}
\end{eqnarray}
 
{\bf hw15-12}: Show that Eq.\ref{eq.15_1} gives Eq.\ref{eq.15_2}.

hint: can you obtain the second and the third steps in Eq.\ref{eq.15_4}?
\begin{eqnarray}
  P^\mu = i \del^\mu
           = i g^{\mu\nu} \del/(\del x^\nu)
           = i \del/(\del x_\mu) \label{eq.15_4}
\end{eqnarray}
  The fundamental commutation relation of quantum mechanics,
\begin{eqnarray}
  [ P^\mu, x^\nu ] = i g^{\mu \nu} \label{eq.15_5}
\end{eqnarray}
  can also be obtained by using the representation Eq.\ref{eq.15_1}:

{\bf hw15-13}: Derive Eq.\ref{eq.15_5} from Eq.\ref{eq.15_1} by using Eq.\ref{eq.15_6} and Eq.\ref{eq.15_7} as hints.
\begin{eqnarray}
  && \del_\mu x^\nu = \del x^\nu / \del x^\mu
  = \delta_\mu^\nu
  = g_\mu^\nu         \label{eq.15_6}         \\
&&\del^\mu x^\nu = g^{\mu\rho} \del_\rho x^\nu
  = g^{\mu\rho} \delta_\rho^\nu   
  = g^{\mu\nu} \label{eq.15_7}
\end{eqnarray}
  When I was a student, I often made a mistake of confusing the upper and
  lower indices of derivative operators. I hope that you confirm the
  rules clearly after doing the above two very simple homework.

  The 4 operators, $p^\mu$, are translation generators, since they shift
  the origin of the four coordinates:
\begin{eqnarray}
  x^\mu \to x'^\mu = e^{i c_a p^a} x^\mu
                    = x^\mu + i c_a  P^a  x^\mu
                    = x^\mu + i c_a [P^a, x^\mu]
                    = x^\mu + i c_a i g^{a\mu}
                    = x^\mu - c^\mu \label{eq.15_8}
\end{eqnarray}

{\bf hw15-14}: Show Eq.\ref{eq.15_8}, either by using the coordinate representation Eq.\ref{eq.15_1} or by using the commutation relation Eq.\ref{eq.15_5}.

  The Poincare group is a group of transformations, including the above
  4 translations and the 6 Lorentz transformations
\begin{eqnarray}
  x^\mu \to x'^\mu = {L^\mu}_\nu x^\nu
  = {(e^{\frac{i}{2} \omega_{ab} M^{ab}})^\mu}_\nu x^\nu
  = x^\mu +\frac{i}{2} (\omega_{ab} M^{ab})^\mu_\nu x^\nu + \cdots
  = x^\mu - {\omega^\mu}_\nu x^\nu + {\cal O}(\omega^2) \label{eq.15_9}
\end{eqnarray}
  The 6 generators of the Lorentz transformations, $M^{ab}$ in Eq.\ref{eq.15_9}, can be expressed as
  \begin{eqnarray}
    M^{ab} = x^a p^b - x^b p^a \label{eq.15_10}
  \end{eqnarray}

{\bf hw15-15}: Show that the operators Eq.\ref{eq.15_10} generates the Lorentz
  transformation Eq.\ref{eq.15_9}, by using the representation Eq.\ref{eq.15_1}, and also by using the commutation relations Eq.\ref{eq.15_5}.

  hint: Following steps will be usefu:
\begin{eqnarray}
  [M^{ab}, x^\mu] = [x^a P^b - x^b P^a, x^\mu]
                     = x^a [P^b, x^\mu] - x^b [P^a, x^\mu]
                     = x^a (i g^{b\mu}) - x^b (i g^{a\mu}) \label{eq.15_11a}
\end{eqnarray}
  Because we study the transformation of coordinates, $x^\mu$,
  the commutation relations Eq.\ref{eq.15_5} and the coordinate representation
  of $P^\mu$ Eq.\ref{eq.15_1} give the same results.

{\bf hw15-16}: Show that the Lorentz transformation Eq.\ref{eq.15_9} can be obtained by using the following ``vector'' representation of the generators;
\begin{eqnarray}
  (M^{ab})^\mu_\nu = i ( g^{a\mu} {g^b}_\nu - g^{b\mu} {g^a}_\nu )\label{eq.15_11b}
\end{eqnarray}

hint: The ``matrix'' representation Eq.\ref{eq.15_11b} is obtained from Eq.\ref{eq.15_11a} as
\begin{eqnarray}
  [M^{ab}, x^\mu] = i ( x^a g^{b\mu} - x^b g^{a\mu} )
                     = i x^\nu ( g^a_\nu g^{b\mu} - g^b_\nu g^{a\mu} )
                     = x^\nu {(M^{ab})_\nu}^\mu
                     = -{(M^{ab})^\mu}_\nu x^\nu \label{eq.15_11c}
\end{eqnarray}
  where $x^\mu$ is transformed into $x^\nu$ by a ``matrix'' $(M^{ab})_\nu^\mu$,
  because $x^\mu$ is annihilated by $M^{ab}$, while $x^\nu$ is created by $M^{ab}$. The relative minus sign between in the last line of Eq.\ref{eq.15_11c}, which fixes the sign of Eq.\ref{eq.15_11b} arises from this exchange of row and column.

  That this relative minus sign is necessary can be confirmed by noting
  that the commutation relations among the generators in the operator
  basis, $M^{ab} = x^a p^b - x^b p^a$, and those among their matrix
  representations should be the same.  (I couldn't find the origin of
  the minus sign between Eq.\ref{eq.15_11c}, straightforward calculation, and Eq.\ref{eq.15_11b},
  the correct matrix representations, for many years.)


{\bf hw15-17}: Confirm that Eq.\ref{eq.15_11b} gives exactly the matrix representations of $J_1$, $J_2$, $J_3$ and $K_1$, $K_2$, $K_3$, that we obtained in the past homework.

hint: Please remember that we defined $J$'s and $K$'s as
\begin{eqnarray}
  \frac{1}{2} M^{ab} \omega_{ab}
  &=& M^{12}\omega_{12} +M^{23}\omega_{23} +M^{31}\omega_{31} +M^{01}\omega_{01} +M^{02}\omega_{02} +M^{03}\omega_{03} \\
  &=& J_3 \theta_3 +J_1 \theta_1 +J_2 \theta_2 +K_1 \eta_1 +K_2 \eta_2 +K_3 \eta_3 \label{eq.15_12}
\end{eqnarray}
  in the matrix representation. Hence
\begin{eqnarray}
  {(\frac{i}{2} M^{ab} \omega_{ab})^\mu}_\nu x^\nu = - {\omega^\mu}_\nu x^\nu \label{eq.15_13}
\end{eqnarray}
  in Eq.\ref{eq.15_9} should first be expressed in the matrix form as
\begin{eqnarray}
  &&x'^\mu - x^\mu =
  \begin{pmatrix}
    0 &   -{\omega^0}_1 & -{\omega^0}_2 & -{\omega^0}_3 \\
    -{\omega^1}_0 & 0 & -{\omega^1}_2 & -{\omega^1}_3 \\
    -{\omega^2}_0 & -{\omega^2}_1 &  0 & -{\omega^2}_3 \\
    -{\omega^3}_0 & -{\omega^3}_1 &  -{\omega^3}_2 &  0  
  \end{pmatrix}
  \begin{pmatrix}
    x^0 \\ x^1 \\ x^2 \\ x^3
  \end{pmatrix} \\
  &=&
  \begin{pmatrix}
    0 &  -\omega_{01} & -\omega_{02} & -\omega_{03} \\
    -\omega_{01} & 0 & \omega_{12} & -\omega_{31} \\
    -\omega_{02} & -\omega_{12} &  0 & \omega_{23} \\
    -\omega_{03} & \omega_{31} &  -\omega_{23} &  0  
  \end{pmatrix}
  \begin{pmatrix}
    x^0 \\ x^1 \\ x^2 \\ x^3
  \end{pmatrix}
  =
  \begin{pmatrix}
    0 &  -\eta_{1} & -\eta_{2} & -\eta_{3} \\
    -\eta_{1} & 0 & \theta_{3} & -\theta_{2} \\
    -\eta_{2} & -\theta_{3} &  0 & \theta_{1} \\
    -\eta_{3} & \theta_{2} &  -\theta_{1} &  0  
  \end{pmatrix}
  \begin{pmatrix}
    x^0 \\ x^1 \\ x^2 \\ x^3
  \end{pmatrix} \label{eq.15_14}
\end{eqnarray}
  Now, the $4\times 4$ matrix above should agree with $i$ times the $4\times 4$ matrix representation of Eq.\ref{eq.15_12}. Since this identification is trivial, your homework is to check Eq.\ref{eq.15_14}.

  The expressions like Eq.\ref{eq.15_8} and Eq.\ref{eq.15_9} where we distinguish upper and lower indices are called tensor representations, which are the same as our matrix representations when applied to a multiplet with single index (such as $x^\mu$, $p^\mu$, $A^\mu$, $\psi_L$, $\psi_R$).  The contraction of upper and lower indices in tensor representations is replaced by the matrix multiplication rule.

  Tensor representation is more powerful, since it can give
  transformations of states or multiplets which have more than one
  indices (such as a tensor meson, a spin 3/2 fermion, or mesons and
  baryon states in QCD).  It has another advantage that we don't need
  to worry about the ordering between operators and the states.  In case
  of matrix representations, we should always place a matrix eperator
  in the left-hand-side of the state column vector.

  Now, it is easy to show that the $4$ generators $P^a$ of the translations
  and the $6$ generators $M^{ab}$ of the Lorentz transformations form the
  Poincare algebra:
\begin{eqnarray}
  &&[ P^a, P^b ]   = 0 \label{eq.15_15a}\\
  &&[ M^{ab}, P^c ]  = i( g^{bc} P^a - g^{ac} P^b  )\label{eq.15_15b}\\
  &&[ M^{ab}, M^{cd} ] = i( g^{ad} M^{bc} +g^{bc} M^{ad} -g^{ac} M^{bd} -g^{bd} M^{ac} )\label{eq.15_15c}
\end{eqnarray}
 
{\bf hw15-18}: By using the representation Eq.\ref{eq.15_1}, or the commutation relations Eq.\ref{eq.15_5}, show Eq.(\ref{eq.15_15a},\ref{eq.15_15b},\ref{eq.15_15c}).

  Since the 10 generators of Translation and Lorentz transformations form
  a closed algebra, Poincare algebra Eq.(\ref{eq.15_15a},\ref{eq.15_15b},\ref{eq.15_15c}), an arbitrary combination of these ten transformations form a group, Poincare group, which dictates the space-time properties of all the fields and their interactions in QFT.  All fundamental fields and particles in QFT are, hence, irreducible representations of the Poincare group.  Those transforming as reducible representations are usually regarded as composite states.

  Let us now confirm that there are two Casimir operators for the
  Poincare group.  Casimir operator is an operator which are made from the
  generators of a group that commutes with all of them.  They play
  essential roles in identifying fields and particles, since the eigen
  values of the Casimir operators are invariant under all Poincare
  transformations.

  There are two Casimir operators for the Poincare group.
\begin{eqnarray}
  P^2 \equiv P^\mu P_\mu \label{eq.15_16}
\end{eqnarray}
  Square-root of its eigenvalue $m^2 (\leq0)$, is called the invariant mass.

{\bf hw15-19}: Prove Eqs.{\ref{eq.15_17a},\ref{eq.15_17b}) by using Poincare algebra Eq.(\ref{eq.15_15a},\ref{eq.15_15b},\ref{eq.15_15c}).
\begin{eqnarray}
  &&[ P^a,  P^\mu P_\mu ] = 0 \label{eq.15_17a} \\
  && [ M^{ab}, P^\mu P_\mu ] = 0 \label{eq.15_17b} 
\end{eqnarray}
  The second Casimir is a bit more difficult to find.
  First, the Pauli-Lubanski vector is defined as
\begin{eqnarray}
  W_\mu = \frac{1}{2} \epsilon_{\mu\nu\rho\sigma} P^\nu M^{\rho\sigma} \label{eq.15_18}
\end{eqnarray}

{\bf hw15-20}: Show that $W_\mu$ Eq.\ref{eq.15_18} satisfies Eq.(\ref{eq.15_19a},\ref{eq.15_19b}).
\begin{eqnarray}
  && [ P^a,  W_\mu ] = 0 \label{eq.15_19a} \\
  && [ M^{ab}, W_\mu ] = -i(g^a_\mu W^b - g^b_\mu W^a ) \label{eq.15_19b}
\end{eqnarray}
hint: Eq.\ref{eq.15_19a} is easy, but Eq.\ref{eq.15_19b} is tough.

Eqs.(\ref{eq.15_19a},\ref{eq.15_19b}) tell that Pauli-Lubanski ``vector'' transform as a four-vector under the Lorentz transformation.
Once Eqs.(\ref{eq.15_19a},\ref{eq.15_19b}) are shown, it is straightforward to show
\begin{eqnarray}
  [ P^a,  W^2 ] = 0, ~~~ [ M^{ab}, W^2 ] = 0 \label{eq.15_20}
\end{eqnarray}
 
{\bf hw15-21}: Show Eq.\ref{eq.15_20} from Eq.(\ref{eq.15_19a},\ref{eq.15_19b}).

  The proof of Eqs.(\ref{eq.15_23a},\ref{eq.15_23b}) from Eqs.(\ref{eq.15_22a},\ref{eq.15_22b}) can be generalized to the Poincare
  invariance of an arbitrary inner product of two four-vectors.

{\bf hw15-22}: Let us do this exercise.  Let there be two four-vectors,
  $A^\mu$ and $B^\mu$, which satisfy:
\begin{eqnarray}
  && [ P^a,  A^\mu ] = 0 \label{eq.15_21a} \\
  && [ M^{ab}, A^\mu ] = -i (g^{a\mu} A^b - g^{b\mu} A^a) \label{eq.15_21b}
\end{eqnarray}
\begin{eqnarray}
  && [ P^a,  B^\mu ] = 0 \label{eq.15_22a} \\
  && [ M^{ab}, B^\mu ] = -i(g^{a\mu} B^b - g^{b\mu} B^a)  \label{eq.15_22b}
\end{eqnarray}
  By using the above ``definitions'' of four vectors, prove
  the invariance of the dot-product $A\cdot B = A_\mu B^\mu$:
\begin{eqnarray}
  && [ P^a,  A_\mu B^\mu ] = 0 \label{eq.15_23a} \\
  &&[ M^{ab}, A_\mu B^\mu ] = 0 \label{eq.15_23b} 
\end{eqnarray}
  Let us examine the physical meaning of $W^2$.
  It is easy for a massive particle, an eigenstate $\ket{p}$ with
\begin{eqnarray}
   P^2 \ket{p} = m^2 \ket{p},~with~m^2 > 0 \label{eq.15_24}
\end{eqnarray}
  In this case, we can study $W^\mu$ in the rest frame,
\begin{eqnarray}
  p^\mu = ( m, 0, 0, 0 ), \label{eq.15_25}
\end{eqnarray}
  where
\begin{eqnarray}
  && W_\mu = m ( 0, -J_x, -J_y, -J_z ) \label{eq.15_25a} \\
  && W^\mu = m ( 0,  J_x,  J_y,  J_z ) \label{eq.15_25b} \\
  && W^2   = - m^2 ( J_x^2 + J_y^2 + J_z^2 ) \label{eq.15_25c}
\end{eqnarray}

{\bf hw15-23}: Show Eq.(\ref{eq.15_25a},\ref{eq.15_25b},\ref{eq.15_25c}).

  Please note that in the above derivation, I used my convention
\begin{eqnarray}
  \epsilon_{0123} = 1 = -\epsilon^{0123} \label{eq.15_26}
\end{eqnarray}
  which differs by sign e.g. from Peskin's textbook.
  Since $W^2$ commutes with all $10$ generators of Poicare group Eq.\ref{eq.15_20}, its eigen value in the rest frame is a Poincare invariant:
\begin{eqnarray}
  W^2 \ket{p,s} = W^2 \ket{m,s} = - m^2 s(s+1) \ket{m,s} \label{eq.15_27}
\end{eqnarray}
  This is the reason why we can use the mass and spin to label a particle.

  Let us now identify the conserved quantity for massless particles,
\begin{eqnarray}
  P^2 \ket{p} = p^2 \ket{p} = 0. \label{eq.15_28}
\end{eqnarray}

  In this case, there is no rest frame of a particle. Instead, let us
  choose the Lorentz frame where the 3-momentum is along the positive
  $z$-axis:
\begin{eqnarray}
  p^\mu = ( E, 0, 0, E ) \label{eq.15_29}
\end{eqnarray}

{\bf hw15-24}: In this frame, please show Eq.\ref{eq.15_30} and  Eq.\ref{eq.15_31}.
\begin{eqnarray}
  W_0 =  E (   J_z       ),~~~W_1 =  E ( - J_x - K_y ),~~~W_2 =  E ( - J_y + K_x ),~~~W_3 =  E ( - J_z  ) \label{eq.15_30}
\end{eqnarray}
 \begin{eqnarray}
  W^2 &=&  E^2 [ (J_z)^2 - (J_x+K_y)^2 - (J_y-K_x)^2 -(-J_z)^2 ] \nonumber \\
  &=& -E^2 [ (J_x+K_y)^2 + (J_y-K_x)^2 ]\label{eq.15_31}
 \end{eqnarray}
  
  The next step is rather difficult.  Please let me follow Steve Weinberg
  [see discussions in pages 69--72 of Weinberg's Quantum Field Theory I]. We define two operators that appear in Eq.\ref{eq.15_30} as
\begin{eqnarray}
  &&A = J_x + K_y \label{eq.15_32a}\\
  &&B = J_y - K_x \label{eq.15_32b}
\end{eqnarray}

{\bf hw15-25}: Show Eq.\ref{eq.15_33}.
\begin{eqnarray}
  [ J_z, A ] =  i B,~~~[ J_z, B ] = -i A,~~~[   A, B ] = 0 \label{eq.15_33}
\end{eqnarray}
  Since $A$ and $B$ commute, a massless particle state with the four-momentum
  Eq.\ref{eq.15_29} can have simultaneous eigen values $(a,b)$
\begin{eqnarray}
  && A \ket{p;a,b} = a \ket{p;a,b} \label{eq.15_34a} \\
  && B \ket{p;a,b} = b \ket{p;a,b} \label{eq.15_34b}
\end{eqnarray}
  It then follows that the states obtained by rotating the above state
  $\ket{p;a,b}$ about the z-axis by an arbitrary azimuthal angle $\phi$,
\begin{eqnarray}
  \ket{p;a,b;\phi} = e^{ -iJ_z \phi } \ket{p;a,b} \label{eq.15_35}
\end{eqnarray}
  is also a simultaneous eigenstates of $A$ and $B$ with eigenvalues
\begin{eqnarray}
  && A \ket{p;a,b;\phi} = (a \cos\phi - b \sin\phi) \ket{p;a,b;\phi} \label{eq.15_36a}\\
  && B \ket{p;a,b;\phi} = (a \sin\phi + b \sin\phi) \ket{p;a,b;\phi} \label{eq.15_36b}
\end{eqnarray}
 
{\bf hw15-26}: Show Eq.(\ref{eq.15_36a},\ref{eq.15_36b}) by using Eq.\ref{eq.15_33}.

hint: Starting from Eq.\ref{eq.15_33}, we obtain $[(J_z)^2, A]$ and $[(J_z)^2,B]$. The rest should be straightforward by using the power expansion of the
  rotation, $e^{-iJ_z \phi} = \sum_{n=0}^\infty (J_z)^n (-i\phi)^n$.

  That the state Eq.\ref{eq.15_35} has simultaneous eigen values of Eq.(\ref{eq.15_36a},\ref{eq.15_36b}) implies that for all the massless particles there is a continuous degree of freedom, $\phi$. Now, please let me borrow what Weinberg says:
  ``Massless particles are not observed to have any continuous degree of
  freedom like $\phi$; to avoid such a continuum of states, we must require
  that physical states are eigen-vectors of $A$ and $B$ with $a=b=0$.''
\begin{eqnarray}
  A \ket{p;m=0} = B \ket{p;m=0} = 0 \label{eq.15_37}
\end{eqnarray}
  For those physical massless states, the Pauli-Lubanski vector
  Eq.\ref{eq.15_30} gives eigenvalues
\begin{eqnarray}
  &&W^0 \ket{p,\lambda} = E J_z \ket{p,\lambda} = E \lambda \ket{p,\lambda} \label{eq.15_38a} \\
  &&W^1 \ket{p,\lambda} = E A   \ket{p,\lambda} = 0\label{eq.15_38b} \\
  &&W^2  \ket{p,\lambda} = E B   \ket{p,\lambda} = 0 \label{eq.15_38c} \\
  &&W^3 \ket{p,\lambda} = E J_z \ket{p,\lambda} = E \lambda \ket{p,\lambda} \label{eq.15_38d} 
\end{eqnarray}
  or simply,
\begin{eqnarray}
  W^\mu \ket{p,\lambda} = \lambda p^\mu \ket{p,\lambda}. \label{eq.15_39}
\end{eqnarray}
  Here $\lambda$ is the eigen value of $J_z$ in the frame where the massless
  particle is moving along the $z$-axis, Eq.\ref{eq.15_29}. In general Lorentz
  frame $J_z$ can be expressed as the helicity operator
\begin{eqnarray}
  h=\frac{{\vec p} \cdot {\vec J}}{|{\vec p}|} \label{eq.15_40}
\end{eqnarray}
  where ${\vec p}$ and $J$ are $3$-vectors. The
  eigenvalue of the operator Eq.\ref{eq.15_40}, $\lambda$, is called helicity.

  Although $W^\mu$ is not invariant, because both $W^\mu$ and $p^\mu$ transform
  identically as a four-vector (we proved this, in Eq.\ref{eq.15_18} and Eq.(\ref{eq.15_22a},\ref{eq.15_22b}), the proportionality constant $\lambda$ in eq.(39) is always the same in an arbitrary Lorentz frame. It is called the helicity conservation.

  It is only for the massless states the proportionality relationship
  Eq.\ref{eq.15_39} holds. Therefore, the conservation of helicity is the
  property only of massless particles.

  Please convince yourself that if there is a mass, no matter how tiny it
  may be, the helicity cannot be conserved under Lorentz transformation.
  The easiest way to convince ourselves is to do the Lorentz boost along
  the momentum direction of a massive particle. Once the boost velocity
  is greater than the original velocity of a particle, the direction of
  the three momentum is reversed, and the helicity changes sign.

  When nonzero neutrino mass was discovered through neutrino oscillation
  phenomena in 1998, we therefore concluded that there are not only
  left-handed (helicity $-1/2$) neutrinos but also are right-handed
  (helicity $+1/2$) neutrinos.  This is definite.

  It is yet to be determined (experimentally) whether the $h = +1/2$ states
  are new particles which did not exist in the SM, or if they are
  anti-particles of the left-handed neutrinos of the SM. In the first
  case, we need new particles (the right-handed neutrinos), and in the
  latter case we need new interactions which mixes particles (left-
  handed neutrinos) and anti-particles (right-handed anti-neutrinos).
  In either way we need to have new physics beyond the SM.

  Because of the above observation, the discovery of neutrino oscillation
  has been declared as a discovery of physics beyond the SM.  Nobel
  prize 2015 was awarded to those physicists who led this discovery.

That's all for hw15.\\

Best regards,\\

Kaoru


\end{document}

\documentclass[12pt]{article}
\usepackage{amsmath,graphicx,color,epsfig,physics}
\usepackage{float}
\usepackage{subfigure}
\usepackage{slashed}
\usepackage{color}
\usepackage{multirow}
\usepackage{feynmp}
\textheight=9.5in \voffset=-1.0in \textwidth=6.5in \hoffset=-0.5in
\parskip=0pt
\def\del{{\partial}}


\begin{document}

\begin{center}
{\large\bf HW14 for Advanced Particle Physics} \\

\end{center}

\vskip 0.2 in

Dear students,

  This week, I introduce the Dirac fermion and its equation of motion
  (the Dirac equation) as a set of two Weyl fermions, $\psi_L$ and $\psi_R$,
  which couple with the Lorentz invariant mass term.  We proved the
  Lorentz invariance of the Dirac Lagrangian, and the Lorentz covariance
  of Dirac equation.  We showed how a set of two Weyl fermions and their
  e.o.m.'s can be combined into one four-spinor, called Dirac spinor,
  and its equation, called Dirac equation.

  We are now ready to learn QFT (Quantum Field Theory), in which all
  the fields that appear in the Lagrangian should be regarded as
  operators which creat and annihilate particles and anti-particles.
  QFT is the only known theory which satisfies both the requirement
  of quantum mechanics (that ensures the stability of atoms) and
  Lorentz invariance of physics. You should master QFT since it is
  the language with which particle physicists communicate with nature.


  Let us start again from the Dirac equations:
\begin{eqnarray}
 && i\del_\mu \sigma_-^\mu \psi_L - m \psi_R = 0 \label{eq.14_1a} \\
 && i\del_\mu \sigma_+^\mu \psi_R - m \psi_L = 0 \label{eq.14_1b} 
\end{eqnarray}
  We multiply Eq.\ref{eq.14_1a} by $-i\del_\nu \sigma_+^\nu$ and find
\begin{eqnarray}
  && -i\del_\nu \sigma_+^\nu
  (i\del_\mu \sigma_-^\mu \psi_L - m \psi_R) = 0 \label{eq.14_2a} \\
  && \del_\nu \del_\mu \sigma_+^\nu \sigma_-^\mu \psi_L
  +i\del_\nu \sigma_+^\nu m \psi_R = 0 \label{eq.14_3a}
\end{eqnarray}
  By using Eq.\ref{eq.14_1b}, the above equation becomes
\begin{eqnarray}
  [\del_\nu \del_\mu \sigma_+^\nu \sigma_-^\mu + m^2] \psi_L = 0. \label{eq.14_4a}
\end{eqnarray}
  Now, by using the symmtric summation of $\nu$ and $\mu$ indices and the
  properties of $\sigma$ matrices,
\begin{eqnarray}
  \del_\nu \del_\mu \sigma_+^\mu \sigma_-^\nu
  = \frac{1}{2} \del_\nu \del_\mu
    (\sigma_+^\mu \sigma_-^\nu + \sigma_+^\nu \sigma_-^\mu)
  = \frac{1}{2} \del_\nu \del_\mu { 2 g^{\mu\nu} }
  = \del_\nu \del_\mu g^{\mu\nu}
  = \del_\mu \del^\mu \label{eq.14_5a}
\end{eqnarray}
  we find
\begin{eqnarray}
  [ \del_\mu \del^\mu + m^2 ] \psi_L = 0. \label{eq.14_6a}
\end{eqnarray}

{\bf hw14-1}: Please show Eq.(\ref{eq.14_3a}, \ref{eq.14_4a}, \ref{eq.14_5a}, \ref{eq.14_6a}).

{\bf hw14-2}: Please multiply Eq.\ref{eq.14_1b} by $-i \del_\nu \sigma_-^\nu$, and obtain the Klein-Gordon equation (Eq.\ref{eq.14_6a}) for $\psi_R$.  
\begin{eqnarray}
  && [ \del_\mu \del^\mu + m^2 ] \psi_L = 0,\label{eq.14_6a2} \\
  && [ \del_\mu \del^\mu + m^2 ] \psi_R = 0. \label{eq.14_6b}
\end{eqnarray}
 
  Since both $\psi_L(x)$ and $\psi_R(x)$ satisfy the same Klein-Gordon
  equations, both of them have two solutions
\begin{eqnarray}
  && e^{-i px} = e^{-i (Et -xp_x -yp_y -zp_z)} = e^{-i(Et -{\vec p}\cdot {\vec x})} \label{eq.14_7a} \\
  && e^{+i px} = e^{+i (Et -xp_x -yp_y -zp_z)} = e^{+i(Et -{\vec p}\cdot {\vec x})} \label{eq.14_7b}
\end{eqnarray}
  where ${\vec p} = (p_x,p_y,p_z)^T$ and ${\vec x} = (x,y,z)^T$, and
\begin{eqnarray}
  E = \sqrt{ p_x^2 + p_y^2 + p_z^2 + m^2 }
  = \sqrt{{\vec p}  \cdot {\vec p} + m^2 } \label{eq.14_8}
\end{eqnarray}

{\bf hw14-3}: Show that both $e^{-ipx}$ in Eq.\ref{eq.14_7a} and $e^{ipx}$ in Eq.\ref{eq.14_7b} are solutions of
  an arbitrary Klein Gordon equations:
\begin{eqnarray}
  [ \del_\mu \del^\mu + m^2 ] \phi(x) = 0. \label{eq.14_9}
\end{eqnarray}
hint: By noting
\begin{eqnarray}
  \del_\mu \del^\mu
  = \left (\frac{\del}{\del t}\right)^2 -\left (\frac{\del}{\del x}\right)^2-\left (\frac{\del}{\del y}\right)^2-\left (\frac{\del}{\del z}\right)^2 \label{eq.14_10}
\end{eqnarray}
  apply each differntial operators to $e^{\pm ipx}$, and then sum all
  the 4 contributions.
  Therefore, we can expand (Fourier transform) an arbitrary solutions
  of the Klein-Gordon equation (Eq.\ref{eq.14_9}) as
\begin{eqnarray}
  \phi(x) = \int d^3p [A(p) e^{-ipx} + B(p) e^{ipx}] \label{eq.14_11}
\end{eqnarray}
  where $A(p)$ and $B(p)$ may be called ``Fourier transform'' of $\phi(x)$ in
  the momentum space.

  Now the so-called ``negative energy solution problem'' is that the
  solution $e^{ipx}$ has negative energy.

{\bf hw14-4}: Please obtain the energy of the wave function $e^{-ipx}$ and
  $e^{ipx}$ by using the Shroedinger equations:
\begin{eqnarray}
  i\frac{d}{dt} \psi(x) = H \psi(x) \label{eq.14_12}
\end{eqnarray}
  when $p^0 = E$ in Eq.\ref{eq.14_8}. Please confirm that $H=E$ for $e^{-ipx}$, while $H=-E$ for $e^{ipx}$.

  [Historically, Dirac tried to solve this negative energy problem of
  relativistic quantum mechanics by thinking the following way:
  The Klein-Gordon equation has negative energy problem because it is
  a double differential equations, as written explicitly in Eq.\ref{eq.14_9}.
  If we obtain a covariant equation of motion which is linear in
  the time derivative, this problem may be solved.  With this motivation,
  he found a 4-spinor wave function, whose covariant equation of motion
  is linear in the time derivative, Eq.(\ref{eq.14_1a},\ref{eq.14_1b}).  Unfortunately, as we showed above, his wave functions automatically satisfy the Klein-Gordon equation, and hence he could not solve the negative-energy problem.

  This problem is solved by quantum field theory (QFT), in which all
  the wave functions are re-interpreted as a set of creation and
  annihilation operators.  Dirac did not give up his original motivation,
  and found his famous `hole theory', which works only for fermions
  since it makes use of Fermi statistics to define the `vacuum' filled
  by the sea of electrons.  His hole theory turns out to be extremely
  useful and powerful in material sciences where the `sea' of electrons
  gives a good approximation for the ground state.]

  In QFT, the Fourier transform of $\phi(x)$ in Eq.\ref{eq.14_11} is expressed as
\begin{eqnarray}
  \phi(x) = \int \frac{d^3p}{2E(2\pi)^3} (a(p) e^{-ipx} + b(p)^\dagger e^{ipx}) \label{eq.14_13}
\end{eqnarray}
  Please note that the only essential change made is that the ``Fourier
  transform'' $A(p)$ and $B(p)$ are now regarded as $a(p)$, an operator which
  annihilates a particle with 3-momentum $p=\vec p$, and as $b(p)^\dagger$, an
  operator which creats its anti-particle with 3-momentum $p=\vec p$.

  The creation and annihilation operators are defined by the following
  sets of commutators (recall the Harmonic oscillators in QM):
\begin{eqnarray}
  &&[ a({\vec p}), a^\dagger({\vec k}) ] = (2\pi)^3 2E \delta^3({\vec p}-{\vec k}) \label{eq.14_14a} \\
  &&[ b({\vec p}), b^\dagger({\vec k}) ] = (2\pi)^3 2E \delta^3({\vec p}-{\vec k}) \label{eq.14_14b}  \\ 
  &&  [ a({\vec p}), b^\dagger({\vec k}) ]= [a({\vec p}), a({\vec k}) ]=[b({\vec p}), b({\vec k}) ]=0 \label{eq.14_14c}
\end{eqnarray}

  With this re-interpretation, the one-particle state with momentum
  ${\vec k}$ can be obtained from the vacuum by applying
  \begin{eqnarray}
   \ket{particle,~{\vec k}}= a^\dagger({\vec k}) \ket{0} \label{eq.14_15}
  \end{eqnarray}
  and its wave function can be expressed as
\begin{eqnarray}
  \bra{0} \phi(x) \ket{particle,~{\vec k}}=  e^{-ikx} \label{eq.14_16}
\end{eqnarray}
  where $k^0 = E = \sqrt{ m^2 + {\vec k}^2 }$, by using the commutators

{\bf hw14-5}: Show Eq.\ref{eq.14_16}.

{\bf hw14-6}: Please note the Lorentz invariance of the Fourier transform
  measure in Eq.\ref{eq.14_13} and in the normalization of the commutators in Eq.(\ref{eq.14_14a},\ref{eq.14_14b},\ref{eq.14_14c}),
  by following the steps Eq.\ref{eq.14_17} to Eq.(\ref{eq.14_18a},\ref{eq.14_18b},\ref{eq.14_18c}) below:
\begin{eqnarray}
  d^4 p \delta( p^2 - m^2 ) \Theta(p^0)
  &=&dp^0dp^1dp^2dp^3 \delta((p^0)^2-(p^1)^2-(p^2)^2-(p^3)^2-m^2) \Theta(p^0) \nonumber \\
  &=&dp^0dp^1dp^2dp^3 \delta( (p^0)^2 -E^2 ) \Theta(p^0) \nonumber\\
  &=&dp^0dp^1dp^2dp^3 \delta( (p^0-E) (p^0+E) ) \Theta(p^0)\nonumber\\
  &=&dp^0dp^1dp^2dp^3 \delta( (p^0-E) (2E) ) \Theta(p^0) \nonumber\\
  &=&    dp^1dp^2dp^3/2E
  =    d^3 p/2E \label{eq.14_17}
\end{eqnarray}
  where we used the definition Eq.\ref{eq.14_8} of $E >0$. The Lorentz invariance of the first line of Eq.\ref{eq.14_17} can be shown by noting
\begin{eqnarray}
  &&d^4 p \to d^4 p' = det(L) d^4 p = d^4 p,~where~p'=Lp \label{eq.14_18a}\\
  &&p^2 - m^2 \to p'^2 - m^2 = p^2 - m^2 \label{eq.14_18b} \\
  &&{\text sign~ of~ p^0 ~can~ never~ change~ under~} p \to p'=Lp\label{eq.14_18c}
\end{eqnarray}
  Are the above conditions clear to you?
  As for the one-antiparticle state with three-momentum $\vec{k}$,
  \begin{eqnarray}
    \ket{anti-particle,~{\vec k}}=b ^\dagger({\vec k}) \ket{0},\label{eq.14_19}
  \end{eqnarray}
  its wave function is obtained by
\begin{eqnarray}
  \bra{0}  \phi(x)^*  \ket{anti-particle,~{\vec k}}= e^{-ikx} \label{eq.14_20}
\end{eqnarray}
  by using the same commutators Eq.(\ref{eq.14_14a},\ref{eq.14_14b},\ref{eq.14_14c}).

{\bf hw14-7}: Show Eq.\ref{eq.14_20} by using the commutators Eq.(\ref{eq.14_14a},\ref{eq.14_14b},\ref{eq.14_14c}).

  The negative-energy problem is thus solved in QFT.  From this very
  foundation of QFT, we can tell that the particle and the anti-particle
  should have exactly the same invariant mass.

  In case of a real scalar boson (like the Higgs boson), the particle
  and the anti-particle are the same, and we express
\begin{eqnarray}
  H(x) = \int \frac{d^3p}{2E(2\pi)^3} [a(p) e^{-ipx} + a^\dagger(p) e^{ipx}]\label{eq.14_21}
\end{eqnarray}
  by using the operators $a(p)$ and $a^\dagger(p)$ only, such that
\begin{eqnarray}
  H^\dagger(x) = H(x). \label{eq.14_22}
\end{eqnarray}
  Please do note here that even though the Higgs field operator is
  Hermetian (Eq.\ref{eq.14_22}), and we call Higgs field as a real field, its one particle wave function, $e^{-ipx}$, is complex, since it is a solution
  of the Shreodinger (Klein-Gordon) equation. We cannot change the
  phase of the Higgs field
\begin{eqnarray}
  H(x) \to H(x)' = e^{iQ\theta} H(x) \label{eq.14_23}
\end{eqnarray}
  for any $Q \neq 0$, because of its Hermeticity (Eq.\ref{eq.14_22}).
  In case of Dirac fermion, the free field expansion is written as
\begin{eqnarray}
  &&\psi_L(x)
  = \sum_{h=\pm1/2} \int \frac{d^3p}{2E(2\pi)^3}
  {a({\vec p},h) e^{-ipx} u_L({\vec p},h) +b^\dagger({\vec p},h) e^{ipx} v_L({\vec p},h)} \label{eq.14_24a}\\ 
  &&\psi_R(x)
  = \sum_{h=\pm1/2} \int \frac{d^3p}{2E(2\pi)^3}
  {a({\vec p},h) e^{-ipx} u_R({\vec p},h) +b^\dagger({\vec p},h) e^{ipx} v_R({\vec p},h)}\label{eq.14_24b}
\end{eqnarray}
  where the annihilation and creation operators satisfy the
  anti-commutation relations,
\begin{eqnarray}
  { a({\vec p},h), a({\vec k},h')^\dagger }
    = { b({\vec p},h), b({\vec k},h')^\dagger }
    = \delta_{h,h'} (2\pi)^3 2E \delta^3( {\vec p} - {\vec k} ) \label{eq.14_25}
\end{eqnarray}
  and all the other combinations anti-commute. With this normalization,
  the one-particle state wave functions are
\begin{eqnarray}
  &&\bra{0} \psi_L(x) a({\vec k},h)^\dagger \ket{0}=e^{-ikx} u_L(k,h)\label{eq.14_26a} \\
  &&\bra{0} \psi_R(x) a({\vec k},h)^\dagger \ket{0}=e^{-ikx} u_R(k,h)\label{eq.14_26b}
\end{eqnarray}
  for a particle, and
  \begin{eqnarray}
   && \bra{0} \psi_L(x)^\dagger b({\vec k},h)^\dagger \ket{0}=e^{-ikx} v_L(k,h)\label{eq.14_27a} \\
    &&\bra{0} \psi_R(x)^\dagger b({\vec k},h)^\dagger \ket{0}=e^{-ikx} v_R(k,h)\label{eq.14_27b}
  \end{eqnarray}
  for an anti-particle. The extra wave functions, $u_L$, $u_R$, $v_L$, $v_R$,
  describe the Lorentz transformation property of the spin $1/2$ fields.
  By using the Dirac equation, Eq.(\ref{eq.14_1a},\ref{eq.14_1b}), we find their e.o.m.'s:
\begin{eqnarray}
  &&k_\mu \sigma_-^\mu u_L(k,h) = m u_R(k,h) \label{eq.14_28a} \\
  && k_\mu \sigma_+^\mu u_R(k,h) = m u_L(k,h) \label{eq.14_28b}
\end{eqnarray}
  for $u$-spinors, and
\begin{eqnarray}
  && -k_\mu v_L(k,h)^\dagger \sigma_-^\mu = m v_R(k,h)^\dagger \label{eq.14_29a} \\
  && -k_\mu v_R(k,h)^\dagger \sigma_-^\mu = m v_L(k,h)^\dagger \label{eq.14_29b}
\end{eqnarray}
  for v-spinors.  Eq.(\ref{eq.14_29a},\ref{eq.14_29b}) can also be written as
\begin{eqnarray}
  &&-k_\mu \sigma_-^\mu v_L(k,h) = m v_R(k,h) \label{eq.14_29c}\\
  && -k_\mu \sigma_+^\mu v_R(k,h) = m v_L(k,h) \label{eq.14_29d}
\end{eqnarray}

{\bf hw14-8}: Derive Eq.(\ref{eq.14_28a},\ref{eq.14_28b}) and Eq.(\ref{eq.14_29a},\ref{eq.14_29b}) by using Dirac eqation Eq.(\ref{eq.14_1a},\ref{eq.14_1b}) for the one particle/anti-particle state in Eq.(\ref{eq.14_26a},\ref{eq.14_26b}) and Eq.(\ref{eq.14_27a},\ref{eq.14_27b}).

  The particle and anti-particle states in Eq.(\ref{eq.14_26a},\ref{eq.14_26b}) and Eq.(\ref{eq.14_27a},\ref{eq.14_27b}), respectively,
  and their e.o.m.'s Eq.(\ref{eq.14_28a},\ref{eq.14_28b}) vs Eq.(\ref{eq.14_29a},\ref{eq.14_29b}), look asymmetric in this formulation.
  In fact, we can show that Dirac equations (and hence QED and QCD)
  are invariant under the exchange of particle and anti-partilce
  (Charge conjugation).

  Now, let us study the last remaining important subject about fermion;
  the charge conjugation transformations.
\begin{eqnarray}
  && \psi_L(x) \to \psi_L(x)^c = (-i\sigma^2) \psi_L(x)^* \label{eq.14_30a}\\
  && \psi_R(x) \to \psi_R(x)^c = (+i\sigma^2) \psi_R(x)^* \label{eq.14_30b}
\end{eqnarray}
  
{\bf hw14-9}: Please confirm that with the above definitions, where the sign
  of $i\sigma^2$ term is opposite between $\psi _L^c$ and $\psi _R^c$, the
  double chanrge conjugations give the original fields:
\begin{eqnarray}
  && (\psi_L(x)^c)^c = \psi_L(x) \label{eq.14_31a}\\
  && (\psi_R(x)^c)^c = \psi_R(x)  \label{eq.14_31b}
\end{eqnarray}
  By applying the above transformations Eq.(\ref{eq.14_30a},\ref{eq.14_30b}) to our free field expansions
  Eq.(\ref{eq.14_24a},\ref{eq.14_24b}), and by requiring that
\begin{eqnarray}
  && \psi_L(x)^c = \psi_R(x) \label{eq.14_32a}\\
  && \psi_R(x)^c = \psi_L(x) \label{eq.14_32b}
\end{eqnarray}
  we obtain v-spinors from the $u$-spinors. Please note that the condition
  Eq.(\ref{eq.14_32a},\ref{eq.14_32b}) is valid only for Majorana fermions. In general, $v$-spinors can
  have independent phase from the $u$-spinors, and their relative phase
  is un-observable as long as the fermion number is conserved. We fix
  the relative phase between u-spinors and $v$-spinors by using the Majorana
  conditions Eq.(\ref{eq.14_32a},\ref{eq.14_32b}), so that we can use the same spinors for both Dirac and Majorana fermions.

  [When I introduced with Dieter Zeppenfeld a numerical program to
  calculate helicity amplitudes in 1985 (HZ:NPB274(1986)1), we did not
  use the condition Eq.(\ref{eq.14_32a},\ref{eq.14_32b})  to fix our $v$-spinors.  When I applied the
  formalism to supersymmetric models with Howard Baer and Xerxes Tata
  in the same year (BHT:PRL57(1986)294), we encountered inconsistency
  between our convention and the conditions Eq.(\ref{eq.14_32a},\ref{eq.14_32b})  imposed for Majorana
  fermions, such as SUSY partners of neutral gauge and Higgs bosons.
  We fixed the $v$-spinor conventions before the HZ paper was published,
  and hence the inconsistent version is no more available.]

{\bf hw14-10}: From the free-field expansions Eq.\ref{eq.14_24a} and Eq.\ref{eq.14_24b}, please obtain
  the expansions for $\psi_L(x)^c$ in Eq.\ref{eq.14_33a}, and $\psi_R^c(x)$ in Eq.\ref{eq.14_33b}:
\begin{eqnarray}
  && \psi_L(x)^c
  = \sum_{h=\pm1/2} \int \frac{d^3p}{2E(2\pi)^3}
  {a^\dagger({\vec p},h) e^{ipx} u_L^c({\vec p},h) +b({\vec p},h) e^{-ipx} v_L^c({\vec p},h)} \label{eq.14_33a} \\
  && \psi_R(x)^c
  = \sum_{h=\pm1/2} \int \frac{d^3p}{2E(2\pi)^3}
  {a^\dagger({\vec p},h) e^{ipx} u_R^c({\vec p},h) +b({\vec p},h) e^{-ipx} v_R^c({\vec p},h)} \label{eq.14_33b} 
\end{eqnarray}

{\bf hw14-11}: Show that the requirement Eq.(\ref{eq.14_32a},\ref{eq.14_32b}) tells
\begin{eqnarray}
  && v_R({\vec p},h) = u_L(p,h)^c = (-i\sigma^2) u_L(p,h)^* \label{eq.14_34a} \\
  && v_L({\vec p},h) = u_R(p,h)^c = (+i\sigma^2) u_R(p,h)^* \label{eq.14_34b}
\end{eqnarray}

  The operators $\psi_L^c(x)$ and $\psi_R^c(x)$ Eq.(\ref{eq.14_33a},\ref{eq.14_33b}) annihilate an
  anti-particle, and create a particle. The role of a particle and
  an anti-particle is inter-changed between the original field operators
  $\psi_L$ and $\psi_R$, and in the Charge conjugated operators
  $\psi_L^c$ and $\psi_R^c$. We show at the bottom of this homework
  that the Dirac Lagrangian is symmetric under Charge conjugation.

  Let us insert an excercise here, so that you know how $u$-spinors and
  $v$-spinors transform under Lorentz transformation. Because $v$-spinors
  are obtained from $u$-spinors by charge conjugations Eq.(\ref{eq.14_34a},\ref{eq.14_34b}), we study
  the Lorentz transformation of $u$-spinors only.

 {\bf hw14-12} First, show that the Dirac equation Eq.(\ref{eq.14_1a},\ref{eq.14_1b}) in the fermion rest frame 
\begin{eqnarray}
  p^\mu = (m,0,0,0)^T \label{eq.14_35}
\end{eqnarray}
   tells that $u_L$ and $u_R$ spinors should be identical:
\begin{eqnarray}
  u_L(p,h) = u_R(p,h). \label{eq.14_36}
\end{eqnarray}
  In the rest frame, we choose the $z$-axis along the direction of the
  three momentum in the observer's frame, such that the helicity
  \begin{eqnarray}
    h = \frac {\vec{J} \cdot \vec{p}}{|\vec{p}|}, \label{eq.14_37}
  \end{eqnarray}
  reduces to $J_z$. In this way, we obtain the wave function which satisfy
\begin{eqnarray}
  &&J_z u_L(p,h) = \frac{\sigma^3}{2} u_L(p,h) = h u_L(p,h) \label{eq.14_38a} \\
  && J_z u_R(p,h) = \frac{\sigma^3}{2} u_R(p,h) = h u_R(p,h) \label{eq.14_38b}
\end{eqnarray}
   with $h=\pm 1/2$.  The solution is,
\begin{eqnarray}
 && u_L(p,+1/2) = u_R(p,+1/2) =\sqrt{m}
  \begin{pmatrix}
    1\\0
  \end{pmatrix} \label{eq.14_39a}\\
  && u_L(p,-1/2) = u_R(p,-1/2) =\sqrt{m}
  \begin{pmatrix}
    0\\ 1
  \end{pmatrix} \label{eq.14_39b}
\end{eqnarray}

{\bf hw14-13}: Show that Eq.(\ref{eq.14_39a},\ref{eq.14_39b}) are the solutions of the Dirac equation Eq.(\ref{eq.14_1a},\ref{eq.14_1b})
  in the rest frame and the quantization condition Eq.(\ref{eq.14_38a},\ref{eq.14_38b}).

  Let us now study how the wave functions Eq.(\ref{eq.14_39a},\ref{eq.14_39b}) changed by Lorentz transformations.  We first make a boost along the positive $z$ axis by the positive rapidity $y$ (so that $h$ becomes the helicity):
\begin{eqnarray}
  p^\mu \to p'^\mu &=& L(0,0,0,0,0,\eta_3=y) p^\mu
                      = e^{ -iy K_3 } p^\mu  \\
                      &=& m( \cosh y, 0, 0, \sinh y    )^T
                      = m( \gamma,   0, 0, \gamma\beta )^T \\
                      &=& (E,          0, 0, p           )^T \label{eq.14_40}
\end{eqnarray}
 and the spinors transform as
\begin{eqnarray}
  u_L(p,+1/2) \to u_L'(p',+1/2)
  &=& S_L(0,0,0,0,0,\eta_3=y) u_L(p,+1/2) \\
  &=& e^{ \frac{\sigma^3}{2} (-y) } \sqrt{m}
  \begin{pmatrix}
    1\\0
  \end{pmatrix}
  =
  \begin{pmatrix}
    e^{-y/2} & 0\\
    0 & e^{y/2}
  \end{pmatrix}
  \sqrt{m}
  \begin{pmatrix}
    1\\0
  \end{pmatrix}\\
 &=&\sqrt{E-p}   
  \begin{pmatrix}
    1\\0
  \end{pmatrix} \label{eq.14_41a} \\
  u_L(p,-1/2) \to u_L'(p',-1/2)&=& \sqrt{E+p} 
  \begin{pmatrix}
    0 \\ 1
  \end{pmatrix} \label{eq.14_41b} \\
  u_R(p,+1/2) \to u_R'(p',+1/2) &=& 
  S_R(0,0,0,0,0,\eta_3=y) u_R(p,+1/2) \\
  &=& e^{ \frac{\sigma^3}{2} (+y) } \sqrt{m}
  \begin{pmatrix}
  1\\0
  \end{pmatrix} 
  = 
  \begin{pmatrix}
  e^{y/2} & 0\\
  0 & e^{-y/2}
\end{pmatrix}
\sqrt{m}
\begin{pmatrix}
  1\\0
\end{pmatrix}\\
  &=& \sqrt{E+p} 
  \begin{pmatrix}
  1\\0
\end{pmatrix} \label{eq.14_41c} \\
u_R(p,-1/2) \to u_R'(p',-1/2)&=& \sqrt{E-p} 
  \begin{pmatrix}
  0 \\ 1
\end{pmatrix} \label{eq.14_41d}                                
\end{eqnarray}
  
{\bf hw14-14}: Show Eq.\ref{eq.14_40} and Eq.(\ref{eq.14_41a},\ref{eq.14_41b},\ref{eq.14_41c},\ref{eq.14_41d}).
  Please observe that in the massless limit, E=p, only the two helicity
  components survive:
\begin{eqnarray}
  && u_L(p,-1/2) = \sqrt{2E}
  \begin{pmatrix}
    0 \\ 1
  \end{pmatrix} \label{eq.14_42a}\\
  &&u_R(p,+1/2) = \sqrt{2E}
  \begin{pmatrix}
    1\\0
  \end{pmatrix} \label{eq.14_42b}
\end{eqnarray}
 The other two combinations vanish in the massless limit ( $m=0$, or $E=p$)
\begin{eqnarray}
  u_L(p,+1/2) = u_R(p,-1/2) = 0 \label{eq.14_43}
\end{eqnarray}
  reproducing the results we obtained for Weyl fermions. We may note that
  in the massless (high energy) limit, only the left-handed (right-handed)
  chirality component survives for the helicity $-1/2$ ($+1/2$) states.

  Let us further operate $R_y(\theta)$ and then $R_z(\phi)$ in order to obtain
  the most general form of free fermion wave functions.  In the frame
  where the fermion momentum takes the generic form:
\begin{eqnarray}
  &&p^\mu \to p{'''}^\mu
             = R_z(\phi) R_y(\theta) B_z(y) p^\mu
             = R_z(\phi) R_y(\theta) m ( \cosh y, 0, 0, \sinh y )^T \\ 
             &=& R_z(\phi) R_y(\theta) ( E, 0, 0, p )^T
             = R_z(\phi) ( E, p\sin\theta, 0, p\cos\theta )^T \\
             &=& ( E, p\sin\theta\cos\phi, p\sin\theta\sin\phi, p\cos\theta )^T\label{eq.14_44}
\end{eqnarray}
  the wave functions are
\begin{eqnarray}
  u_L{'''}(p{'''},+1/2) &=& R_z(\phi)R_y(\theta)B_z(y) p^\mu u_L(p,+1/2)
  =\sqrt{E-p}
  \begin{pmatrix}
    e^{-i\phi/2} \cos\frac{\theta}{2} \\
    e^{i\phi/2}  \sin\frac{\theta}{2}
  \end{pmatrix} \label{eq.14_45a} \\
  u_L{'''}(p{'''},-1/2) &=& R_z(\phi)R_y(\theta)B_z(y) p^\mu u_L(p,-1/2)
  =\sqrt{E+p}
  \begin{pmatrix}
    -e^{-i\phi/2} \sin\frac{\theta}{2} \\
    e^{i\phi/2}  \cos\frac{\theta}{2}
  \end{pmatrix} \label{eq.14_45b} \\
  u_R{'''}(p{'''},+1/2) &=& R_z(\phi)R_y(\theta)B_z(y) p^\mu u_R(p,+1/2)
  = \sqrt{E+p}
  \begin{pmatrix}
    e^{-i\phi/2} \cos\frac{\theta}{2} \\
    e^{i\phi/2}  \sin\frac{\theta}{2}
  \end{pmatrix} \label{eq.14_45c} \\
  u_R{'''}(p{'''},-1/2) &=& R_z(\phi)R_y(\theta)B_z(y) p^\mu u_R(p,-1/2)
  = \sqrt{E-p}
  \begin{pmatrix}
    -e^{-i\phi/2} \sin\frac{\theta}{2} \\
    e^{i\phi/2}  \cos\frac{\theta}{2}
  \end{pmatrix} \label{eq.14_45d}
 \end{eqnarray}
Here I introduced a notation for finite Lorentz transformations
\begin{eqnarray}
  &&B_z(y)      = L(0,0,0,0,0,\eta_3=y)  \label{eq.14_46a} \\
  && R_y(\theta) = L(0,\theta_2=\theta,0,0,0,0)\label{eq.14_46b} \\
  && R_z(\phi)   = L(0,0,\theta_3=\phi,0,0,0)\label{eq.14_46c}
\end{eqnarray}
   for vectors ($4\times 4$ matrices for $p^\mu$ or any $V^\mu$),
\begin{eqnarray}
  && B_z(y)      = S_L(0,0,0,0,0,\eta_3=y)  \label{eq.14_47a} \\
  && R_y(\theta) = S_L(0,\theta_2=\theta,0,0,0,0) \label{eq.14_47b} \\
  && R_z(\phi)   = S_L(0,0,\theta_3=\phi,0,0,0) \label{eq.14_47c}
\end{eqnarray}
  for left-handed spinors ($2\times 2$ matrices for $u_L$ or $v_L$),
\begin{eqnarray}
  && B_z(y)      = S_R(0,0,0,0,0,\eta_3=y) \label{eq.14_48a} \\
  && R_y(\theta) = S_R(0,\theta_2=\theta,0,0,0,0) \label{eq.14_48b} \\
  && R_z(\phi)   = S_R(0,0,\theta_3=\phi,0,0,0) \label{eq.14_48c}
\end{eqnarray}
  for right-handed spinors ($2\times 2$ matrices for $u_R$ or $v_R$).

{\bf hw14-15}: Please obtain Eq.(\ref{eq.14_46a}$\to$\ref{eq.14_48c}), and derive Eq.\ref{eq.14_44} and Eq.(\ref{eq.14_45a},\ref{eq.14_45b},\ref{eq.14_45c},\ref{eq.14_45d}).

  By using the charge conjugation relations in Eq.(\ref{eq.14_34a},\ref{eq.14_34b}), we obtain the $v$-spinors:
\begin{eqnarray}
  && v_L(p, 1/2) = (+i\sigma^2) u_R^*(p, 1/2) \label{eq.14_49a} \\
  && v_R(p, 1/2) = (-i\sigma^2) u_L^*(p, 1/2) \label{eq.14_49b} \\
  && v_L(p,-1/2) = (+i\sigma^2) u_R^*(p,-1/2) \label{eq.14_49c} \\
  && v_R(p,-1/2) = (-i\sigma^2) u_L^*(p,-1/2) \label{eq.14_49d} 
\end{eqnarray}

{\bf hw14-16}: Please obtain explicit form of $v$-spinors Eq.(\ref{eq.14_49a},\ref{eq.14_49b},\ref{eq.14_49c},\ref{eq.14_49d}).

  [As I wrote above, the numerical program for the above wave functions
  were introduced in 1985 (HZ:NPB274(1986)1), which was later adopted
  by HELAS (1991) and by MadGraph (1994).  If you read the original
  HZ paper, you may notice that we dropped a phase factor $e^{\pm \phi/2}$
  in order to make the numerical program simple.  This dropped phase
  should be recovered when we study Lorentz transformations of the wave
  functions (e.g. in the paper EPJC71(2011)1529 where we studied spin
  $3/2$ particles), or when we study relative phase among helicity
  amplitudes (see e.g. Appendix B of arXiv:1712.09953).]

  Now, we show that the Dirac Lagrangian
\begin{eqnarray}
  {\cal L}_{Dirac} = \psi_L^\dagger i\del_\mu \sigma_-^\mu \psi_L
                + \psi_R^\dagger i\del_\mu \sigma_+^\mu \psi_R
                - m (\psi_L^\dagger \psi_R + \psi_R^\dagger \psi_L) \label{eq.14_50}
\end{eqnarray}
  is invariant under Charge conjugation transformations. Because ${\cal L}_{Dirac}$ is a singlet in the spinor space, it is equal to its transpose:
\begin{eqnarray}
  {\cal L}_{Dirac}= (  {\cal L}_{Dirac} )^T
  = ( \psi_L^\dagger i\del_\mu \sigma_-^\mu \psi_L )^T
  + ( \psi_R^\dagger i\del_\mu \sigma_+^\mu \psi_R )^T
  - m ( \psi_L^\dagger \psi_R + \psi_R^\dagger \psi_L )^T\label{eq.14_51}
\end{eqnarray}
  Because Fermi-operators anti-commutes,
\begin{eqnarray}
  {\cal L}_{Dirac} &=& - i\del_\mu \psi_L^T{\sigma_-^\mu}^T {\psi_L^\dagger}^T - i\del_\mu \psi_R^T{\sigma_+^\mu}^T {\psi_R^\dagger}^T
  + m( \psi_R^T {\psi_L^\dagger}^T + \psi_L^T {\psi_R^\dagger}^T )\\
&=& - i\del_\mu \psi_L^T{\sigma_-^\mu}^T \psi_L^*
  - i\del_\mu \psi_R^T {\sigma_+^\mu}^T \psi_R^*
  + m( \psi_R^T \psi_L^* + \psi_L^T \psi_R^* ) \label{eq.14_52}
\end{eqnarray}

{\bf hw14-17}: Confirm Eq.\ref{eq.14_52} is equivalent to Eq.\ref{eq.14_50}.

  By ignoring the total derivative terms in ${\cal L}$,
\begin{eqnarray}
  {\cal L}_{Dirac}=\psi_L^T{\sigma_-^\mu}^T( i\del_\mu \psi_L^*)+\psi_R^T {\sigma_+^\mu}^T(i\del_\mu \psi_R^*) + m( \psi_R^T \psi_L^* + \psi_L^T \psi_R^* ) \label{eq.14_53}
\end{eqnarray}

{\bf hw14-18}: Confirm Eq.\ref{eq.14_53} is equivalent to Eq.\ref{eq.14_52}.

  By inserting $(\sigma^2)^2 = (-i\sigma^2)(i\sigma^2) = 1$ between $\psi$'s
  and $\sigma_\pm^\mu$, we find
\begin{eqnarray}
  {\cal L}_{Dirac} &=& {\psi_L^c}^\dagger \sigma^2 {\sigma_-^\mu}^T \sigma^2 ( i\del_\mu {\psi_L^c})+{\psi_R^c}^\dagger \sigma^2 {\sigma_+^\mu}^T \sigma^2(i\del_\mu {\psi_R^c}) - m( {\psi_R^c}^\dagger{\psi_L^c} + {\psi_L^c}^\dagger {\psi_R^c})  \label{eq.14_54}
\end{eqnarray}

{\bf hw14-19}: Confirm Eq.\ref{eq.14_54} is equivalent to Eq.\ref{eq.14_53}.

  Finally, by noting
\begin{eqnarray}
  &&(\sigma^2) {\sigma_-^\mu}^T (\sigma^2) = \sigma_+^\mu \label{eq.14_55a} \\
  && (\sigma^2) {\sigma_+^\mu}^T (\sigma^2) = \sigma_-^\mu \label{eq.14_55b}
\end{eqnarray}

{\bf hw14-20}: Show Eq.(\ref{eq.14_55a},\ref{eq.14_55b}).

we find
\begin{eqnarray}
  {\cal L}_{Dirac}= {\psi_L^c}^\dagger i\del_\mu {\sigma_+^\mu} {\psi_L^c}+{\psi_R^c}^\dagger i\del_\mu  {\sigma_+^\mu} {\psi_R^c} - m( {\psi_R^c}^\dagger{\psi_L^c} + {\psi_L^c}^\dagger {\psi_R^c})\label{eq.14_56}
\end{eqnarray}

{\bf hw14-21}: Confirm  Eq.\ref{eq.14_56} is equivalent to Eq.\ref{eq.14_54}.

  Compared to the original $ {\cal L}_{Dirac}$ in Eq.\ref{eq.14_50}, Eq.\ref{eq.14_56} shows that exactly
  the same Lagrangian is obtained by the replacements
\begin{eqnarray}
  \psi_L \to \psi_R^c, ~~~\psi_R \to \psi_L^c \label{eq.14_57}
\end{eqnarray}
  The transformation Eq.\ref{eq.14_57} is called the Charge conjugation transformation. As we showed above, in the charge conjugated field operators, an anti-particle is annihilated, and a particle is created.  For instance,
  for QED, the field operators
\begin{itemize}
  \item $\psi_{L,R}$: annihilate electron,  creat positron;
  \item $\psi_{L,R}^c$: annihilate positrion, creat electron.
\end{itemize}
  The role of the electron and positron are interchanged under the
  Charge conjugation transformation.

  It is important for you to recognize that the charge-conjugation
  relationship is obtained from the simple fact that the Lagrangian
  density is a singlet in the spinor space, and hence it is equal
  to its transpose, Eq.\ref{eq.14_50}=Eq.\ref{eq.14_51}. Because this is always true, the Charge conjugation plays an important role in the SM and in its extentions.

  That's all for hw14.\\

Best regards,\\

Kaoru


\end{document}

\documentclass[12pt]{article}
\usepackage{amsmath,graphicx,color,epsfig,physics}
\usepackage{float}
\usepackage{subfigure}
\usepackage{slashed}
\usepackage{color}
\usepackage{multirow}
\usepackage{feynmp}
\textheight=9.5in \voffset=-1.0in \textwidth=6.5in \hoffset=-0.5in
\parskip=0pt
\def\del{{\partial}}
\def\dgr{\dagger}
\def\eps{\epsilon}
\def\lmd{\lambda}
\def\th{\theta}
\begin{document}

\begin{center}
{\large\bf HW22 for Advanced Particle Physics} \\
  
\end{center}

\vskip 0.2 in

Dear students,\\

This week, I explain a simple collision process e+ e- -> f fbar.
The first part of this homework contains Z -> f fbar amplitudes,
and the calculation of the Z boson partial and total widths.
In the second part (hw23), we obtain e+e- to f fbar amplitudes in
the second order of S-matrix expansion, with the Z-boson propagator.
I will explain Schwinger-Dyson summation of one-particle-irreducible
(1PI) two-point functions in the full Z-boson propagator, and
show how Unitarity determines the Z-boson decay width.

==================
E: e+ e- -> f fbar
==================

Please recall the covariant derivative of the EW currents:

1) D_\mu = \del_\mu
          + i g_W/\sqrt{2} [T^+ W^+_\mu  + T^- W^-_\mu]
          + i g_Z [ T^3 - Q \sin^2\theta_W ] Z_\mu
          + i e Q A_\mu

For simplicity, let us take the Z-boson contribution only.
The relevant interaction Lagrangian is then

2) L_\int = \Psibar i(D_\mu-\del_\mu) \gamma^\mu \Psi
          = -gZ Z_\mu \Psibar \gamma^\mu (T^3 - Q \sw2) \Psi + ...

where I introduced abbreviations gZ=g_Z=g/\cw=e/\sw\cw,
\sw=\sin\theta_W, \cw=\cos\theta_W, \sw2=\sw^2, etc.  As always,
e = g\sw = g'\cw = g_Z\cw\sw is worth memorizing.

In the SM, we have 7 Zff coupling factors

3a) u_L     gZ (T^3 - Q\sw2) = gZ( 1/2 - 2/3 \sw2) = gZuL
3b) u_R     gZ (T^3 - Q\sw2) = gZ( 0   - 2/3 \sw2) = gZuR
3c) d_L     gZ (T^3 - Q\sw2) = gZ(-1/2 + 1/3 \sw2) = gZdL
3d) d_R     gZ (T^3 - Q\sw2) = gZ( 0   + 1/3 \sw2) = gZdR
3e) n_L     gZ (T^3 - Q\sw2) = gZ( 1/2           ) = gZnL
3f) l_L     gZ (T^3 - Q\sw2) = gZ(-1/2 +     \sw2) = gZlL
3g) l_R     gZ (T^3 - Q\sw2) = gZ( 0   +     \sw2) = gZlR

hw22-1: Please give the magnitudes of e,g,gZ,g' when
e = \sqrt{ 4\pi\alpha }, with \alpha = \alpha(mZ) = 1/128,
and \sw2 = 0.233.  Also, please give the magnitudes of
gZf/gZ and (gZf/gZ)^2 for f = uL, uR, dL, dR, nL, lL, lR.
Do you observe something interesting among the 7 couplings?

For the on-shell Z boson (mZ=92GeV), the relevant fermions are,
u=u,c, d=d,s,b, n=n1,n2,n3 or ne,nmu,ntau, l=e,mu,tau.
Only the top-quark (u=t) cannot appear in the Z boson decays.

Please remember from the previous homework that all the neutral
currents are diagonal in the SM, since the unitary matrix which
relates the mass-eigenstates with arbitrary current states
cancel out for the neutral currents.  Therefore, the couplings
are the same in the mass-eigenstate basis of nu1, nu2, nu3,
or the charged-lepton current basis of nue, numu, nutau.

hw22-2: Prove the above statements.

Let us first review the Z boson decay rate into a fermion pair, f+fbar,
which is very similar to the W boson decay we studied before.  The
initial state is a one-Z boson state

3) |i> = a^\dagger_{Z,p,\lambda} |0>

Let us take the Z boson rest-frame

4) p^\mu = (mZ, 0, 0, 0)

and the polarization along the z-direction

5) J_z \epsilon_\mu(p,h) = h \epsilon_\mu(p,h)

with h=+1, or 0.  The final state is a pair of f and fbar in the same
frame:

6) |f> = a^\dagger_{f,k1,h1} b^\dagger_{fbar,k2,h2} |0>

where

7) k1 = E(1,  \sin\theta, 0,  \cos\theta)
   k2 = E(1, -\sin\theta, 0, -\cos\theta)

from the energy momentum conservation, we can tell

8) E = mZ/2.

The helicities, h1 and h2, should be chosen according to the chirality
of f,

9) h1=-h2=+1/2 for f=uR,dR,lR
   h1=-h2=-1/2 for f=uL,dL,lL,nL

For simplicity, we ignore all the masses of fermions that appear in
Z-boson decays (the heaviest one is the b-quark, whose mass is about
5GeV, which is about 4.5% of mZ, giving (4.5%)^2=0.2% order corrections
to the decay rate).

hw22-3: Please confirm the h1=-h2 rule for massless fermions, when
a pair of f and fbar couples to a vector boson, in the initial state
(Z -> f fbar) or in the final state (f fbar -> Z).

Let us first obtain the amplitude

10) <f(h1)fbar(-h1)| T\exp{i\Int d^4x L_int(x)} |Z(h)>
= i (2pi)^4 \delta^4(pZ-pf-pf~) M(Z(h) > f(h1) f~(-h1))

The operators in L_int(x) that contribute to the amplitudes (10) are

11) Z_\mu(x)
= \Sum_h \Int d^3k/2E
  { a_{Z:k,h}         \eps_\mu(k,h)   e^{-ikx}
  + a_{Z:k,h}^\dagger \eps_\mu(k,h)^* e^{ikx} }_{k^0=E_k} }

where the sum is over h=+1,0,-1, and E_k=\sqrt{mZ^2+k^2}.  And

12a) \Psi_f(x)
= \Sum_h \Int d^3k/2E
  { a_{f:k,h}         u(k,h) e^{-ikx}
  + b_{f:k,h}^\dagger v(k,h) e^{ikx} }_{k^0=E_k} }

12b) \Psi_f(x)^\dagger
= \Sum_h \Int d^3k/2E
  { a_{f:k,h}^\dagger u(k,h)^\dagger e^{ikx}
  + b_{f:k,h}         v(k,h)^\dagger e^{-ikx} }_{k^0=E_k} }

Please note that Z boson is real, Z_\mu^\dagger = Z_\mu, whereas the
SM fermions other than the neutrinos are not.  Neutrinos may or may
not be self-conjugate (Majorana).  In eqs.(12), I intentionally write
the complex conjugates (\Psi^\dagger) rather than the Dirac conjugate
(\Psibar = \Psi^\dagger \gamma^0), since the latter is simply obtained
multiplying \gamma^0 from the right-hand-side of (12b).  We already
know that the purpose of the Dirac conjugate is simply to choose the
correct chiral components, fL or fR, when making the Lorentz
invariants (fL^\dagger fR) or (fR^\dagger fL), or making the vectors
(fL^\dagger \sigma_-^\mu fL) or (fR^\dagger \sigma_+^\mu fR).
Therefore, as long as we can ignore the fermion masses,
it is far easier to compute the amplitudes in the Weyl basis.

Now in the expression (10), the transition amplitudes receive non-zero
contribution at n=1,

13) T \exp{i\Int d^4x L_int(x)}
= T {  1 + i\Int d^4x L_int(x) + ...
= T {  1 + i\Int d^4x
 {-gZ Z_\mu(x) \Psibar(x) \gamma^\mu (T^3 - Q \sw2) \Psi(x)} + ...
= 1 + \Int d^4x Z_\mu(x)
 {(-igZfL \fL(x)^\dagger \sigma_-^\mu \fL(x)
 +(-igZfR \fR(x)^\dagger \sigma_+^\mu \fR(x) } + ...

In the last term, I drop the T-ordering since the n=1 term has only
one time, and express the Dirac fermion fields in terms of their two
chiral component fields, fL(x) and fR(x).  The symbol `f' stands for
f=u,d,l,nu, and the coupling gZfL and gZfR are for Z-fL-fL and
Z-fR-fR couplings, respectively.  Also, the factor of i multiplying
\Int d^4x L_{int}(x) is moved to multiply the coupling factor.

Now, we can evaluate the matrix elements of the above operator

14) <ffbar| S |Z>
= <ffbar| T\exp{i\Int d^4x L_int(x)} |Z>
= <0| b_{fbar,k2,h2} a_{f,k1,h1} S a^\dagger_{Z,p,\lambda} |0>
= (2pi)^4 \delta^4(p_Z-p_f-p_fbar) iM(Z>ffbar)

by inserting eqs.(3) and (6) into (10).  Please confirm that only the
following terms survive after using the definition of the vacuum,

15) a_{any particle, any momentum, any spin} |0> = 0
    b_{any particle, any momentum, any spin} |0> = 0

and an appropriate normalization, <0|0> = 1, which depends on
how we normalize the creation/annihilation operators.  The
covariant normalizations for the commutator for Z, anti-commutators
for f and fbar:

16a) [ a_{k}, a^\dagger_{k'} ] = 2E \delta^3(k - k')
16b) [ b_{k}, b^\dagger_{k'} ] = 2E \delta^3(k - k')

are common, and the normalizations of one-particle states etc
follow.  In any case, the space-time integral \Int d^4x gives
the overall 4-momentum conservation,

17) \Int d^4x e^{-i \Sum P_initial x} e^{ +i \Sum P_final x}
  = (2\pi)^4 \delta^4( \Sum_initial p_k - \Sum_final p_l )

Because this overall 4-momentum conservation is common to all the
S-matrix element, we usually drop this term from the scattering
amplitudes, iM, as written in eq.(14).  We find:

18a) iM(Z > fL fLbar)
= (-igZfL) \uL(k1,L)^\dagger \sigma_-^\mu \vL(k2,R)
                                 \eps_\mu(P,\lambda)

18b) iM(Z > fR fRbar)
= (-igZfR) \uR(k1,R)^\dagger \sigma_+^\mu \vR(k2,L)
                                 \eps_\mu(P,\lambda)

hw22-4:  Derive eqs.(18) by yourself.

I denote the helicity +1/2 as R, and the helicity -1/2 as L.
The factor of i in front of M, cancels the factor of i
in the `Feynman rules'

19a) (-igZfL) \sigma_-^\mu    for Z-fL-fL vertex
19b) (-igZfR) \sigma_+^\mu    for Z-fR-fR vertex

and the tree-amplitude is real, rather than pure imaginary.

Now in the rest frame of the Z boson (mZ)

20) P^\mu  = (mZ, 0, 0, 0)

the 3 polarization vectors are (see past homeworks)

21a) \eps^\mu(P,+1) = ( 0, -1, -i, 0 )/\rt2
21b) \eps^\mu(P, 0) = ( 0,  0,  0, 1 )
21c) \eps^\mu(P,-1) = ( 0, +1, -i, 0 )/\rt2

In the massless limit, the fermion wave functions are

22a) uL(p,L) = \sqrt{2E} [ 0 ]
                         [ 1 ]
22b) uR(p,R) = \sqrt{2E} [ 1 ]
                         [ 0 ]
22c) vL(p,R) = uR(p,R)^c =  i\sigma^2 uR(p,R)^* = \rt{2E} [  0 ]
                                                          [ -1 ]
22d) vR(p,L) = uL(p,L)^c = -i\sigma^2 uL(p,L)^* = \rt{2E} [ -1 ]
                                                          [  0 ]

in the frame where

23) p^\mu = (E, 0, 0, E)

In the Z boson rest frame, both k1 and k2 have energies E=mZ/2.
k1^\mu is obtained from p^\mu by R_y(\theta), whereas k2^\mu is
obtained from p^\mu by R_y(\theta+\pi).  Therefore,

24a) uL(k1,L) = R_y(\theta) uL(p,L)
              = \rt{2E} [c(th/2) -s(th/2)] [0] = \rt{2E} [-s(th/2)]
                        [s(th/2)  c(th/2)] [1]           [ c(th/2)]
24b) uR(k1,R) = R_y(\theta) uR(p,R)
              = \rt{2E} [c(th/2) -s(th/2)] [1] = \rt{2E} [ c(th/2)]
                        [s(th/2)  c(th/2)] [0]           [ s(th/2)]
24c) vL(k2,R) = R_y(\pi+\theta) vL(p,R)
              = \rt{2E} [c(pi/2+th/2) -s(pi/2+th/2)] [ 0]
                        [s(pi/2+th/2)  c(pi/2+th/2)] [-1]
              = \rt{2E} [-s(th/2) -c(th/2)] [ 0] = \rt{2E} [ c(th/2)]
                        [ c(th/2) -s(th/2)] [-1]           [ s(th/2)]
24d) vR(k2,L) = R_y(\pi+\theta) vR(p,L)
              = \rt{2E} [c(pi/2+th/2) -s(pi/2+th/2)] [-1]
                        [s(pi/2+th/2)  c(pi/2+th/2)] [ 0]
              = \rt{2E} [-s(th/2) -c(th/2)] [-1] = \rt{2E} [ s(th/2)]
                        [ c(th/2) -s(th/2)] [ 0]           [-c(th/2)]

hw22-5:  Derive eqs.(24).  What happens if we use R_y(\theta-\pi)
instead of R_y(\theta+\pi) to obtain fbar wave functions?

Now the fermion currents are (by noting (\rt{2E})^2 = 2E = mZ):

25a) uL(k1,L)^\dagger \sigma_-^\mu vL(k2,R)
   = mZ[-s(th/2),c(th/2)] [1,-\sigma^1,-\sigma^2,-\sigma^3] [c(th/2)]
                                                            [s(th/2)]
   = mZ[ 0, -c(th),  -i,  s(th) ]

25b) uR(k1,R)^\dagger \sigma_+^\mu vR(k2,L)
   = mZ[c(th/2),s(th/2)] [1, \sigma^1, \sigma^2, \sigma^3] [ s(th/2)]
                                                           [-c(th/2)]
   = mZ[ 0, -c(th),  i,  s(th) ]

Now the amplitudes (18a,b) are easily calculated as
(I denote a^\mu b_mu = a^mu * b^mu):

26a) M(Z > fL fLbar)
=(-gZfL) \uL(k1,L)^\dagger \sigma_-^\mu \vL(k2,R) \eps_\mu(P,\lambda)
= -gZfL mZ [0,-c(th),-i,s(th)]*[0,-1,-i, 0]/\rt2  for \lambda=+1
= -gZfL mZ [0,-c(th),-i,s(th)]*[0, 0, 0, 1]/\rt2  for \lambda= 0
= -gZfL mZ [0,-c(th),-i,s(th)]*[0,+1,-i, 0]/\rt2  for \lambda=-1
= -gZfL mZ (0 -c(th) +1)/\rt2   for \lambda=+1
= -gZfL mZ (0 -0  -0 -s(th))    for \lambda= 0
= -gZfL mZ (0 +c(th) +1)/\rt2   for \lambda=-1

26b) M(Z > fR fRbar)
=(-gZfR) \uR(k1,R)^\dagger \sigma_+^\mu \vR(k2,L) \eps_\mu(P,\lambda)
= -gZfR mZ [0,-c(th),+i,s(th)]*[0,-1,-i, 0]/\rt2  for \lambda=+1
= -gZfR mZ [0,-c(th),+i,s(th)]*[0, 0, 0, 1]       for \lambda= 0
= -gZfR mZ [0,-c(th),+i,s(th)]*[0,+1,-i, 0]/\rt2  for \lambda=-1
= -gZfL mZ (0 -c(th) -1)/\rt2   for \lambda=+1
= -gZfL mZ (0 -0  -0 -s(th))    for \lambda= 0
= -gZfL mZ (0 +c(th) -1)/\rt2   for \lambda=-1

The amplitudes are just a (real) function of the opening angle between
the Z boson spin quantization axis and the fL/fR momentum direction in
the Z boson rest frame.

hw22-6:  Please confirm eqs.(26), and express them by using the
d-functions we introduced in the last homework.

The decay width can be calculated as

27a) \Gamma(Z > fL+fLbar)
= 1/2mZ \Int |M(Z>fL+fLbar;\lambda,\theta)|^2 d\PS_2(P=k1+k2)

27b) \Gamma(Z > fR+fRbar)
= 1/2mZ \Int |M(Z>fR+fRbar;\lambda,\theta)|^2 d\PS_2(P=k1+k2)

where the integral is over the 2-body phase space of fL and fR:

28) d\PS_2(P=k1+k2)
= (2pi)^4 \delta^4(P-k1-k2) \d^3k1/(2E1)(2pi)^3 d^3k2/(2E2)(2pi)^3

In general, n-body phase space is defined as

29) d\PS_n(P=\Sum_i k_i)
= (2pi)^4 \delta^4(P-\Sum_i k_i) \Pi_j { d^3k_j/[(2E_j)(2pi)^3 }

where \Pi_j A_j = A_1 * A_2 * ... * A_n, and

30) E_j = \sqrt{ k_j^2 + m_j^2 }

is the on-shell energy of the particle j of mass m_j.  In the two-body
rest frame, we can determine the k_1 and k_2 4-momentum by using the
k_1 3-momentum (which are chosen as the integration variable) as
(for massless particles)

31a) k_1^\mu = (E, Esin(th)cos(phi), Esin(th)sin(phi), Ecos(th))
31b) k_2^\mu = (E,-Esin(th)cos(phi),-Esin(th)sin(phi),-Ecos(th))

The remaining delta function and the integrations are

32) delta(mZ-E-E) d^3 k_1 = delta(mZ-2E) E^2 dE dcos(th) d(phi)
                          = mZ^2/8 dcos(th) d(phi)

where E=mZ/2 is fixed when the last delta function is used.  We find

33) dPS_2 = (1/8pi) dcos(th)/2 d(phi)/2pi

when both particles are massless, k_1^2=k_2^2=0.  The second
parametrization is for general mass case, k_j^2=mj^2:

34) dPS_2 = (1/8pi) (2p^*/M) dcos(th)/2 d(phi)/2pi

where p^* is the magnitude of the common 3-momentum of the
two particles in the rest frame:

35a) k_1^\mu = (E1, p^* s(th)c(phi), p^* s(th)s(phi), p^* c(th))
     k_2^\mu = (E2,-p^* s(th)c(phi),-p^* s(th)s(phi),-p^* c(th))

35b) E1 = \sqrt{ (p^*)^2 + m_1^2 }
     E2 = \sqrt{ (p^*)^2 + m_2^2 }
     E1+E2 = M

35c) 2p^*/M
   = \sqrt{1+(m1/M)^4+(m2/M)^4-2(m1/M)^2-2(m2/M)^2-2(m1m2/M^2)^2}

Note that (35c) can be written as

36) 2Mp^*
  = \sqrt{M^4+(m1)^4+(m2)^4-2(m1*M)^2-2(m2*M)^2-2(m1*m2)^2}
  = \sqrt{M^4-2M^2[(m1)^2+(m2)^2]+[(m1)^2-(m2)^2]^2}
  = \sqrt{[M^2-(m1)^2-(m2)^2]^2-[(m1)^2+(m2)^2]^2+[(m1)^2-(m2)^2]^2}
  = \sqrt{[M^2-(m1)^2-(m2)^2]^2-4(m1)^2(m2)^2}
  = \sqrt{[M^2-(m1)^2-(m2)^2+2(m1)(m2)][M^2-(m1)^2-(m2)^2-2(m1)(m2)]}
  = \sqrt{[M^2-(m1-m2)^2][M^2-(m1+m2)^2]}
  = \sqrt{(M-m1+m2)(M+m1-m2)(M+m1+m2)(M-m1-m2)}

or

37) 2p^*/M
  = \sqrt{[1+(m1+m2)/M][1-(m1+m2)/M][1+(m1-m2)/M][1-(m1-m2)/M]}

The above factor dictates the 2-body phase space suppression factor
for massive particles, and it vanishes when

38) M = m1+m2

which is the two particle production threshold.  Since the magnitude of
the momentum is fixed by energy-momentum conservation, the whole phase
space is the region

39) -1 < \cos\theta < 1, 0 < \phi < 2pi

The total volume of the 2-body phase space is hence

40) PS_2 = 1/8pi 2p^*/M -> 1/8pi (for m1=m2=0).

It is quite useful to memorize this.

hw22-7:  The parametrization of the two body phase space has been done
in the past homework.  You may repeat the proof of (33), (34), (37),
as homework only if you cannot yet write them down without viewing
my derivation.

Since our amplitudes, (25a-b) do not depend on the azimuthal angle \phi,
the differential decay rate is

41) d\Gamma = 1/2M |M(\lambda,\theta)|^2 (1/8pi) dcos(th)/2

or

42) d\Gamma      1    |M(\lambda,\theta)|^2
    -------- = -----  ---------------------
    dcos(th)    16pi          2M

The Lorentz invariance of physics tells that the decay rates of
a particle do not depend on the direction of the polarization.
If a particle decays faster when it is polarized in a particular
direction, it tells that there is a special direction in our universe.

hw22-8: Please obtain all the partial decay widths (43) in terms of
the Z boson mass and the couplings.

43) \Gamma(Z > fL + fLbar), and \Gamma(Z > fR + fRbar)

by integrating out the \cos\theta dependence of the squared amplitudes,
and show that they are independent of the Z boson polarization,
\lambda.

h22-9: Please obtain numerically the following partial widths:

44a) \Gamma(Z > uL + uLbar)  (u=u,c;   sum over 3 colors)
44b) \Gamma(Z > uR + uRbar)  (u=u,c;   sum over 3 colors)
44c) \Gamma(Z > dL + dLbar)  (d=d,s,b; sum over 3 colors)
44d) \Gamma(Z > dR + dRbar)  (d=d,s,b; sum over 3 colors)
44e) \Gamma(Z > lL + lLbar)  (l=e,\mu,\tau)
44f) \Gamma(Z > lR + lRbar)  (l=e,\mu,\tau)
44g) \Gamma(Z > vL + vLbar)  (v=v_e,v_\mu,v_\tau or v=v_1,v_2,v_3)

Please use mZ = 92GeV, e=\sqrt{4pi\alpha} with \alpha=1/128,
sW^2=0.233.  Please compute

45a) \Gamma(Z > u+ubar) = \Gamma(Z>uL+uLbar) + \Gamm(Z>uR+uRbar)
45b) \Gamma(Z > d+dbar) = \Gamma(Z>dL+dLbar) + \Gamm(Z>dR+dRbar)
45c) \Gamma(Z > l+lbar) = \Gamma(Z>lL+lLbar) + \Gamm(Z>lR+lRbar)
45d) \Gamma(Z > v+vbar) = \Gamma(Z>vL+vLbar)

Please obtain the total decay width of the Z boson as

46) \Gamma_Z = 2*\Gamma(Z>uubar)
              +3*\Gamma(Z>ddbar)
              +3*\Gamma(Z>llbar)
              +3*\Gamma(Z>vvbar)

and compare your result with the data compiled by the PDG.

Please also obtain the Z boson decay branching fractions

47a) B(Z>uubar)=B(Z>ccbar)=\Gamma(Z>uubar)/\Gamma_Z
47b) B(Z>ddbar)=B(Z>ssbar)=B(Z>bbbar)=\Gamma(Z>ddbar)/\Gamma_Z
47c) B(Z>eebar)=B(Z>mumubar)=B(Z>tautaubar)=\Gamma(Z>uubar)/\Gamma_Z
47d) B(Z>invisible)=3*B(Z>vvbar)=3*\Gamma(Z>vvbar)/\Gamma_Z

Please compare your results with the data compiled by the PDG.

The major cause of the difference comes from the QCD radiative
corrections, which enhances all the decay widths into quarks
by about a few %.  Smaller effects are due to finite masses
of the b-quark and tau-leptons.  After correcting for them,
the observed magnitude of the invisible width is very accurately
3 times the calculated width of Z > nu nubar.  This confirmed
that there are just 3 neutrinos in the world.

The Z-bosons produced in e+e- collisions are polarized in
the Z-boson rest frame, along the colliding e+e- beam direction.

The chirality conservation in the massless electron limit tells
that only eL and eL^c(right-handed positron), and
eR and eR^c(left-handed positron) can annihilate to produce a
Z boson.  The angular momentum sum along the e+e- beam
direction tells (if we take the e- beam direction as our
z-axis), then the initial angular momentum is

48a) J_z = (-1/2) - (+1/2) = -1  when eL collides with eL^c
48b) J_z = (+1/2) - (-1/2) = +1  when eR collides with eR^c

The produced Z boson hence has J_z=-1 or J_z=+1.
Because J_z=0 Z boson is never produced, the Z boson is
polarized even when the beam is not polarized (half eL
and half eR).  Because of this, the angular distribution
of the process

49) e- + e+ > Z > f + fbar

has a definite \cos\theta dependence, as we found in the
amplitudes (26a)-(26b).

hw22-10:  Please show that IF exactly the same number of
\lambda = +1, 0, -1 Z bosons are there, then all the
angular distributions become flat:

50) d\Gamma/d\cos(th) = 1/3 d\Gamma(\lambda=+1)/d\cos(th)
                       +1/3 d\Gamma(\lambda= 0)/d\cos(th)
                       +1/3 d\Gamma(\lambda=-1)/d\cos(th)

This exercise tells that the equal weight sum of the 3 polarized
states is the same as the un-polarized Z boson.  Since the un-polarized
Z boson has no preferential direction, the decay angular distributions
can only be flat (in both \cos\theta and \phi).

Now, when the colliding e- and e+ are un-polarized, the produced
Z boson is still polarized, and hence its decay distribution is
NOT flat:

51) d\Gamma\d\cos(th) = 1/2 d\Gamma(\lambda=+1)/d\cos(th)
                       +1/2 d\Gamma(\lambda=-1)/d\cos(th)

These distributions have been measured at LEP and SLC, and from the
angular distributions, the magnitude of the weak mixing parameter
sw2 = \sin^2\theta_W has been measured.

hw22-11:  Please compute 51) for Z > l+lbar, and show that the
Forward-Backward asymmetry can determine the magnitude of sw2:

            \Gamma(\cos(th)>0) - \Gamma(\cos(th)<0)
52) A_FB =  ---------------------------------------
            \Gamma(\cos(th)>0) + \Gamma(\cos(th)<0)

Here

53a) \Gamma(cos(th)>0) = \Int_0^1  dcos(th) d\Gamma/dcos(th)
53b) \Gamma(cos(th)<0) = \Int_-1^0 dcos(th) d\Gamma/dcos(th)

hw22-12:  Please repeat the above exercise for Z > b+bbar mode, and
compute A_FB for Z > b+bbar and Z > l+lbar when sw2=0.233.  Assuming
that we can use three charged leptons to measure sw2, which process,
Z > llbar or Z > bbar, is more sensitive to sw2 ?  For definiteness,
let us assume that we gather 10^7 Z bosons.  What is the expected
statistical error of sw2 from the lepton mode and from the b mode?

That's all for hw22.

Best regards,

Kaoru




\end{document}
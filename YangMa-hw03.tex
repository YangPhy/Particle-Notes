\documentclass[11pt]{article}
\usepackage{amsmath,graphicx,color,epsfig,physics}
%\usepackage{pstricks}
\usepackage{float}
\usepackage{subfigure}
\usepackage{slashed}
\usepackage{color}
\usepackage{multirow}
\usepackage{feynmp}
\usepackage[top=1in, bottom=1in, left=1.2in, right=1.2in]{geometry}
\begin{document}
\title{Particle physics HW3}
\author{Yang Ma}

\maketitle


\section{ }
The $\Gamma/m$ ratio
\begin{eqnarray}
  &&{\rm top}:~ 8.15 \times 10^{-3}\\
  &&{\rm Higgs}:~ 1.04 \times 10^{-4}\\
  &&{\rm Z}: ~ 2.74 \times  10^{-2}\\
  &&{\rm W}^\pm: ~ 2.60 \times 10^{-2}\\
  &&{\tau}: ~ 1.28 \times 10^{-9}\\
  &&{\mu}: ~ 2.85 \times 10^{-15}\\
  &&{n}: ~ 7.96 \times  10^{-28} \\
  &&{\pi}^+: ~ 1.81 \times  10^{-16} \\
  &&{\pi}^0: ~ 5.73 \times  10^{-8} \\
  &&{ p}: < 1.06 \times  10^{-61}\\
  &&{e}: < 6.34 \times  10^{-58}
\end{eqnarray}

\section{ }
In general, the smaller the $\Gamma/m$ ratio is, the more stable the given particle is.
And, by comparing the $\Gamma/m$ ratio between different particle, e.g. $\tau$ and $\mu$, we notice that
\begin{eqnarray}
  \Gamma(l \to e \nu_e \nu_l) \propto G^2 m^5_l,
\end{eqnarray}
where $l$ stands for $\tau$ or $\mu$. It can be seen that (since both processes are weak interaction) $\Gamma/m \propto m_l^4$, which agrees with our numerical results
\begin{eqnarray}
  &&\frac{m_{\tau}^4}{m_{\mu}^4} = 8.2 \times 10^4, \\
  &&\frac{(\Gamma/m)_{\tau}}{(\Gamma/m)_{\mu}} = 4.5 \times 10^5.
\end{eqnarray}
Note 
\begin{eqnarray}
  \frac{m_{\tau}^4}{m_{\mu}^4} \sim \frac{(\Gamma/m)_{\tau} B (\tau \to e \nu_e \nu_l )}{(\Gamma/m)_{\mu}},
\end{eqnarray}
where $B (\tau \to e \nu_e \nu_l ) \sim 0.2$ is the branching ratio.

For the difference between $\pi^+$ and $\pi^0$ is quite larege, $(\Gamma/m)_{\pi^+} \ll (\Gamma/m)_{\pi^0}$, which is understandable that $\pi^0 \to \gamma \gamma$ is EW interaction and $\pi^+$ decay is weak interaction.

\section{ }

\begin{itemize}
  \item Since
  \begin{eqnarray}
  Q
  \begin{pmatrix}
    1\\0
  \end{pmatrix}
  =(T^3+Y)
  \begin{pmatrix}
    1\\0
  \end{pmatrix}
  =(\frac{1}{2}+\frac{1}{6})
  \begin{pmatrix}
    1\\0
  \end{pmatrix},
  \end{eqnarray}
  \begin{eqnarray}
    Q
    \begin{pmatrix}
      0\\1
    \end{pmatrix}
    =(T^3+Y)
    \begin{pmatrix}
      0\\1
    \end{pmatrix}
    =(-\frac{1}{2}+\frac{1}{6})
    \begin{pmatrix}
      0\\1
    \end{pmatrix},
    \end{eqnarray}
     we see
      \begin{eqnarray}
        Q u_{Li}
        \begin{pmatrix}
          1\\0
        \end{pmatrix}
        =\frac{2}{3} u_{Li}
        \begin{pmatrix}
          1\\0
        \end{pmatrix},
        \end{eqnarray}
        \begin{eqnarray}
          Q d_{Li}
          \begin{pmatrix}
            0\\1
          \end{pmatrix}
          =-\frac{1}{3} d_{Li}
          \begin{pmatrix}
            0\\1
          \end{pmatrix}.
          \end{eqnarray}
  The results show that $Q=\frac{2}{3},~-\frac{1}{3}$.

  \item Likewise, we can also write
  \begin{eqnarray}
    Q v_{Li}
    \begin{pmatrix}
      1\\0
    \end{pmatrix}
    =\frac{2}{3} v_{Li}
    \begin{pmatrix}
      1\\0
    \end{pmatrix},
    \end{eqnarray}
    \begin{eqnarray}
      Q l_{Li}
      \begin{pmatrix}
        0\\1
      \end{pmatrix}
      =-\frac{1}{3} l_{Li}
      \begin{pmatrix}
        0\\1
      \end{pmatrix},
      \end{eqnarray}
      where $Y=-1/2$ and the results show that $Q=1,~0$.
\end{itemize}

\section{ }
The Pauli matrix are
\begin{eqnarray}
  \sigma^1=
  \begin{pmatrix}
    0&1\\1&0
  \end{pmatrix},~
  \sigma^2=
  \begin{pmatrix}
    0&-i \\i&0
  \end{pmatrix},~
  \sigma^3=
  \begin{pmatrix}
    1&0\\0&-1
  \end{pmatrix}.
\end{eqnarray}
Then we can write
\begin{eqnarray}
  (\sigma^1)^2&=&
  \begin{pmatrix}
    0&1\\1&0
  \end{pmatrix}
  \begin{pmatrix}
    0&1\\1&0
  \end{pmatrix}=
  \begin{pmatrix}
    1&0\\0&1
  \end{pmatrix} \nonumber \\
  (\sigma^2)^2&=&
  \begin{pmatrix}
    0&-i \\i&0
  \end{pmatrix}
  \begin{pmatrix}
    0&-i \\i&0
  \end{pmatrix}
  =\begin{pmatrix}
    1&0\\0&1
  \end{pmatrix} \nonumber \\
  (\sigma^3)^2&=&
  \begin{pmatrix}
    1&0\\0&-1
  \end{pmatrix}
  \begin{pmatrix}
    1&0\\0&-1
  \end{pmatrix}
  =\begin{pmatrix}
    1&0\\0&1
  \end{pmatrix}.
\end{eqnarray}
    
\section{ } \label{sec:udu}
Since we know $(\sigma^k)^\dagger = \sigma^k$ and $T^k=\frac{1}{2}\sigma^k$, we see
\begin{eqnarray}
  (T^k)^\dagger=T^k,
\end{eqnarray}
and can then write
\begin{eqnarray}
  U (\theta_1,\theta_2,\theta_3)&=&e^{i\sum_k T^k \theta_k},\nonumber \\
  U^\dagger (\theta_1,\theta_2,\theta_3)&=&e^{-i\sum_k (T^k)\dagger \theta_k}= e^{-i\sum_k T^k \theta_k}.
\end{eqnarray}
Note that $[i\sum_k T^k \theta_k,-i\sum_k T^k \theta_k]=0$, therefore
\begin{eqnarray}
  U^\dagger (\theta_1,\theta_2,\theta_3) U (\theta_1,\theta_2,\theta_3) = e^{i\sum_k T^k \theta_k -i\sum_k T^k \theta_k}=1.
\end{eqnarray}

\section{  }

\begin{itemize}
  \item U(1) transfromation:
  \begin{eqnarray}
    \phi \to \phi'= U(\theta_0) \phi= e^{i Y \theta_0} \phi, ~ Y=\frac{1}{2},
  \end{eqnarray}
 so we know
 \begin{eqnarray}
  \phi^* \to (\phi^*)' &=& (\phi')^*= (U(\theta_0) \phi)^*= e^{-i Y \theta_0} \phi^*\nonumber \\
  &=& U(-\theta_0) \phi^*.
 \end{eqnarray}
  \item SU(2) transfromation:
  \begin{eqnarray}
    \phi \to \phi'=U(\theta_1,\theta_2,\theta_3)\phi =e^{i\sum_k T^k \theta_k} \phi,
  \end{eqnarray}
  we then write
  \begin{eqnarray}
    (\phi^*)'= (\phi^*)'=(U(\theta_1,\theta_2,\theta_3)\phi)^* =e^{-i\sum_k (T^k)^* \theta_k} \phi^*.
  \end{eqnarray}
Note that
\begin{eqnarray}
  (T^1)^*=T^1,~~(T^2)^*=T^2,~~(T^3)^*=T^3,
\end{eqnarray}
so 
\begin{eqnarray}
  (\phi^*)'=e^{-i\sum_k (T^k)^* \theta_k} \phi^*=U(-\theta_1,\theta_2,-\theta_3) \phi^*.
\end{eqnarray}
\end{itemize}

\section{ }
Under SU(2) transfromation,
\begin{eqnarray}
  (\phi^\dagger)' \phi = (U \phi)^\dagger U \phi= \phi\dagger U^\dagger U \phi.
\end{eqnarray}
With $U^\dagger U U =d$ proved in \ref{sec:udu}, we have
\begin{eqnarray}
  (\phi^\dagger)' \phi = \phi\dagger \phi.
\end{eqnarray}

\section{ } 
Define $\phi^c=i\sigma^2 \phi^*$,
then we have the SU(2) transfromation
\begin{eqnarray}
  \phi^c \to (\phi^c)' &=& (\phi')^c = i\sigma^2 (\phi')^* \nonumber \\
  &=& i\sigma^2 U^* \phi^*\nonumber \\
  &=& i\sigma^2 U^* (-i\sigma^2) i\sigma^2 \phi^*\nonumber \\
  &=&i\sigma^2 U^* (-i\sigma^2) \phi^c,
\end{eqnarray}
where we used $-i\sigma^2 i\sigma^2=1$.
Now we need to prove
\begin{eqnarray}
  i\sigma^2 U^* (-i\sigma^2)= \sigma^2 U^* \sigma^2=U.
\end{eqnarray}
Now we look at the infinitesimal transfromations
\begin{eqnarray}
  U &=& e^{ i(T^1 \theta_1 + T^2 \theta_2 + T^3 \theta_3) }\\
&=& 1 + i(T^1 \theta_1 + T^2 \theta_2 + T^3 \theta_3) + {\cal O}( (T^k \theta_k)^2)\\
U^* &=& e^{ -i (T^1* \theta_1 + T^2* \theta_2 + T^3* \theta_3) }\\
&=& 1 - i(T^1* \theta_1 + T^2* \theta_2 + T^3* \theta_3) + {\cal O}( (T^{k*} \theta_k)^2),
\end{eqnarray}
and see we need only to prove
\begin{eqnarray}
  \sigma^2 T^{k*} \sigma^2=-T^k,
\end{eqnarray}
or
\begin{eqnarray}
  \sigma^2 \sigma^{k*} \sigma^2=-\sigma^k.
\end{eqnarray}
One can directly write out
\begin{eqnarray}
  \sigma^2 \sigma^{1*} \sigma^2=
  \begin{pmatrix}
    0&-i\\i&0
  \end{pmatrix}
  \begin{pmatrix}
    0&1\\1&0
  \end{pmatrix}
  \begin{pmatrix}
    0&-i\\i&0
  \end{pmatrix}
  =  \begin{pmatrix}
    0&-1\\-1&0
  \end{pmatrix}
  =-\sigma^1,
\end{eqnarray}
\begin{eqnarray}
  \sigma^2 \sigma^{2*} \sigma^2=
  \begin{pmatrix}
    0&-i\\i&0
  \end{pmatrix}
  \begin{pmatrix}
    0&-i\\i&0
  \end{pmatrix}
  \begin{pmatrix}
    0&-i\\i&0
  \end{pmatrix}
  =  \begin{pmatrix}
    0&i\\-i&0
  \end{pmatrix}
  =-\sigma^2,
\end{eqnarray}
\begin{eqnarray}
  \sigma^2 \sigma^{3*} \sigma^2=
  \begin{pmatrix}
    0&-i\\i&0
  \end{pmatrix}
  \begin{pmatrix}
    1&0\\0&-1
  \end{pmatrix}
  \begin{pmatrix}
    0&-i\\i&0
  \end{pmatrix}
  =  \begin{pmatrix}
    -1&0\\0&1
  \end{pmatrix}
  =-\sigma^3.
\end{eqnarray}
As a summary, we now have
\begin{eqnarray}
  \phi^c \to (\phi^c)'=U(\theta_1,\theta_2,\theta_3) \phi^c.
\end{eqnarray}


\end{document}
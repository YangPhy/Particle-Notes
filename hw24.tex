\documentclass[12pt]{article}
\usepackage{amsmath,graphicx,color,epsfig,physics}
\usepackage{float}
\usepackage{subfigure}
\usepackage{slashed}
\usepackage{color}
\usepackage{multirow}
\usepackage{feynmp}
\textheight=9.5in \voffset=-1.0in \textwidth=6.5in \hoffset=-0.5in
\parskip=0pt
\def\del{{\partial}}
\def\dgr{\dagger}
\def\eps{\epsilon}
\def\lmd{\lambda}
\def\th{\theta}
\begin{document}

\begin{center}
{\large\bf HW24 for Advanced Particle Physics} \\
  
\end{center}

\vskip 0.2 in

Dear students,\\

Please let me send you my last homework today (04.10), so that you will
have more than a week to send me your report by my last lecture on
Apr/19.  By the last lecture and the homework 23, I introduced all the
most basic knowledge and technique that I think are useful for your
research in particle physics.  Materials were chosen based on my own
experiences as a particle physics researchers in the past 40 years.
What I taught in my lectures (and in hw01 to hw23) are all those
materials which were useful for my own research at some stage of my
past research.  I hope that they will be useful for you as well.

There is another reason for selecting certain materials in my lectures.
You may have felt that I emphasized many times that before the EW
symmetry breaking, all particles in the SM (except the Higgs boson)
are massless, and hence massless representation for fermions (the
Weyl representation) and massless gauge bosons are more fundamental
than Dirac fermions and massive weak bosons which we observe.  It is
because, in my opinion, the particular representations of the SM
fermions, the sextet Q, two triplets uR and dR, one doublet L, and
on singlet lR, and their very particular values of the hypercharge,
1/6, 2/3, -1/3, -1/2, and -1, are the most important hint of physics
beyond the SM.  I planned my lectures in such a way that in my last
lecture and hw, I can explain the classical idea of SU(5) model by
using only those materials which were introduced during the course
of my lectures.  Whether it is true or not, I am sure that it is the
most important hint that the SM of particle physics gave us.

                               *****

In the last lecture, I will make use of all what you learned from my
past lectures.  In addition, I try to explain how radiative corrections
affect effective gauge couplings in a heuristic manner.  You should
study quantization of gauge fields, technique to calculate quantum
(radiative) corrections, renormalization of UV sensitive terms, and
the renormalization scale dependence of the effective couplings,
which simply follows from the requirement that physics (as we observe)
should not depend on how we renormalize our QFT.

Before introducing SU(5) theory, let us make a preparation.
Out motivation is try to find a possibility that the fundamental
fermions of the SM, in each generation,

1) Q, uR, dR, L, lR

which we may write 6+3+3+2+1=15, can be components of bigger multiplets,
just like uL and dL are upper and lower components of Q.  In order to
study such possibility, we should note that all components of such
bigger multiplet should have exactly the same Lorentz transformation
properties.  Among the 5 multiplets of (1), Q and L transform as left
handed Weyl spinor, while uR, dR, lR transform as right-handed Weyl
spinor.  Since we already learn that the charge conjugate of
right-handed Weyl spinor transform as left-handed Weyl-spinor, let
us first write down the SM Lagrangian in terms of left-handed Weyl
spinors only:

2) Q, uR^c, dR^c, L, lR^c

The fermions appear only in two parts of the SM Lagrangian, which are

3) L_fermion
= Q^\dagger  iD_\mu \sigma_-^\mu Q
+ uR^\dagger iD_\mu \sigma_+^\mu uR
+ dR^\dagger iD_\mu \sigma_+^\mu dR
+ L^\dagger  iD_\mu \sigma_-^\mu L
+ lR^\dagger iD_\mu \sigma_+^\mu lR

and

4) L_Yukawa
= y^u Q^\dagger uR \phi^c
+ y^d Q^\dagger dR \phi
+ y^l L^\dagger lR \phi
+ h.c.

where I dropped the neutrino mass term, by going back in time to 1973.

hw24-1: Please write (3) and (4) solely in terms of left-handed Weyl
fermions, (2).

Let me try, e.g., for the 2nd term in (3), and for the 3rd term in (4)

5) uR^\dagger iD_\mu \sigma_+^\mu uR
= (uR^\dagger iD_\mu \sigma_+^\mu uR)^T
= -(iD_\mu \sigma_+^\mu uR)^T uR^*
= -((i\del_\mu -gT^aA^a_\mu)\sigma_+^\mu uR)^T uR^*
= (uR)^T (i\del_\mu + g(T^a)^T A^a_\mu) (\sigma_+^\mu)^T uR^*
= (uR)^T (-i\sigma^2)(i\del_\mu + g(T^a)^T A^a_\mu)
          (i\sigma^2) (\sigma_+^\mu)^T (-i\sigma^2)(i\sigma^2) uR^*
= (uR^c)^\dagger (i\del_\mu + g(T^a)^T A^a_\mu) \sigma_-^\mu uR^c
= (uR^c)^\dagger iD_\mu^* \sigma_-^\mu uR^c

I hope you can follow all the steps by now.  If not, please go back
to the relevant hw, and repeat it.  In the last step of (5), I
introduce complex conjugate of the covariant derivative,

6a) D_\mu   = \del_\mu + ig  T^a    A^a_\mu
6b) D_\mu^* = \del_\mu - ig (T^a)^* A^a_\mu

Although I wrote only SU(3)_c coupling part, the same applies for both
SU(2)_L and U(1)_Y couplings.  All the charges should change the sign,
and the SU(3) and SU(2) generators should be replaced by complex
conjugates (or transpose).  Most importantly, the chiral \sigma
four vectors  are now all left-handed.

The Yukawa interactions can be expressed the same way.  For instance,

7) y^l  L^\dagger lR \phi
=  y^l (L^\dagger lR \phi)^T
=  -y^l (lR)^T (L^\dagger \phi)^T
=  -y^l (lR)^T \phi^T L^*
=  -y^l (lR)^T (-i\sigma^2)(i\sigma^2) \phi^T L^*
=  -y^l (lR^c)^\dagger (i\sigma^2) \phi^T L^*
=  y^l (lR^c)^\dagger (-i\sigma^2) \phi^T L^*

The Yukawa interaction now looks like the Majorana couplings between
two left-handed Weyl fermions.  The Hermetian conjugate of (7) gives

8) y^l  L^\dagger lR \phi + h.c.
=  y^l (lR^c)^\dagger (-i\sigma^2) \phi^T L^*  + h.c.
=  y^l    (lR^c)^\dagger (-i\sigma^2) \phi^T L^*
+ (y^l)^* (lR^c)^T        (i\sigma^2) \phi^\dagger L

Although the appearance of (i\sigma^2) between two left-handed Weyl
fermions look like Majorana couplings, they are still Dirac couplings
because the two left-handed Weyl fermions are different particles.

The only lesson I would like you to learn from the above hw24-1
is that we can express the whole SM Lagrangian in terms of left-handed
(or right-handed) Weyl fermions only.

Now we are read to introduce the SU(5) model of Georgi-Glashow (1973).

==================================
SU(5) model (original: 2007.10.12)
==================================

In the SU(5) theory, we can write down the covariant derivative as

9) D_\mu = \del_\mu + i g T^a A^a_\mu

where the superscript a runs from 1 to 24.

Let us identify T^1 to T^8 as the SU(3) generators in the
(i,j)=(1,2,3) corner of the 5*5 matrix.

Let us identify T^9 to T^11 as the SU(2) generators in the
(i,j)=(4,5) corner of the matrix.

Let us identify T^12 as the hypercharge generator.

With these identifications, 12 out of 24 gauge bosons of the SU(5)
theory are identified as the SM gauge bosons:

10a) A^1_\mu to A^8_\mu are gluons
10b) A^9_\mu=W^1_\mu, A^10_\mu=W^2_\mu, A^11_\mu=W^3_\mu
10c) A^12_\mu=B_\mu

When we compare the SU(5) covariant derivative with that of the SM,
we immediately notice the unification condition

11) g_s = g_W = g

but a care is needed for the hypercharge coupling g_Y.

This is because the hypercharge coupling is not normalized in the SM,
being the U(1) charge.

In the SU(5) theory, T^12 is one of the 24 generators.  All the 24
generators satisfy

12a) Tr( T^a ) = 0
12b) Tr( T^a T^b ) = (1/2) \delta^{ab}

The condition (12b) determines the normalization N of the generator
T^12,

13) T^12 = N \diag(1/3, 1/3, 1/3, -1/2, -1/2)

hw24-2: Obtain N from the condition (12b).

hw24-3: Obtain all 24 generators as 5x5 Hermetian matrices that satisfy
(12a) and (12b).

hint: T^1 to T^8 are QCD generators, so

14a) T^1 = 1/2 [ 0  1 0 0 0 ]
               [ 1  0 0 0 0 ]
               [ 0  0 0 0 0 ]
               [ 0  0 0 0 0 ]
               [ 0  0 0 0 0 ]
14b) T^2 = 1/2 [ 0 -i 0 0 0 ]
               [ i  0 0 0 0 ]
               [ 0  0 0 0 0 ]
               [ 0  0 0 0 0 ]
               [ 0  0 0 0 0 ]
14c) T^3 = 1/2 [ 1  0 0 0 0 ]
               [ 0 -1 0 0 0 ]
               [ 0  0 0 0 0 ]
               [ 0  0 0 0 0 ]
               [ 0  0 0 0 0 ]
14d) T^4 = 1/2 [ 0 0  1 0 0 ]
               [ 0 0  0 0 0 ]
               [ 1 0  0 0 0 ]
               [ 0 0  0 0 0 ]
               [ 0 0  0 0 0 ]
14e) T^5 = 1/2 [ 0 0 -i 0 0 ]
               [ 0 0  0 0 0 ]
               [ i 0  0 0 0 ]
               [ 0 0  0 0 0 ]
               [ 0 0  0 0 0 ]
14f) T^6 =
14g) T^7 =
14h) T^8 = 1/2\rt{3} [ 1 0  0 0 0 ]
                     [ 0 1  0 0 0 ]
                     [ 0 0 -2 0 0 ]
                     [ 0 0  0 0 0 ]
                     [ 0 0  0 0 0 ]

T^9 to T^11 are SU(2) generators, and hence

14i) T^9 = 1/2 [ 0 0 0 0  0 ]
               [ 0 0 0 0  0 ]
               [ 0 0 0 0  0 ]
               [ 0 0 0 0  1 ]
               [ 0 0 0 1  0 ]
14j) T^10 =
14k) T^11 =

T^12 is obtained as (5) and hw24-2.

T^13 to T^24 can be obtained simply as the Hermetian matrices
that have components in i=1-3 and j=4,5, or i=4,5 and j=1-3.

14l) T^13 = 1/2 [ 0 0 0  1 0 ]
                [ 0 0 0  0 0 ]
                [ 0 0 0  0 0 ]
                [ 1 0 0  0 0 ]
                [ 0 0 0  0 0 ]
14m) T^14 = 1/2 [ 0 0 0 -i 0 ]
                [ 0 0 0  0 0 ]
                [ 0 0 0  0 0 ]
                [ i 0 0  0 0 ]
                [ 0 0 0  0 0 ]
14n) T^15 = 1/2 [ 0 0 0  0 0 ]
                [ 0 0 0  1 0 ]
                [ 0 0 0  0 0 ]
                [ 0 1 0  0 0 ]
                [ 0 0 0  0 0 ]
14o) T^16 =
14p) T^17 =
14q) T^18 =
14r) T^19 = 1/2 [ 0 0 0 0  1 ]
                [ 0 0 0 0  0 ]
                [ 0 0 0 0  0 ]
                [ 0 0 0 0  0 ]
                [ 1 0 0 0  0 ]
14s) T^20 = 1/2 [ 0 0 0 0 -i ]
                [ 0 0 0 0  0 ]
                [ 0 0 0 0  0 ]
                [ 0 0 0 0  0 ]
                [ i 0 0 0  0 ]
14t) T^21 =
14u) T^22 =
14v) T^23 =
14w) T^24 = 1/2 [ 0 0 0 0  0 ]
                [ 0 0 0 0  0 ]
                [ 0 0 0 0 -i ]
                [ 0 0 0 0  0 ]
                [ 0 0 i 0  0 ]

Since I will later use the above generator numbers for our SU(5) gauge
boson numbers, A^1 to A^24, please follow my notation for all the 24
matrices.

hw24-4: Please identify 4 diagonal generators among 24 generators.
Please confirm that 2 of them are SU(3) QCD generators, 1 of them
is the SU(2) generator, and the last one is T^12.  Please show that
the number of diagonal generators is n-1 for SU(n).  The number of
diagonal generators is called `rank' of the Group.  The rank of U(1)
group is 1.  The rank of the SM and SU(5) are both 4.

Because the diagonal generators give neutral currents, the rank of the
group gives the number of neutral currents.

Because the SM has 4 neutral currents (2 of which are gluons), the
unifying group should at least have rank 4.  The SU(5) model is hence
the smallest possible unification group, which has no additional
neutral current.

Let us choose the 5^* representation of SU(5) as

15) 5^* = ( (d_R^c)^1, (d_R^c)^2, (d_R^c)^3, \l_L, -\nu_L )^T
        = (      F*^1,      F*^2,      F*^3, F*^4,   F*^5 )^T

which should transform under SU(5) as a 5^* representation,

16a) F  > F'  = U   F
16b) F* > F*' = U^* F*

where I symbolically write F and F* for the column vector with 5
components.

Please note the hypercharge of the above 5 components can be expressed
as a diagonal 5x5 matrix

17) Y F*= \diag(1/3, 1/3, 1/3, -1/2, -1/2) F*

Please observe that the above Y matrix is traceless!!!  This is the
reason why the Hypercharge gauge boson (the B boson) can be interpreted
as one of the SU(5) gauge bosons.

Please observe that all 5 components of (15) transform as left-handed
Weyl fermions under the Lorentz transformation.  Therefore, they
transform as a 5-plet under the Lorentz transformation.  This is why
I spent lots of time explaining the Lorentz transformation of the
Weyl fermions and that of their charge conjugates.

Please note that since (\nu_L, \l_L)^T is an SU(2) doublet,
i\sigma^2 (\nu_L, \l_L)^T = (\l_L, -\nu_L)^T transforms as an SU(2)
anti-doublet, or as 2^*.  Please remember how we obtain an SU(2)
doublet Higgs boson, \phi^c, from the SU(2) doublet Higgs boson \phi.
This tells that a doublet (\psi^c)^* = -i\sigma^2 \phi, transform as
an anti-doublet, or as 2^*.

hw24-5: Please confirm that the 1-3 components of F* transform as 3^*
of SU(3), and the 4-5 components of F* transform as 2^* of SU(2).

For the fundamental n representation of SU(n), I denote
f_i (i=1,2,...,n) for the fundamental representation, and
f^i (i=1,2,...,n) for the complex conjugate, or n^* representation
(without putting a bar on f).  For instance, the quark fields may be
denoted as q^i (i=1,2,3), and the anti-quark fields may be denoted
as qbar_i (i=1,2,3).  In the tensor representation, the lowering of
indices changes n to n^* representations, and hence we don't need
to put a bar: q_i transform as n^* if q^i transforms as n.

The transformation laws of (16a) and (16b) are now written as

16a) F  > F'  = U   F   => 18a) F^i > F^i' = U^i_j F^j
16b) F* > F*' = U^* F*  => 18b) F_i > F_i' = U_i^j F_j

The transformation tensor/matrix are related as

19a) U^i_j =  U_{ij}
19b) U_i^j = (U^*)_{ij} = (U^\dgr)_{ji} = (U^{-1})_{ji}

which satisfy

20a) U^i_j U_k^j  = \delta^i_k
20b) U     U^\dgr = 1

hw24-6: Please confirm the relationship between the matrix representation
and tensor representation as given in eqs.(16)-(20).

We can now identify the relation between the hyper-charge coupling
g_Y in the SM and the SU(5) coupling g as follows:

21) g T^12 B_\mu = g_Y Y B_\mu

or, by using the normalization factor N of hw24-2.

22) N g = g_Y

hw24-7: Please obtain the SU(5) value of the weak mixing angle
as follows:

23a) \tan\theta_W = g_Y/g_W = (N g)/g = N
23b) \sin^2\theta_W = 1 -\cos^2\theta_W
                    = 1 -1/(1+\tan^2\theta_W)
                    = 1 -1/(1+N^2)

hw24-8: Please compare the ratio of the 3 couplings in the
SU(5) theory

24) g_s^2 : g_W^2 : g_Y^2 = g^2 : g^2 : (N g)^2
                          = 1   : 1   : N^2

with the observed values from the PDG tables

25) \alpha_s : \alpha/\sin^2\theta_W : \alpha/\cos^2\theta_W
    = 1 : x : xy

Please observe that x is significantly smaller than 1, and y is
significantly smaller than N^2.  Therefore, the SU(5) theory does
not agree with the observation at this level.

Nevertheless, Georgi and Glashow were excited in 1973 to find that
the quantization of the hypercharge follows from the tracelessness
of the SU(5) generator, (5).

Most importantly, they found out that the mysterious factor of the 1/3
unit of quark charges is a direct consequence of the fact that quarks
are color-triplets, once quarks and leptons are in a same multiplet.
Even more, they found that the right-handed down quark should be SU(2)
singlet if the left-handed leptons form a SU(2) doublet.

hw24-9: Please observe that for the 5-plet state, F* (7), its i=1-3
components transform as 3^* of SU(3), does NOT transform under SU(2)
(and hence they are SU(2) singlet), whereas the i=4-5 components trans
form as 2^* of SU(2), whereas they don't transform under SU(3)
(color singlet).

They further found that the remaining 15-5=10 fermions of the SM, i.e.,
the left-handed quark doublet (6), the right-handed up quark (3),
and the right-handed lepton (1) makes up the 10 plet of SU(5).

10 plet has two SU(5) indices

26) T_[ij]

where ij are anti-symmetric.  It can hence be expressed as 5x5
anti-symmetric matrix, which has just 10 components.

hw24-10: Please confirm that there are 10 independent components in
the 5x5 anti-symmetric matrix.

Please note that F^i and T_ij transform as

27a) T_ij -> T_kl = U_k^i U_l^j T_ij
27b) F^i  -> F^j  = U^j_i F^i

where

28a) U_i^j = [U]_ij
28b) U^i_j = [U^*]_ij

in terms of the 5*5 unitary matrix U.

Please note that it is not trivial to express the transformation
(27a) by using the matrix notation.  I find

29a) [T]  -> [T']  = [U] [T] [U^T]
29b) [F*] -> [F*'] = [U^*] [F*]

in the matrix representations.

hw24-11: Please prove (29).

hw24-12: Show that

30) [T_14, T_24, T_34] = [u_L_1, u_L_2, u_L_3]
    [T_15, T_25, T_35]   [d_L_1, d_L_2, d_L_3]

transform as SU(3) triplet for the (1,2,3) indices,
and as SU(2) doublet for the (4,5) indices.

Please also show that

31) [T_12, T_23, T_31] = [u_R^c^3, u_R^c^1, u_R^c^2]

by using the fact that

32) \epsilon_{ijk} q_i q_j q_k

is a color SU(3) singlet when i,j,k are the SU(3) indices.

hw24-13: Prove that (32) is SU(3) invariant by using the property of
the SU(n) transformation matrix, det[U]=1.

It then follows that

33) \epsilon_{ijk} q_i q_j

transforms as q^k (anti-quark).  Finally, we can identify

34) T_45 = l_R^c

hw24-14: Please calculate the hypercharge of all the components of
T_ij, and show that the above identification agrees with the SM.

In this way, all the 15 components of the SM generation are identifies
with 5^* and 10 plets of the SU(5).

Let us now assume that the remaining 12 gauge bosons of the SU(5)
theory, A^13 to A^24 should all be heavy.  Because we exhausts all
the block-diagonal generators by T^1 to T^12, all the remaining
generators have only block-off-diagonal components.  We may write

35) \Sum_{a=13 to 24} T^a A^a = 1/\sqrt{2}
                              * [ 0   0   0   X Y ]
                                [ 0   0   0   X Y ]
                                [ 0   0   0   X Y ]
                                [ X^* X^* X^* 0 0 ]
                                [ Y^* Y^* Y^* 0 0 ]

This is very similar to the SU(2) case, where we had

36) \Sum_{a=1,2} T^a W^a = 1/\sqrt{2} * [ 0   W^+ ]
                                        [ W^- 0   ]

Just like we can express W^+ and W^- in terms of the real gauge bosons
W^1 and W^2, we can express X^{-4/3} and Y^{-1/3} in terms of A^a
(a=13,...,24).

hw24-15: Please express the charged gauge bosons X^{-4/3} and Y^{-1/3}
in terms of the 12 neutral gauge bosons, A^13 to A^24.

********************************************************************
This is my last question, for which I expect to receive your report.
Please read the following parts, which I gave to my students at KEK
to play.  You may enjoy making many plots, and may want to learn how
to evaluate quantum corrections by yourself, which was the subject
of my next advanced course at KEK.
********************************************************************

Let us now assume that all X and Y have a common mass M.  Because at
scales far above the breakdown of SU(2), we should assume the SU(2)
invariance to hold, and hence this assumption of the same mass for
X and Y are necessary condition for the theory to contain the SM.

Below the mass scale of M, X and Y bosons do not contribute to radiative
corrections (decoupling of heavy particles).  Only the light bosons
(such as gluons, W's and B, Higgs, and quarks and leptons) contribute
to radiative corrections.

For instances, the gluon propagator receives corrections from gluons
and quarks, while the weak-boson propagator receives corrections
from the electroweak bosons, quarks and leptons.

Therefore, even though the couplings are the same in the tree-level,
the radiative corrections at the scale \mu << M will be very different:

37a) q-q scattering
                      q|    |q       q|         |q
                       |    |         |         |
                 =    g|----|g   +   g|---[A]---|g
                       | g  |         | g     g |
                      q|    |q       q|         |q


37b) nu-lL scattering
                     nu|    |lL     nu|         |lL
                       |    |         |         |
                 =    g|----|g   +   g|---[B]---|g
                       | W  |         | W     W |
                     lL|    |nu     lL|         |nu


37c) lR-lR scattering
                     lR|    |lR     lR|         |lR
                       |    |         |         |
                 =    g|----|g   +   g|---[C]---|g
                       | B  |         | B     B |
                     lR|    |lR     lR|         |lR


In the tree-level, the first term in the r.h.s. of eqs.(37a,b,c),
all the above 3 scattering amplitudes have the same magnitude in
the SU(5) theory.

However, the radiative corrections at the scale \mu << M will be very
different.  In [A], only gluons and quarks can contribute.  In [B],
only electroweak gauge bosons and left-handed quarks and leptons can
contribute.  In [C], no gauge bosons can contribute while all the
quarks and leptons will contribute.

The scattering amplitudes at the scale \mu can be expressed as the sum
of the tree-level and the loop-level amplitudes, the first and the
second terms in the r.h.s. of eqs.(37a,b,c), and they can be expressed
as `effective couplings' for each scattering at the scale \mu:

38a) g_3^2(\mu) = g^2 + g^2*[A] + ...
38b) g_2^2(\mu) = g^2 + g^2*[B] + ...
38c) g_1^2(\mu) = g^2 + g^2*[C] + ...

where

39a) g_3^2(\mu) = g_s^2(\mu)     = 4\pi\alpha_s(\mu)
39b) g_2^2(\mu) = g_W^2(\mu)     = 4\pi\alpha_W(\mu)
39c) g_1^2(\mu) = g_Y^2(\mu)/N^2 = 4\pi\alpha_Y(\mu)/N^2

The terms A, B, C, are obtained from the loop corrections in the
diagrams in (37a,b,c), respectively.  We find

40a) A = g^2/4pi^2 * b_3 * \ln(M/\mu) + ...
40b) B = g^2/4pi^2 * b_2 * \ln(M/\mu) + ...
40c) C = g^2/4pi^2 * b_1 * \ln(M/\mu) + ...

The radiative corrections are renormalized such that they vanish at
the scale \mu=M, recovering the unification.  Indeed, above the
unification scale, \mu > M, all the SU(5) gauge bosons as well as
scalar/fermion multiplets contribute to the loops, making the
coefficients the same (b_5) for all the processes.

At \mu << M, however, because all the particles with mass \sim M
decouple, the coefficients b_3,2,1 reflect the number of light
bosons and fermions which couple to each gauge boson.

By inserting (40) back into (38), we find

41a) 4\pi^2/g_3^2(\mu) = 4\pi^2/g^2 - b_3 * \ln(M/\mu) + ...
41b) 4\pi^2/g_2^2(\mu) = 4\pi^2/g^2 - b_2 * \ln(M/\mu) + ...
41c) 4\pi^2/g_1^2(\mu) = 4\pi^2/g^2 - b_1 * \ln(M/\mu) + ...

Can you derive (41) from (38)?

If only the SM particles are light as compared to M, we obtain

42a) b_3 = (11/6)*T_A -(4/3)*T_F*nG
         =  11/2          -(2/3)*nG
42b) b_2 = (11/6)*T_A -(4/3)*T_F*nG -(1/12)*nH
         =  11/3          -(2/3)*nG -(1/12)*nH
42c) b_1 =            -(4/3)*T_F*nG -(1/20)*nH
         =                -(2/3)*nG -(1/20)*nH

Here T_R = T(R) are group theory constants defined by

43) Tr[ T^a(R) T^b(R) ] = T(R) \delta^{ab}

for the generators in the R-representation.  R=F is fundamental
representation, for which T(F)=T_F=1/2 for all SU(n).  R=A is
adjoint, for which T(A)=n for SU(n).  T(A)=0 for U(1).

In eqs.(42a,b), the first term is the contribution from the gauge
bosons in the loop, the second term which is proportional to T_F is
the contribution from the fermions, the sum of quarks and leptons,
and the last term is the contribution from the scalar bosons, the
Higgs doublet.  nG is the number of quark and lepton generations,
nH is the number of Higgs doublets.

It is instructive to note how quarks and leptons contribute to b_n:

44a) b_3: -(2/3)*T_F each from up-quark and down-quark
44b) b_2: -1*T_F from the quark doublet Q, -(1/3)*T_F from L.
44c) b_1: -(1/5)*[6*(1/6)^2+3*(2/3)^2+3*(1/3)^2+2*(1/2)^2+1^2]=-2/3

Because quarks and leptons of each generation have the same (zero)
mass for the complete SU(5) multiplets, 5^* and 10, their contri-
bution to the running of the gauge couplings are identical.

It will be fun to confirm that, b_3=b_2=b_1=-(2/3)*nG, if there
are nG generations of 15 fermions.

It is the contributions of split representations, where one SU(5)
multiplet splits into particles of mass M and massless particles,
which make the radiative corrections different for each coupling.

The first terms in (42a,b) comes from the gauge boson sector, where
the SU(5) multiplet of 24 splits into 12 massive bosons (X,Y,X^*,Y^*)
and 12 massless gauge bosons (8 gluons, 3 W's, 1 B). Only the massless
gauge bosons contribute to the radiative corrections at \mu << M.
The sign of the first term for SU(3) and SU(2) couplings show the
asymptotic freedom [the couplings become small at high energies].
Because of the splitting of the SU(5) multiplet (with 24 gauge bosons)
into massive (12) and massless (12) bosons, the gauge boson
contributions to the radiative corrections differ for the 3 processes.

The last terms in (42b,c) comes from the Higgs boson sector, where
the SU(5) multiplet of 5 splits into 3 massive bosons and 2 massless
bosons (the Higgs boson of the SM).  nH is the number of the Higgs
doublets.  nH=1 for the minimum SM.  Again, it is the splitting of
the SU(5) multiplet of Higgs bosons into massive (3) and massless (2)
bosons, the Higgs boson contribution to the radiative corrections
differ for the 3 processes.

It is easy to show that the 3 equations of (41) can be solved as:

45a) \sin^2\theta_W(\mu)

          3(b_3-b_2)              5(b_2-b_1)          \alpha(\mu)
   = --------------------- + --------------------- * -------------
     5(b_3-b_1)+3(b_3-b_2)   5(b_3-b_1)+3(b_3-b_2)   \alpha_3(\mu)

45b) \ln(M/\mu)

               8               3     \pi            \pi
   = --------------------- * [ - * ----------- - ------------- ]
     3(b_3-b_2)+5(b_3-b_1)     8   \alpha(\mu)   \alpha_3(\mu)

Eq.(45a) gives the SU(5) prediction of the weak mixing angle \theta_W,
e.g. at the \mu=mZ scale, once we know \alpha(mZ) and \alpha_3(mZ).
Eq.(45b) gives the GUT scale M, in terms of the couplings measured
at \mu=mZ.

Please use as inputs the observed value of mZ (in the PDG tables) and

46a) 1/\alpha(mZ) = 128.12
46b) 1/\alpha_3(mZ) = 9.09

I give the numerical values in (46), because it is not easy to find
them in the reviews of the PDG book.  For advanced students, the above
values are the measured mean values of the running coupling constants
in the MSbar scheme, including the top-quark contribution.  Their
magnitudes differ significantly from those which appear in the SM
Lagrangian in the weak boson mass scale, since the latter couplings
are obtained in the effective theory without the top quark (the
effective 5-quark theory).  Because in the MSbar scheme the heavy
quarks do not decouple from the beta function, the magnitude of the
MSbar couplings with or without the top-quark differ significantly
from each other (their magnitudes are determined by the requirement
that the full theory with the top quark, and the effective theory
without the top quark give quantitatively the same predictions for
all the low energy phenomena below the top quark mass scale).

If one uses the MSbar coupling values without the top-quark
contribution at the scale \mu = mZ, in the following analysis, one
should use the 5-quark RG equations in the region mZ<\mu<mt, and
then the 6-quark RG equation at mt<\mu<M.  This is an unnecessary
complication, if we use the 6-quark MSbar couplings in the whole range.
In order to obtain the same physics outputs, the coupling values
should be shifted as above.

You can now find the SU(5) predictions of \sin^2\theta_W(\mz) and M
with the above inputs (46), for

47) nH = 0, 1 (minimum SM), 2, 3, 4, 5, 6, 7, 8.

If my memory is correct, the SU(5) prediction for the mixing angle
agrees with the observation

48) \sin^2\theta_W(mZ) = 0.2314

when nH \sim 7.   Please make a table of your results as

49)  nH     \sin^2\theta_W(mZ)   M         1/\alpha_GUT    \tau_p

   0     ...                   ... (GeV)   ...          ... (year)
   1     ...                   ... (GeV)   ...          ... (year)
   2     ...                   ... (GeV)   ...          ... (year)

by estimating the proton decay life time as follows:

50a) \Gamma = 1/(2m) 1/2 |(g^2/2)/M^2|^2 m^6 1/8\pi
            = g^4/128\pi m^5/M^4
            = \pi/8 (\alpha_GUT)^2 m^5/M^4

50b) \tau = 1/Gamma

with m=m_proton.  g^2=g^2(\mu=M) should be estimated from eq.(51):

51a) 4\pi^2/g^2     = 4\pi^2/g_3^2(mZ) + b_3 * \ln(M/mZ)
51b) \pi/\alpha_GUT = \pi/\alpha_3(mZ) + b_3 * \ln(M/mZ)

Please note that the above estimate for the proton decay width (50a)
is extremely rough, where I used an order of magnitude estimate for
the proton to lepton plus meson matrix elements as m(proton)^3.
There have been careful estimates of the perturbative renormalization
of the operator (from the scale M down to the scale m(proton)),
and the matrix elements are now evaluated by Lattice QCD.

Now, please work out the running of the 3 couplings in the
opposite direction.  Please start from the `observed' values

52) 1/\alpha_1(mZ) = 59.08
    1/\alpha_2(mZ) = 29.65
    1/\alpha_3(mZ) = 9.09

and examine the running of the 3 gauge couplings at \mu >> mZ.
In the leading order, the solution of the RG equations, (41a,b,c),
is linear in \ln\mu for 1/\alpha_i(\mu).  Therefore, please make
a plot by taking the \log scale in the horizontal direction,
and 1/\alpha_i(\mu) in the vertical direction.

How the curves look like, say, for nH=1,3,5,7 ?

Let us repeat this last analysis in the supersymmetric SM.
The \beta functions of (34) change as follows:

53a) b_3 = (11/6)*T_A -(1/3)*T_A -(4/3+2/3)*T_F*nG
         =  (7/6)*3                      -2*T_F*nG
53b) b_2 = (11/6)*T_A -(1/3)*T_A -(4/3+2/3)*T_F*nG -(1/12+2/12)*nH
         =  (7/6)*2                      -2*T_F*nG       -(1/4)*nH
53c) b_1 =                               -2*T_F*nG      -(3/20)*nH

It is very easy to obtain the MSSM result (53) from the SM one (42).
In the expression for b_3, -(1/3)*T_A is the gluino contribution,
which is just like the standard quark contribution with the adjoint
color factor T_A instead of T_F.

For the squark contribution, it suffices to know that a complex scalar
boson contributes just half of the corresponding Weyl fermion.  Hence
the quark and lepton contributions are multiplied by (1+1/2)=3/2.
The Higgs contributions are multiplied by (1+2)=3, because the
Higgsino contribution is twice as large as the Higgs contribution,
by the same token.

In addition, we note that only even number of the Higgs doublets are
acceptable for the supersymmetric SM.  [This is essentially because
we cannot make a supersymmetric Lagrangian from \phi and \phi^c.  If
a super-partner of \phi is a left-handed Weyl fermion, then a
super-partner of \phi^c is a right-handed Weyl spinor.  If we have
both left- and right-handed Weyl spinors, we cannot make the Yukawa
interactions supersymmetric.  Therefore, instead of using \phi and
\phi^c to give masses to both up and down quarks, we should introduce
two Higgs doublets \phi_u and \phi_d with opposite Hyper charge, both
having left-handed Weyl spinors, the Higgsinos \tilde{\phi}_u and
\tilde{\phi}_d, respectively.]

You may draw a new figure of 1/\alpha_i(\mu) vs \log\mu, for
the supersymmetric SM, but for nH = 0, 2, 4.  nH=2 is the minimal
SUSY SM (MSSM).  Observe that only the MSSM satisfies the GUT condition.

Finally, please obtain the GUT scale M, and the GUT coupling
1/\alpha_GUT for the MSSM by using the same formula which you used
to make a table for the SM.  How large is the proton lifetime for
the MSSM (nH=2) ?  Please estimate the size (in units of ton's) of
the detector which can observe one proton decay in one year.

This is the last homework of my PT1 lectures.

Best regards,

Kaoru




\end{document}
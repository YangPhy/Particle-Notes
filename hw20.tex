\documentclass[12pt]{article}
\usepackage{amsmath,graphicx,color,epsfig,physics}
\usepackage{float}
\usepackage{subfigure}
\usepackage{slashed}
\usepackage{color}
\usepackage{multirow}
\usepackage{feynmp}
\textheight=9.5in \voffset=-1.0in \textwidth=6.5in \hoffset=-0.5in
\parskip=0pt
\def\del{{\partial}}
\def\dgr{\dagger}
\def\eps{\epsilon}
\def\lmd{\lambda}
\def\th{\theta}
\begin{document}

\begin{center}
{\large\bf HW20 for Advanced Particle Physics} \\
  
\end{center}

\vskip 0.2 in

Dear students,

  Last week, I introduce perturbative QFT methods to compute transition
  amplitudes for 1 > 2 (1 particle to 2 particles) processes, which
  appear in the very first expansion of our S matrix operator

  S = T e^{i \Int d^4x L_{int}(x)}
    = 1 +  i \Int d^4x L_{int}(x)} + ...

  As examples, I plan to show the following three cases,

  H   > tau+ + tau-   ( spin= 0  )
  t   > t    + W^+    ( spin= 1  )
  W^+ > l^+  + \nu_l  ( spin= 1/2)

  Although I gave detailed homework only for W decays, I may come
  back to the other two processes if time allows.

  This week, I proceed to compute transition matrix elements that are
  obtained from the 2'nd expansion of the S matrix exponent,

  T (1/2!) [i\Int d^4x L_\int(x)]^2
  =
  [i\Int d^4x1 L_\int(x1)] [i\Int d^4x2 L_\int(x2)] \Theta(x1^0-x2^0)

  At this order, we have the time-ordering of the operators, and a
  propagator appears.  I show how Feynman introduced his causal
  (Feynman) propagator by using the analyticity of the function in
  the p^0 plane.

  We examine two processes carefully,

  A: \mu \to \numu e \nuebar

  where the 3 body phase space is evaluated carefully, and
  we compare its life-time with the prediction of the SM.

  As a supplement, I mention the two processes:

  A-1: \tau \to \nutau e \nuebar
  A-2: n    \to p      e \nuebar

  briefly.  A-1 shows the scaling law

  \Gamma(\tau \to \nutau e \nuebar)   (m_\tau)^5
  --------------------------------- = ----------
  \Gamma(\mu  \to \numu  e \nuebar)    (m_\mu)^5

  A-2 is to show the importance of phase space suppression
  due to the very small mass difference

  m(n) - m(p) = 1.29MeV

  which is not much larger than m(e)=0.51MeV.

  B: e- e+ \to f fbar

  Here, we introduce the definition of the scattering cross section,
  unitarization of the Z-boson propagator (which gives the normalization
  of the Z-boson decay width which is similar to the W width we calculated
  in the last lecture), and again the d-functions for spin 1 particle
  production and decay that govern the angular distributions.

  If possible, I may try to cover the following two processes:

  C: u dbar \to W^+ \gamma

  which contains the WW\gamma coupling, and a single top quark production
  by the W boson exchange

  D: u b \to d t

  followed by the top quark decay

  E: t \to b l^+ \nu_l.

  They are all very interesting processes, where you can learn a lot by
  calculating the helicity amplitudes explicitly.

  ======================
  A: mu -> numu e nuebar
  ======================

  There are just two terms in L_\int(x) which contribute to this amplitude
  in the lowest (n=2) order of the expansion:

  01) L_\int(x)
  = W^+_\mu(x) \nl(x)^\dgr \sigma_-^\mu (-g_L^{Wnl}) \lL(x)
  + W^-_\mu(x) \lL(x)^\dgr \sigma_-^\mu (-g_L^{Wln}) \nl(x)
  = W^+_\mu(x) J_{Wnl}^\mu(x)
  + W^-_\mu(x) J_{Wln}^\mu(x)

  where I used the generic notation of hw19 for the couplings.  The first
  term give the transition from the initial muon into \numu and (virtual)
  W^- (l=\mu), whereas the second term gives the transition of this
  (virtual) W^- into electron and \nuebar.

  In the above paragraph, I used the expression which became standard
  after Feynman proposed his Feynman's propagator (and all of us
  accepted his method of calculating transition amplitudes), please note
  at this moment that the first operator,

  01a) W^+_\mu(x) \nl(x)^\dgr \sigma_-^\mu (-g_L^{Wnl}) \lL(x)

  can annihilate W^+ and l^-, and create \nu\l at the spacetime point
  x^\mu, whereas the second operator

  01b) W^-_\mu(x) \lL(x)^\dgr \sigma_-^\mu (-g_L^{Wln}) \nl(x)

  can create W^+, l^-, \nubar_l from the vacuum at x^\mu.

hw20-1: Please select (01) from our SM Lagrangian, and explain that no
  other terms in the Lagrangian can contribute to the process A, in the
  second order of perturbation theory.

  Let us parametrize the helicity and momenta of all the external
  particles as:

  02) \mu(p,h) -> e(p1,-1/2) + \nuebar(p2,+/12) + \numu(p3,-1/2)

  where h = +-1/2 gives the Jz eigenvalue of the polarized muon, and the
  helicities of all the final state particles, e, nuebar, numu, are fixed
  in the massless limit.

  I neglect the electron mass

  03)  [m(e)/m(mu)]^2 = (0.51/106)^2 = (1/207)^2 = 2.3*10^{-5}

  in this study.  Note that the present measurement accuracy of the muon
  life time, or that of the Fermi constant G_F is 2 times 10^{-5}, see
  RPP, and hence the electron mass cannot be neglected in precision
  physics.

  The 4-momentum conservation gives

  04a) p = p1 + p2 + p3

  with

  04b) p^2 = M^2 (M=m(\mu)), p1^2 = p2^2 = p3^2 = 0

  The initial and final states are

  05a) |i> = a^\dagger_{mu,      p,   h} |0>
  05b) |f> = a^\dagger_{e,      p1,-1/2}
             b^\dagger_{\nuebar,p2,+1/2}
             a^\dagger_{\numu,  p3,-1/2} |0>

  05c) <f| = <0| a_{\numu,  p3,-1/2}
                 b_{\nuebar,p2,+1/2}
                 a_{e,      p1,-1/2}

  The leading order amplitude is

  06) iT_{fi}
  = <f| S |i>
  = <f| T e^{i\Int d^4x L_int(x)} |i>
  = <f| (1/2!) T (i\Int d^4x L_int(x))^2 |i>         + ...
  = <f| T [i\Int d^4x W^-_\mu(x) J_{Wln}^\mu(x)]
          [i\Int d^4y W^+_\nu(y) J_{Wnl}^\nu(y)] |i> + ...

  where the factor of 1/2 has been eliminated by retaining only the
  J_{Wln} term in L_\int(x), and J_{Wnl} term in L_\int(y).

hw20-2: Please show that (06) gives the correct normalization by using
  the definition of the time ordering:

  07a) T A(x) B(y) = A(x) B(y) \Theta(x^0-y^0)
                   + B(y) A(x) \Theta(y^0-x^0)

  when A(x) and B(x) commutes,

  07b) T A(x) B(y) = A(x) B(y) \Theta(x^0-y^0)
                   - B(y) A(x) \Theta(y^0-x^0)

  when A(c) and B(x) anti-commutes.  Because all terms in L_{int}(x) are
  Lorentz scalar, only the commuting case (07a) appears in the expansion
  of our S matrix.

hint: Please note that the factor of 1/2! is compensated by the fact
  that two contributing terms in L_{int}(x), say, A(x) and B(x), are
  needed to give non-zero amplitudes, where both TA(x)B(y) and TB(x)A(y)
  contributes with the same way.  Schematically, you should prove

  07c) T   L_int(x)    *   L_int(y)
    =  T (A(x) + B(x)) * (A(y) + B(y))
    =    (T A(x)*B(y)) + (T B(x)*A(y))
    =  2 (T A(x)*B(y)).

  Since this is the first time we encounter the t-ordered product of
  field operators, let us show the free field expansions of all the
  fields again:

  08a) W^-_\mu(x)
  = \Sum_h \Int d^3q1/(2pi)^3/2E(q1) \Sum_{h1}
    { a_{W^-:q1,h1}         \eps_\mu(q1,h1)   e^{-iq1x}
    + b_{W^+:q1,h1}^\dagger \eps_\mu(q1,h1)^* e^{ iq1x} }_{q1^0=E(q1)} }

  08b) \Psi(x)_{l=\mu}
  = \Sum_h \Int d^3k/(2pi)^3/2E(k) \Sum_{h'}
    { a_{mu:k,h'}         u(k,h') e^{-ikx}
    + b_{mu:k,h'}^\dagger v(k,h') e^{+ikx} }_{k^0=E(k)} }

  08c) \Psibar(x)_{n=\numu}
  = \Sum_h \Int d^3k3/(2pi)^3/2E(k3)
    { a_{\numu:k3}^\dagger ubar(k1,-1/2) e^{+ik3x}
    + b_{\numu:k3}         vbar(k1,+1/2) e^{-ik3x} }_{k3^0=E(k3)} }

  in L_\int(x), and

  08d) W^+_\mu(y)
  = \Sum_h \Int d^3q2/(2pi)^3/2E(q2) \Sum_{h2}
    { b_{W^+:q2,h2}         \eps_\mu(q2,h2)   e^{-iq2y}
    + a_{W^-:q2,h2}^\dagger \eps_\mu(q2,h2)^* e^{ iq2y} }_{q2^0=E(q2)} }

  08e) \Psibar(y)_{l=e}
  = \Sum_h \Int d^3k1/(2pi)^3/2E(k1) \Sum_{h1}
    { a_{e:k1,h1}^\dagger ubar(k1,h1) e^{+ik1y}
    + b_{e:k1,h1}         vbar(k1,h1) e^{-ik1y} }_{k1^0=E(k1)} }

  08f) \Psi(y)_{n=\nue}
  = \Sum_h \Int d^3k2/(2pi)^3/2E(k2)
    { a_{\nue:k2}         u(k,-1/2) e^{-ik2y}
    + b_{\nue:k2}^\dagger v(k,+1/2) e^{+ik2y} }_{k2^0=E(k2)} }

  in L_\int(y).  Here, I used Dirac fermion notation, in order to avoid
  giving L and R indices for the field operators and u and v spinors.
  Please note that the Dirac conjugates simply gives complex conjugates
  (^\dgr) for uL, uR, vL, vR spinors.

  Now, by making use of the anti-commutators of the fermion operators,

  09a) {a_{mu:k,h'},a^\dg_{mu:p,h}}=(2pi)^3 2E(k)\delta^3(p-k) \d_{h,h'}
  09b) {a_{e:p1,-1/2},a^\dg_{e:k1}}=(2pi)^3 2E(k1)\delta^3(p1-k1)
  09c) {b_{nue:p2,+1/2},b^\dg_{nue:k2}}=(2pi)^3 2E(k2)\delta^3(p2-k2)
  09d) {a_{num:p3,-1/2},a^\dg_{num:k3}}=(2pi)^3 2E(k3)\delta^3(p3-k3)

  the amplitude (07) can be immediately reduced to

  10) iT_{fi}
  = <f| (1/2!) T (i\Int d^4x L_int(x))^2 |i>         + ...
  = <f| T [i\Int d^4x W^-_\mu(x) J_{Wln}^\mu(x)]
          [i\Int d^4y W^+_\nu(y) J_{Wnl}^\nu(y)] |i> + ...
  = \Int d^4x \Int d^4y
    <0| T W^-_\mu(x) W^+_\nu(y) |0>
    \ubar(e:p1) \gamma^\mu (-ig_L^{Wln}) P_L v(nue:p2) e^{-ix(-p1-p2)}
    \ubar(num:p3) \gamma^\nu (-ig_L^{Wnl}) P_L u(mu:p,h) e^{-iy(p-p3)}

hw20-3: Derive (10).  This is just a review of our last homework.

  Please note that the factor of i has been included in each coupling
  factor, in accordance with the rule (the Feynman rule) we introduced
  in the last homework.

  In eq.(10), we have a new term, the propagator, which connects the
  interaction at a spacetime point x^\mu and the other interaction which
  occurs at another spacetime point y^\mu.  The time ordering tells

  11) <0| T W^-_\mu(x) W^+_\nu(y) |0>
  = \Theta(x^0-y^0) <0| W^-_\mu(x) W^+_\nu(y) |0>
  + \Theta(y^0-x^0) <0| W^+_\nu(y) W^-_\mu(x) |0>

  which ensures that the time flows from the right (the initial state at
  t=-\infty) to the left (the final state at t=+\infty).

  The relative sign between the two terms in (11) is plus, because the
  W field is a bosonic operator.

  Now, let us evaluate the propagator (11) by using the free field
  expansions, (08a) and (08b).  The only possible surviving terms are:

  12a) <0| a_{W^-:q1,h1} a_{W^-:q2,h2}^\dagger |0>
     = [a_{W^-:q1,h1}, a_{W^-:q2,h2}^\dagger]
     = (2pi)^3 (2E(q2) \delta^3(q1-q2) \delta_{h1,h2}

  12b) <0| b_{W^+:q2,h2} b_{W^+:q1,h1}^\dagger |0>
     = [b_{W^+:q2,h2}, b_{W^+:q1,h1}^\dagger]
     = (2pi)^3 (2E(q2) \delta^3(q1-q2) \delta_{h1,h2}

hw20-4: Show that only the term (12b) survives in the first term of (11),
  whereas only the term (12a) survives in the second term of (11).

  Let us start from the first term in eq.(11)

  13) \Theta(x^0-y^0) <0| W^-_\mu(x) W^+_\nu(y) |0>
  = \Theta(x^0-y^0) <0| W^-_\mu(x) W^+_\nu(y) |0>
  = \Theta(x^0-y^0)
    \Int d^3q1/(2pi)^32E(q1) e^{-iq1(x-y)}_{q1^0=E(q1)}
    \Sum_{h1} \eps_\mu(q1,h1) \eps_\nu(q1,h1)^*
  = \Theta(x^0-y^0)
    \Int d^3q1/2E(q1) e^{-iq1(x-y)}_{q1^0=E(q1)}
    (-g_{\mu\nu} + q1_\mu q1_\nu/m_W^2)

hw20-5: Derive (13), except for the last part, which is explained below.

  The identity

  14) \Sum_{h=+-1,0} \eps_\mu(q,h) \eps_\nu(q,h)^*
    = (-g_{\mu\nu} + q_\mu q_\nu/m_W^2)

  has been shown in previous homework when q^2=m_W^2 (on-shell).  When
  q^2 \neq m_W^2 (off-shell), the use of the polarization sum (14) is
  called `Unitary gauge'.  We can show that the Unitary gauge propagator
  can reproduce the spin=0 component of off-shell weak bosons in the
  tree level (such as the virtual W contribution to \pi^\pm and K^\pm
  decays into muon and muon-neutrino).

  In order to understand the reason  why eq.(14) gives a consistent
  description of the sum of off-shell vector boson wave functions,
  you need to study quantization of gauge bosons in spontaneously
  broken theories.

  Now, what Feynman showed is that the remaining part of the propagator
  can be expressed simply as (I denote q1 as q below)

  15) \Theta(x^0-y^0)
               d^3 q
      \Int -------------- e^{-iq(x-y)}_{q^0=E(q)}
           (2pi)^3 2E(q)
  =
      \Theta(x^0-y^0)
               d^4 q             i
      \int  ------------  ---------------  e^{-iq(x-y)}
              (2pi)^4      q^2 - M^2 +i0

  It is relatively easy to show the upper expression (which we obtained
  from the commutation relations) from the lower expression, which is
  a part of the Feynman's propagator.

  We first note that

  16) q^2 - M^2 = (q^0)^2 - |qvec|^2 - M^2
                = (q^0)^2 - E(qvec)^2
                = [(q^0) + E(qvec)] [(q^0) - E(qvec)]

  and hence the the function

  16') 1/(q^2-M^2)

  has two poles in the complex q^0 plane.  Feynman's +i0 prescription,
  moves the pole from the real q^0 axis to the complex plane,

            1                     1
  17) --------------- = ------------------------
      q^2 - M^2 + i0    (q^0)^2 - E(qvec)^2 + i0
                                  1
                      = ------------------------
                        (q^0 + E)(q^0 - E) + i0
                                  1
                      = --------------------------
                        (q^0 + E -i0)(q^0 - E +i0)

  since E=E(qvec)>0.  Now, the region of the q^0 integral in the d^4 q is

  18) \Int_{-\infty}^{+\infty d q^0

  we want to close this q^0 integration contour such that the contribution
  from the far distance |q^0| \to \infty is regular, and the q^0 = E pole
  is inside.  Because

  19) (x^0-y^0) > 0

  due to the Theta function, the exponential function

  20) e^{-i(q^0)(x^0-y^0)}

  is regular in the lower (negative imaginary part) half plane.   We can
  close the contour by starting from -\infty in the real q^0 axis, move
  to +\infty along the real axis, then make a large half circle in the
  negative imaginary plane, and goes back to the original -\infty on the
  real axis.  This is a counter-clockwise circle (which gives a negative
  sign when applying the residue theorem).  Now, the only pole which
  exists inside this contour is the one at

  21) q^0 = E-i0

  The residue theorem tells

  22) \Int_{clockwise} dz f(z) = 2\pi i c_{-1}

  and hence

  23) \Int_{counter-clockwise} dz f(z) = 2\pi (-i) c_{-1}

  Here c_n is the n'th pole in the expansion

  24) f(z) = \Sum_{n=all integers} c_n (z-z0)^n

  By applying the residue theorem (23), we find

  25)                                       1
  \Int_{counter-clockwise} dq^0 --------------------------
                                (q^0 + E -i0)(q^0 - E +i0)
                                          1
  = \Int_{counter-clockwise} dq^0 ------------------
                                   2E (q^0 - E +i0)
  = 2\pi (-i)/(2E)

  Inserting this into the bottom expression of (15) gives the starting
  expression of (15).

hw20-6:  Please confirm the above proof by yourself, and show that
  exactly the same propagator is found for the y^0-x^0 > 0 case of
  eq.(11):

  26) \Theta(y^0-x^0) <0| W^+_\nu(y) W^-_\mu(x) |0>
  = \Theta(y^0-x^0)
    \Int d^3q1/(2pi)^32E(q1) e^{-iq1(y-x)}_{q1^0=E(q1)}
    \Sum_{h1} \eps_\nu(q1,h1) \eps_\mu(q1,h1)^*
  = \Theta(y^0-x^0)
    \Int d^3q1/2E(q1) e^{-iq1(y-x)}_{q1^0=E(q1)}
    (-g_{\mu\nu} + q1_\mu q1_\nu/m_W^2)

  where the anti-particle (W^+ in our case) is produced at x and
  annihilates at y.  Since the only difference is the exchange of x
  and y everywhere, we find, instead of (15)

  27) \Theta(y^0-x^0)
               d^3 q
      \Int -------------- e^{-iq(y-x)}_{q^0=E(q)}
           (2pi)^3 2E(q)
  =
      \Theta(y^0-x^0)
               d^4 q             i
      \Int  ------------  ---------------  e^{-iq(y-x)}
              (2pi)^4      q^2 - M^2 +i0

  We can now find the whole expression for the propagator (11):

  28) <0| T W^-_\mu(x) W^+_\nu(y) |0>
  = \Theta(x^0-y^0) <0| W^-_\mu(x) W^+_\nu(y) |0>
  + \Theta(y^0-x^0) <0| W^+_\nu(y) W^-_\mu(x) |0>
          d^4 q       i                          q_\mu q_\nu
  = \Int ------- --------------- (-g_{\mu\nu} + -------------)
         (2pi)^4  q^2 - M^2 + i0                    M^2
    \times { \Theta(x^0-y^0) e^{-iq(x-y)
           + \Theta(y^0-x^0) e^{-iq(y-x) }

  Inserting (28) into our amplitude (10), we find

  29) iT_{fi}
  = <f| (1/2!) T (i\Int d^4x L_int(x))^2 |i>         + ...
  = <f| T [i\Int d^4x W^-_\mu(x) J_{Wln}^\mu(x)]
          [i\Int d^4y W^+_\nu(y) J_{Wnl}^\nu(y)] |i> + ...

  = \Int d^4x \Int d^4y
    \ubar(e:p1) \gamma^\mu (-ig_L^{Wln}) P_L v(nue:p2) e^{-ix(-p1-p2)}
    \ubar(num:p3) \gamma^\nu (-ig_L^{Wnl}) P_L u(mu:p,h) e^{-iy(p-p3)}
    { \Theta(x^0-y^0) e^{-iq(x-y) + \Theta(y^0-x^0) e^{-iq(y-x) }
          d^4 q       i                          q_\mu q_\nu
    \Int ------- --------------- (-g_{\mu\nu} + -------------)
         (2pi)^4  q^2 - M^2 + i0                    M^2

  = \ubar(e:p1) \gamma^\mu (-ig_L^{Wln}) P_L v(nue:p2)
    \ubar(num:p3) \gamma^\nu (-ig_L^{Wnl}) P_L u(mu:p,h)
    { \Theta(x^0-y^0) (2pi)^8 \delta^4(q-p1-p2) \delta^4(p-q3-q)
    + \Theta(y^0-x^0) (2pi)^8 \delta^4(q+p1+p2) \delta^4(p-p3+q) }
          d^4 q       i                          q_\mu q_\nu
    \Int ------- --------------- (-g_{\mu\nu} + -------------)
         (2pi)^4  q^2 - M^2 + i0                    M^2

  = \ubar(e:p1) \gamma^\mu (-ig_L^{Wln}) P_L v(nue:p2)
    \ubar(num:p3) \gamma^\nu (-ig_L^{Wnl}) P_L u(mu:p,h)
    { \Theta(x^0-y^0) (2pi)^4 \delta^4(p-p3-p1-p2)
    + \Theta(y^0-x^0) (2pi)^4 \delta^4(-p+p3+p1+p2) }
            i                          (p1+p2)_\mu (p1+p2)_\nu
    -------------------- (-g_{\mu\nu} + -----------------------)
    (p1+p2)^2 - M^2 + i0                         M^2

  = (2pi)^4 \delta^4(p-p1-p2-p3)
    \ubar(e:p1) \gamma^\mu (-ig_L^{Wln}) P_L v(nue:p2)
    \ubar(num:p3) \gamma^\nu (-ig_L^{Wnl}) P_L u(mu:p,h)
            i                          (p1+p2)_\mu (p1+p2)_\nu
    -------------------- (-g_{\mu\nu} + -----------------------)
    (p1+p2)^2 - M^2 + i0                         M^2

  since

  30) \Theta(x^0-y^0) + \Theta(y^0-x^0) = 1.

hw20-7: Please follow my long derivation of eq.(29).

  It is indeed amazing that the contributions of the W^- propagation and
  W^+ propagation sum up to the above extremely simple formula.

  Let me try to follow the time-ordered perturbation theory calculation
  for the above example.

  When x^0>y^0, the term L_int(y) in eq.(06) annihilates \mu, and creates
  \nu_\mu and W^- at the spacetime point y^mu, and then L_int(x)
  annihilates W^-, creates e and \nubar_e at the spacetime point x^\mu.
  The process may be expressed as (the time flows from left to right):

  <-- <-- <-- <-- <-- <-- <-- <-- time <-- <-- <-- <-- <-- <-- <-- <--

  \nu_\mu <----------------------------------------- y <---------  \mu
                                                   / W^-
                                                /    is created
                                             /       at y^\mu
                                          /
                                       /
  e       <------------------------- x W^- is annihilated
                                   /       at x^\mu
                                /
                             /
                          /
  \nubar_e  <--------- /

  The propagation of W^- from y^\mu to x^\mu is shown.

  Now, when y^0 > x^0, the ordering of the two operators change, and
  the process can be depicted as

  <-- <-- <-- <-- <-- <-- <-- <-- time <-- <-- <-- <-- <-- <-- <-- <--

  \nu_\mu <--------- y <-----------------------------------------  \mu
                       -  W^+ is annihilated
                          -    at y^\mu
                             -
                                 -
                                     -
  e       <---------------------------- x W^+ is created
                                     /       at x^\mu
                                 /
                             /
                          /
  \nubar_e  <--------- /

  The propagation of W^+ from x^\mu to y^\mu is shown.

  Therefore the S-matrix elements are the sum of the amplitudes depicted
  by the above two plots:  One where W^- is created first at y^\mu by
  annihilating \mu and creating \nu_\mu, and this W^- is annihilated
  at a later time at x^\mu, creating e and \nubar_e.  The other one is
  when W^+ is created with e and \nubar_e from the vacuum at x^\mu,
  and at a later time, W^+ is annihilated at y^\mu, where \mu is also
  annihilated and \nu_\mu is created.

  Before Feynman, we calculated the above two contributions separately,
  and their sum is the transition amplitudes.  Feynman showed that the
  sum of the above two contributions can be expressed by a single
  amplitudes with Feynman's propagator (I use a scalar boson propagator
  to avoid the polarization sum term):

  31) <0| T \Phi(x) \Phi^*(y) |0>
  = \Theta(x^0-y^0) <0| \Phi(x) \Phi^*(y) |0>
  + \Theta(y^0-x^0) <0| \Phi^*(y) \Phi(x) |0>
  =
                         d^4 q         i
  \Theta(x^0-y^0) \int  -------  -------------  e^{-iq(x-y)}
                        (2pi)^4  q^2 - M^2 +i0
  +
                         d^4 p         i
  \Theta(y^0-x^0) \int  -------  -------------  e^{-ip(y-x)}
                        (2pi)^4  p^2 - M^2 +i0

  =
                         d^4 q         i
  \Theta(x^0-y^0) \int  -------  -------------  e^{-iq(x-y)}
                        (2pi)^4  q^2 - M^2 +i0
  +
                         d^4 q         i
  \Theta(y^0-x^0) \int  -------  -------------- e^{-iq(x-y)}
                        (2pi)^4  q^2 - M^2 +i0
  =
                         d^4 q         i
                        -------  -------------  e^{-iq(x-y)}
                        (2pi)^4  q^2 - M^2 +i0

  In the next-to-the-last step above, I set p=-q.  Amazingly, the two
  contributions, the particle propagation from y^\mu to x^\mu when
  x^0>y^0, and the anti-particle propagation from x^\mu to y^\mu when
  y^0>x^0, sum up to a very simple term which is now called Feynman.

  Feynman's discovery allows us to forget about the time-ordering of
  the operators in the evaluation of the S matrix elements, because
  the only operator products whose v.e.v. is non-vanishing are expressed
  as bi-linear combination of the same field operator, which can all be
  expressed as above, except for the polarization sum factors which do
  not depend on the operator ordering.  By using the Feynman propagator,
  the two `processes' depicted above can be expressed by one `diagram'

  <-- <-- <-- <-- <-- <-- <-- <-- time <-- <-- <-- <-- <-- <-- <-- <--

  \nu_\mu <----------------------------------------- y <---------  \mu
                                                   /
                                                /
                                             / Feynman's propagator
                                          /    of W^\pm
                                       /
  e       <------------------------- x
                                   /
                                /
                             /
                          /
  \nubar_e  <--------- /

  where the initial state at t=-\infty and the final state at t=+\infty
  are still ordered, but the time-ordering between x^\mu and y^\mu no
  more exists.  The Feynman's propagator above gives the sum of

  W^- is produced at y^\mu and annihilated at x^\mu when x^0>y^0

  and

  W^+ is produced at x^\mu and annihilated at y^\mu when y^0>x^0

  Therefore, when we draw Feynman diagrams, there is no time-ordering
  between the interaction points.  More accurately, in the n-th order,
  the total sum of all the possible time-ordering among n-interaction
  points (n! cases) can be expressed by just one diagram !!!

  No wonder how everybody was delighted by Feynman's discovery, and
  shortly after his work, everybody started calling our transition
  matrix elements as Feynman amplitudes, and the corresponding
  plots as above as Feynman diagrams.

                                   *****

  Now with our standard amplitude convention,

  32)
  iT_{fi} = (2pi)^4 \delta^4(\Sum{p_i}-\Sum{p_f}) iM_{fi}

  the Feynman amplitudes for the process (02) reads

  33) iM_{fi}
  = \ubar(e:p1)   \gamma^\mu (-ig_L^{Wln}) P_L v(nue:p2)
    \ubar(num:p3) \gamma^\nu (-ig_L^{Wnl}) P_L u(mu:p,h)
            i                          (p1+p2)_\mu (p1+p2)_\nu
    -------------------- (-g_{\mu\nu} + -----------------------)
    (p1+p2)^2 - M^2 + i0                         M^2

  = (-ig_L^{Wln}) uL(e:p1)^\dagger   \sigma_-^\mu vL(nue:p2)
    (-ig_L^{Wnl}) uL(num:p3)^\dagger \sigma_-^\nu uL(mu:p,h)
            i                          (p1+p2)_\mu (p1+p2)_\nu
    -------------------- (-g_{\mu\nu} + -----------------------)
    (p1+p2)^2 - M^2 + i0                         M^2

  We call the expression (33) as the Feynman amplitude, which follows
  from the Feynman rules.  As expected, the factor of i cancel (it is
  important to note that Feynman's propagator has a factor of i, coming
  from the residue theorem).  After removing the common factor of i,
  we find

  34) M_{fi}
  = -(g_L^{Wln} g_L^{Wnl})
    uL(e:p1)^\dagger   \sigma_-^\mu vL(nue:p2)
    uL(num:p3)^\dagger \sigma_-^\nu uL(mu:p,h)
            1                          (p1+p2)_\mu (p1+p2)_\nu
    -------------------- (-g_{\mu\nu} + -----------------------)
    (p1+p2)^2 - M^2 + i0                         M^2

  which is usually the starting point of our amplitude calculation.

hw20-8: Obtain (34) directly from the `Feynman rule'.

  Let us make a few simplification.  Since we do not distinguish the
  three neutrino mass eigenstates in this measurements (neutrinos in
  the final states are measured only through their missing energy-momenta,
  whose accuracy cannot distinguish tiny mass-squared differences of
  the order of (0.1eV)^2), we can ignore the MNS mixing terms.

  35) g_L^{Wln} = g_L^{Wnl} = g_W/\rt2

  Next, since we ignore the electron mass, the second term in the W boson
  polarization vector sum do not contribute:

  36)
  uL(e:p1)^\dagger \sigma_-^\mu vL(nue:p2) (p1+p2)_\mu = 0.

hw20-9: Show (36) by using the Dirac equations.

  Third, since

  37) (p1+p2)^2 < m_\mu^2 << M^2 (M=m_W)

  the on-shell pole of the Feynman propagator can never be reached.
  Therefore +i0 can be set to zero.

  After these simplification, the amplitude reads

  38) M_{fi}
  = (g_W/\rt2)^2/[(p1+p2)^2 - M^2]
    uL(e:p1)^\dagger   \sigma_-^\mu vL(nue:p2)
    uL(num:p3)^\dagger \sigma_-_\mu uL(mu:p,h)

  We can further drop the (p1+p2)^2 piece in the W propagator because

  39) (p1+p2)^2/M^2 < (m_\mu/m_W)^2 = (0.106/80.4)^2 = 2*10^{-6}

  which is smaller than the experimental error of the muon life-time.

  40) M_{fi}
  = -g_W^2/(2m_W^2)
    uL(e:p1)^\dagger   \sigma_-^\mu vL(nue:p2)
    uL(num:p3)^\dagger \sigma_-_\mu uL(mu:p,h)

  We can now calculate the above amplitudes, and obtain the muon life
  time by integrating their absolute value square's over the three body
  phase space.

                          *****

That's all for hw20.

Best regards,

Kaoru



\end{document}
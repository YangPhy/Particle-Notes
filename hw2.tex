\documentclass[12pt]{article}
\usepackage{amsmath}
\textheight=9.5in \voffset=-1.0in \textwidth=6.5in \hoffset=-0.5in
\parskip=0pt
\begin{document}

\begin{center}
{\large\bf HW2 for Advanced Particle Physics} \\

\end{center}

\vskip 0.2 in

Dear students:\\

Please let me first give a few comments from hw01:\\

\begin{enumerate}
  \item When you read/quote experimental data, such as
\begin{eqnarray}
\tau(\mu)&=& (1.2969811\pm 0.0000022) \times 10^{-6} s\\
        &=& \rm mean \pm \rm error
\end{eqnarray}
please always note that the accuracy of the measurement is\\
\begin{eqnarray}                 
\rm accuracy = \frac{\rm error}{\rm mean}                 
\end{eqnarray}
which cannot change when we change the units from second
to 1/GeV or to cm.  Also, please note that the digits
after the decimal point should be the same between the
mean and the error.  For instance,\\
\begin{eqnarray} 
100 \pm 0.01
\end{eqnarray} 
is not acceptable in physics.  If the error is +-0.01,
then, the mean should be given as\\
\begin{eqnarray} 
100.00 \pm 0.01
\end{eqnarray} 
The accuracy of 1/10000.  There is another standard
method to quote errors, which reads\\
\begin{eqnarray} 
\tau(\mu) = 1.2969811(22) \times 10^{-6} s
\end{eqnarray} 
The rule is that the number in () gives the 1-sigma
error for the last digits of the mean.  In case of
my example above, it becomes 100.00(1).
I like this method since I don't need to count the
number of 0's for high precision data.\\

In the above I expressed the ``error'' as ``1-sigma error''
Here 1-sigma stands for one-standard-deviation, in
the Gaussian probability distribution,\\
\begin{eqnarray} 
P(x) = 1/\sqrt{2\pi\sigma^2} \exp\{-(x-\mu)^2/2\sigma^2\}
\end{eqnarray} 
which is normalized to 1,\\
\begin{eqnarray} 
\int_{-\infty}^{\infty} dx ~ P(x) = 1
\end{eqnarray} 

{\bf hw02-1}: prove the above identity
and the mean of x is $\mu$,
\begin{eqnarray} 
\langle x\rangle = \int_{-\infty}^{\infty}dx~ xP(x) = \mu
\end{eqnarray} 

{\bf hw02-2}: prove the above identity\\

The probability of x being inside the 1-sigma range,
\begin{eqnarray} 
\mu -\sigma < x < \mu +\sigma
\end{eqnarray} 
is about $67\%$.  When the measurement tells that the true
value of x is in this range, we write\\
\begin{eqnarray} 
x = \mu \pm \sigma.
\end{eqnarray} 

{\bf hw02-3}: please try to show that the above estimate of $67\%$ is fine.\\
\begin{itemize}
  \item hint: If you know numerical integration method, please obtain
  the probability numerically up to 3 digits.  If you don't
  know it, you may draw a curve of P(x) vs x, and try to
  estimate the area under P(x) approximately.  In this
  case, if you can obtain say, about 2/3, it is fine.
\end{itemize}


\item When a particle decays in flight, its mean decay-in-flight
length is\\
\begin{eqnarray} 
\langle l\rangle = \tau c \gamma \beta,
\end{eqnarray} 
where $\gamma = E/m$, $\beta = \sqrt{1-(1/\gamma)^2}$.
Here are excercises:

{\bf hw02-4}:
Cosmic rays hit the atmosphere at about 20km above
the ground/sea level, and produce many $\pi$, which
decay in flight.\\
\begin{eqnarray} 
&&\pi^+ \to \mu^+ + \nu_\mu \\
&&\pi^- \to \mu^- + {\overline \nu_\mu} \\
&&\pi^0 \to \gamma \gamma \\
\end{eqnarray} 
Charged pion has a long lifetime, and hence flies
significantly before decay.  If charged $\pi$ energy
is 1.4GeV ($\gamma = E/m = 10$), how long does it fly
before it decays into $\mu$ and neutrino ?\\

{\bf hw02-5}:
Let us assume that muon from $\pi$ decay has energy
1GeV (again $\gamma = 10$).  How long does it fly
before it decays ?   What is the energy of muon
for which the mean decay-in-flight distance is 20km ?\\

{\bf hw02-6}:
Super-Kamiokande, the world largest Water Cerenkov
neutrino detector started its operation in 1996,
20 years ago. Its fiducial volume is 22 kton.\\

Let us assume that the proton decays mainly into
\begin{eqnarray} 
p \to e^+ + \pi^0
\end{eqnarray} 
with the lifetime of
\begin{eqnarray} 
\tau(p) = 8 \times 10^{33} ~\rm yrs
\end{eqnarray} 
which is roughly the present $90\%$ CL lower bound from
the PDG table.\\

Here $90\%$ CL lower bound means that the probability
that the proton life time is longer than
$8 \times 10^{33}$ years is $90\%$.  In other words,
the probability that it is shorter than this value
is $10\%$. \\

How many protons in Super-Kamiokande have decayed
in the past 20 years ? \\

Hint: first count the number of protons in 22 kton
of water.  Then I think the rest is straightforward.\\

This number tells the detection efficiency and the
background rejection capability of the detector.
\end{enumerate}

{\bf Now, here is the homework for the 2nd lecture.}\\

Today, I introduced fundamental particles of the SM.
I explained that all the interactions of the fundamental
particles in the SM have definite quantum numbers (generalized
charges) under the symmetry group
$SU(3)_{\rm color} \times SU(2)_{\rm left} \times U(1)_Y$,
where I will explain each gauge group carefully in the
following lectures.\\

Please let me start by reviewing the group U(1),
which is the group that governs the electromagnetic
interactions (dictated by Maxwell's equations).\\

Let us start with a complex number
\begin{eqnarray} 
z = x + iy,
\end{eqnarray} 
where x and y are real numbers, and $i=\sqrt{-1}$ is the
unit of imaginary part.  The U(1) transformation is
defined as the transformation of z to another complex
number z', which is obtained by the operation,
\begin{eqnarray} 
z \to z' = e^{i\theta} z
\end{eqnarray} 
where $\theta$ is a real number (angle).  The real number
$\theta$ is called angle, because the transformation
$e^{i2\pi} = 1$
is the identity.  Because of this, the operation is periodic
in $\theta$,
\begin{eqnarray} 
e^{i(\theta + 2\pi)} = e^{i\theta}
\end{eqnarray} 
and the transformation is determined by $\theta$ in its
`defined region',
\begin{eqnarray} 
0 \leq \theta < 2\pi
\end{eqnarray} 
or any other region whose ``length'' is just $2\pi$.\\
We immediately notice that under the above transformation,
the norm (abolute value) of z remain un-changed, i.e. $|z'|^2 = |z|^2$.\\

{\bf hw02-7}: prove the above identity.\\

The conservation of the norm is called Unitarity (or the
conservation of Probability, since in quantum mechanics,
the absolute value squared of the amplitudes corresponds
to the probability).  When there is one complex number
involved in the transformation, the group is called $U(1)$.\\

Let us show that U(1) transformation is the same as the
rotation inside a plane.  Let us start from the identity
\begin{eqnarray} 
e^{i\theta} = \cos\theta + i\sin\theta
\end{eqnarray} 
{\bf hw02-8}: prove the above identity.\\

hint: there are many ways to prove it, but I usually use the
power expansion of the functions:
\begin{eqnarray} 
e^{ix} = \sum_{n=0}^{n=\infty} x^n/n!
\end{eqnarray} 
where we split the sum into even (n=2m) and odd (n=2m+1)
power terms,
\begin{eqnarray} 
e^{ix}&=&\sum_{n=0}^{n=\infty} (ix)^n/n!\\
&=&\sum_{m=0}^{m=\infty}  (ix)^{2m}/{2m}! + (ix)^{2m+1}/{2m+1}!\\
&=&\sum_{m=0}^{m=\infty} x^{2m}/{2m}!+i \sum_{m=0}^{n=\infty} x^{2m+1}/{2m+1}!
\end{eqnarray}
and then show that the first term is the expansion of $\cos(x)$
and the latter is that of $i\sin(x)$.\\

{\bf hw02-9}: When we write
\begin{eqnarray} 
z &=& x + iy\\
z' &=& x' + iy'
\end{eqnarray} 
with real numbers x,y,x',y', and when z' are obtained from
z by the U(1) transformation,
$z \to z' = e^{i\theta} z$
please obtain the transformation from (x,y) to (x',y') as
\begin{equation}
  \label{eq:3}
  (x,y)^T \to (x',y')^T = O (x,y)^T
\end{equation}
with 2x2 real matrix O.  Here, $^T$ stands for `transpose',
meaning
\begin{equation}
  \label{eq:1}
  (x,y)^T =
  \begin{pmatrix}
    x\\y
  \end{pmatrix}
\end{equation}
I often use this notation in my homework, since it is
difficult to write the column vector in text.  I write
the row vector, and put $^T$, to show that it is a column
vector.  In short, the above expression means
\begin{equation}
  \label{eq:2}
  \begin{pmatrix}
    x\\y 
  \end{pmatrix} \rightarrow
  \begin{pmatrix}
    x'\\y'
  \end{pmatrix} = 
  \begin{pmatrix}
    O_{11}&O_{12}\\
    O_{21}&O_{22}
  \end{pmatrix}
  \begin{pmatrix}
    x\\y
  \end{pmatrix}
\end{equation}
{\bf hw02-10}: Show
\begin{equation}
  \label{eq:4}
  O^T = O^{-1}
\end{equation}
or
\begin{equation}
  \label{eq:5}
O^T O = 1  
\end{equation}


The above property of the 2x2 transformation matrix
with real numbers is called Orthogonal.  The group
generated by the Orthogonal 2x2 matrix is called SO(2).\\

Here, O stands for orthogonal, while S stands for
special, meaning
$\det{O} = 1.$
This ``special'' condition tells that all the transformations
can be obtained from the identity transformation,
\begin{equation}
  \label{eq:6}
  I = {\rm diag}\{1,1\} =
  \begin{pmatrix}
    1&0\\
    0&1
  \end{pmatrix}
\end{equation}
by using the genrator formalism:
\begin{equation}
  \label{eq:7}
O(\theta) = e^{i J \theta} I
\end{equation}  
{\bf hw02-11}: Show
\begin{equation}
  \label{eq:8}
J = \sigma^2 =
\begin{pmatrix}
  0&-i\\
  i&0
\end{pmatrix}
\end{equation}
gives all the orthogonal transformation matrix which
corresponds to U(1) transformation.\\

hint: Please remember that the $\exp$ (or any
funcation) of a matrix is defined by power expansion:
\begin{equation}
  \label{eq:9}
e^{A} = \sum_{n=0}^{n=\infty} A^n/n!,
\end{equation}
where $A^n$ can be calculated as matrix multiplications.

hint2: If $A^2$ or $A^4$ is proportional to the identity, $I$,
then we can sum over all the n to infinity to each
component of the marrix.\\

In short, with the above homework, we showed that
the group U(1) is the same as the group SO(2).\\

Let me remind you of the mathmatical definition of
a group.  A group has elements,
$a, b, c, ...$
which has a multiplication rule
\begin{eqnarray}
  a\times b = c,
\end{eqnarray}
where an arbitrary product of two elements of the group
is an element of the group.  The multiplication rule
usually satisfy
\begin{eqnarray}
(a\times b)\times c = a\times (b\times c).
\end{eqnarray}
The element should contain identity, $I$, for which
\begin{eqnarray}
a\times I = I\times a
\end{eqnarray}
for all elements, and also for all element a, there
exists its inverse element, $a^{-1}$, which satisfy
\begin{eqnarray}
a^{-1} \times a = I.
\end{eqnarray}

{\bf hw02-12}: Show that all the above properties are satisfied
by the element of U(1) transformation,
$U(\theta) = e^{i\theta}$
and also by the element of SO(2) transformation,
\begin{equation}
O(\theta) =
\begin{pmatrix}
  *&*\\
  *&*
\end{pmatrix}
\end{equation}
Please show the identity transformation for both U(1)
and SO(2), and also the inverse transformations.\\

Now, let me go on to the next complex Unitary group,
SU(2). Just like the transformation U(1) can be
expressed compactly as a transformation on a complex
number, SU(2) can be expressed compactly as a transformation
of a set of two complex numbers:
\begin{equation}
  \label{eq:10}
  (z_1, z_2)^T \to (z_1',z_2')^T = U (z_1,z_2)^T
\end{equation}
This looks similar to the SO(2) transformation as I
introduced above, but now, both $z_1$ and $z_2$ are complex
numbers, and the transformation U is a $2\times 2$ matrix
of complex numbers.  The transformation is called
SU(2), Special Unitary transformation with 2 complex
degrees of freedom when
\begin{equation}
  \label{eq:11}
U^\dagger = U^{-1}~\text{or} ~ U^\dagger U = 1 ~~\rm (Unitary)  
\end{equation}
and
$\det{\{U\}} = 1$ (Special).\\

{\bf hw02-13}: Show that the SU(2) transformation preserves
the norm of complex vectors $(z_1,z_2)^T$,
\begin{eqnarray}
(z_1,z_2)^* (z_1,z_2)^T = |z_1|^2 + |z_2|^2
\end{eqnarray}
Please note that we usually introduce a doublet of
two complex number
\begin{eqnarray}
z = (z_1,z_2)^T
\end{eqnarray}
and for this column vector z, the unitarity condition is
\begin{eqnarray}
z'^\dagger z' = z^\dagger z.
\end{eqnarray}\\


Please let me stop here.\\

Best regards,\\

Kaoru



\end{document}
